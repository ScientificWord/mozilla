
\documentclass{article}
%%%%%%%%%%%%%%%%%%%%%%%%%%%%%%%%%%%%%%%%%%%%%%%%%%%%%%%%%%%%%%%%%%%%%%%%%%%%%%%%%%%%%%%%%%%%%%%%%%%%%%%%%%%%%%%%%%%%%%%%%%%%%%%%%%%%%%%%%%%%%%%%%%%%%%%%%%%%%%%%%%%%%%%%%%%%%%%%%%%%%%%%%%%%%%%%%%%%%%%%%%%%%%%%%%%%%%%%%%%%%%%%%%%%%%%%%%%%%%%%%%%%%%%%%%%%
\usepackage{amssymb}
\usepackage{amsmath}
\usepackage{revsymb}
\usepackage{eurosym}
\usepackage{graphicx}

\setcounter{MaxMatrixCols}{10}
%TCIDATA{OutputFilter=LATEX.DLL}
%TCIDATA{Version=5.00.0.2606}
%TCIDATA{<META NAME="SaveForMode" CONTENT="1">}
%TCIDATA{BibliographyScheme=Manual}
%TCIDATA{LastRevised=Monday, September 27, 2004 16:26:10}
%TCIDATA{<META NAME="ViewSettings" CONTENT="31">}
%TCIDATA{<META NAME="GraphicsSave" CONTENT="32">}
%TCIDATA{CSTFile=LaTeX article (bright).cst}
%TCIDATA{PageSetup=72,72,72,72,0}
%TCIDATA{EvenPages=
%H=36,\PARA{038<p type="texpara" tag="Body Text" >samplearticle.tex}
%F=36,\PARA{038<p type="texpara" tag="Body Text" >\hfill Page \thepage \hfill }
%}

%TCIDATA{OddPages=
%H=36,\PARA{038<p type="texpara" tag="Body Text" >\hfill samplearticle.tex}
%F=36,\PARA{038<p type="texpara" tag="Body Text" >\hfill Page \thepage \hfill }
%}

%TCIDATA{FirstPage=
%H=36
%F=36
%}


\newtheorem{theorem}{Theorem}
\newtheorem{acknowledgement}[theorem]{Acknowledgement}
\newtheorem{algorithm}[theorem]{Algorithm}
\newtheorem{axiom}[theorem]{Axiom}
\newtheorem{case}[theorem]{Case}
\newtheorem{claim}[theorem]{Claim}
\newtheorem{conclusion}[theorem]{Conclusion}
\newtheorem{condition}[theorem]{Condition}
\newtheorem{conjecture}[theorem]{Conjecture}
\newtheorem{corollary}[theorem]{Corollary}
\newtheorem{criterion}[theorem]{Criterion}
\newtheorem{definition}[theorem]{Definition}
\newtheorem{example}[theorem]{Example}
\newtheorem{exercise}[theorem]{Exercise}
\newtheorem{lemma}[theorem]{Lemma}
\newtheorem{notation}[theorem]{Notation}
\newtheorem{problem}[theorem]{Problem}
\newtheorem{proposition}[theorem]{Proposition}
\newtheorem{remark}[theorem]{Remark}
\newtheorem{summary}[theorem]{Summary}
\newenvironment{proof}[1][]{\textbf{Proof.} }{}
% Macros for Scientific Word and Scientific WorkPlace 5.5 documents saved with the LaTeX filter.
% Copyright (C) 2005 Mackichan Software, Inc.

\typeout{TCILATEX Macros for Scientific Word and Scientific WorkPlace 5.5 <06 Oct 2005>.}
\typeout{NOTICE:  This macro file is NOT proprietary and may be 
freely copied and distributed.}
%
\makeatletter

%%%%%%%%%%%%%%%%%%%%%
% pdfTeX related.
\ifx\pdfoutput\relax\let\pdfoutput=\undefined\fi
\newcount\msipdfoutput
\ifx\pdfoutput\undefined
\else
 \ifcase\pdfoutput
 \else 
    \msipdfoutput=1
    \ifx\paperwidth\undefined
    \else
      \ifdim\paperheight=0pt\relax
      \else
        \pdfpageheight\paperheight
      \fi
      \ifdim\paperwidth=0pt\relax
      \else
        \pdfpagewidth\paperwidth
      \fi
    \fi
  \fi  
\fi

%%%%%%%%%%%%%%%%%%%%%
% FMTeXButton
% This is used for putting TeXButtons in the 
% frontmatter of a document. Add a line like
% \QTagDef{FMTeXButton}{101}{} to the filter 
% section of the cst being used. Also add a
% new section containing:
%     [f_101]
%     ALIAS=FMTexButton
%     TAG_TYPE=FIELD
%     TAG_LEADIN=TeX Button:
%
% It also works to put \defs in the preamble after 
% the \input tcilatex
\def\FMTeXButton#1{#1}
%
%%%%%%%%%%%%%%%%%%%%%%
% macros for time
\newcount\@hour\newcount\@minute\chardef\@x10\chardef\@xv60
\def\tcitime{
\def\@time{%
  \@minute\time\@hour\@minute\divide\@hour\@xv
  \ifnum\@hour<\@x 0\fi\the\@hour:%
  \multiply\@hour\@xv\advance\@minute-\@hour
  \ifnum\@minute<\@x 0\fi\the\@minute
  }}%

%%%%%%%%%%%%%%%%%%%%%%
% macro for hyperref and msihyperref
%\@ifundefined{hyperref}{\def\hyperref#1#2#3#4{#2\ref{#4}#3}}{}

\def\x@hyperref#1#2#3{%
   % Turn off various catcodes before reading parameter 4
   \catcode`\~ = 12
   \catcode`\$ = 12
   \catcode`\_ = 12
   \catcode`\# = 12
   \catcode`\& = 12
   \catcode`\% = 12
   \y@hyperref{#1}{#2}{#3}%
}

\def\y@hyperref#1#2#3#4{%
   #2\ref{#4}#3
   \catcode`\~ = 13
   \catcode`\$ = 3
   \catcode`\_ = 8
   \catcode`\# = 6
   \catcode`\& = 4
   \catcode`\% = 14
}

\@ifundefined{hyperref}{\let\hyperref\x@hyperref}{}
\@ifundefined{msihyperref}{\let\msihyperref\x@hyperref}{}




% macro for external program call
\@ifundefined{qExtProgCall}{\def\qExtProgCall#1#2#3#4#5#6{\relax}}{}
%%%%%%%%%%%%%%%%%%%%%%
%
% macros for graphics
%
\def\FILENAME#1{#1}%
%
\def\QCTOpt[#1]#2{%
  \def\QCTOptB{#1}
  \def\QCTOptA{#2}
}
\def\QCTNOpt#1{%
  \def\QCTOptA{#1}
  \let\QCTOptB\empty
}
\def\Qct{%
  \@ifnextchar[{%
    \QCTOpt}{\QCTNOpt}
}
\def\QCBOpt[#1]#2{%
  \def\QCBOptB{#1}%
  \def\QCBOptA{#2}%
}
\def\QCBNOpt#1{%
  \def\QCBOptA{#1}%
  \let\QCBOptB\empty
}
\def\Qcb{%
  \@ifnextchar[{%
    \QCBOpt}{\QCBNOpt}%
}
\def\PrepCapArgs{%
  \ifx\QCBOptA\empty
    \ifx\QCTOptA\empty
      {}%
    \else
      \ifx\QCTOptB\empty
        {\QCTOptA}%
      \else
        [\QCTOptB]{\QCTOptA}%
      \fi
    \fi
  \else
    \ifx\QCBOptA\empty
      {}%
    \else
      \ifx\QCBOptB\empty
        {\QCBOptA}%
      \else
        [\QCBOptB]{\QCBOptA}%
      \fi
    \fi
  \fi
}
\newcount\GRAPHICSTYPE
%\GRAPHICSTYPE 0 is for TurboTeX
%\GRAPHICSTYPE 1 is for DVIWindo (PostScript)
%%%(removed)%\GRAPHICSTYPE 2 is for psfig (PostScript)
\GRAPHICSTYPE=\z@
\def\GRAPHICSPS#1{%
 \ifcase\GRAPHICSTYPE%\GRAPHICSTYPE=0
   \special{ps: #1}%
 \or%\GRAPHICSTYPE=1
   \special{language "PS", include "#1"}%
%%%\or%\GRAPHICSTYPE=2
%%%  #1%
 \fi
}%
%
\def\GRAPHICSHP#1{\special{include #1}}%
%
% \graffile{ body }                                  %#1
%          { contentswidth (scalar)  }               %#2
%          { contentsheight (scalar) }               %#3
%          { vertical shift when in-line (scalar) }  %#4

\def\graffile#1#2#3#4{%
%%% \ifnum\GRAPHICSTYPE=\tw@
%%%  %Following if using psfig
%%%  \@ifundefined{psfig}{\input psfig.tex}{}%
%%%  \psfig{file=#1, height=#3, width=#2}%
%%% \else
  %Following for all others
  % JCS - added BOXTHEFRAME, see below
    \bgroup
	   \@inlabelfalse
       \leavevmode
       \@ifundefined{bbl@deactivate}{\def~{\string~}}{\activesoff}%
        \raise -#4 \BOXTHEFRAME{%
           \hbox to #2{\raise #3\hbox to #2{\null #1\hfil}}}%
    \egroup
}%
%
% A box for drafts
\def\draftbox#1#2#3#4{%
 \leavevmode\raise -#4 \hbox{%
  \frame{\rlap{\protect\tiny #1}\hbox to #2%
   {\vrule height#3 width\z@ depth\z@\hfil}%
  }%
 }%
}%
%
\newcount\@msidraft
\@msidraft=\z@
\let\nographics=\@msidraft
\newif\ifwasdraft
\wasdraftfalse

%  \GRAPHIC{ body }                                  %#1
%          { draft name }                            %#2
%          { contentswidth (scalar)  }               %#3
%          { contentsheight (scalar) }               %#4
%          { vertical shift when in-line (scalar) }  %#5
\def\GRAPHIC#1#2#3#4#5{%
   \ifnum\@msidraft=\@ne\draftbox{#2}{#3}{#4}{#5}%
   \else\graffile{#1}{#3}{#4}{#5}%
   \fi
}
%
\def\addtoLaTeXparams#1{%
    \edef\LaTeXparams{\LaTeXparams #1}}%
%
% JCS -  added a switch BoxFrame that can 
% be set by including X in the frame params.
% If set a box is drawn around the frame.

\newif\ifBoxFrame \BoxFramefalse
\newif\ifOverFrame \OverFramefalse
\newif\ifUnderFrame \UnderFramefalse

\def\BOXTHEFRAME#1{%
   \hbox{%
      \ifBoxFrame
         \frame{#1}%
      \else
         {#1}%
      \fi
   }%
}


\def\doFRAMEparams#1{\BoxFramefalse\OverFramefalse\UnderFramefalse\readFRAMEparams#1\end}%
\def\readFRAMEparams#1{%
 \ifx#1\end%
  \let\next=\relax
  \else
  \ifx#1i\dispkind=\z@\fi
  \ifx#1d\dispkind=\@ne\fi
  \ifx#1f\dispkind=\tw@\fi
  \ifx#1t\addtoLaTeXparams{t}\fi
  \ifx#1b\addtoLaTeXparams{b}\fi
  \ifx#1p\addtoLaTeXparams{p}\fi
  \ifx#1h\addtoLaTeXparams{h}\fi
  \ifx#1X\BoxFrametrue\fi
  \ifx#1O\OverFrametrue\fi
  \ifx#1U\UnderFrametrue\fi
  \ifx#1w
    \ifnum\@msidraft=1\wasdrafttrue\else\wasdraftfalse\fi
    \@msidraft=\@ne
  \fi
  \let\next=\readFRAMEparams
  \fi
 \next
 }%
%
%Macro for In-line graphics object
%   \IFRAME{ contentswidth (scalar)  }               %#1
%          { contentsheight (scalar) }               %#2
%          { vertical shift when in-line (scalar) }  %#3
%          { draft name }                            %#4
%          { body }                                  %#5
%          { caption}                                %#6


\def\IFRAME#1#2#3#4#5#6{%
      \bgroup
      \let\QCTOptA\empty
      \let\QCTOptB\empty
      \let\QCBOptA\empty
      \let\QCBOptB\empty
      #6%
      \parindent=0pt
      \leftskip=0pt
      \rightskip=0pt
      \setbox0=\hbox{\QCBOptA}%
      \@tempdima=#1\relax
      \ifOverFrame
          % Do this later
          \typeout{This is not implemented yet}%
          \show\HELP
      \else
         \ifdim\wd0>\@tempdima
            \advance\@tempdima by \@tempdima
            \ifdim\wd0 >\@tempdima
               \setbox1 =\vbox{%
                  \unskip\hbox to \@tempdima{\hfill\GRAPHIC{#5}{#4}{#1}{#2}{#3}\hfill}%
                  \unskip\hbox to \@tempdima{\parbox[b]{\@tempdima}{\QCBOptA}}%
               }%
               \wd1=\@tempdima
            \else
               \textwidth=\wd0
               \setbox1 =\vbox{%
                 \noindent\hbox to \wd0{\hfill\GRAPHIC{#5}{#4}{#1}{#2}{#3}\hfill}\\%
                 \noindent\hbox{\QCBOptA}%
               }%
               \wd1=\wd0
            \fi
         \else
            \ifdim\wd0>0pt
              \hsize=\@tempdima
              \setbox1=\vbox{%
                \unskip\GRAPHIC{#5}{#4}{#1}{#2}{0pt}%
                \break
                \unskip\hbox to \@tempdima{\hfill \QCBOptA\hfill}%
              }%
              \wd1=\@tempdima
           \else
              \hsize=\@tempdima
              \setbox1=\vbox{%
                \unskip\GRAPHIC{#5}{#4}{#1}{#2}{0pt}%
              }%
              \wd1=\@tempdima
           \fi
         \fi
         \@tempdimb=\ht1
         %\advance\@tempdimb by \dp1
         \advance\@tempdimb by -#2
         \advance\@tempdimb by #3
         \leavevmode
         \raise -\@tempdimb \hbox{\box1}%
      \fi
      \egroup%
}%
%
%Macro for Display graphics object
%   \DFRAME{ contentswidth (scalar)  }               %#1
%          { contentsheight (scalar) }               %#2
%          { draft label }                           %#3
%          { name }                                  %#4
%          { caption}                                %#5
\def\DFRAME#1#2#3#4#5{%
  \vspace\topsep
  \hfil\break
  \bgroup
     \leftskip\@flushglue
	 \rightskip\@flushglue
	 \parindent\z@
	 \parfillskip\z@skip
     \let\QCTOptA\empty
     \let\QCTOptB\empty
     \let\QCBOptA\empty
     \let\QCBOptB\empty
	 \vbox\bgroup
        \ifOverFrame 
           #5\QCTOptA\par
        \fi
        \GRAPHIC{#4}{#3}{#1}{#2}{\z@}%
        \ifUnderFrame 
           \break#5\QCBOptA
        \fi
	 \egroup
  \egroup
  \vspace\topsep
  \break
}%
%
%Macro for Floating graphic object
%   \FFRAME{ framedata f|i tbph x F|T }              %#1
%          { contentswidth (scalar)  }               %#2
%          { contentsheight (scalar) }               %#3
%          { caption }                               %#4
%          { label }                                 %#5
%          { draft name }                            %#6
%          { body }                                  %#7
\def\FFRAME#1#2#3#4#5#6#7{%
 %If float.sty loaded and float option is 'h', change to 'H'  (gp) 1998/09/05
  \@ifundefined{floatstyle}
    {%floatstyle undefined (and float.sty not present), no change
     \begin{figure}[#1]%
    }
    {%floatstyle DEFINED
	 \ifx#1h%Only the h parameter, change to H
      \begin{figure}[H]%
	 \else
      \begin{figure}[#1]%
	 \fi
	}
  \let\QCTOptA\empty
  \let\QCTOptB\empty
  \let\QCBOptA\empty
  \let\QCBOptB\empty
  \ifOverFrame
    #4
    \ifx\QCTOptA\empty
    \else
      \ifx\QCTOptB\empty
        \caption{\QCTOptA}%
      \else
        \caption[\QCTOptB]{\QCTOptA}%
      \fi
    \fi
    \ifUnderFrame\else
      \label{#5}%
    \fi
  \else
    \UnderFrametrue%
  \fi
  \begin{center}\GRAPHIC{#7}{#6}{#2}{#3}{\z@}\end{center}%
  \ifUnderFrame
    #4
    \ifx\QCBOptA\empty
      \caption{}%
    \else
      \ifx\QCBOptB\empty
        \caption{\QCBOptA}%
      \else
        \caption[\QCBOptB]{\QCBOptA}%
      \fi
    \fi
    \label{#5}%
  \fi
  \end{figure}%
 }%
%
%
%    \FRAME{ framedata f|i tbph x F|T }              %#1
%          { contentswidth (scalar)  }               %#2
%          { contentsheight (scalar) }               %#3
%          { vertical shift when in-line (scalar) }  %#4
%          { caption }                               %#5
%          { label }                                 %#6
%          { name }                                  %#7
%          { body }                                  %#8
%
%    framedata is a string which can contain the following
%    characters: idftbphxFT
%    Their meaning is as follows:
%             i, d or f : in-line, display, or floating
%             t,b,p,h   : LaTeX floating placement options
%             x         : fit contents box to contents
%             F or T    : Figure or Table. 
%                         Later this can expand
%                         to a more general float class.
%
%
\newcount\dispkind%

\def\makeactives{
  \catcode`\"=\active
  \catcode`\;=\active
  \catcode`\:=\active
  \catcode`\'=\active
  \catcode`\~=\active
}
\bgroup
   \makeactives
   \gdef\activesoff{%
      \def"{\string"}%
      \def;{\string;}%
      \def:{\string:}%
      \def'{\string'}%
      \def~{\string~}%
      %\bbl@deactivate{"}%
      %\bbl@deactivate{;}%
      %\bbl@deactivate{:}%
      %\bbl@deactivate{'}%
    }
\egroup

\def\FRAME#1#2#3#4#5#6#7#8{%
 \bgroup
 \ifnum\@msidraft=\@ne
   \wasdrafttrue
 \else
   \wasdraftfalse%
 \fi
 \def\LaTeXparams{}%
 \dispkind=\z@
 \def\LaTeXparams{}%
 \doFRAMEparams{#1}%
 \ifnum\dispkind=\z@\IFRAME{#2}{#3}{#4}{#7}{#8}{#5}\else
  \ifnum\dispkind=\@ne\DFRAME{#2}{#3}{#7}{#8}{#5}\else
   \ifnum\dispkind=\tw@
    \edef\@tempa{\noexpand\FFRAME{\LaTeXparams}}%
    \@tempa{#2}{#3}{#5}{#6}{#7}{#8}%
    \fi
   \fi
  \fi
  \ifwasdraft\@msidraft=1\else\@msidraft=0\fi{}%
  \egroup
 }%
%
% This macro added to let SW gobble a parameter that
% should not be passed on and expanded. 

\def\TEXUX#1{"texux"}

%
% Macros for text attributes:
%
\def\BF#1{{\bf {#1}}}%
\def\NEG#1{\leavevmode\hbox{\rlap{\thinspace/}{$#1$}}}%
%
%%%%%%%%%%%%%%%%%%%%%%%%%%%%%%%%%%%%%%%%%%%%%%%%%%%%%%%%%%%%%%%%%%%%%%%%
%
%
% macros for user - defined functions
\def\limfunc#1{\mathop{\rm #1}}%
\def\func#1{\mathop{\rm #1}\nolimits}%
% macro for unit names
\def\unit#1{\mathord{\thinspace\rm #1}}%

%
% miscellaneous 
\long\def\QQQ#1#2{%
     \long\expandafter\def\csname#1\endcsname{#2}}%
\@ifundefined{QTP}{\def\QTP#1{}}{}
\@ifundefined{QEXCLUDE}{\def\QEXCLUDE#1{}}{}
\@ifundefined{Qlb}{\def\Qlb#1{#1}}{}
\@ifundefined{Qlt}{\def\Qlt#1{#1}}{}
\def\QWE{}%
\long\def\QQA#1#2{}%
\def\QTR#1#2{{\csname#1\endcsname {#2}}}%
  %	Add aliases for the ulem package
  \let\QQQuline\uline
  \let\QQQsout\sout
  \let\QQQuuline\uuline
  \let\QQQuwave\uwave
  \let\QQQxout\xout
\long\def\TeXButton#1#2{#2}%
\long\def\QSubDoc#1#2{#2}%
\def\EXPAND#1[#2]#3{}%
\def\NOEXPAND#1[#2]#3{}%
\def\PROTECTED{}%
\def\LaTeXparent#1{}%
\def\ChildStyles#1{}%
\def\ChildDefaults#1{}%
\def\QTagDef#1#2#3{}%

% Constructs added with Scientific Notebook
\@ifundefined{correctchoice}{\def\correctchoice{\relax}}{}
\@ifundefined{HTML}{\def\HTML#1{\relax}}{}
\@ifundefined{TCIIcon}{\def\TCIIcon#1#2#3#4{\relax}}{}
\if@compatibility
  \typeout{Not defining UNICODE  U or CustomNote commands for LaTeX 2.09.}
\else
  \providecommand{\UNICODE}[2][]{\protect\rule{.1in}{.1in}}
  \providecommand{\U}[1]{\protect\rule{.1in}{.1in}}
  \providecommand{\CustomNote}[3][]{\marginpar{#3}}
\fi

\@ifundefined{lambdabar}{
      \def\lambdabar{\errmessage{You have used the lambdabar symbol. 
                      This is available for typesetting only in RevTeX styles.}}
   }{}

%
% Macros for style editor docs
\@ifundefined{StyleEditBeginDoc}{\def\StyleEditBeginDoc{\relax}}{}
%
% Macros for footnotes
\def\QQfnmark#1{\footnotemark}
\def\QQfntext#1#2{\addtocounter{footnote}{#1}\footnotetext{#2}}
%
% Macros for indexing.
%
\@ifundefined{TCIMAKEINDEX}{}{\makeindex}%
%
% Attempts to avoid problems with other styles
\@ifundefined{abstract}{%
 \def\abstract{%
  \if@twocolumn
   \section*{Abstract (Not appropriate in this style!)}%
   \else \small 
   \begin{center}{\bf Abstract\vspace{-.5em}\vspace{\z@}}\end{center}%
   \quotation 
   \fi
  }%
 }{%
 }%
\@ifundefined{endabstract}{\def\endabstract
  {\if@twocolumn\else\endquotation\fi}}{}%
\@ifundefined{maketitle}{\def\maketitle#1{}}{}%
\@ifundefined{affiliation}{\def\affiliation#1{}}{}%
\@ifundefined{proof}{\def\proof{\noindent{\bfseries Proof. }}}{}%
\@ifundefined{endproof}{\def\endproof{\mbox{\ \rule{.1in}{.1in}}}}{}%
\@ifundefined{newfield}{\def\newfield#1#2{}}{}%
\@ifundefined{chapter}{\def\chapter#1{\par(Chapter head:)#1\par }%
 \newcount\c@chapter}{}%
\@ifundefined{part}{\def\part#1{\par(Part head:)#1\par }}{}%
\@ifundefined{section}{\def\section#1{\par(Section head:)#1\par }}{}%
\@ifundefined{subsection}{\def\subsection#1%
 {\par(Subsection head:)#1\par }}{}%
\@ifundefined{subsubsection}{\def\subsubsection#1%
 {\par(Subsubsection head:)#1\par }}{}%
\@ifundefined{paragraph}{\def\paragraph#1%
 {\par(Subsubsubsection head:)#1\par }}{}%
\@ifundefined{subparagraph}{\def\subparagraph#1%
 {\par(Subsubsubsubsection head:)#1\par }}{}%
%%%%%%%%%%%%%%%%%%%%%%%%%%%%%%%%%%%%%%%%%%%%%%%%%%%%%%%%%%%%%%%%%%%%%%%%
% These symbols are not recognized by LaTeX
\@ifundefined{therefore}{\def\therefore{}}{}%
\@ifundefined{backepsilon}{\def\backepsilon{}}{}%
\@ifundefined{yen}{\def\yen{\hbox{\rm\rlap=Y}}}{}%
\@ifundefined{registered}{%
   \def\registered{\relax\ifmmode{}\r@gistered
                    \else$\m@th\r@gistered$\fi}%
 \def\r@gistered{^{\ooalign
  {\hfil\raise.07ex\hbox{$\scriptstyle\rm\text{R}$}\hfil\crcr
  \mathhexbox20D}}}}{}%
\@ifundefined{Eth}{\def\Eth{}}{}%
\@ifundefined{eth}{\def\eth{}}{}%
\@ifundefined{Thorn}{\def\Thorn{}}{}%
\@ifundefined{thorn}{\def\thorn{}}{}%
% A macro to allow any symbol that requires math to appear in text
\def\TEXTsymbol#1{\mbox{$#1$}}%
\@ifundefined{degree}{\def\degree{{}^{\circ}}}{}%
%
% macros for T3TeX files
\newdimen\theight
\@ifundefined{Column}{\def\Column{%
 \vadjust{\setbox\z@=\hbox{\scriptsize\quad\quad tcol}%
  \theight=\ht\z@\advance\theight by \dp\z@\advance\theight by \lineskip
  \kern -\theight \vbox to \theight{%
   \rightline{\rlap{\box\z@}}%
   \vss
   }%
  }%
 }}{}%
%
\@ifundefined{qed}{\def\qed{%
 \ifhmode\unskip\nobreak\fi\ifmmode\ifinner\else\hskip5\p@\fi\fi
 \hbox{\hskip5\p@\vrule width4\p@ height6\p@ depth1.5\p@\hskip\p@}%
 }}{}%
%
\@ifundefined{cents}{\def\cents{\hbox{\rm\rlap c/}}}{}%
\@ifundefined{tciLaplace}{\def\tciLaplace{\ensuremath{\mathcal{L}}}}{}%
\@ifundefined{tciFourier}{\def\tciFourier{\ensuremath{\mathcal{F}}}}{}%
\@ifundefined{textcurrency}{\def\textcurrency{\hbox{\rm\rlap xo}}}{}%
\@ifundefined{texteuro}{\def\texteuro{\hbox{\rm\rlap C=}}}{}%
\@ifundefined{euro}{\def\euro{\hbox{\rm\rlap C=}}}{}%
\@ifundefined{textfranc}{\def\textfranc{\hbox{\rm\rlap-F}}}{}%
\@ifundefined{textlira}{\def\textlira{\hbox{\rm\rlap L=}}}{}%
\@ifundefined{textpeseta}{\def\textpeseta{\hbox{\rm P\negthinspace s}}}{}%
%
\@ifundefined{miss}{\def\miss{\hbox{\vrule height2\p@ width 2\p@ depth\z@}}}{}%
%
\@ifundefined{vvert}{\def\vvert{\Vert}}{}%  %always translated to \left| or \right|
%
\@ifundefined{tcol}{\def\tcol#1{{\baselineskip=6\p@ \vcenter{#1}} \Column}}{}%
%
\@ifundefined{dB}{\def\dB{\hbox{{}}}}{}%        %dummy entry in column 
\@ifundefined{mB}{\def\mB#1{\hbox{$#1$}}}{}%   %column entry
\@ifundefined{nB}{\def\nB#1{\hbox{#1}}}{}%     %column entry (not math)
%
\@ifundefined{note}{\def\note{$^{\dag}}}{}%
%
\def\newfmtname{LaTeX2e}
% No longer load latexsym.  This is now handled by SWP, which uses amsfonts if necessary
%
\ifx\fmtname\newfmtname
  \DeclareOldFontCommand{\rm}{\normalfont\rmfamily}{\mathrm}
  \DeclareOldFontCommand{\sf}{\normalfont\sffamily}{\mathsf}
  \DeclareOldFontCommand{\tt}{\normalfont\ttfamily}{\mathtt}
  \DeclareOldFontCommand{\bf}{\normalfont\bfseries}{\mathbf}
  \DeclareOldFontCommand{\it}{\normalfont\itshape}{\mathit}
  \DeclareOldFontCommand{\sl}{\normalfont\slshape}{\@nomath\sl}
  \DeclareOldFontCommand{\sc}{\normalfont\scshape}{\@nomath\sc}
\fi

%
% Greek bold macros
% Redefine all of the math symbols 
% which might be bolded	 - there are 
% probably others to add to this list

\def\alpha{{\Greekmath 010B}}%
\def\beta{{\Greekmath 010C}}%
\def\gamma{{\Greekmath 010D}}%
\def\delta{{\Greekmath 010E}}%
\def\epsilon{{\Greekmath 010F}}%
\def\zeta{{\Greekmath 0110}}%
\def\eta{{\Greekmath 0111}}%
\def\theta{{\Greekmath 0112}}%
\def\iota{{\Greekmath 0113}}%
\def\kappa{{\Greekmath 0114}}%
\def\lambda{{\Greekmath 0115}}%
\def\mu{{\Greekmath 0116}}%
\def\nu{{\Greekmath 0117}}%
\def\xi{{\Greekmath 0118}}%
\def\pi{{\Greekmath 0119}}%
\def\rho{{\Greekmath 011A}}%
\def\sigma{{\Greekmath 011B}}%
\def\tau{{\Greekmath 011C}}%
\def\upsilon{{\Greekmath 011D}}%
\def\phi{{\Greekmath 011E}}%
\def\chi{{\Greekmath 011F}}%
\def\psi{{\Greekmath 0120}}%
\def\omega{{\Greekmath 0121}}%
\def\varepsilon{{\Greekmath 0122}}%
\def\vartheta{{\Greekmath 0123}}%
\def\varpi{{\Greekmath 0124}}%
\def\varrho{{\Greekmath 0125}}%
\def\varsigma{{\Greekmath 0126}}%
\def\varphi{{\Greekmath 0127}}%

\def\nabla{{\Greekmath 0272}}
\def\FindBoldGroup{%
   {\setbox0=\hbox{$\mathbf{x\global\edef\theboldgroup{\the\mathgroup}}$}}%
}

\def\Greekmath#1#2#3#4{%
    \if@compatibility
        \ifnum\mathgroup=\symbold
           \mathchoice{\mbox{\boldmath$\displaystyle\mathchar"#1#2#3#4$}}%
                      {\mbox{\boldmath$\textstyle\mathchar"#1#2#3#4$}}%
                      {\mbox{\boldmath$\scriptstyle\mathchar"#1#2#3#4$}}%
                      {\mbox{\boldmath$\scriptscriptstyle\mathchar"#1#2#3#4$}}%
        \else
           \mathchar"#1#2#3#4% 
        \fi 
    \else 
        \FindBoldGroup
        \ifnum\mathgroup=\theboldgroup % For 2e
           \mathchoice{\mbox{\boldmath$\displaystyle\mathchar"#1#2#3#4$}}%
                      {\mbox{\boldmath$\textstyle\mathchar"#1#2#3#4$}}%
                      {\mbox{\boldmath$\scriptstyle\mathchar"#1#2#3#4$}}%
                      {\mbox{\boldmath$\scriptscriptstyle\mathchar"#1#2#3#4$}}%
        \else
           \mathchar"#1#2#3#4% 
        \fi     	    
	  \fi}

\newif\ifGreekBold  \GreekBoldfalse
\let\SAVEPBF=\pbf
\def\pbf{\GreekBoldtrue\SAVEPBF}%
%

\@ifundefined{theorem}{\newtheorem{theorem}{Theorem}}{}
\@ifundefined{lemma}{\newtheorem{lemma}[theorem]{Lemma}}{}
\@ifundefined{corollary}{\newtheorem{corollary}[theorem]{Corollary}}{}
\@ifundefined{conjecture}{\newtheorem{conjecture}[theorem]{Conjecture}}{}
\@ifundefined{proposition}{\newtheorem{proposition}[theorem]{Proposition}}{}
\@ifundefined{axiom}{\newtheorem{axiom}{Axiom}}{}
\@ifundefined{remark}{\newtheorem{remark}{Remark}}{}
\@ifundefined{example}{\newtheorem{example}{Example}}{}
\@ifundefined{exercise}{\newtheorem{exercise}{Exercise}}{}
\@ifundefined{definition}{\newtheorem{definition}{Definition}}{}


\@ifundefined{mathletters}{%
  %\def\theequation{\arabic{equation}}
  \newcounter{equationnumber}  
  \def\mathletters{%
     \addtocounter{equation}{1}
     \edef\@currentlabel{\theequation}%
     \setcounter{equationnumber}{\c@equation}
     \setcounter{equation}{0}%
     \edef\theequation{\@currentlabel\noexpand\alph{equation}}%
  }
  \def\endmathletters{%
     \setcounter{equation}{\value{equationnumber}}%
  }
}{}

%Logos
\@ifundefined{BibTeX}{%
    \def\BibTeX{{\rm B\kern-.05em{\sc i\kern-.025em b}\kern-.08em
                 T\kern-.1667em\lower.7ex\hbox{E}\kern-.125emX}}}{}%
\@ifundefined{AmS}%
    {\def\AmS{{\protect\usefont{OMS}{cmsy}{m}{n}%
                A\kern-.1667em\lower.5ex\hbox{M}\kern-.125emS}}}{}%
\@ifundefined{AmSTeX}{\def\AmSTeX{\protect\AmS-\protect\TeX\@}}{}%
%

% This macro is a fix to eqnarray
\def\@@eqncr{\let\@tempa\relax
    \ifcase\@eqcnt \def\@tempa{& & &}\or \def\@tempa{& &}%
      \else \def\@tempa{&}\fi
     \@tempa
     \if@eqnsw
        \iftag@
           \@taggnum
        \else
           \@eqnnum\stepcounter{equation}%
        \fi
     \fi
     \global\tag@false
     \global\@eqnswtrue
     \global\@eqcnt\z@\cr}


\def\TCItag{\@ifnextchar*{\@TCItagstar}{\@TCItag}}
\def\@TCItag#1{%
    \global\tag@true
    \global\def\@taggnum{(#1)}%
    \global\def\@currentlabel{#1}}
\def\@TCItagstar*#1{%
    \global\tag@true
    \global\def\@taggnum{#1}%
    \global\def\@currentlabel{#1}}
%
%%%%%%%%%%%%%%%%%%%%%%%%%%%%%%%%%%%%%%%%%%%%%%%%%%%%%%%%%%%%%%%%%%%%%
%
\def\QATOP#1#2{{#1 \atop #2}}%
\def\QTATOP#1#2{{\textstyle {#1 \atop #2}}}%
\def\QDATOP#1#2{{\displaystyle {#1 \atop #2}}}%
\def\QABOVE#1#2#3{{#2 \above#1 #3}}%
\def\QTABOVE#1#2#3{{\textstyle {#2 \above#1 #3}}}%
\def\QDABOVE#1#2#3{{\displaystyle {#2 \above#1 #3}}}%
\def\QOVERD#1#2#3#4{{#3 \overwithdelims#1#2 #4}}%
\def\QTOVERD#1#2#3#4{{\textstyle {#3 \overwithdelims#1#2 #4}}}%
\def\QDOVERD#1#2#3#4{{\displaystyle {#3 \overwithdelims#1#2 #4}}}%
\def\QATOPD#1#2#3#4{{#3 \atopwithdelims#1#2 #4}}%
\def\QTATOPD#1#2#3#4{{\textstyle {#3 \atopwithdelims#1#2 #4}}}%
\def\QDATOPD#1#2#3#4{{\displaystyle {#3 \atopwithdelims#1#2 #4}}}%
\def\QABOVED#1#2#3#4#5{{#4 \abovewithdelims#1#2#3 #5}}%
\def\QTABOVED#1#2#3#4#5{{\textstyle 
   {#4 \abovewithdelims#1#2#3 #5}}}%
\def\QDABOVED#1#2#3#4#5{{\displaystyle 
   {#4 \abovewithdelims#1#2#3 #5}}}%
%
% Macros for text size operators:
%

\def\tint{\msi@int\textstyle\int}%
\def\tiint{\msi@int\textstyle\iint}%
\def\tiiint{\msi@int\textstyle\iiint}%
\def\tiiiint{\msi@int\textstyle\iiiint}%
\def\tidotsint{\msi@int\textstyle\idotsint}%
\def\toint{\msi@int\textstyle\oint}%


\def\tsum{\mathop{\textstyle \sum }}%
\def\tprod{\mathop{\textstyle \prod }}%
\def\tbigcap{\mathop{\textstyle \bigcap }}%
\def\tbigwedge{\mathop{\textstyle \bigwedge }}%
\def\tbigoplus{\mathop{\textstyle \bigoplus }}%
\def\tbigodot{\mathop{\textstyle \bigodot }}%
\def\tbigsqcup{\mathop{\textstyle \bigsqcup }}%
\def\tcoprod{\mathop{\textstyle \coprod }}%
\def\tbigcup{\mathop{\textstyle \bigcup }}%
\def\tbigvee{\mathop{\textstyle \bigvee }}%
\def\tbigotimes{\mathop{\textstyle \bigotimes }}%
\def\tbiguplus{\mathop{\textstyle \biguplus }}%
%
%
%Macros for display size operators:
%

\newtoks\temptoksa
\newtoks\temptoksb
\newtoks\temptoksc


\def\msi@int#1#2{%
 \def\@temp{{#1#2\the\temptoksc_{\the\temptoksa}^{\the\temptoksb}}}%   
 \futurelet\@nextcs
 \@int
}

\def\@int{%
   \ifx\@nextcs\limits
      \typeout{Found limits}%
      \temptoksc={\limits}%
	  \let\@next\@intgobble%
   \else\ifx\@nextcs\nolimits
      \typeout{Found nolimits}%
      \temptoksc={\nolimits}%
	  \let\@next\@intgobble%
   \else
      \typeout{Did not find limits or no limits}%
      \temptoksc={}%
      \let\@next\msi@limits%
   \fi\fi
   \@next   
}%

\def\@intgobble#1{%
   \typeout{arg is #1}%
   \msi@limits
}


\def\msi@limits{%
   \temptoksa={}%
   \temptoksb={}%
   \@ifnextchar_{\@limitsa}{\@limitsb}%
}

\def\@limitsa_#1{%
   \temptoksa={#1}%
   \@ifnextchar^{\@limitsc}{\@temp}%
}

\def\@limitsb{%
   \@ifnextchar^{\@limitsc}{\@temp}%
}

\def\@limitsc^#1{%
   \temptoksb={#1}%
   \@ifnextchar_{\@limitsd}{\@temp}%   
}

\def\@limitsd_#1{%
   \temptoksa={#1}%
   \@temp
}



\def\dint{\msi@int\displaystyle\int}%
\def\diint{\msi@int\displaystyle\iint}%
\def\diiint{\msi@int\displaystyle\iiint}%
\def\diiiint{\msi@int\displaystyle\iiiint}%
\def\didotsint{\msi@int\displaystyle\idotsint}%
\def\doint{\msi@int\displaystyle\oint}%

\def\dsum{\mathop{\displaystyle \sum }}%
\def\dprod{\mathop{\displaystyle \prod }}%
\def\dbigcap{\mathop{\displaystyle \bigcap }}%
\def\dbigwedge{\mathop{\displaystyle \bigwedge }}%
\def\dbigoplus{\mathop{\displaystyle \bigoplus }}%
\def\dbigodot{\mathop{\displaystyle \bigodot }}%
\def\dbigsqcup{\mathop{\displaystyle \bigsqcup }}%
\def\dcoprod{\mathop{\displaystyle \coprod }}%
\def\dbigcup{\mathop{\displaystyle \bigcup }}%
\def\dbigvee{\mathop{\displaystyle \bigvee }}%
\def\dbigotimes{\mathop{\displaystyle \bigotimes }}%
\def\dbiguplus{\mathop{\displaystyle \biguplus }}%

\if@compatibility\else
  % Always load amsmath in LaTeX2e mode
  \RequirePackage{amsmath}
\fi

\def\ExitTCILatex{\makeatother\endinput}

\bgroup
\ifx\ds@amstex\relax
   \message{amstex already loaded}\aftergroup\ExitTCILatex
\else
   \@ifpackageloaded{amsmath}%
      {\if@compatibility\message{amsmath already loaded}\fi\aftergroup\ExitTCILatex}
      {}
   \@ifpackageloaded{amstex}%
      {\if@compatibility\message{amstex already loaded}\fi\aftergroup\ExitTCILatex}
      {}
   \@ifpackageloaded{amsgen}%
      {\if@compatibility\message{amsgen already loaded}\fi\aftergroup\ExitTCILatex}
      {}
\fi
\egroup

%Exit if any of the AMS macros are already loaded.
%This is always the case for LaTeX2e mode.


%%%%%%%%%%%%%%%%%%%%%%%%%%%%%%%%%%%%%%%%%%%%%%%%%%%%%%%%%%%%%%%%%%%%%%%%%%
% NOTE: The rest of this file is read only if in LaTeX 2.09 compatibility
% mode. This section is used to define AMS-like constructs in the
% event they have not been defined.
%%%%%%%%%%%%%%%%%%%%%%%%%%%%%%%%%%%%%%%%%%%%%%%%%%%%%%%%%%%%%%%%%%%%%%%%%%
\typeout{TCILATEX defining AMS-like constructs in LaTeX 2.09 COMPATIBILITY MODE}
%%%%%%%%%%%%%%%%%%%%%%%%%%%%%%%%%%%%%%%%%%%%%%%%%%%%%%%%%%%%%%%%%%%%%%%%
%  Macros to define some AMS LaTeX constructs when 
%  AMS LaTeX has not been loaded
% 
% These macros are copied from the AMS-TeX package for doing
% multiple integrals.
%
\let\DOTSI\relax
\def\RIfM@{\relax\ifmmode}%
\def\FN@{\futurelet\next}%
\newcount\intno@
\def\iint{\DOTSI\intno@\tw@\FN@\ints@}%
\def\iiint{\DOTSI\intno@\thr@@\FN@\ints@}%
\def\iiiint{\DOTSI\intno@4 \FN@\ints@}%
\def\idotsint{\DOTSI\intno@\z@\FN@\ints@}%
\def\ints@{\findlimits@\ints@@}%
\newif\iflimtoken@
\newif\iflimits@
\def\findlimits@{\limtoken@true\ifx\next\limits\limits@true
 \else\ifx\next\nolimits\limits@false\else
 \limtoken@false\ifx\ilimits@\nolimits\limits@false\else
 \ifinner\limits@false\else\limits@true\fi\fi\fi\fi}%
\def\multint@{\int\ifnum\intno@=\z@\intdots@                          %1
 \else\intkern@\fi                                                    %2
 \ifnum\intno@>\tw@\int\intkern@\fi                                   %3
 \ifnum\intno@>\thr@@\int\intkern@\fi                                 %4
 \int}%                                                               %5
\def\multintlimits@{\intop\ifnum\intno@=\z@\intdots@\else\intkern@\fi
 \ifnum\intno@>\tw@\intop\intkern@\fi
 \ifnum\intno@>\thr@@\intop\intkern@\fi\intop}%
\def\intic@{%
    \mathchoice{\hskip.5em}{\hskip.4em}{\hskip.4em}{\hskip.4em}}%
\def\negintic@{\mathchoice
 {\hskip-.5em}{\hskip-.4em}{\hskip-.4em}{\hskip-.4em}}%
\def\ints@@{\iflimtoken@                                              %1
 \def\ints@@@{\iflimits@\negintic@
   \mathop{\intic@\multintlimits@}\limits                             %2
  \else\multint@\nolimits\fi                                          %3
  \eat@}%                                                             %4
 \else                                                                %5
 \def\ints@@@{\iflimits@\negintic@
  \mathop{\intic@\multintlimits@}\limits\else
  \multint@\nolimits\fi}\fi\ints@@@}%
\def\intkern@{\mathchoice{\!\!\!}{\!\!}{\!\!}{\!\!}}%
\def\plaincdots@{\mathinner{\cdotp\cdotp\cdotp}}%
\def\intdots@{\mathchoice{\plaincdots@}%
 {{\cdotp}\mkern1.5mu{\cdotp}\mkern1.5mu{\cdotp}}%
 {{\cdotp}\mkern1mu{\cdotp}\mkern1mu{\cdotp}}%
 {{\cdotp}\mkern1mu{\cdotp}\mkern1mu{\cdotp}}}%
%
%
%  These macros are for doing the AMS \text{} construct
%
\def\RIfM@{\relax\protect\ifmmode}
\def\text{\RIfM@\expandafter\text@\else\expandafter\mbox\fi}
\let\nfss@text\text
\def\text@#1{\mathchoice
   {\textdef@\displaystyle\f@size{#1}}%
   {\textdef@\textstyle\tf@size{\firstchoice@false #1}}%
   {\textdef@\textstyle\sf@size{\firstchoice@false #1}}%
   {\textdef@\textstyle \ssf@size{\firstchoice@false #1}}%
   \glb@settings}

\def\textdef@#1#2#3{\hbox{{%
                    \everymath{#1}%
                    \let\f@size#2\selectfont
                    #3}}}
\newif\iffirstchoice@
\firstchoice@true
%
%These are the AMS constructs for multiline limits.
%
\def\Let@{\relax\iffalse{\fi\let\\=\cr\iffalse}\fi}%
\def\vspace@{\def\vspace##1{\crcr\noalign{\vskip##1\relax}}}%
\def\multilimits@{\bgroup\vspace@\Let@
 \baselineskip\fontdimen10 \scriptfont\tw@
 \advance\baselineskip\fontdimen12 \scriptfont\tw@
 \lineskip\thr@@\fontdimen8 \scriptfont\thr@@
 \lineskiplimit\lineskip
 \vbox\bgroup\ialign\bgroup\hfil$\m@th\scriptstyle{##}$\hfil\crcr}%
\def\Sb{_\multilimits@}%
\def\endSb{\crcr\egroup\egroup\egroup}%
\def\Sp{^\multilimits@}%
\let\endSp\endSb
%
%
%These are AMS constructs for horizontal arrows
%
\newdimen\ex@
\ex@.2326ex
\def\rightarrowfill@#1{$#1\m@th\mathord-\mkern-6mu\cleaders
 \hbox{$#1\mkern-2mu\mathord-\mkern-2mu$}\hfill
 \mkern-6mu\mathord\rightarrow$}%
\def\leftarrowfill@#1{$#1\m@th\mathord\leftarrow\mkern-6mu\cleaders
 \hbox{$#1\mkern-2mu\mathord-\mkern-2mu$}\hfill\mkern-6mu\mathord-$}%
\def\leftrightarrowfill@#1{$#1\m@th\mathord\leftarrow
\mkern-6mu\cleaders
 \hbox{$#1\mkern-2mu\mathord-\mkern-2mu$}\hfill
 \mkern-6mu\mathord\rightarrow$}%
\def\overrightarrow{\mathpalette\overrightarrow@}%
\def\overrightarrow@#1#2{\vbox{\ialign{##\crcr\rightarrowfill@#1\crcr
 \noalign{\kern-\ex@\nointerlineskip}$\m@th\hfil#1#2\hfil$\crcr}}}%
\let\overarrow\overrightarrow
\def\overleftarrow{\mathpalette\overleftarrow@}%
\def\overleftarrow@#1#2{\vbox{\ialign{##\crcr\leftarrowfill@#1\crcr
 \noalign{\kern-\ex@\nointerlineskip}$\m@th\hfil#1#2\hfil$\crcr}}}%
\def\overleftrightarrow{\mathpalette\overleftrightarrow@}%
\def\overleftrightarrow@#1#2{\vbox{\ialign{##\crcr
   \leftrightarrowfill@#1\crcr
 \noalign{\kern-\ex@\nointerlineskip}$\m@th\hfil#1#2\hfil$\crcr}}}%
\def\underrightarrow{\mathpalette\underrightarrow@}%
\def\underrightarrow@#1#2{\vtop{\ialign{##\crcr$\m@th\hfil#1#2\hfil
  $\crcr\noalign{\nointerlineskip}\rightarrowfill@#1\crcr}}}%
\let\underarrow\underrightarrow
\def\underleftarrow{\mathpalette\underleftarrow@}%
\def\underleftarrow@#1#2{\vtop{\ialign{##\crcr$\m@th\hfil#1#2\hfil
  $\crcr\noalign{\nointerlineskip}\leftarrowfill@#1\crcr}}}%
\def\underleftrightarrow{\mathpalette\underleftrightarrow@}%
\def\underleftrightarrow@#1#2{\vtop{\ialign{##\crcr$\m@th
  \hfil#1#2\hfil$\crcr
 \noalign{\nointerlineskip}\leftrightarrowfill@#1\crcr}}}%
%%%%%%%%%%%%%%%%%%%%%

\def\qopnamewl@#1{\mathop{\operator@font#1}\nlimits@}
\let\nlimits@\displaylimits
\def\setboxz@h{\setbox\z@\hbox}


\def\varlim@#1#2{\mathop{\vtop{\ialign{##\crcr
 \hfil$#1\m@th\operator@font lim$\hfil\crcr
 \noalign{\nointerlineskip}#2#1\crcr
 \noalign{\nointerlineskip\kern-\ex@}\crcr}}}}

 \def\rightarrowfill@#1{\m@th\setboxz@h{$#1-$}\ht\z@\z@
  $#1\copy\z@\mkern-6mu\cleaders
  \hbox{$#1\mkern-2mu\box\z@\mkern-2mu$}\hfill
  \mkern-6mu\mathord\rightarrow$}
\def\leftarrowfill@#1{\m@th\setboxz@h{$#1-$}\ht\z@\z@
  $#1\mathord\leftarrow\mkern-6mu\cleaders
  \hbox{$#1\mkern-2mu\copy\z@\mkern-2mu$}\hfill
  \mkern-6mu\box\z@$}


\def\projlim{\qopnamewl@{proj\,lim}}
\def\injlim{\qopnamewl@{inj\,lim}}
\def\varinjlim{\mathpalette\varlim@\rightarrowfill@}
\def\varprojlim{\mathpalette\varlim@\leftarrowfill@}
\def\varliminf{\mathpalette\varliminf@{}}
\def\varliminf@#1{\mathop{\underline{\vrule\@depth.2\ex@\@width\z@
   \hbox{$#1\m@th\operator@font lim$}}}}
\def\varlimsup{\mathpalette\varlimsup@{}}
\def\varlimsup@#1{\mathop{\overline
  {\hbox{$#1\m@th\operator@font lim$}}}}

%
%Companion to stackrel
\def\stackunder#1#2{\mathrel{\mathop{#2}\limits_{#1}}}%
%
%
% These are AMS environments that will be defined to
% be verbatims if amstex has not actually been 
% loaded
%
%
\begingroup \catcode `|=0 \catcode `[= 1
\catcode`]=2 \catcode `\{=12 \catcode `\}=12
\catcode`\\=12 
|gdef|@alignverbatim#1\end{align}[#1|end[align]]
|gdef|@salignverbatim#1\end{align*}[#1|end[align*]]

|gdef|@alignatverbatim#1\end{alignat}[#1|end[alignat]]
|gdef|@salignatverbatim#1\end{alignat*}[#1|end[alignat*]]

|gdef|@xalignatverbatim#1\end{xalignat}[#1|end[xalignat]]
|gdef|@sxalignatverbatim#1\end{xalignat*}[#1|end[xalignat*]]

|gdef|@gatherverbatim#1\end{gather}[#1|end[gather]]
|gdef|@sgatherverbatim#1\end{gather*}[#1|end[gather*]]

|gdef|@gatherverbatim#1\end{gather}[#1|end[gather]]
|gdef|@sgatherverbatim#1\end{gather*}[#1|end[gather*]]


|gdef|@multilineverbatim#1\end{multiline}[#1|end[multiline]]
|gdef|@smultilineverbatim#1\end{multiline*}[#1|end[multiline*]]

|gdef|@arraxverbatim#1\end{arrax}[#1|end[arrax]]
|gdef|@sarraxverbatim#1\end{arrax*}[#1|end[arrax*]]

|gdef|@tabulaxverbatim#1\end{tabulax}[#1|end[tabulax]]
|gdef|@stabulaxverbatim#1\end{tabulax*}[#1|end[tabulax*]]


|endgroup
  

  
\def\align{\@verbatim \frenchspacing\@vobeyspaces \@alignverbatim
You are using the "align" environment in a style in which it is not defined.}
\let\endalign=\endtrivlist
 
\@namedef{align*}{\@verbatim\@salignverbatim
You are using the "align*" environment in a style in which it is not defined.}
\expandafter\let\csname endalign*\endcsname =\endtrivlist




\def\alignat{\@verbatim \frenchspacing\@vobeyspaces \@alignatverbatim
You are using the "alignat" environment in a style in which it is not defined.}
\let\endalignat=\endtrivlist
 
\@namedef{alignat*}{\@verbatim\@salignatverbatim
You are using the "alignat*" environment in a style in which it is not defined.}
\expandafter\let\csname endalignat*\endcsname =\endtrivlist




\def\xalignat{\@verbatim \frenchspacing\@vobeyspaces \@xalignatverbatim
You are using the "xalignat" environment in a style in which it is not defined.}
\let\endxalignat=\endtrivlist
 
\@namedef{xalignat*}{\@verbatim\@sxalignatverbatim
You are using the "xalignat*" environment in a style in which it is not defined.}
\expandafter\let\csname endxalignat*\endcsname =\endtrivlist




\def\gather{\@verbatim \frenchspacing\@vobeyspaces \@gatherverbatim
You are using the "gather" environment in a style in which it is not defined.}
\let\endgather=\endtrivlist
 
\@namedef{gather*}{\@verbatim\@sgatherverbatim
You are using the "gather*" environment in a style in which it is not defined.}
\expandafter\let\csname endgather*\endcsname =\endtrivlist


\def\multiline{\@verbatim \frenchspacing\@vobeyspaces \@multilineverbatim
You are using the "multiline" environment in a style in which it is not defined.}
\let\endmultiline=\endtrivlist
 
\@namedef{multiline*}{\@verbatim\@smultilineverbatim
You are using the "multiline*" environment in a style in which it is not defined.}
\expandafter\let\csname endmultiline*\endcsname =\endtrivlist


\def\arrax{\@verbatim \frenchspacing\@vobeyspaces \@arraxverbatim
You are using a type of "array" construct that is only allowed in AmS-LaTeX.}
\let\endarrax=\endtrivlist

\def\tabulax{\@verbatim \frenchspacing\@vobeyspaces \@tabulaxverbatim
You are using a type of "tabular" construct that is only allowed in AmS-LaTeX.}
\let\endtabulax=\endtrivlist

 
\@namedef{arrax*}{\@verbatim\@sarraxverbatim
You are using a type of "array*" construct that is only allowed in AmS-LaTeX.}
\expandafter\let\csname endarrax*\endcsname =\endtrivlist

\@namedef{tabulax*}{\@verbatim\@stabulaxverbatim
You are using a type of "tabular*" construct that is only allowed in AmS-LaTeX.}
\expandafter\let\csname endtabulax*\endcsname =\endtrivlist

% macro to simulate ams tag construct


% This macro is a fix to the equation environment
 \def\endequation{%
     \ifmmode\ifinner % FLEQN hack
      \iftag@
        \addtocounter{equation}{-1} % undo the increment made in the begin part
        $\hfil
           \displaywidth\linewidth\@taggnum\egroup \endtrivlist
        \global\tag@false
        \global\@ignoretrue   
      \else
        $\hfil
           \displaywidth\linewidth\@eqnnum\egroup \endtrivlist
        \global\tag@false
        \global\@ignoretrue 
      \fi
     \else   
      \iftag@
        \addtocounter{equation}{-1} % undo the increment made in the begin part
        \eqno \hbox{\@taggnum}
        \global\tag@false%
        $$\global\@ignoretrue
      \else
        \eqno \hbox{\@eqnnum}% $$ BRACE MATCHING HACK
        $$\global\@ignoretrue
      \fi
     \fi\fi
 } 

 \newif\iftag@ \tag@false
 
 \def\TCItag{\@ifnextchar*{\@TCItagstar}{\@TCItag}}
 \def\@TCItag#1{%
     \global\tag@true
     \global\def\@taggnum{(#1)}%
     \global\def\@currentlabel{#1}}
 \def\@TCItagstar*#1{%
     \global\tag@true
     \global\def\@taggnum{#1}%
     \global\def\@currentlabel{#1}}

  \@ifundefined{tag}{
     \def\tag{\@ifnextchar*{\@tagstar}{\@tag}}
     \def\@tag#1{%
         \global\tag@true
         \global\def\@taggnum{(#1)}}
     \def\@tagstar*#1{%
         \global\tag@true
         \global\def\@taggnum{#1}}
  }{}

\def\tfrac#1#2{{\textstyle {#1 \over #2}}}%
\def\dfrac#1#2{{\displaystyle {#1 \over #2}}}%
\def\binom#1#2{{#1 \choose #2}}%
\def\tbinom#1#2{{\textstyle {#1 \choose #2}}}%
\def\dbinom#1#2{{\displaystyle {#1 \choose #2}}}%

% Do not add anything to the end of this file.  
% The last section of the file is loaded only if 
% amstex has not been.
\makeatother
\endinput

\def\TextSymbolUnavailable#1{\textbf{???}}
\begin{document}

\title{A Sample Article}
\author{Dr. Tane Hunter \\
%EndAName
TCI Software Research}
\maketitle

\begin{abstract}
This article illustrates many features of a mathematics document created
with Scientific Word or Scientific WorkPlace.
\end{abstract}

\tableofcontents
\listoffigures

\section{A Sample Article}

This document illustrates a wide range of the symbols and constructs
available to you. This document is set up to compile with the standard and
multilingual format files. In particular, certain characters from the
Latin-1, Latin Extended-A, and General Punctuation symbol panels will
typeset as \textbf{???} if a multilingual format file such as TrueTeX
MultiLingual is not used.

\subsection{Mathematics and Text}

Let $H$ be a Hilbert space, $C$ be a closed bounded convex subset of $H$, $T$
a nonexpansive self map of $C$. Suppose that as $n\rightarrow \infty $, $%
a_{n,k}\rightarrow 0$ for each $k$, and $\gamma _{n}=\sum_{k=0}^{\infty
}\left( a_{n,k+1}-a_{n,k}\right) ^{+}\rightarrow 0$ Then for each $x$ in $C$%
, $A_{n}x=\sum_{k=0}^{\infty }a_{n,k}T^{k}x$ converges weakly to a fixed
point of $T$ \cite{dunford}.

In this situation. we would also like to cite \cite%
{black,blyth,Chang,condorcet,funkenbusch}, but we do so only to demonstrate
a citation with multiple entries.

\subsection{In-line and Displayed Mathematics}

The equation 
\begin{equation}
u_{tt}-\Delta u+u^{5}+u\left| u\right| ^{p-2}=0\text{ in }\mathbf{R}%
^{3}\times \left[ 0,\infty \right[ .  \tag{1}  \label{wave}
\end{equation}
is numbered and it also has the label ``wave''. You can use this label to
jump to this equation using hypertext links. You can reference this equation
within your document as equation \ref{wave}.

There are two sets of \LaTeX{} parameters governing mathematical displays.
The spacing left above and below a display depends on whether the lines
above or below are short or long.

Short above 
\begin{equation*}
x^2+y^2=z^2
\end{equation*}
short below.

Long above. This may depend on your margins. Avoid wrapping. 
\begin{equation*}
\sin ^{2}\theta +\cos ^{2}\theta =1
\end{equation*}
and long below. You don't have to worry about this line being long enough in
most circumstances because we've tried to ensure that it will wrap no matter
how wide you have your margins set.

\subsection{Multi-Line Displays}

\emph{Scientific Word and WorkPlace} provide a range of alignment options
for multiline mathematical displays. Here is a series of multiline displays. 
\begin{eqnarray}
x &=&17y \\
y &>&a+b+c+d+e+f+g+h+i+j+  \notag \\
&&k+l+m+n+o+p
\end{eqnarray}
\begin{eqnarray*}
x &\ll &y_{1}+\cdots +y_{n} \\
&\leq &z
\end{eqnarray*}
\begin{eqnarray*}
y &=&a+b+c+d+e+f+g+h+i+j \\
&&\,+k+l+m+n+o+p
\end{eqnarray*}
\begin{eqnarray*}
w+x+y+z &=& \\
&&a+b+c+d+e+f+g+h+i+j+ \\
&&k+l+m+n+o+p
\end{eqnarray*}

If $f(x)=x+1,$ then we will have 
\begin{equation*}
f([x+1]/[x+2])=\{[x+1]/[x+2]\}+1=(2x+3)/(x+2)
\end{equation*}
\begin{eqnarray*}
1+2+3+4+5+6+7+8+9+10+11+12 && \\
+13+14+15+16+17+18+19+20 &=&190
\end{eqnarray*}

\begin{eqnarray}
\max (f,g) &=&\frac{f+g+\left| f-g\right| }{2}\text{,} \\
\max (f,-g) &=&\frac{f-g+\left| f+g\right| }{2}\text{.}
\end{eqnarray}
\qquad 
\begin{eqnarray}
(a+b)^{n+1} &=&(a+b)(a+b)^{n}=(a+b)\sum_{j=0}^{n}\binom{n}{j}a^{n-1}b^{j}
\label{JoyA} \\
&=&\sum_{j=0}^{n}\binom{n}{j}a^{n+1-j}b^{j}+\sum_{j=1}^{n}\binom{n}{j-1}%
a^{n-1}b^{j}  \label{JoyB} \\
&=&\sum_{j=0}^{n}\binom{n+1}{j}a^{n+1-j}b^{j}\text{.}  \label{JoyC}
\end{eqnarray}

The equations in the display immediately above are (\ref{JoyA}), (\ref{JoyB}%
), and (\ref{JoyC}).

You can have subequation numbering generated automatically, illustrated in
the following:

\begin{subequations}
\begin{eqnarray}
x &=&a+b \\
y &=&c+d \\
z &=&e+f
\end{eqnarray}
This is text between the two sets of equations. The following set of
equations continues the numbering from the set above.

\begin{eqnarray}
u &=&g+h \\
v &=&i+j \\
w &=&k+l
\end{eqnarray}

Test the overriding of automatic equation labels: 
\end{subequations}
\begin{eqnarray}
x &=&y+z  \TCItag{A--1} \\
&=&k+m  \TCItag{A--2}
\end{eqnarray}
Test the overriding of automatic equation labels, suppressing annotation: 
\begin{eqnarray}
x &=&y+z  \TCItag*{A--1} \\
&=&k+m  \TCItag*{A--2}
\end{eqnarray}
The labels in the first set of equations should be parenthesized, while the
labels in the second set should not be parenthesized.

There is a significant difference between the standard \LaTeX{} equation
array and the AMS align environment. SW/SWP does not show this difference on
screen. To check this difference, look at the Typeset Preview versions of
the following. 
\begin{eqnarray*}
x &=&y \\
&=&z
\end{eqnarray*}
\begin{align*}
x& =y \\
& =z
\end{align*}

The space around the = sign in the first set of equations (the \LaTeX{}
equation array) is much greater than in the second set (the AMS align
environment). The spacing in the second set is preferred, as it agrees with
the spacing in a single-line display:

\begin{equation*}
x=y
\end{equation*}

To convert the \LaTeX{} equation array to the AMS align environment, select
the equations, then from the Edit menu, choose Properties. Select Advanced
and then select Enable Alignment.

\section{Mathematics in section heads $\protect\int_{\protect\alpha }^{%
\protect\beta }\ln tdt$}

You ought to be able to put mathematics in section heads. It might be a
problem in style with running headers and table of contents entries.

\subsection{Theorems, Lemmata, Etc.}

\label{theorems}You can automatically number theorems and other proclamations%
\footnote{%
Such as propositions, lemmas and corollaries. You can create your own
theorem-like environments to extend the basic set provided with \textsl{SW}.
\par
Footnotes, such as this, can have several paragraphs. Each paragraph can be
as large as you want. You can include mathematics, graphics, or anything
else you can enter in main-document paragraphs. It is even possible to
compare the difference between an in-line sum $\sum_{n=1}^{\infty }\frac{1}{%
n^{2}}$ and a displayed sum 
\begin{equation*}
\sum_{n=1}^{\infty }\frac{1}{n^{2}}
\end{equation*}%
}. This is a marginal marginal note.%
\marginpar{
Marginal notes are like call-outs in graphics. You can bring the reader's
attention to a point made in the text.
\par
Marginal notes can have multiple paragraphs.} These theorem-like
environments are available.

\begin{acknowledgement}
This is an acknowledgement
\end{acknowledgement}

\begin{algorithm}
This is an algorithm
\end{algorithm}

\begin{axiom}
All mathematics is fun.
\end{axiom}

\begin{case}
This is a case
\end{case}

\begin{claim}
This is a claim
\end{claim}

\begin{conclusion}
This is a conclusion
\end{conclusion}

\begin{condition}
This is a condition
\end{condition}

\begin{conjecture}
Some people don't like mathematics.
\end{conjecture}

\begin{corollary}
This is a corollary
\end{corollary}

\begin{criterion}
This is a criterion
\end{criterion}

\begin{definition}
This is a definition
\end{definition}

\begin{example}
This is an example
\end{example}

\begin{exercise}
This is an exercise
\end{exercise}

\begin{lemma}
The plural of lemma is lemmata, so they say. Why all these lemmas?
\end{lemma}

\begin{proof}
This is the proof of the lemma.
\end{proof}

\begin{notation}
This is notation
\end{notation}

\begin{problem}
This is a problem
\end{problem}

\begin{proposition}
This is a proposition
\end{proposition}

\begin{remark}
This is a remark
\end{remark}

\begin{summary}
This is a summary
\end{summary}

\begin{theorem}[Main Theorem]
\label{existence}Suppose $u_{0}\in C^{3}\left( \mathbf{R}^{3}\right)
,u_{1}\in C^{2}\left( \mathbf{R}^{3}\right) $ have finite energy 
\begin{equation}
\int_{\mathbf{R}^{3}}\left( \frac{\left| u_{1}\right| ^{2}+\left| \nabla
u_{0}\right| ^{2}}{2}+\frac{\left| u_{0}\right| ^{6}}{6}\right) dx<\infty 
\tag{9}
\end{equation}
and suppose the solution $u^{(0)}$ to the homogeneous wave equation with
initial data $u_{0},u_{1}$ is uniformly bounded. Then there exists $\epsilon
_{0}>0$ such that for $\left| \epsilon \right| <\epsilon _{0}$ the initial
value problem (\ref{wave}) with data $\epsilon u_{0},\epsilon u_{1}$ admits
a global $C^{2}$-solution.
\end{theorem}

\begin{proof}[Proof of the Main Theorem]
This is the proof.
\end{proof}

\section{Section Heading}

This is body text following a section heading. \label{sectA}

\subsection{Subsection Heading}

This is body text following a subsection heading.\label{ssectA}

\subsubsection{Subsubsection Heading}

This is body text following a subsubsection heading.\label{sssectA}

\paragraph{Subsubsubsection Heading}

This is body text following a subsubsubsection heading.\label{ssssectA}

\subparagraph{Subsubsubsubsection Heading}

This is body text following a subsubsubsubsection heading.\label{sssssectA}

\section{Another Section Heading}

If this section is numbered, check that it is numbered correctly.\label%
{sectB}

\subsection{Another Subsection Heading}

If this subsection heading is numbered, check that it is numbered correctly.%
\label{ssectB}

\subsubsection{Another Subsubsection Heading}

If this subsubsection heading is numbered, check that it is numbered
correctly.\label{sssectB}

\paragraph{Another Subsubsubsection Heading}

If this subsubsubsection heading is numbered, check that it is numbered
correctly.\label{ssssectB}

\subparagraph{Another Subsubsubsubsection Heading}

If this subsubsubsubsection heading is numbered, check that it is numbered
correctly.\label{sssssectB}

The sections above are sections \ref{sectA}, \ref{ssectA}, \ref{sssectA}, %
\ref{ssssectA}, \ref{sssssectA}, \ref{sectB}, \ref{ssectB}, \ref{sssectB}, %
\ref{ssssectB}, \ref{sssssectB}.

\section*{Starred Sections}

With version 4, SW/SWP now provides user access to the \verb|\section*|
command. To adjust this on an existing section heading, position the cursor
on the left side of the section. Use edit properties or the context menu
from the right mouse button to bring up the Section Properties dialog.
Starred sections are not numbered and do not appear in the table of contents
when a document is typeset. A better way to handle this is to set the
section numbering counter depth.

\subsection*{Starred Subsection}

\subsubsection*{Starred Subsubsection}

\paragraph*{Starred Subsubsubsection}

\subparagraph*{Starred Subsubsubsubsection}

\section[Short Form]{Sections with Optional Arguments}

Optional arguments to sections are entered by putting them in square
brackets at the start of the heading text. These optional arguments are used
for the short form of a heading in running heads and in the table of contents

\subsection[Short Form]{Subsection}

\subsubsection[Short Form]{Subsubsection}

\paragraph[Short Form]{Subsubsubsection}

\subparagraph[Short Form]{Subsubsubsubsection}

\subsection{Lists}

\subsubsection{Numbered Lists}

\begin{enumerate}
\item First numbered item, level 1.\label{A}

\begin{enumerate}
\item First numbered item, level 2.,\label{AA}

\begin{enumerate}
\item First numbered item, level 3.\label{AAA}

\begin{enumerate}
\item First numbered item, level 4.\label{AAAA}

\item Second numbered item, level 4.\label{AAAB}
\end{enumerate}

\item Second numbered item, leve1 3.\label{AAB}
\end{enumerate}

\item Second numbered item, level 2.\label{AB}
\end{enumerate}

\item Second numbered item, level 1.\label{B}

\begin{enumerate}
\item First numbered item, level 2.\label{BA}

\begin{enumerate}
\item First numbered item, level 3.\label{BAA}

\begin{enumerate}
\item First numbered item, level 4.\label{BAAA}

\item Second numbered item, level 4.\label{BAAB}
\end{enumerate}

\item Second numbered item, leve1 3.\label{BAB}
\end{enumerate}

\item Second numbered item, level 2.\label{BB}
\end{enumerate}
\end{enumerate}

The items above are numbered \ref{A}, \ref{AA}, \ref{AAA}, \ref{AAAA}, \ref%
{AAAB}, \ref{AAB}, \ref{AB}, \ref{B}, \ref{BA}, \ref{BAA}, \ref{BAAA}, \ref%
{BAAB}, \ref{BAB}, \ref{BB} in that order.

Here is a list with multiple levels on one line. There should be five lines,
only. Items should not be split out on separate lines.

\begin{enumerate}
\item First item, normal.

\item 
\begin{enumerate}
\item Second item, both numbers.
\end{enumerate}

\item 
\begin{enumerate}
\item 
\begin{enumerate}
\item Third item, three numbers.
\end{enumerate}
\end{enumerate}

\item 
\begin{enumerate}
\item 
\begin{enumerate}
\item 
\begin{enumerate}
\item Fourth item, four numbers.
\end{enumerate}
\end{enumerate}
\end{enumerate}

\item 
\begin{enumerate}
\item 
\begin{enumerate}
\item 
\begin{enumerate}
\item Fifth item. Second, third, and fourth numbers.
\end{enumerate}
\end{enumerate}
\end{enumerate}
\end{enumerate}

Here is a list with continuation items.

\begin{enumerate}
\item Level 1, first paragraph.

Level 1, second paragraph.

\begin{enumerate}
\item Level 2, first paragraph.

Level 2, second paragraph.

\begin{enumerate}
\item Level 3, first paragraph.

Level 3, second paragraph.

\begin{enumerate}
\item Level 4, first paragraph.

Level 4, second paragraph.
\end{enumerate}
\end{enumerate}
\end{enumerate}
\end{enumerate}

\subsubsection{Bulleted Lists}

\begin{itemize}
\item Bullet item, level 1.

\begin{itemize}
\item Bullet item, level 2.

\begin{itemize}
\item Bullet item, level 3.

\begin{itemize}
\item Bullet item, level 4.
\end{itemize}
\end{itemize}
\end{itemize}

\item Bullet item, level 1.

This is a continuation paragraph in a bullet item. The paragraph has been
made long enough to wrap, and so wrap it will. The question is, do you like
wrap? Is it your taste in music?

\begin{itemize}
\item Bullet item, level 2.

This is a continuation paragraph in a bullet item. The paragraph has been
made long enough to wrap, and so wrap it will. The question is, do you like
wrap?

\begin{itemize}
\item Bullet item, level 3.

This is a continuation paragraph in a bullet item. The paragraph has been
made long enough to wrap, and so wrap it will. The question is, do you like
wrap?

\begin{itemize}
\item Bullet item, level 4.

This is a continuation paragraph in a bullet item. The paragraph has been
made long enough to wrap, and so wrap it will. The question is, do you like
wrap?
\end{itemize}
\end{itemize}
\end{itemize}
\end{itemize}

\subsubsection{Custom Lists}

\begin{description}
\item[gnat] A small animal, found in the North Woods, that causes no end of
trouble.

\item[gnu] A large animal, found in crossword puzzles, that causes no end of
trouble.

\item[armadillo] A medium-sized animal, named after a medium-sized Texas
city. A medium-sized animal, named after a medium-sized Texas city.

\item[A very long custom label here] There should be enough text following
to allow the paragraph to wrap to the next line.

Here is a continuation paragraph of the above custom item.
\end{description}

\subsubsection{Lists with custom lead-ins.}

\paragraph{Numbered lists with custom lead-ins.}

\begin{enumerate}
\item[Number 1:] First numbered item

\item[Number 2:] Second numbered item

\item[Number 3:] Third numbered item

\begin{enumerate}
\item[Number 3.a:] First item under Number 3

\begin{enumerate}
\item[Number 3.a.i:] First item under Number 3.a:

\begin{enumerate}
\item[Number 3.a.i.1:] First item under Number 3.a.i:

\item[Number 3.a.i.2:] Second item under Number 3.a.i:

Continuation paragraph under the second item under Number 3.a.i:
\end{enumerate}
\end{enumerate}
\end{enumerate}
\end{enumerate}

\paragraph{Bulleted lists with custom lead-ins.}

\begin{itemize}
\item[Bullet:] First item

\item[Bullet:] Second item

\item[Bullet:] Third item

\begin{itemize}
\item[Square:] First item under the third item

\begin{itemize}
\item[Triangle:] First item under the first item under the third item

\begin{itemize}
\item[elgnairT:] First item under the first item under the first item under
the third item

\item[elgnairT:] Second item under the first item under the first item under
the third item

Continuation paragraph under the second item under the first item under the
third item
\end{itemize}
\end{itemize}
\end{itemize}
\end{itemize}

\paragraph{Lists with multiple lead-ins per line.}

\begin{enumerate}
\item First

\item 
\begin{enumerate}
\item Second, First
\end{enumerate}

\item 
\begin{enumerate}
\item 
\begin{enumerate}
\item Third, First, First
\end{enumerate}
\end{enumerate}

\item 
\begin{enumerate}
\item 
\begin{enumerate}
\item 
\begin{enumerate}
\item Fourth, First, First, First
\end{enumerate}
\end{enumerate}
\end{enumerate}
\end{enumerate}

\begin{enumerate}
\item Reset to First

\item 
\begin{enumerate}
\item Second, First

\item 
\begin{enumerate}
\item None, Second, First

\item 
\begin{enumerate}
\item None, None, Second, First
\end{enumerate}
\end{enumerate}
\end{enumerate}
\end{enumerate}

\paragraph{Lists with complex custom items.}

\begin{enumerate}
\item[$\protect\int \sin xdx$] Lead-in is $\int \sin xdx$

\item[$\mathfrak{ABC}$] Lead-in is $\mathfrak{ABC}$


%%JCS\item[$\sqrt[3]{3}$]
\item[{$\sqrt[3]{3}$}]If the leadin contains an item with an optional
parameter, that item should be in a group to prevent the closing bracket of
the optional argument from closing the item brackets.

\item[{\protect\FRAME{itbpF}{0.3528in}{0.3105in}{0in}{}{}{swplogo.wmf}{%
\special{language "Scientific Word";type "GRAPHIC";maintain-aspect-ratio
TRUE;display "PICT";valid_file "F";width 0.3528in;height 0.3105in;depth
0in;original-width 0pt;original-height 0pt;cropleft "0";croptop
"1";cropright "1";cropbottom "0";filename
'../Graphics/swplogo.wmf';file-properties "XNPEU";}}}] Lead-in is \FRAME{%
itbpF}{0.3528in}{0.3105in}{0in}{}{}{swplogo.wmf}{\special{language
"Scientific Word";type "GRAPHIC";maintain-aspect-ratio TRUE;display
"PICT";valid_file "F";width 0.3528in;height 0.3105in;depth
0in;original-width 0.3226in;original-height 0.2811in;cropleft "0";croptop
"1";cropright "1";cropbottom "0";filename
'../Graphics/swplogo.wmf';file-properties "XNPEU";}}
\end{enumerate}

\paragraph{Lists whose items contain notes}

\begin{enumerate}
\item A footnote\footnote{%
This footnote is in a list item}

\item A margin note%
\marginpar{
This marginal note is in a list item}

\begin{enumerate}
\item A footnote with lists. \footnote{%
A list in a footnote:
\par
\begin{enumerate}
\item First item
\par
\item Second item
\par
\begin{enumerate}
\item First item
\par
\item Second item
\end{enumerate}
\end{enumerate}
}

\item A margin note with lists. 
\marginpar{
A list in a margin note:
\par
\begin{enumerate}
\item First item
\par
\item Second item
\par
\begin{enumerate}
\item First item
\par
\item Second item
\end{enumerate}
\end{enumerate}
}
\end{enumerate}
\end{enumerate}

\subsection{Block Quote and Centered}

In addition to Body Text, there are paragraph tags named Block Quote and
Center. Here is a Block Quote paragraph.

\begin{quotation}
Why, there, there, there, there! A diamond gone, cost me two thousand ducats
in Frankfort! The curse never fell upon our nation till now; I never felt it
till now. Two thousand ducats in that, and other precious, precious jewels.
I would my daughter were dead at my foot, and the jewels in her ear! Would
she were hearsed at my foot, and the jewels in her coffin! No news of them?
Why, thou loss upon loss! The thief gone with so much, and so much to find
the thief, and no satisfaction, no revenge! Nor no ill luck stirring but
what lights o' my shedding.
\end{quotation}

Here is a centered paragraph:

\begin{center}
Nay, that's true, that's very true. Go, Tubal, fee me an officer; bespeak
him a fortnight before. I will have the heart of him if he forfeit, for were
he out of Venice I can make what merchandise I will. Go, Tubal, and meet me
at our synagogue; go, good Tubal; at our synagogue, Tubal.
\end{center}

\subsection{Footnotes and Marginal Notes}

In this section, we insert multiple footnotes and multiple marginal notes.
This illustrates the various parameters used to separate multiple footnotes
and marginal paragraphs.\footnote{%
The first in a series of consecutive footnotes.}\footnote{%
The second in a series of consecutive footnotes.}\footnote{%
The third in a series of consecutive footnotes.}\footnote{%
The fourth in a series of consecutive footnotes.}%
\marginpar{
The first in a series of marginal notes.}%
\marginpar{
The second in a series of marginal notes.}%
\marginpar{
The third in a series of marginal notes.}%
\marginpar{
The fourth in a series of marginal notes.}

\section{Cross-References and Their Ilk}

\label{crossrefs}The section on theorems is numbered \ref{theorems} and is
on page \pageref{theorems}. The current section (on cross-references and
their ilk) is numbered \ref{crossrefs} and is on page \pageref{crossrefs}.

\section{Text Tags}

Text tags are the tags you apply to text within a paragraph, to distinguish
phrases. There are tags for physical phrase markup and tags for logical
phrase markup.

This section discusses the differences between some aspects of the \LaTeX{}
coding \textsl{SW\/} uses in \LaTeX{}2.09 and \LaTeX{}2e documents. In \LaTeX%
{}2.09, the commands for what we call \emph{tagged runs}, including
boldface, italics, and emphasis, are switches. In \LaTeX{}2e, the switch
commands are still available, but use of the new macro forms is strongly
encouraged. We at TCI agree with this wholeheartedly---switches are an
abomination and have caused us no end of trouble in the past. The new macro
forms of the tagged runs also have additional capabilities, most of which
are supported in \textsl{SW}.

\subsection{Tagged Runs in Text}

Here is a comparison of the tagged runs in text.

\begin{center}
\begin{tabular}{llll}
\textbf{Name} & \textbf{\LaTeX{}2.09} & \textbf{\LaTeX{}2e} & \textbf{Example%
} \\ \hline
Italic & \{\TEXTsymbol{\backslash}it text\} & \TEXTsymbol{\backslash}%
textit\{text\} & \textit{text} \\ 
Bold & \{\TEXTsymbol{\backslash}bf text\} & \TEXTsymbol{\backslash}%
textbf\{text\} & \textbf{text} \\ 
Small Caps & \{\TEXTsymbol{\backslash}sc text\} & \TEXTsymbol{\backslash}%
textsc\{text\} & \textsc{text} \\ 
Sans Serif & \{\TEXTsymbol{\backslash}sf text\} & \TEXTsymbol{\backslash}%
textsf\{text\} & \textsf{text} \\ 
Slant & \{\TEXTsymbol{\backslash}sl text\} & \TEXTsymbol{\backslash}%
textsl\{text\} & \textsl{text} \\ 
Typewriter & \{\TEXTsymbol{\backslash}tt text\} & \TEXTsymbol{\backslash}%
texttt\{text\} & \texttt{text} \\ 
Emphasis & \{\TEXTsymbol{\backslash}em text\} & \{\TEXTsymbol{\backslash}em
text\} & \emph{text} \\ 
Roman & \{\TEXTsymbol{\backslash}rm text\} & \TEXTsymbol{\backslash}%
textrm\{text\} & \textrm{text}%
\end{tabular}
\end{center}

\subsection{Tagged Runs in Mathematics}

All of the text tagged runs are also valid in mathematics. Since \textsl{SW}
works very hard to provide a minimal nesting level of tags, these must be
shown in an enviroment that forces math. Within that environment (here, a
pair of empty expanding brackets), we select text in addition to the tagged
run.

\begin{center}
\begin{tabular}{llll}
\textbf{Name} & \textbf{\LaTeX{}2.09} & \textbf{\LaTeX{}2e} & \textbf{Example%
} \\ \hline
Italic & \$\TEXTsymbol{\backslash}text\{\TEXTsymbol{\backslash}it text\}\$ & 
\$\TEXTsymbol{\backslash}textit\{text\}\$ & $\left. \text{\textit{text}}%
\right. $ \\ 
Bold & \$\TEXTsymbol{\backslash}text\{\TEXTsymbol{\backslash}bf text\}\$ & \$%
\TEXTsymbol{\backslash}textbf\{text\}\$ & $\left. \text{\textbf{text}}%
\right. $ \\ 
Small Caps & \$\TEXTsymbol{\backslash}text\{\TEXTsymbol{\backslash}sc
text\}\$ & \$\TEXTsymbol{\backslash}textsc\{text\}\$ & $\left. \text{\textsc{%
text}}\right. $ \\ 
Sans Serif & \$\TEXTsymbol{\backslash}text\{\TEXTsymbol{\backslash}sf
text\}\$ & \$\TEXTsymbol{\backslash}textsf\{text\}\$ & $\left. \text{\textsf{%
text}}\right. $ \\ 
Slant & \$\TEXTsymbol{\backslash}text\{\TEXTsymbol{\backslash}sl text\}\$ & 
\$\TEXTsymbol{\backslash}textsl\{text\}\$ & $\left. \text{\textsl{text}}%
\right. $ \\ 
Typewriter & \$\TEXTsymbol{\backslash}text\{\TEXTsymbol{\backslash}tt
text\}\$ & \$\TEXTsymbol{\backslash}texttt\{text\}\$ & $\left. \text{\texttt{%
text}}\right. $ \\ 
Emphasis & \$\TEXTsymbol{\backslash}text\{\TEXTsymbol{\backslash}em text\}\$
& \$\{\TEXTsymbol{\backslash}em text\}\$ & $\left. \text{\emph{text}}\right. 
$ \\ 
Roman & \$\TEXTsymbol{\backslash}text\{\TEXTsymbol{\backslash}rm text\}\$ & 
\$\TEXTsymbol{\backslash}textrm\{text\}\$ & $\left. \text{\textrm{text}}%
\right. $%
\end{tabular}
\end{center}

The coding of the above tagged runs presents a number of problems for the
filter. There is now an option to use \TEXTsymbol{\backslash}textrm or 
\TEXTsymbol{\backslash}mbox in place of the \TEXTsymbol{\backslash}text
command.

\begin{center}
\begin{tabular}{llll}
\textbf{Name} & \textbf{\LaTeX{}2.09} & \textbf{\LaTeX{}2e} & \textbf{Example%
} \\ \hline
Italic & \{\TEXTsymbol{\backslash}it text\} & \TEXTsymbol{\backslash}%
textit\{text\} & $\mathit{text}$ \\ 
Bold & \{\TEXTsymbol{\backslash}bf text\} & \TEXTsymbol{\backslash}%
textbf\{text\} & $\mathbf{text}$ \\ 
Small Caps & \{\TEXTsymbol{\backslash}sc text\} & \TEXTsymbol{\backslash}%
textsc\{text\} & $\QTR{sc}{text}$ \\ 
Sans Serif & \{\TEXTsymbol{\backslash}sf text\} & \TEXTsymbol{\backslash}%
textsf\{text\} & $\mathsf{text}$ \\ 
Slant & \{\TEXTsymbol{\backslash}sl text\} & \TEXTsymbol{\backslash}%
textsl\{text\} & $\QTR{sl}{text}$ \\ 
Typewriter & \{\TEXTsymbol{\backslash}tt text\} & \TEXTsymbol{\backslash}%
texttt\{text\} & $\mathtt{text}$ \\ 
Emphasis & NA & NA & NA \\ 
Roman & \{\TEXTsymbol{\backslash}rm text\} & \TEXTsymbol{\backslash}%
textrm\{text\} & $\mathrm{text}$%
\end{tabular}
\end{center}

\textsc{Small Caps Here!}

\textsl{This sentence is slanted.}

This is math $\left[ x+\mathit{italic}+y+\mathsf{sansserif}+\mathbf{bold}+%
\mathrm{roman}\right] $

Here is a display 
\begin{equation*}
x+\text{text}+y+\mathrm{roman}+\text{\textrm{roman text}}
\end{equation*}
and then we finish off the paragraph with great alacrity.

\textit{Now we use italics. Here is a display} 
\begin{equation*}
x+\text{text}+y+\mathrm{roman}+\text{\textrm{roman text}}
\end{equation*}
\textit{and then we finish off the paragraph with great alacrity.}

\textit{This is a paragraph of italic text with a stretch of \textrm{roman}
in the middle.}

You obtain \TEXTsymbol{\backslash}mathrm by applying roman to a selection in
math, and you obtain \TEXTsymbol{\backslash}textrm by applying text and then
applying roman. $\left[ x+\mathrm{roman}+\text{\textrm{textroman}}\right] $

\subsection{Bold Symbol}

Bold Symbol is a special tag you can apply in math to build italic bold
symbols. For example: $\boldsymbol{m}$. (Compare with $\mathbf{m}$). Another
examples: $\boldsymbol{2\pi i\neq }$ $\boldsymbol{\dot{x}}$.

\subsection{Calligraphic, Fraktur and Blackboard Bold}

These are math--only tags.

$\mathcal{CALLIGRAPHIC}$

$\mathfrak{fraktur}$

$\mathbb{BLACKBOARDBOLD}$

\section{The \texttt{\TEXTsymbol{\backslash}text} Command}

\textsl{SW} writes text in mathematics using the AMS \TEXTsymbol{\backslash}%
text command. This is the only command for text in mathematics that reduces
size appropriately when smaller fonts are called for in \LaTeX{}2.09. The 
\TEXTsymbol{\backslash}rm switch also works in \LaTeX{}2e. Here is text in
mathematics at various sizes: 
\begin{equation*}
X^{x+\text{text}^{x+\text{text}}}+\text{text}
\end{equation*}
and the same, attempted with \TEXTsymbol{\backslash}rm: 
\begin{equation*}
X^{x+\mathrm{text}^{x+\mathrm{text}}}+\mathrm{text}
\end{equation*}
and with \TEXTsymbol{\backslash}mbox: 
\begin{equation*}
X^{x+\mbox{text}^{x+\mbox{text}}}+\mbox{text}
\end{equation*}

\section{Operators}

The following is a list of operators from the \textsc{Insert, Operator}...
menu selection.

\begin{tabular}{|l|l|l|l|l|l|l|l|l|}
\hline
$\int $ & $\iint $ & $\iiint $ & $\iiiint $ & $\idotsint $ & $\oint $ & $%
\sum $ & $\prod $ & $\bigcap $ \\ \hline
$\bigwedge $ & $\bigoplus $ & $\bigodot $ & $\bigsqcup $ & $\coprod $ & $%
\bigcup $ & $\bigvee $ & $\bigotimes $ & $\biguplus $ \\ \hline
\end{tabular}

The following is a list of operators from the \textsc{Insert, Operator}...
menu selection. In this example, all operators are set to ``Big,'' while all
limit positions are set ``at right.''

\begin{tabular}{|l|l|l|l|l|}
\hline
$\dint\nolimits_{b}^{a}$ & $\diint\nolimits_{A}^{b}$ & $\diiint\nolimits_{%
\Delta }dV$ & $\diiiint\nolimits_{what?}dT$ & $\didotsint\nolimits_{\ldots
}d\cdots $ \\ \hline
$\dbigwedge\nolimits_{i}x_{i}$ & $\dbigoplus\nolimits^{100}X_{i}$ & $%
\dbigodot\nolimits_{\nu \neq v}\Psi _{\nu }$ & $\dbigsqcup\nolimits_{\xi
=\left( 
\begin{array}{ll}
1 & 2 \\ 
3 & 4%
\end{array}%
\right) }\xi $ & $\dcoprod\nolimits^{\text{A small paragraph}}\Xi _{i}$ \\ 
\hline
\end{tabular}
\vspace{1.019pc}

\begin{tabular}{|l|l|l|l|}
\hline
$\doint\nolimits_{\Gamma }d\gamma $ & $\dsum\nolimits_{x=1}^{5}\frac{1}{x^{2}%
}$ & $\dprod\nolimits_{x=5}^{7}\left( 1-\sqrt{x}\right) $ & $%
\dbigcap\nolimits_{A\in \mathbf{A}}A$ \\ \hline
$\dbigcup\nolimits_{x^{x^{x}}}U_{x}$ & $\dbigvee\nolimits_{\int i}y^{i}$ & $%
\dbigotimes\nolimits_{\mu \in \{1,2,5,7\}}\Phi \mu $ & $\dbiguplus%
\nolimits_{Blow}^{Joe}B$ \\ \hline
\end{tabular}

The following is a list of operators from the \textsc{Insert, Operator}...
menu selection. In this example, all operators are set to ``Small,'' while
all limit positions are set ``at right.''

\begin{tabular}{|l|l|l|l|l|}
\hline
$\tint\nolimits_{b}^{a}xdx$ & $\tiint\nolimits_{A}\frac{dxdy}{xy}$ & $%
\tiiint\nolimits_{\Delta }dV$ & $\tiiiint\nolimits_{what?}dT$ & $%
\tidotsint\nolimits_{\ldots }d\cdots $ \\ \hline
$\tbigwedge\nolimits_{i}x_{i}$ & $\tbigoplus\nolimits^{100}X_{i}$ & $%
\tbigodot\nolimits_{\nu \neq v}\Psi _{\nu }$ & $\tbigsqcup\nolimits_{\xi
=\left( 
\begin{array}{ll}
1 & 2 \\ 
3 & 4%
\end{array}%
\right) }\xi $ & $\tcoprod\nolimits^{\text{A small paragraph}}\Xi _{i}$ \\ 
\hline
\end{tabular}
\vspace{1.019pc}

\begin{tabular}{|l|l|l|l|}
\hline
$\toint\nolimits_{\Gamma }d\gamma $ & $\tsum\nolimits_{x=1}^{5}\frac{1}{x^{2}%
}$ & $\tprod\nolimits_{x=5}^{7}\left( 1-\sqrt{x}\right) $ & $%
\tbigcap\nolimits_{A\in \mathbf{A}}A$ \\ \hline
$\tbigcup\nolimits_{x^{x^{x}}}U_{x}$ & $\tbigvee\nolimits_{\int i}y^{i}$ & $%
\tbigotimes\nolimits_{\mu \in \{1,2,5,7\}}\Phi \mu $ & $\tbiguplus%
\nolimits_{Blow}^{Joe}B$ \\ \hline
\end{tabular}

The following is a list of operators from the \textsc{Insert, Operator}...
menu selection. In this example, all operators are set to ``Big,'' while all
limit positions are set ``Above/below.''

\begin{tabular}{|l|l|l|l|l|}
\hline
$\dint\limits_{b}^{a}xdx$ & $\diint\limits_{A}\frac{dxdy}{xy}$ & $%
\diiint\limits_{\Delta }dV$ & $\diiiint\limits_{what?}dT$ & $%
\didotsint\limits_{\ldots }d\cdots $ \\ \hline
$\dbigwedge\limits_{i}x_{i}$ & $\dbigoplus\limits^{100}X_{i}$ & $%
\dbigodot\limits_{\nu \neq v}\Psi _{\nu }$ & $\dbigsqcup\limits_{\xi =\left( 
\begin{array}{ll}
1 & 2 \\ 
3 & 4%
\end{array}%
\right) }\xi $ & $\dcoprod\limits^{\text{A small paragraph}}\Xi _{i}$ \\ 
\hline
\end{tabular}
\vspace{1.019pc}

\begin{tabular}{|l|l|l|l|}
\hline
$\doint\limits_{\Gamma }d\gamma $ & $\dsum\limits_{x=1}^{5}\frac{1}{x^{2}}$
& $\dprod\limits_{x=5}^{7}\left( 1-\sqrt{x}\right) $ & $\dbigcap\limits_{A%
\in \mathbf{A}}A$ \\ \hline
$\dbigcup\limits_{x^{x^{x}}}U_{x}$ & $\dbigvee\limits_{\int i}y^{i}$ & $%
\dbigotimes\limits_{\mu \in \{1,2,5,7\}}\Phi \mu $ & $\dbiguplus%
\limits_{Blow}^{Joe}B$ \\ \hline
\end{tabular}

\subsection{Multi-Line Subscripts and Superscripts}

$\sum_{\substack{ 1<i<10  \\ 1<j<10}}^{{}}2^{i+j}$ is more reasonable than $%
\Gamma _{1_{\substack{ 2_{\substack{ 3  \\ 4}}  \\ 5^{\substack{ 6  \\ 7}}}}%
}^{1 ^{\substack{ 5^{\substack{ 7  \\ 6}}  \\ 2_{\substack{ 4  \\ 5}}}}}$

\section{Brackets}

The following is a list of the different brackets from the \textsc{Insert,
Brackets}... menu selection.

%%JCS$%
%%JCS\begin{tabular}{|l|l|l|l|l|l|}
%%JCS\hline
%%JCS$\left( \frac{a}{\frac{x}{y}}\right) $ & $\left\lfloor \frac{E}{X}%
%%JCS\right\rfloor $ & $\left/ \frac{\mathbf{I}}{\mathbf{J}}\right/ $ & $%
%%JCS\left\Updownarrow \frac{m}{\frac{m}{m}}\right\Updownarrow $ & $%
%%JCS\left\Downarrow \frac{Q}{\frac{Q}{\frac{Q}{Q}}}\right\Downarrow $ & $\left\} 
%%JCS\mathsf{U}\right\{ $ \\ \hline
%%JCS$\left[ \frac{b}{\frac{x}{y}}\right] $ & $\left\lceil \frac{F}{X}%
%%JCS\right\rceil $ & $\left\backslash \frac{\mathbf{J}}{\mathbf{L}}%
%%JCS\right\backslash $ & $\left\uparrow \frac{n}{\frac{n}{n}}\right\uparrow $ & $%
%%JCS\left\rceil \frac{R}{\frac{R}{\frac{R}{R}}}\right\lceil $ & $\left] \mathsf{V%
%%JCS}\right[ $ \\ \hline
%%JCS$\left\{ \frac{c}{\frac{x}{y}}\right\} $ & $\left\vert \frac{G}{X}%
%%JCS\right\vert $ & $\left\updownarrow \frac{\mathbf{K}}{\mathbf{L}}%
%%JCS\right\updownarrow $ & $\left\Uparrow \frac{o}{\frac{o}{o}}\right\Uparrow $
%%JCS& $\left\rfloor \frac{S}{\frac{S}{\frac{S}{S}}}\right\lfloor $ & $\left) 
%%JCS\mathsf{W}\right( $ \\ \hline
%%JCS$\left\langle \frac{d}{\frac{x}{y}}\right\rangle $ & $\left\Vert \frac{H}{X}%
%%JCS\right\Vert $ & $\left. \frac{\mathbf{L}}{\mathbf{X}}\right. $ & $%
%%JCS\left\downarrow \frac{p}{\frac{p}{p}}\right\downarrow $ & $\left\rangle 
%%JCS\frac{T}{\frac{T}{\frac{T}{T}}}\right\langle $ &  \\ \hline
%%JCS\end{tabular}%
%%JCS\ $

\section{Math Names}

The following is a sample of names from the \textsc{Insert, Math Name}...
menu selection. Some are standard, built-in names: $\ln x$, $\sin ^{2}x$, $%
\arg \tanh \pi $. Or you can make up your own: $\limfunc{above}\limits^{a\in
B}\left( \left\{ a\right\} \right) $, $\limfunc{right}\nolimits_{b\in
A}\left( \left\{ b\right\} \right) $,$\ \func{func}\left( x\right) $.

\section{Binomials}

The following is a list of operators from the \textsc{Insert, Binomials}...
menu selection.

\noindent 
\begin{tabular}{|l|l|l|l|l|l|l|l|l|l|l|l|}
\hline
$\QOVERD( ) {2}{3}$ & $\QOVERD[ ] {3}{4}$ & $\QOVERD\{ \} {4}{5}$ & $%
\QOVERD\langle \rangle {5}{6}$ & $\QOVERD\lfloor \rfloor {6}{7}$ & $%
\QOVERD\lceil \rceil {7}{8}$ & $\QOVERD\vert \vert {8}{9}$ & $\QOVERD\Vert
\Vert {9}{0}$ & $\QOVERD/ / {0}{1}$ & $\QOVERD\backslash \backslash {1}{2}$
& $\QOVERD\updownarrow \updownarrow {2}{3}$ & $\QOVERD. . {3}{4}$ \\ \hline
$\QOVERD\Updownarrow \Updownarrow {4}{5}$ & $\QOVERD\uparrow \uparrow {5}{6}$
& $\QOVERD\Uparrow \Uparrow {6}{7}$ & $\QOVERD\downarrow \downarrow {7}{8}$
& $\QOVERD\Downarrow \Downarrow {8}{9}$ & $\QOVERD\rceil \lceil {9}{0}$ & $%
\QOVERD\rfloor \lfloor {0}{1}$ & $\QOVERD\rangle \langle {1}{2}$ & $%
\QOVERD\} \{ {2}{3}$ & $\QOVERD] [ {3}{4}$ & $\QOVERD) ( {4}{5}$ &  \\ \hline
\end{tabular}

You can modify the thickness of the line: none $\binom{q}{p}$, normal $%
\QOVERD( ) {q}{p}$, thick $\QABOVED( ) {1pt}{q}{p}$. And the size : auto $%
\binom{q}{p}$, big $\dbinom{q}{p}$, small $\tbinom{q}{p}$. In a display, the
auto should become big:

\begin{equation*}
\text{auto }\binom{q}{p}\text{, big }\dbinom{q}{p}\text{, small }\tbinom{q}{p%
}
\end{equation*}%
And you can make binomials with no delimiters: none $\QATOP{q}{p}$, normal $%
\frac{q}{p}$, thick $\QABOVE{1pt}{q}{p}$.

\section{Labels}

Use \textsc{Insert, Label}... to build labels above or below. $A\overset{%
f\left( x\right) }{\longrightarrow }B$ or $A^{\ast }\underset{f^{-1}}{%
\Longleftarrow }B^{\ast }$.

\section{Decorations}

The following is a list of decorations from the \textsc{Insert, Decorations}%
... menu selection.

%% jcs$%
%% jcs\begin{tabular}{|l|l|l|l|l|l|l|}
%% jcs\hline
%% jcs$\overline{\frac{1}{1+a^{2}}}$ & $\underleftarrow{\frac{1}{1+d^{2}}}$ & $%
%% jcs\overleftrightarrow{\frac{1}{1+g^{2}}}$ & $\underbrace{\frac{1}{1+j^{2}}}$ & 
%% jcs$\widetilde{\frac{1}{1+m^{2}}}$ & $\underline{\frac{1}{1+b^{2}}}$ & $%
%% jcs\overrightarrow{\frac{1}{1+e^{2}}}$ \\ \hline
%% jcs$\frac{1}{1+h^{2}}$ & $\widehat{\frac{1}{1+k^{2}}}$ & $\frame{$\frac{1}{%
%% jcs1+n^{2}}$}$ & $\overleftarrow{\frac{1}{1+c^{2}}}$ & $\underrightarrow{\frac{1%
%% jcs}{1+f^{2}}}$ & $\overbrace{\frac{1}{1+i^{2}}}$ & $\fbox{$\frac{1}{1+l^{2}}$}$
%% jcs\\ \hline
%% jcs\end{tabular}%
%% jcs\ $

\section{The Symbols}

Here are the symbols on the panels.

\textbf{Note: }If you typeset this document and do not use a multilingual
format file such as TrueTeX MultiLingual, several characters in the Latin-1,
Latin Extended-A, and General Punctuation panels will not typeset properly
and will instead appear as \textbf{???} in the typeset document.

\subsection{Lowercase Greek}

\begin{tabular}{llll}
$\alpha $ & $\beta $ & $\gamma $ & $\delta $ \\ 
$\epsilon $ & $\varepsilon $ & $\zeta $ & $\eta $ \\ 
$\theta $ & $\vartheta $ & $\iota $ & $\kappa $ \\ 
$\lambda $ & $\mu $ & $\nu $ & $\xi $ \\ 
$\pi $ & $\varpi $ & $\rho $ & $\sigma $ \\ 
$\varsigma $ & $\tau $ & $\upsilon $ & $\phi $ \\ 
$\varphi $ & $\chi $ & $\psi $ & $\omega $ \\ 
$\varkappa $ & $\varrho $ &  & 
\end{tabular}

\subsection{Uppercase Greek}

\begin{tabular}{ll}
$\Gamma $ & $\Delta $ \\ 
$\Theta $ & $\Lambda $ \\ 
$\Xi $ & $\Pi $ \\ 
$\Sigma $ & $\Upsilon $ \\ 
$\Phi $ & $\Psi $ \\ 
$\Omega $ & $\digamma $%
\end{tabular}

\subsection{Binary Operations}

\begin{tabular}{lllll}
$\pm $ & $\mp $ & $\times $ & $\cdot $ & $\ast $ \\ 
$\cup $ & $\cap $ & $\circ $ & $\bullet $ & $\div $ \\ 
$\sqcup $ & $\sqcap $ & $\bigcirc $ & $\setminus $ & $\smallsetminus $ \\ 
$\oplus $ & $\ominus $ & $\otimes $ & $\odot $ & $\circledast $ \\ 
$\vee $ & $\wedge $ & $\circledcirc $ & $\lhd $ & $\rhd $ \\ 
$\veebar $ & $\barwedge $ & $\wr $ & $\triangleleft $ & $\triangleright $ \\ 
$\uplus $ & $\doublebarwedge $ & $\amalg $ & $\unlhd $ & $\unrhd $ \\ 
$\bigtriangledown $ & $\bigtriangleup $ & $\diamond $ & $\dagger $ & $%
\ddagger $ \\ 
$\boxplus $ & $\boxminus $ & $\boxtimes $ & $\boxdot $ & $\oslash $ \\ 
$\Cup $ & $\Cap $ & $\ltimes $ & $\rtimes $ & $\star $ \\ 
$\curlyvee $ & $\curlywedge $ & $\leftthreetimes $ & $\rightthreetimes $ & $%
\intercal $ \\ 
$\dotplus $ & $\circleddash $ & $\divideontimes $ & $\centerdot $ & $%
\smallint $ \\ 
$\And $ &  &  &  & 
\end{tabular}

\subsection{Binary Relations}

\begin{tabular}{llllll}
$\leq $ & $\prec $ & $\preceq $ & $\ll $ & $\subset $ & $\subseteq $ \\ 
$\sqsubset $ & $\sqsubseteq $ & $\in $ & $\vdash $ & $\smile $ & $\frown $
\\ 
$\equiv $ & $\sim $ & $\simeq $ & $\asymp $ & $\approx $ & $\geq $ \\ 
$\succ $ & $\succeq $ & $\gg $ & $\supset $ & $\supseteq $ & $\sqsupset $ \\ 
$\sqsupseteq $ & $\ni $ & $\dashv $ & $\mid $ & $\parallel $ & $\perp $ \\ 
$\cong $ & $\bowtie $ & $\propto $ & $\models $ & $\doteq $ & $\Join $ \\ 
$\Vdash $ & $\Vvdash $ & $\vDash $ & $\circeq $ & $\succsim $ & $\gtrsim $
\\ 
$\gtrapprox $ & $\therefore $ & $\because $ & $\doteqdot $ & $\triangleq $ & 
$\precsim $ \\ 
$\lesssim $ & $\lessapprox $ & $\eqslantless $ & $\eqslantgtr $ & $%
\curlyeqprec $ & $\curlyeqsucc $ \\ 
$\preccurlyeq $ & $\leqq $ & $\leqslant $ & $\lessgtr $ & $\risingdotseq $ & 
$\fallingdotseq $ \\ 
$\succcurlyeq $ & $\geqq $ & $\geqslant $ & $\gtrless $ & $\vartriangleright 
$ & $\vartriangleleft $ \\ 
$\trianglerighteq $ & $\trianglelefteq $ & $\between $ & $%
\blacktriangleright $ & $\blacktriangleleft $ & $\eqcirc $ \\ 
$\lesseqgtr $ & $\gtreqless $ & $\lesseqqgtr $ & $\gtreqqless $ & $%
\varpropto $ & $\eqcirc $ \\ 
$\smallfrown $ & $\Subset $ & $\Supset $ & $\subseteqq $ & $\supseteqq $ & $%
\bumpeq $ \\ 
$\Bumpeq $ & $\lll $ & $\ggg $ & $\pitchfork $ & $\backsim $ & $\backsimeq $
\\ 
$\eqsim $ & $\lessdot $ & $\gtrdot $ & $\shortmid $ & $\shortparallel $ & $%
\thicksim $ \\ 
$\thickapprox $ & $\approxeq $ & $\succapprox $ & $\precapprox $ & $%
\backepsilon $ & $\vartriangle $%
\end{tabular}

\subsection{Negated Relations}

\begin{tabular}{lllll}
$\neq $ & $\notin $ & $\lvertneqq $ & $\gvertneqq $ & $\nleq $ \\ 
$\ngeq $ & $\nless $ & $\ngtr $ & $\nprec $ & $\nsucc $ \\ 
$\lneqq $ & $\gneqq $ & $\nleqslant $ & $\ngeqslant $ & $\lneq $ \\ 
$\gneq $ & $\npreceq $ & $\nsucceq $ & $\precnsim $ & $\succnsim $ \\ 
$\lnsim $ & $\gnsim $ & $\nleqq $ & $\ngeqq $ & $\precneqq $ \\ 
$\succneqq $ & $\precnapprox $ & $\succnapprox $ & $\lnapprox $ & $\gnapprox 
$ \\ 
$\nsim $ & $\ncong $ & $\varsubsetneq $ & $\varsupsetneq $ & $\nsubseteqq $
\\ 
$\nsupseteqq $ & $\subsetneqq $ & $\supsetneqq $ & $\varsubsetneqq $ & $%
\varsupsetneqq $ \\ 
$\subsetneq $ & $\supsetneq $ & $\nsubseteq $ & $\nsupseteq $ & $\nparallel $
\\ 
$\nmid $ & $\nshortmid $ & $\nshortparallel $ & $\nvdash $ & $\nVdash $ \\ 
$\nvDash $ & $\nVDash $ & $\ntrianglerighteq $ & $\ntrianglelefteq $ & $%
\ntriangleright $ \\ 
$\ntriangleleft $ & $\nleftarrow $ & $\nrightarrow $ & $\nLeftarrow $ & $%
\nRightarrow $ \\ 
$\nLeftrightarrow $ & $\nleftrightarrow $ &  &  & 
\end{tabular}

\subsection{Arrows}

\begin{tabular}{lllll}
$\leftarrow $ & $\Leftarrow $ & $\rightarrow $ & $\Rightarrow $ & $%
\leftrightarrow $ \\ 
$\Leftrightarrow $ & $\mapsto $ & $\hookleftarrow $ & $\leftharpoonup $ & $%
\leftharpoondown $ \\ 
$\uparrow $ & $\Uparrow $ & $\downarrow $ & $\Downarrow $ & $\updownarrow $
\\ 
$\longleftarrow $ & $\Longleftarrow $ & $\longrightarrow $ & $%
\Longrightarrow $ & $\longleftrightarrow $ \\ 
$\Longleftrightarrow $ & $\longmapsto $ & $\hookrightarrow $ & $%
\rightharpoonup $ & $\rightharpoondown $ \\ 
$\leadsto $ & $\Updownarrow $ & $\nearrow $ & $\searrow $ & $\swarrow $ \\ 
$\nwarrow $ & $\circlearrowright $ & $\circlearrowleft $ & $%
\rightleftharpoons $ & $\leftrightharpoons $ \\ 
$\twoheadrightarrow $ & $\twoheadleftarrow $ & $\leftleftarrows $ & $%
\rightrightarrows $ & $\upuparrows $ \\ 
$\downdownarrows $ & $\upharpoonright $ & $\downharpoonright $ & $%
\upharpoonleft $ & $\downharpoonleft $ \\ 
$\rightarrowtail $ & $\leftarrowtail $ & $\leftrightarrows $ & $%
\rightleftarrows $ & $\Lsh $ \\ 
$\Rsh $ & $\rightsquigarrow $ & $\leftrightsquigarrow $ & $\looparrowleft $
& $\looparrowright $ \\ 
$\multimap $ & $\Rrightarrow $ & $\Lleftarrow $ & $\curvearrowleft $ & $%
\curvearrowright $ \\ 
$\dashrightarrow $ & $\dashleftarrow $ & $\implies $ & $\impliedby $ & $\iff 
$%
\end{tabular}

\subsection{Miscellaneous}

\begin{tabular}{llllll}
$\infty $ & $\partial $ & \ldots & $\cdots $ & $\vdots $ & $\ddots $ \\ 
\i & \j & $\ell $ & $\wp $ & $\hbar $ & $\hslash $ \\ 
$\forall $ & $\exists $ & $\nexists $ & $\Re $ & $\Im $ & $\aleph $ \\ 
$\prime $ & $\backprime $ & $\emptyset $ & $\varnothing $ & $\diagup $ & $%
\diagdown $ \\ 
$\nabla $ & $\surd $ & $\top $ & $\bot $ & $\eth $ & $\lambdabar $ \\ 
$\angle $ & $\measuredangle $ & $\sphericalangle $ & $\circledS $ & $%
\tciLaplace $ & $\tciFourier $ \\ 
$\lnot $ & $\complement $ & $\Finv $ & $\Game $ & $\bigstar $ & $\Vert $ \\ 
$\square $ & $\blacksquare $ & $\Diamond $ & $\lozenge $ & $\blacklozenge $
& \TEXTsymbol{\backslash} \\ 
$\triangle $ & $\blacktriangle $ & $\triangledown $ & $\blacktriangledown $
& \dag & \ddag \\ 
$\beth $ & $\gimel $ & $\daleth $ & $\sharp $ & $\flat $ & $\natural $ \\ 
$\clubsuit $ & $\diamondsuit $ & $\heartsuit $ & $\spadesuit $ & $\checkmark 
$ & $\maltese $ \\ 
$\Bbbk $ & $%
%TCIMACRO{\U{2102} }%
%BeginExpansion
\mathbb{C}
%EndExpansion
$ & $%
%TCIMACRO{\U{2115} }%
%BeginExpansion
\mathbb{N}
%EndExpansion
$ & $%
%TCIMACRO{\U{211a} }%
%BeginExpansion
\mathbb{Q}
%EndExpansion
$ & $%
%TCIMACRO{\U{211d} }%
%BeginExpansion
\mathbb{R}
%EndExpansion
$ & $%
%TCIMACRO{\U{2124} }%
%BeginExpansion
\mathbb{Z}
%EndExpansion
$ \\ 
\cents  & \pounds  & \textcurrency  & \yen  & \euro & \textfranc \\ 
\textlira & \textpeseta  & \texttrademark & $\mho $ & $\colon $ & 
\end{tabular}

\subsection{Delimiters}

\begin{tabular}{ll}
$\lfloor $ & $\rfloor $ \\ 
$\lceil $ & $\rceil $ \\ 
$\langle $ & $\rangle $ \\ 
$\ulcorner $ & $\urcorner $ \\ 
$\llcorner $ & $\lrcorner $%
\end{tabular}

\subsection{Latin-1}

\begin{tabular}{llllll}
~ & $%
%TCIMACRO{\U{b0}}%
%BeginExpansion
{{}^\circ}%
%EndExpansion
$ & \`{A} & \DH  & \`{a} & \dh  \\ 
\textexclamdown  & $\pm $ & \'{A} & \~{N} & \'{a} & \~{n} \\ 
\cents  & 
%TCIMACRO{\U{b2} }%
%BeginExpansion
${{}^2}$
%EndExpansion
& \^{A} & \`{O} & \^{a} & \`{o} \\ 
\pounds  & 
%TCIMACRO{\U{b3} }%
%BeginExpansion
${{}^3}$
%EndExpansion
& \~{A} & \'{O} & \~{a} & \'{o} \\ 
\textcurrency  & 
%TCIMACRO{\U{b4} }%
%BeginExpansion
\'{}
%EndExpansion
& \"{A} & \^{O} & \"{a} & \^{o} \\ 
\yen  & 
%TCIMACRO{\U{b5} }%
%BeginExpansion
$\mu$
%EndExpansion
& \AA  & \~{O} & \aa  & \~{o} \\ 
$%
%TCIMACRO{\U{a6}}%
%BeginExpansion
{\vert}%
%EndExpansion
$ & \P  & \AE  & \"{O} & \ae  & \"{o} \\ 
\S  & \textperiodcentered & \c{C} & $\times $ & \c{c} & $\div $ \\ 
%TCIMACRO{\U{a8} }%
%BeginExpansion
\"{}
%EndExpansion
& 
%TCIMACRO{\U{b8} }%
%BeginExpansion
\c{}
%EndExpansion
& \`{E} & \O  & \`{e} & \o  \\ 
\copyright  & 
%TCIMACRO{\U{b9} }%
%BeginExpansion
${{}^1}$
%EndExpansion
& \'{E} & \`{U} & \'{e} & \`{u} \\ 
%TCIMACRO{\U{aa} }%
%BeginExpansion
${{}^a}$
%EndExpansion
& 
%TCIMACRO{\U{ba} }%
%BeginExpansion
${{}^o}$
%EndExpansion
& \^{E} & \'{U} & \^{e} & \'{u} \\ 
\guillemotleft  & \guillemotright  & \"{E} & \^{U} & \"{e} & \^{u} \\ 
$\lnot $ & 
%TCIMACRO{\U{bc} }%
%BeginExpansion
$\frac14$
%EndExpansion
& \`{I} & \"{U} & \`{\i} & \"{u} \\ 
- & 
%TCIMACRO{\U{bd} }%
%BeginExpansion
$\frac12$
%EndExpansion
& \'{I} & \'{Y} & \'{\i} & \'{y} \\ 
\registered  & 
%TCIMACRO{\U{be} }%
%BeginExpansion
$\frac34$
%EndExpansion
& \^{I} & \TH  & \^{\i} & \th  \\ 
%TCIMACRO{\U{af} }%
%BeginExpansion
\={}
%EndExpansion
& \textquestiondown  & \"{I} & \ss  & \"{\i} & \"{y}%
\end{tabular}

\subsection{Latin Extended-A}

\begin{tabular}{llllllll}
\={A} & \DJ  & \.{G} & \.{I} & 
%TCIMACRO{\U{140} }%
%BeginExpansion
l{\hskip.15em}\llap{\raise.25ex\hbox{$\cdot$}}
%EndExpansion
& \H{O} & \v{S} & \H{U} \\ 
\={a} & \dj  & \.{g} & \i & \L  & \H{o} & \v{s} & \H{u} \\ 
\u{A} & \={E} & \c{G} & 
%TCIMACRO{\U{132} }%
%BeginExpansion
\symbol{156}
%EndExpansion
& \l  & \OE  & \c{T} & \k{U} \\ 
\u{a} & \={e} & \c{g} & 
%TCIMACRO{\U{133} }%
%BeginExpansion
\symbol{188}
%EndExpansion
& \'{N} & \oe  & \c{t} & \k{u} \\ 
\k{A} & \u{E} & \^{H} & \^{J} & \'{n} & \'{R} & \v{T} & \^{W} \\ 
\k{a} & \u{e} & \^{h} & \^{\j} & \c{N} & \'{r} & \v{t} & \^{w} \\ 
\'{C} & \.{E} & 
%TCIMACRO{\U{126} }%
%BeginExpansion
H\llap{\protect\rule[1.1ex]{.735em}{.1ex}}
%EndExpansion
& \c{K} & \c{n} & \c{R} & 
%TCIMACRO{\U{166} }%
%BeginExpansion
T{\hskip-.05em}\llap{\protect\rule[.7ex]{.6em}{.1ex}}{\hskip.05em}
%EndExpansion
& \^{Y} \\ 
\'{c} & \.{e} & 
%TCIMACRO{\U{127} }%
%BeginExpansion
h{\hskip-.2em}\llap{\protect\rule[1.1ex]{.325em}{.1ex}}{\hskip.2em}
%EndExpansion
& \c{k} & \v{N} & \c{r} & 
%TCIMACRO{\U{167} }%
%BeginExpansion
t{\hskip-.05em}\llap{\protect\rule[.7ex]{.36em}{.1ex}}{\hskip.05em}
%EndExpansion
& \^{y} \\ 
\^{C} & \k{E} & \~{I} & \U{138} & \v{n} & \v{R} & \~{U} & \"{Y} \\ 
\^{c} & \k{e} & \~{\i} & \'{L} & 
%TCIMACRO{\U{149} }%
%BeginExpansion
n{\hskip-.3em}\llap{\raise1.5ex\hbox{,}}{\hskip.3em}
%EndExpansion
& \v{r} & \~{u} & \'{Z} \\ 
\.{C} & \v{E} & \={I} & \'{l} & \NG  & \'{S} & \={U} & \'{z} \\ 
\.{c} & \v{e} & \={\i} & \c{L} & \ng  & \'{s} & \={u} & \.{Z} \\ 
\v{C} & \^{G} & \u{I} & \c{l} & \={O} & \^{S} & \u{U} & \.{z} \\ 
\v{c} & \^{g} & \u{\i} & \v{L} & \={o} & \^{s} & \u{u} & \v{Z} \\ 
\v{D} & \u{G} & \k{I} & \v{l} & \u{O} & \c{S} & \r{U} & \v{z} \\ 
\v{d} & \u{g} & \k{i} & 
%TCIMACRO{\U{13f} }%
%BeginExpansion
L{\hskip-.05em}\llap{\raise.5ex\hbox{$\cdot$}}{\hskip.05em}
%EndExpansion
& \u{o} & \c{s} & \r{u} & 
\end{tabular}

\subsection{General Punctuation}

\begin{tabular}{ll}
-- & --- \\ 
` & ' \\ 
\quotesinglbase & \quotedblbase \\ 
\textquotedblleft & \textquotedblright \\ 
\guilsinglleft  & \guilsinglright 
\end{tabular}

\begin{center}
\TeX{} converts the following pairs and triples to single characters called
ligatures

ff fi fl ffi ffl -- \textbf{ff fi fl ffi ffl} -- \textit{ff fi fl ffi ffl}

Kerning is the subtle adjustment of certain pairs of characters.

To, Wo, Ro -- \textbf{To, Wo, Ro }\textit{-- To, Wo, Ro}
\end{center}

\section{Type Styles and Kerns}

The text in this section tests the available \LaTeX{} type styles (plain,
bold, italic, and bold italic) for each of the basic fonts (Roman-Times,
Sans Serif-Arial, and Typewriter-Courier).

\subsection{Type Styles for Roman, Sans Serif, and Typewriter}

Roman \textbf{bold}, \textit{italic}, and \textbf{\textit{bold italic}}.

\textsf{Sans Serif \textbf{bold}, \textit{italic}, and \textbf{\textit{bold
italic}}.}

\texttt{Typewriter \textbf{bold,} \textit{italic,} and\textbf{\ \textit{bold
italic}.}}

{\tiny tiny roman \textbf{bold},\textit{\ italic}, and \textbf{\textit{bold
italic}}}

\textsf{{\tiny tiny Sans Serif \textbf{bold},\textit{\ italic}, and \textbf{%
\textit{bold italic}}}}

\texttt{{\tiny tiny Typewriter \textbf{bold},\textit{\ italic}, and \textbf{%
\textit{bold italic}}}}

{\scriptsize scriptsize roman \textbf{bold},\textit{\ italic}, and \textbf{%
\textit{bold italic}}}

\textsf{{\scriptsize scriptsize Sans Serif \textbf{bold},\textit{\ italic},
and \textbf{\textit{bold italic}}}}

\texttt{{\scriptsize scriptsize Typewriter \textbf{bold},\textit{\ italic},
and \textbf{\textit{bold italic}}}}

{\footnotesize footnotesize roman \textbf{bold},\textit{\ italic}, and 
\textbf{\textit{bold italic}}}

\textsf{{\footnotesize footnotesize Sans Serif \textbf{bold},\textit{\ italic%
}, and \textbf{\textit{bold italic}}}}

\texttt{{\footnotesize footnotesize Typewriter \textbf{bold},\textit{\ italic%
}, and \textbf{\textit{bold italic}}}}

{\small small roman \textbf{bold},\textit{\ italic}, and \textbf{\textit{%
bold italic}}}

\textsf{{\small small Sans Serif \textbf{bold},\textit{\ italic}, and 
\textbf{\textit{bold italic}}}}

\texttt{{\small small Typewriter \textbf{bold},\textit{\ italic}, and 
\textbf{\textit{bold italic}}}}

{\normalsize normalsize roman \textbf{bold},\textit{\ italic}, and \textbf{%
\textit{bold italic}}}

\textsf{{\normalsize normalsize Sans Serif \textbf{bold},\textit{\ italic},
and \textbf{\textit{bold italic}}}}

\texttt{{\normalsize normalsize Typewriter \textbf{bold},\textit{\ italic},
and \textbf{\textit{bold italic}}}}

{\large large roman \textbf{bold},\textit{\ italic}, and \textbf{\textit{%
bold italic}}}

\textsf{{\large large Sans Serif \textbf{bold},\textit{\ italic}, and 
\textbf{\textit{bold italic}}}}

\texttt{{\large large Typewriter \textbf{bold},\textit{\ italic}, and 
\textbf{\textit{bold italic}}}}

{\Large Large roman \textbf{bold},\textit{\ italic}, and \textbf{\textit{%
bold italic}}}

\textsf{{\Large Large Sans Serif \textbf{bold},\textit{\ italic}, and 
\textbf{\textit{bold italic}}}}

\texttt{{\Large Large Typewriter \textbf{bold},\textit{\ italic}, and 
\textbf{\textit{bold italic}}}}

{\LARGE LARGE roman \textbf{bold},\textit{\ italic}, and \textbf{\textit{%
bold italic}}}

\textsf{{\LARGE LARGE Sans Serif \textbf{bold},\textit{\ italic}, and 
\textbf{\textit{bold italic}}}}

\texttt{{\LARGE LARGE Typewriter \textbf{bold},\textit{\ italic}, and 
\textbf{\textit{bold italic}}}}

{\huge huge roman \textbf{bold},\textit{\ italic}, and \textbf{\textit{bold
italic}}}

\textsf{{\huge huge Sans Serif \textbf{bold},\textit{\ italic}, and \textbf{%
\textit{bold italic}}}}

\texttt{{\huge huge Typewriter \textbf{bold},\textit{\ italic}, and \textbf{%
\textit{bold italic}}}}

{\Huge Huge roman \textbf{bold},\textit{\ italic}, and \textbf{\textit{bold
italic}}}

\textsf{{\Huge Huge Sans Serif \textbf{bold},\textit{\ italic}, and \textbf{%
\textit{bold italic}}}}

\texttt{{\Huge Huge Typewriter \textbf{bold},\textit{\ italic}, and \textbf{%
\textit{bold italic}}}}

\subsubsection{Bold Greek Letters}

Here are uppercase and lowercase greek letters, normal and bold, together
with nabla, in a variety of positions:

\begin{equation*}
x+\gamma +\mathbf{\alpha }+\mathbf{\Gamma }(\lambda )\Psi ^{\mathbf{\beta }%
^\delta }+\nabla (\mathbf{\kappa }_{\nu ^{\mathbf{\mu }}}\mathbf{)}+\mathbf{%
\nabla }
\end{equation*}

Here is a more comprehensive listing:

$%
\begin{array}{ccccccc}
\alpha ^{\alpha ^\alpha }\mathbf{\alpha }^{\mathbf{\alpha }^{\mathbf{\alpha }%
}} & \beta ^{\beta ^\beta }\mathbf{\beta }^{\mathbf{\beta }^{\mathbf{\beta }%
}} & \gamma ^{\gamma ^\gamma }\mathbf{\gamma }^{\mathbf{\gamma }^{\mathbf{%
\gamma }}} & \delta ^{\delta ^\delta }\mathbf{\delta }^{\mathbf{\delta }^{%
\mathbf{\delta }}} & \epsilon ^{\epsilon ^\epsilon }\mathbf{\epsilon }^{%
\mathbf{\epsilon }^{\mathbf{\epsilon }}} & \varepsilon ^{\varepsilon
^\varepsilon }\mathbf{\varepsilon }^{\mathbf{\varepsilon }^{\mathbf{%
\varepsilon }}} & \zeta ^{\zeta ^\zeta }\mathbf{\zeta }^{\mathbf{\zeta }^{%
\mathbf{\zeta }}} \\ 
\eta ^{\eta ^\eta }\mathbf{\eta }^{\mathbf{\eta }^{\mathbf{\eta }}} & \theta
^{\theta ^\theta }\mathbf{\theta }^{\mathbf{\theta }^{\mathbf{\theta }}} & 
\vartheta ^{\vartheta ^\vartheta }\mathbf{\vartheta }^{\mathbf{\vartheta }^{%
\mathbf{\vartheta }}} & \iota ^{\iota ^\iota }\mathbf{\iota }^{\mathbf{\iota 
}^{\mathbf{\iota }}} & \kappa ^{\kappa ^\kappa }\mathbf{\kappa }^{\mathbf{%
\kappa }^{\mathbf{\kappa }}} & \lambda ^{\lambda ^\lambda }\mathbf{\lambda }%
^{\mathbf{\lambda }^{\mathbf{\lambda }}} & \mu ^{\mu ^\mu }\mathbf{\mu }^{%
\mathbf{\mu }^{\mathbf{\mu }}}%
\end{array}
$

Here are superscript greek letters, normal and bold, following normal and
bold numbers:

$%
\begin{array}{ccccccc}
1^{\alpha ^{\alpha }}1\mathbf{\alpha }^{\mathbf{\alpha }^{\mathbf{\alpha }}}
& 1^{\beta ^{\beta }}\mathbf{1}^{\mathbf{\beta }^{\mathbf{\beta }}} & 
1^{\gamma ^{\gamma }}\mathbf{1}^{\mathbf{\gamma }^{\mathbf{\gamma }}} & 
1^{\delta ^{\delta }}\mathbf{1}^{\mathbf{\delta }^{\mathbf{\delta }}} & 
1^{\epsilon ^{\epsilon }}\mathbf{1}^{\mathbf{\epsilon }^{\mathbf{\epsilon }}}
& 1^{\varepsilon ^{\varepsilon }}\mathbf{1}^{\mathbf{\varepsilon }^{\mathbf{%
\varepsilon }}} & 1^{\zeta ^{\zeta }}\mathbf{1}^{\mathbf{\zeta }^{\mathbf{%
\zeta }}} \\ 
1^{\eta ^{\eta }}\mathbf{1}^{\mathbf{\eta }^{\mathbf{\eta }}} & 1^{\theta
^{\theta }}\mathbf{1}^{\mathbf{\theta }^{\mathbf{\theta }}} & 1^{\vartheta
^{\vartheta }}\mathbf{1}^{\mathbf{\vartheta }^{\mathbf{\vartheta }}} & 
1^{\iota ^{\iota }}\mathbf{1}^{\mathbf{\iota }^{\mathbf{\iota }}} & 
1^{\kappa ^{\kappa }}\mathbf{1}^{\mathbf{\kappa }^{\mathbf{\kappa }}} & 
1^{\lambda ^{\lambda }}\mathbf{1}^{\mathbf{\lambda }^{\mathbf{\lambda }}} & 
1^{\mu ^{\mu }}\mathbf{1}^{\mathbf{\mu }^{\mathbf{\mu }}}%
\end{array}
$

Here are subscript greek letters, normal and bold, following normal and bold
numbers:

$%
\begin{array}{ccccccc}
1_{\alpha _{\alpha }}\mathbf{1}_{\mathbf{\alpha }_{\mathbf{\alpha }}} & 
1_{\beta _{\beta }}\mathbf{1}_{\mathbf{\beta }_{\mathbf{\beta }}} & 
1_{\gamma _{\gamma }}\mathbf{1}_{\mathbf{\gamma }_{\mathbf{\gamma }}} & 
1_{\delta _{\delta }}\mathbf{1}_{\mathbf{\delta }_{\mathbf{\delta }}} & 
1_{\epsilon _{\epsilon }}\mathbf{1}_{\mathbf{\epsilon }_{\epsilon }} & 
1_{\varepsilon _{\varepsilon }}\mathbf{1}_{\mathbf{\varepsilon }_{\mathbf{%
\varepsilon }}} & 1_{\zeta _{\zeta }}\mathbf{1}_{\mathbf{\zeta }_{\mathbf{%
\zeta }}} \\ 
1_{\eta _{\eta }}\mathbf{1}_{\mathbf{\eta }_{\mathbf{\eta }}} & 1_{\theta
_{\theta }}\mathbf{1}_{\mathbf{\theta }_{\mathbf{\theta }}} & 1_{\vartheta
_{\vartheta }}\mathbf{1}_{\mathbf{\vartheta }_{\mathbf{\vartheta }}} & 
1_{\delta _{\delta }}1_{\mathbf{\delta }_{\mathbf{\delta }}} & 1_{\kappa
_{\kappa }}\mathbf{1}_{\mathbf{\kappa }_{\mathbf{\kappa }}} & 1_{\lambda
_{\lambda }}\mathbf{1}_{\mathbf{\lambda }_{\mathbf{\lambda }}} & 1_{\mu
_{\mu }}\mathbf{1}_{\mathbf{\mu }_{\mathbf{\mu }}}%
\end{array}
$

\section{Font Size Commands}

The \LaTeX{} font size switches are not directly supported by \textsl{SW}.
This is deliberate, because we attempt to create objects that encapsulate
such switches. A switch ``in the clear'' causes the filter to turn that
switch on and off for every paragraph. Thus, all such switches should be
encapsulated like this: \{\TEXTsymbol{\backslash}huge huge text\}. We
strongly discourage the use of such switches. Instead, you should create
appropriate tags in a style, giving the size changes content-oriented
meaning. The following table is included here to check the font sizes and
shapes provided by the style.

\begin{center}
\begin{tabular}{lll}
\textbf{\LaTeX{} Command} & \textbf{Style Editor Size} & \textbf{Example} \\ 
\hline
\TEXTsymbol{\backslash}tiny & tiny & {\tiny Gnu} \\ 
\TEXTsymbol{\backslash}scriptsize & script & {\scriptsize Gnu} \\ 
\TEXTsymbol{\backslash}footnotesize & footnote & {\footnotesize Gnu} \\ 
\TEXTsymbol{\backslash}small & small & {\small Gnu} \\ 
\TEXTsymbol{\backslash}normal & normal & {\normalsize Gnu} \\ 
\TEXTsymbol{\backslash}large & large-1 & {\large Gnu} \\ 
\TEXTsymbol{\backslash}Large & large-2 & {\Large Gnu} \\ 
\TEXTsymbol{\backslash}LARGE & large-3 & {\LARGE Gnu} \\ 
\TEXTsymbol{\backslash}huge & large-4 & {\huge Gnu} \\ 
\TEXTsymbol{\backslash}Huge & large-5 & {\Huge Gnu}%
\end{tabular}
\end{center}

\section{Spacing Objects}

\subsection{Horizontal}

Each horizontal spacing object is placed between reversed brackets, where
this makes sense.

Required space \rbrack\ \lbrack

Non-breaking space \rbrack ~\lbrack

Em space (quad) \rbrack \quad \lbrack

2-Em space (double quad) \rbrack \qquad \lbrack

Thin space \rbrack \thinspace \lbrack

Thick space \rbrack \ \lbrack

Italic correction \rbrack \/\lbrack

Negative thin space \rbrack \negthinspace \lbrack

Zero space \rbrack {}\lbrack

\noindent No indent

Custom space (1 inch) \rbrack \hspace{1in}\lbrack

Custom space (stretchy, 1.0) \rbrack \hfill \lbrack

Custom spaces (stretchy, 1.0, 0.5) \rbrack \hfill \lbrack \rbrack \hspace{%
\stretch{0.5000}}\lbrack

Custom space (stretchy, line) \rbrack \hrulefill \lbrack

Custom space (stretchy, dots) \rbrack \dotfill \lbrack

\subsection{Vertical}

Small skip:\smallskip

Medium skip: \medskip

Big skip: \bigskip

Strut \lbrack \strut \rbrack

Math strut $\sqrt{\mathstrut }$ Radical without math strut $\sqrt{}$

Custom (1 inch): \vspace{1in}

\subsection{Rule}

Rule, raised 0.25 inch, 1 inch wide, .25 inch high \rule[0.25in]{1in}{0.25in}

\subsection{Breaks}

Allowbreak \rbrack \allowbreak \lbrack

Discretionary hyphen \rbrack \-\lbrack

No break \rbrack \nolinebreak \lbrack

Page break\rbrack \pagebreak \lbrack

New page \rbrack \newpage

\lbrack

Line break \rbrack \linebreak \lbrack

New line \rbrack \newline
\lbrack

Custom new line, 1 inch\rbrack \\[1in]
\lbrack

\section{Verb and Verbatim}

The \LaTeX{} \TEXTsymbol{\backslash}verb command is available as a fragment.
You enter the fragment, then choose Edit Properties to change the contents.
A \TEXTsymbol{\backslash}verb: \verb"`~!@#$^&*()_+|\{}"

The \LaTeX{} verbatim environment is translated to the Body Verbatim
paragraph enviroment in \textsl{Scientific Word\/}{}. This environment is
valuable for displaying fragments of program code:
\begin{verbatim}
\long\def\@caption#1[#2]#3{%
  \par
   %\edef\@tempa{\csname #1TOCLabel\endcsname}
   \AddContentsLine
   {\csname ext@#1\endcsname}%
   {#1@toc}%
   {\csname #1toclabel\endcsname}%
   {\ignorespaces #2}%
   \begingroup
     \@parboxrestore
     \normalsize
     \csname @makecaption#1\endcsname{\ignorespaces #3}\par
   \endgroup}
\end{verbatim}

Active \LaTeX{} characters can cause problems in verbatim translations:
\begin{verbatim}
  `~!@#$%^&*()_-+=[]{}\|;:'",<.>/?
\end{verbatim}

Another approach to program code is illustrated by the following, which uses
a table.

\begin{center}
\begin{tabular}{l}
\textbf{for $j:=2$ step $1$ until $n$ do} \\ 
\quad \textbf{begin} $\mathit{accum}:=A\lbrack j\rbrack $; $k:=j-1$; $%
A\lbrack 0\rbrack :=\mathit{accum}$; \\ 
\quad \textbf{while $A\lbrack k\rbrack >\mathit{accum}$ do} \\ 
\qquad \textbf{begin} $A\lbrack k+1\rbrack :=A\lbrack k\rbrack $; $k:=k-1$;
\\ 
\qquad \textbf{end}; \\ 
\quad $A\lbrack k+1\rbrack :=\mathit{accum}$; \\ 
\quad \textbf{end}.%
\end{tabular}
\end{center}

\section{Comments}

You enter \LaTeX{} comments using the comment fragment. Enter the fragment,
then choose Edit Properties to change the body of the comment.%
%This comment is the fragment of TeX code from the Body Verbatim example:
%
%\long\def\@caption#1[#2]#3{%
%  \par
%  %\edef\@tempa{\csname #1TOCLabel\endcsname}
%  \AddContentsLine
%     {\csname ext@#1\endcsname}%
%     {#1@toc}%
%     {\csname #1toclabel\endcsname}%
%     {\ignorespaces #2}%
%  \begingroup
%    \@parboxrestore
%    \normalsize
%    \csname @makecaption#1\endcsname{\ignorespaces #3}\par
%  \endgroup}
%

\section{Tables and Arrays}

\subsection{Tables}

There are many different tables that you can create with the table editor.
Some are included here to test

\begin{center}
\begin{tabular}{|lrrrrrr|}
\hline
&  &  &  &  &  & \% Change \\ 
Ethnicity & 1986 & 1987 & 1988 & 1989 & 1990 & 1986-1990 \\ \hline
Amer. Indian & 294 & 277 & 298 & 328 & 361 & 23.0\% \\ 
Black & 192 & 206 & 199 & 183 & 225 & 17.0\% \\ 
Hispanic & 3,108 & 3,131 & 3,433 & 3,637 & 4,038 & 30.0\% \\ 
Oriental & 67 & 65 & 71 & 77 & 96 & 30.0\% \\ 
Other & 10,057 & 10,324 & 10,283 & 10,075 & 10,089 & 0.3\% \\ \hline
\end{tabular}

\begin{tabular}{|l|rrrrr|}
\hline
Course & 1987 & 1988 & 1989 & 1990 & 1991 \\ \hline
MATH 280 & 68 & 61 & 58 & 61 & 84 \\ \hline
MATH 480 & 36 & 26 & 41 & 44 & 53 \\ \hline
\end{tabular}
\end{center}

\subsubsection{From the \LaTeX{} User's Guide and Reference Manual}

\begin{center}
\begin{tabular}{||l|lr||}
\hline
gnats & gram & \$13.65 \\ \cline{2-3}
& each & .01 \\ \hline
gnu & stuffed & 92.50 \\ \cline{1-1}\cline{3-3}
emur &  & 33.33 \\ \hline
armadillo & frozen & 8.99 \\ \hline
\end{tabular}

\begin{tabular}{|l|l|r|}
\hline\hline
\emph{type} & \multicolumn{2}{c|}{\emph{style}} \\ \hline
smart & red & short \\ 
rather silly & puce & tall \\ \hline\hline
\end{tabular}
\end{center}

\subsubsection{From Jane Hahn's \LaTeX{} for Everyone}

\begin{center}
\begin{tabular}{|l|r|c|r|}
\hline
\multicolumn{4}{|c|}{\textbf{Overall Heading of Table}} \\ \hline
\multicolumn{1}{|c|}{Left-justified} & \multicolumn{1}{c|}{Right-justified}
& Centered & \multicolumn{1}{|c|}{Right-justified} \\ \hline\hline
one & two & three & four \\ \cline{2-4}
1 & 2 & 3 & 4 \\ 
i & ii & iii & iv \\ \hline
\end{tabular}

\begin{tabular}{|l|r@{.}l|}
\hline
\multicolumn{3}{|c|}{\textbf{Household Budget}} \\ \hline\hline
& \multicolumn{2}{c|}{\textbf{\% of}} \\ 
\textbf{Item} & \multicolumn{2}{c|}{\textbf{Budget}} \\ \hline
Housing & 50 & 0 \\ 
Food & 25 & 0 \\ 
Toys & 10 & 5 \\ 
Pet Supplies & 7 & 5 \\ 
Clothes & 5 & 25 \\ 
Charity & 1 & 75 \\ \hline
\end{tabular}

\begin{tabular}{|l|cr|}
\hline
\multicolumn{3}{|c|}{\textbf{PORTABLE HOOK-ON CHAIRS}} \\ \hline\hline
\textbf{BRAND} & \textbf{Graco} & \textbf{Strolee} \\ \hline
\textbf{MODEL} & Tot Loc Chair & Meal Mate \\ \hline
\textbf{PRICE} & \$23 & \$32 \\ \hline
\textbf{OVERALL JUDGEMENT} & Satisfactory & Satisfactory \\ \hline
\textbf{WEIGHT} & 7 lb. 9 oz. & 6 lb. 8 oz. \\ \hline
\end{tabular}

\begin{tabular}{|ll|}
\hline
\multicolumn{2}{|c|}{\textbf{THE CATECHISM}} \\ \hline
Q. & What are we by nature? \\ 
A. & We are part of God's creation, made in the image of God. \\[3mm] 
Q. & What does it mean to be created in the image of God? \\ 
A. & It means that we are free to make choices: to love; to create, to
reason, \\[3mm] 
& and to live in harmony with creation and with God. \\ 
Q. & Why then do we live apart from God and out of harmony with creation? \\ 
A. & From the beginning, human beings have misused their freedom and \\ 
& made wrong choices. \\ \hline
\end{tabular}
\end{center}

\subsubsection{The Economist}

Here is a table that is too wide for the page:

\begin{center}
\begin{tabular}{llllcllcl}
&  &  &  & \multicolumn{5}{c}{\textbf{\% change on}} \\ \cline{5-9}
&  & \multicolumn{2}{c}{\textbf{1993/94}} & \textbf{one} & \textbf{one} & 
\textbf{record} & \multicolumn{2}{c}{\textbf{Dec 31st 1992}} \\ \cline{3-4}
& \textbf{Feb 22nd} & \textbf{high} & \textbf{low} & \textbf{week} & \textbf{%
year} & \textbf{high} & \textbf{in local} & \textbf{in \$} \\ 
&  &  &  & \multicolumn{1}{l}{} &  &  & \textbf{currency} & \textbf{terms}
\\ \hline
\textbf{Australia} & \multicolumn{1}{r}{2,202.5} & \multicolumn{1}{r}{2,340.6
} & \multicolumn{1}{r}{1,495.0} & \multicolumn{1}{l}{$-$ 1.4} & $+$ 36.9 & $%
- $ 5.9 & \multicolumn{1}{l}{$+$42.1} & $+$48.6 \\ \hline
\textbf{Austria} & \multicolumn{1}{r}{450.7} & \multicolumn{1}{r}{460.7} & 
\multicolumn{1}{r}{300.3} & \multicolumn{1}{l}{$+$ 0.1} & $+$ 29.8 & $-$35.9
& \multicolumn{1}{l}{$+$44.0} & $+$35.0 \\ \hline
\textbf{Belgium} & \multicolumn{1}{r}{1,504.5} & \multicolumn{1}{r}{1,542.7}
& \multicolumn{1}{r}{1,125.5} & \multicolumn{1}{l}{$-$ 1.2} & $+$ 24.3 & $-$
2.5 & \multicolumn{1}{l}{$+$33.5} & $+$24.8 \\ \hline
\textbf{Britain} & \multicolumn{1}{r}{3,333.7} & \multicolumn{1}{r}{3,520.3}
& \multicolumn{1}{r}{2,737.6} & \multicolumn{1}{l}{$-$ 1.8} & $+$ 18.3 & $-$
5.3 & \multicolumn{1}{l}{$+$17.1} & $+$14.4 \\ \hline
\textbf{Canada} & \multicolumn{1}{r}{4,371.1} & \multicolumn{1}{r}{4,591.3}
& \multicolumn{1}{r}{3,275.8} & \multicolumn{1}{l}{$-$ 1.0} & $+$ 26.6 & $-$
4.8 & \multicolumn{1}{l}{$+$30.5} & $+$23.5 \\ \hline
\textbf{Denmark} & \multicolumn{1}{r}{407.4} & \multicolumn{1}{r}{415.8} & 
\multicolumn{1}{r}{261.9} & \multicolumn{1}{l}{$+$ 0.3} & $+$ 44.4 & $-$ 2.0
& \multicolumn{1}{l}{$+$55.7} & $+$45.0 \\ \hline
\textbf{France *} & \multicolumn{1}{r}{1,506.6} & \multicolumn{1}{r}{1,585.2}
& \multicolumn{1}{r}{1,114.2} & \multicolumn{1}{l}{$-$ 1.0} & $+$ 23.1 & $-$
5.0 & \multicolumn{1}{l}{$+$32.1} & $+$24.4 \\ \hline
\textbf{Germany} & \multicolumn{1}{r}{2,107.6} & \multicolumn{1}{r}{2,268.0}
& \multicolumn{1}{r}{1,516.5} & \multicolumn{1}{l}{$-$ 0.4} & $+$ 26.8 & $-$
7.1 & \multicolumn{1}{l}{$+$36.4} & $+$27.9 \\ \hline
\textbf{Holland} & \multicolumn{1}{r}{285.6} & \multicolumn{1}{r}{294.8} & 
\multicolumn{1}{r}{198.6} & \multicolumn{1}{l}{$-$ 1.4} & $+$ 35.7 & $-$ 3.1
& \multicolumn{1}{l}{$+$44.2} & $+$35.4 \\ \hline
\textbf{Italy} & \multicolumn{1}{r}{672.2} & \multicolumn{1}{r}{689.0} & 
\multicolumn{1}{r}{446.3} & \multicolumn{1}{l}{$-$ 0.6} & $+$ 33.7 & $-$26.0
& \multicolumn{1}{l}{$+$50.6} & $+$32.0 \\ \hline
\textbf{Japan} & \multicolumn{1}{r}{19,342.6} & \multicolumn{1}{r}{21,148.1}
& \multicolumn{1}{r}{16,078.7} & \multicolumn{1}{l}{$+$ 1.9} & $+$ 14.7 & $-$%
50.3 & \multicolumn{1}{l}{$+$14.3} & $+$35.3 \\ \hline
\textbf{Spain} & \multicolumn{1}{r}{339.6} & \multicolumn{1}{r}{358.3} & 
\multicolumn{1}{r}{215.6} & \multicolumn{1}{l}{$-$ 2.0} & $+$ 49.0 & $-$ 5.2
& \multicolumn{1}{l}{$+$58.5} & $+$29.2 \\ \hline
\textbf{Sweden} & \multicolumn{1}{r}{1,565.8} & \multicolumn{1}{r}{1,603.9}
& \multicolumn{1}{r}{879.1} & \multicolumn{1}{l}{$+$ 2.1} & $+$ 58.5 & $-$
2.4 & \multicolumn{1}{l}{$+$71.6} & $+$53.0 \\ \hline
\textbf{Switzerland} & \multicolumn{1}{r}{2,982.8} & \multicolumn{1}{r}{
3,178.4} & \multicolumn{1}{r}{2,049.5} & \multicolumn{1}{l}{$-$ 0.3} & $+$
45.5 & $-$ 6.2 & \multicolumn{1}{l}{$+$41.6} & $+$43.1 \\ \hline
\textbf{United States} & \multicolumn{1}{r}{3,911.7} & \multicolumn{1}{r}{
3,978.4} & \multicolumn{1}{r}{3,242.0} & \multicolumn{1}{l}{$-$ 0.4} & $+$
17.7 & $-$ 1.7 & \multicolumn{1}{l}{$+$18.5} & $+$18.5 \\ \hline
\textbf{World }$\dagger $ & \multicolumn{1}{r}{621.2} & \multicolumn{1}{r}{
641.0} & \multicolumn{1}{r}{488.6} & \multicolumn{1}{l}{$-$ 0.7} & $+$ 23.1
& $-$ 3.1 & \multicolumn{1}{l}{$+$25.0} & $+$25.0 \\ \hline
\end{tabular}
\end{center}

\section{Matrices}

Matrices have similar behavior to tables, but are in math mode.

Inline matrix: $%
\begin{array}{cc}
1 & \frac{1}{2} \\ 
\frac{1}{2} & \frac{1}{3}%
\end{array}%
$. Generally, these are hard to read without surrounding fences, like $%
\left( 
\begin{array}{cc}
1 & \frac{1}{2} \\ 
\frac{1}{2} & \frac{1}{3}%
\end{array}%
\right) $ or $\left[ 
\begin{array}{cc}
1 & \frac{1}{2} \\ 
\frac{1}{2} & \frac{1}{3}%
\end{array}%
\right] $.

Displayed matrix:

\begin{equation*}
\begin{array}{cc}
1 & \frac{1}{2} \\ 
\frac{1}{2} & \frac{1}{3}%
\end{array}%
\text{or }\left( 
\begin{array}{cc}
1 & \frac{1}{2} \\ 
\frac{1}{2} & \frac{1}{3}%
\end{array}%
\right) \text{ or }\left[ 
\begin{array}{cc}
1 & \frac{1}{2} \\ 
\frac{1}{2} & \frac{1}{3}%
\end{array}%
\right]
\end{equation*}

Try various column alignment options: $\ $left$%
\begin{array}{ll}
1 & \frac{1}{4}+\frac{1}{4} \\ 
\frac{1}{4}+\frac{1}{4} & \frac{1}{3}%
\end{array}%
$, center$%
\begin{array}{cc}
1 & \frac{1}{4}+\frac{1}{4} \\ 
\frac{1}{4}+\frac{1}{4} & \frac{1}{3}%
\end{array}%
$, right $%
\begin{array}{rr}
1 & \frac{1}{4}+\frac{1}{4} \\ 
\frac{1}{4}+\frac{1}{4} & \frac{1}{3}%
\end{array}%
$.

Try various baseline alignment options: $\ $top row$%
\begin{array}[t]{cc}
1 & \frac{1}{2} \\ 
\frac{1}{2} & \frac{1}{3}%
\end{array}%
$, center$%
\begin{array}{cc}
1 & \frac{1}{2} \\ 
\frac{1}{2} & \frac{1}{3}%
\end{array}%
$, bottom row $%
\begin{array}[b]{cc}
1 & \frac{1}{2} \\ 
\frac{1}{2} & \frac{1}{3}%
\end{array}%
$.

Finally, you can use built-in delimiters which may look better in certain
situations. brackets $%
\begin{bmatrix}
1 & \frac{1}{2} \\ 
\frac{1}{2} & \frac{1}{3}%
\end{bmatrix}%
$ vs. $\left[ 
\begin{array}{cc}
1 & \frac{1}{2} \\ 
\frac{1}{2} & \frac{1}{3}%
\end{array}%
\right] $ (standard matrix with fence -- the screen appearance is the same
in SWP). braces $%
\begin{Bmatrix}
1 & \frac{1}{2} \\ 
\frac{1}{2} & \frac{1}{3}%
\end{Bmatrix}%
$ vs. $\left\{ 
\begin{array}{cc}
1 & \frac{1}{2} \\ 
\frac{1}{2} & \frac{1}{3}%
\end{array}%
\right\} $. vertical bar $%
\begin{vmatrix}
1 & \frac{1}{2} \\ 
\frac{1}{2} & \frac{1}{3}%
\end{vmatrix}%
$ vs. $\left\vert 
\begin{array}{cc}
1 & \frac{1}{2} \\ 
\frac{1}{2} & \frac{1}{3}%
\end{array}%
\right\vert $. double vertical bar $%
\begin{Vmatrix}
1 & \frac{1}{2} \\ 
\frac{1}{2} & \frac{1}{3}%
\end{Vmatrix}%
$ vs. $\left\Vert 
\begin{array}{cc}
1 & \frac{1}{2} \\ 
\frac{1}{2} & \frac{1}{3}%
\end{array}%
\right\Vert $. parentheses $%
\begin{pmatrix}
1 & \frac{1}{2} \\ 
\frac{1}{2} & \frac{1}{3}%
\end{pmatrix}%
$ vs. $\left( 
\begin{array}{cc}
1 & \frac{1}{2} \\ 
\frac{1}{2} & \frac{1}{3}%
\end{array}%
\right) $.

\section{Graphics}

Here is a displayed graphic.\FRAME{dhFU}{2.3298in}{1.5489in}{0pt}{\Qcb{%
Triangle}}{\Qlb{triangle}}{triangle.wmf}{\special{language "Scientific
Word";type "GRAPHIC";maintain-aspect-ratio TRUE;display "FULL";valid_file
"F";width 2.3298in;height 1.5489in;depth 0pt;original-width
166.9375pt;original-height 110.3125pt;cropleft "0";croptop
"1.0004";cropright "0.9981";cropbottom "0";filename
'../Graphics/triangle.wmf';file-properties "XNPEU";}}Displayed graphic.%
\FRAME{dtbpF}{0.3494in}{0.2992in}{0pt}{}{}{swplogo.wmf}{\special{language
"Scientific Word";type "GRAPHIC";maintain-aspect-ratio TRUE;display
"FULL";valid_file "F";width 0.3494in;height 0.2992in;depth
0pt;original-width 23.3125pt;original-height 20.3125pt;cropleft "0";croptop
"0.9851";cropright "1.0129";cropbottom "0";filename
'../Graphics/swplogo.wmf';file-properties "XNPEU";}}

In-line graphic in a centered paragraph.

\begin{center}
\FRAME{itbpF}{0.3494in}{0.2992in}{0in}{}{}{swplogo.wmf}{\special{language
"Scientific Word";type "GRAPHIC";maintain-aspect-ratio TRUE;display
"FULL";valid_file "F";width 0.3494in;height 0.2992in;depth 0in;cropleft
"0";croptop "0.9851";cropright "1.0129";cropbottom "0";filename
'../Graphics/swplogo.wmf';file-properties "XNPEU";}}
\end{center}

Various inline graphics. On baseline\FRAME{itbpF}{0.3494in}{0.2992in}{0in}{}{%
}{swplogo.wmf}{\special{language "Scientific Word";type
"GRAPHIC";maintain-aspect-ratio TRUE;display "FULL";valid_file "F";width
0.3494in;height 0.2992in;depth 0in;original-width 0.3226in;original-height
0.2811in;cropleft "0";croptop "0.9851";cropright "1.0129";cropbottom
"0";filename '../Graphics/swplogo.wmf';file-properties "XNPEU";}}, raised%
\FRAME{itbpF}{0.3494in}{0.2992in}{-0.3122in}{}{}{swplogo.wmf}{\special%
{language "Scientific Word";type "GRAPHIC";maintain-aspect-ratio
TRUE;display "FULL";valid_file "F";width 0.3494in;height 0.2992in;depth
-0.3122in;original-width 0.3226in;original-height 0.2811in;cropleft
"0";croptop "0.9851";cropright "1.0129";cropbottom "0";filename
'../Graphics/swplogo.wmf';file-properties "XNPEU";}}, lowered\FRAME{itbpF}{%
0.3494in}{0.2992in}{0.2811in}{}{}{swplogo.wmf}{\special{language "Scientific
Word";type "GRAPHIC";maintain-aspect-ratio TRUE;display "FULL";valid_file
"F";width 0.3494in;height 0.2992in;depth 0.2811in;original-width
0.3226in;original-height 0.2811in;cropleft "0";croptop "0.9851";cropright
"1.0129";cropbottom "0";filename '../Graphics/swplogo.wmf';file-properties
"XNPEU";}}, with caption\FRAME{itbpFU}{0.3494in}{0.2992in}{0in}{\Qcb{%
Butterfly}}{}{swplogo.wmf}{\special{language "Scientific Word";type
"GRAPHIC";maintain-aspect-ratio TRUE;display "FULL";valid_file "F";width
0.3494in;height 0.2992in;depth 0in;original-width 23.3125pt;original-height
20.3125pt;cropleft "0";croptop "0.9875";cropright "1.0150";cropbottom
"0";filename '../Graphics/swplogo.wmf';file-properties "XNPEU";}}, without
frame\FRAME{itbpF}{0.3494in}{0.2992in}{0in}{}{}{swplogo.wmf}{\special%
{language "Scientific Word";type "GRAPHIC";maintain-aspect-ratio
TRUE;display "PICT";valid_file "F";width 0.3494in;height 0.2992in;depth
0in;original-width 23.3125pt;original-height 20.3125pt;cropleft "0";croptop
"0.9875";cropright "1.0150";cropbottom "0";filename
'../Graphics/swplogo.wmf';file-properties "XNPEU";}}, frame only\FRAME{itbpF%
}{0.3494in}{0.2992in}{0in}{}{}{swplogo.wmf}{\special{language "Scientific
Word";type "GRAPHIC";maintain-aspect-ratio TRUE;display "FRAME";valid_file
"F";width 0.3494in;height 0.2992in;depth 0in;original-width
0.3226in;original-height 0.2811in;cropleft "0";croptop "0.9875";cropright
"1.0150";cropbottom "0";filename '../Graphics/swplogo.wmf';file-properties
"XNPEU";}}, iconified\FRAME{itbpF}{0.3494in}{0.2992in}{0in}{}{}{swplogo.wmf}{%
\special{language "Scientific Word";type "GRAPHIC";maintain-aspect-ratio
TRUE;display "ICON";valid_file "F";width 0.3494in;height 0.2992in;depth
0in;original-width 0.3226in;original-height 0.2811in;cropleft "0";croptop
"0.9875";cropright "1.0150";cropbottom "0";filename
'../Graphics/swplogo.wmf';file-properties "XNPEU";}}, using the default 
\FRAME{itbpF}{0.3494in}{0.2992in}{0in}{}{}{swplogo.wmf}{\special{language
"Scientific Word";type "GRAPHIC";maintain-aspect-ratio TRUE;display
"FULL";valid_file "F";width 0.3494in;height 0.2992in;depth
0in;original-width 23.3125pt;original-height 20.3125pt;cropleft "0";croptop
"0.9875";cropright "1.0150";cropbottom "0";filename
'../Graphics/swplogo.wmf';file-properties "XNPEU";}}.

Different graphic types:

WMF :\FRAME{itbpF}{0.7507in}{0.2162in}{0in}{}{}{greeks.wmf}{\special%
{language "Scientific Word";type "GRAPHIC";maintain-aspect-ratio
TRUE;display "PICT";valid_file "F";width 0.7507in;height 0.2162in;depth
0in;original-width 0.7187in;original-height 0.1877in;cropleft "0";croptop
"1";cropright "1";cropbottom "0";filename
'../Graphics/greeks.wmf';file-properties "XNPEU";}} \ \ GIF:\FRAME{itbpF}{%
1.913in}{0.8553in}{0.2292in}{}{}{nb2.gif}{\special{language "Scientific
Word";type "GRAPHIC";maintain-aspect-ratio TRUE;display "PICT";valid_file
"F";width 1.913in;height 0.8553in;depth 0.2292in;original-width
1.8749in;original-height 0.8233in;cropleft "0";croptop "1";cropright
"1";cropbottom "0";filename '../Graphics/NB2.GIF';file-properties "XNPEU";}}
\ JPEG: \FRAME{itbpF}{0.6123in}{0.8475in}{0.2006in}{}{}{barry.jpg}{\special%
{language "Scientific Word";type "GRAPHIC";maintain-aspect-ratio
TRUE;display "PICT";valid_file "F";width 0.6123in;height 0.8475in;depth
0.2006in;original-width 0.4566in;original-height 0.6434in;cropleft
"0";croptop "1";cropright "1";cropbottom "0";filename
'../Graphics/barry.jpg';file-properties "XNPEU";}}

AI: \FRAME{dtbpF}{2.3177in}{1.817in}{0pt}{}{}{020112.ai}{\special{language
"Scientific Word";type "GRAPHIC";maintain-aspect-ratio TRUE;display
"PICT";valid_file "F";width 2.3177in;height 1.817in;depth 0pt;original-width
2.2779in;original-height 1.7798in;cropleft "0";croptop "1";cropright
"1";cropbottom "0";filename '../Graphics/020112.AI';file-properties "XNPEU";}%
}

Here is a floating graphic.

\FRAME{ftbpFU}{0.3632in}{0.3736in}{0pt}{\Qcb{This is the caption text. It's
below. It also contains a marker (\label{fig:key2})}}{\Qlb{fig:key1}}{%
nblogo.wmf}{\special{ language "Scientific Word"; type "GRAPHIC";
maintain-aspect-ratio TRUE; display "PICT"; valid_file "F"; width 0.3632in;
height 0.3736in; depth 0pt; original-width 0.3329in; original-height
0.3433in; cropleft "0"; croptop "1"; cropright "1"; cropbottom "0"; filename
'../Graphics/NBlogo.wmf';file-properties "XNPEU";}}

Reference to fig:key1 (\ref{fig:key1}) and to fig:key2 (\ref{fig:key2})

Here is a floating graphic with very long keys.

\FRAME{ftbpFU}{0.3632in}{0.3736in}{0pt}{\Qcb{This is the caption text. It's
below. It also contains a marker (\label{fig: This is a very long key that
can cause problems1})}}{\Qlb{fig: This is a very long key that can cause
problems2}}{nblogo.wmf}{\special{ language "Scientific Word"; type
"GRAPHIC"; maintain-aspect-ratio TRUE; display "PICT"; valid_file "F"; width
0.3632in; height 0.3736in; depth 0pt; original-width 0.3329in;
original-height 0.3433in; cropleft "0"; croptop "1"; cropright "1";
cropbottom "0"; filename '../Graphics/NBlogo.wmf';file-properties "XNPEU";}}

References to (\ref{fig: This is a very long key that can cause problems1})
and to (\ref{fig: This is a very long key that can cause problems2})\FRAME{%
ftbpFU}{0.3632in}{0.3736in}{0pt}{\Qcb{This is the caption text. It is a long
caption so that we can observe line wrap in this case of a long long long
caption longer than the line width.}}{}{nblogo.wmf}{\special{language
"Scientific Word";type "GRAPHIC";maintain-aspect-ratio TRUE;display
"PICT";valid_file "F";width 0.3632in;height 0.3736in;depth
0pt;original-width 0.3329in;original-height 0.3433in;cropleft "0";croptop
"1";cropright "1";cropbottom "0";filename
'../Graphics/NBlogo.wmf';file-properties "XNPEU";}}\FRAME{ftbpFU}{0.3632in}{%
0.3736in}{0pt}{\Qcb{This is a caption containing mathematics. $\protect\int_{%
\protect\alpha }^{\protect\beta }\ln udu$}}{}{nblogo.wmf}{\special{language
"Scientific Word";type "GRAPHIC";maintain-aspect-ratio TRUE;display
"PICT";valid_file "F";width 0.3632in;height 0.3736in;depth
0pt;original-width 0.3329in;original-height 0.3433in;cropleft "0";croptop
"1";cropright "1";cropbottom "0";filename
'../Graphics/NBlogo.wmf';file-properties "XNPEU";}}

\section{Standard \LaTeX{} Float Environments}

The standard \LaTeX{} float environments are figure and table. These are not
directly supported by the SW/SWP input and output filters. However, these
environments are preserved, and you can edit their contents using Edit
Properties. All label statements are also visible in SW/SWP for
cross-references.

\begin{figure}[ht]
\caption{Caption First}
\label{fig:in2}
\begin{center}
\fbox{\ldots{} figure body \ldots}
\par
\fbox{\ldots{} figure body \ldots}
\end{center}
\caption{Caption Second}
\label{fig:in4}
\end{figure}

\begin{table}[ht]
\caption{Caption First}
\label{tab:in2}
\begin{center}
\fbox{\ldots{} table body \ldots}
\par
\fbox{\ldots{} table body \ldots}
\end{center}
\caption{Caption Second}
\label{tab:in4}
\end{table}

\section{Scientific Notebook Items}

These constructs were introduced with the first version of Scientific
Notebook. They are now available in SW/SWP. They are provided here to test
the ability of SW/SWP to create a document containing these constructs that
can also be compiled with \LaTeX{}.

\subsection{Hypertext Links}

Hypertext link with text: 
%TCIMACRO{%
%\hyperref{MacKichan Software, Inc. home page}{}{}{http://www.mackichan.com}}%
%BeginExpansion
\msihyperref{MacKichan Software, Inc. home page}{}{}{http://www.mackichan.com}%
%EndExpansion

Hypertext link with graphic: 
%TCIMACRO{%
%\hyperref{\FRAME{itbpF}{0.499in}{0.2257in}{0in}{}{}{nb2.gif}{\special{language "Scientific Word";type "GRAPHIC";maintain-aspect-ratio TRUE;display "PICT";valid_file "F";width 0.499in;height 0.2257in;depth 0in;original-width 1.8749in;original-height 0.8233in;cropleft "0";croptop "1";cropright "1";cropbottom "0";filename '../Graphics/Nb2.gif';file-properties "XNPEU";}}}{}{}{http://www.scinotebook.com}}%
%BeginExpansion
\msihyperref{\FRAME{itbpF}{0.499in}{0.2257in}{0in}{}{}{nb2.gif}{\special{language "Scientific Word";type "GRAPHIC";maintain-aspect-ratio TRUE;display "PICT";valid_file "F";width 0.499in;height 0.2257in;depth 0in;original-width 1.8749in;original-height 0.8233in;cropleft "0";croptop "1";cropright "1";cropbottom "0";filename '../Graphics/Nb2.gif';file-properties "XNPEU";}}}{}{}{http://www.scinotebook.com}%
%EndExpansion

Hypertext links can go to external documents specified by a URI as show
above, or to markers or Hypertext targets, as illustrated here: 
%TCIMACRO{\hyperref{go to target}{}{}{my target}}%
%BeginExpansion
\msihyperref{go to target}{}{}{my target}%
%EndExpansion

\subsection{Hypertext Targets}

Hypertext target at the end of this sentence.\label{my target} These become
visible by turning on markers.

\subsection{Formulas}

Formulas allow you to create elements which are recalculated every time they
are displayed on screen. If the expression uses defined quantities, then the
formula will change every time you redefine the quantity.

Formula: $%
%TCIMACRO{\FORMULA{\int ax^{2}dx}{\frac{1}{3}ax^{3}}{evaluate}}%
%BeginExpansion
\frac{1}{3}ax^{3}%
%EndExpansion
$

Now, define $a=\pi $ and see what happens.

\subsection{Units}

%TCIMACRO{\TeXButton{Reverse noindent}{\hspace{\parindent}}}%
%BeginExpansion
\hspace{\parindent}%
%EndExpansion
\begin{tabular}{lll}
\multicolumn{3}{c}{\textbf{Base SI Units}} \\ \hline
\textit{Physical quantity} &  & \textit{Symbol} \\ \hline
length &  & $\unit{m}$ \\ 
mass &  & $\unit{kg}$ \\ 
time &  & $\unit{s}$ \\ 
electric current &  & $\unit{A}$ \\ 
thermodynamic temperature &  & $\unit{K}$ \\ 
amount of substance &  & $\unit{mol}$ \\ 
luminous intensity &  & $\unit{cd}$ \\ \hline
\end{tabular}

\begin{tabular}{lll}
\multicolumn{3}{c}{\textbf{Supplementary SI Units}} \\ \hline
\textit{Physical quantity} & \textit{Name} & \textit{Symbol} \\ \hline
plane angle & radian & $\unit{rad}$ \\ 
solid angle & steradian & $\unit{sr}$ \\ \hline
\end{tabular}

\begin{tabular}{lll}
\multicolumn{3}{c}{\textbf{Derived SI Units with Special Names}} \\ \hline
\textit{Physical quantity} &  & \textit{Symbol} \\ \hline
frequency &  & $\unit{Hz}$ \\ 
energy &  & $\unit{J}$ \\ 
force &  & $\unit{N}$ \\ 
power &  & $\unit{W}$ \\ 
pressure &  & $\unit{Pa}$ \\ 
electric charge &  & $\unit{C}$ \\ 
electric potential difference &  & $\unit{V}$ \\ 
electric resistance &  & $\unit{%
%TCIMACRO{\U{3a9}}%
%BeginExpansion
\Omega%
%EndExpansion
}$ \\ 
electric conductance &  & $\unit{S}$ \\ 
electric capacitance &  & $\unit{F}$ \\ 
magnetic flux &  & $\unit{Wb}$ \\ 
inductance &  & $\unit{H}$ \\ 
magnetic flux density &  & $\unit{T}$ \\ 
luminous flux &  & $\unit{lm}$ \\ 
illumination &  & $\unit{lx}$ \\ \hline
\end{tabular}

The following examples use Unicode characters.

MicroFarad: $\unit{%
%TCIMACRO{\U{3bc}}%
%BeginExpansion
\mu%
%EndExpansion
F}$

Degree Celcius: $\unit{%
%TCIMACRO{\U{2103}}%
%BeginExpansion
{}^{\circ}{\rm C}%
%EndExpansion
}$

Angular Measure: $30\unit{%
%TCIMACRO{\U{b0}}%
%BeginExpansion
{{}^\circ}%
%EndExpansion
}10\unit{%
%TCIMACRO{\U{2032}}%
%BeginExpansion
{}^{\prime }%
%EndExpansion
}12\unit{%
%TCIMACRO{\U{2033}}%
%BeginExpansion
{}^{\prime \prime}%
%EndExpansion
}$

Ohm: $\unit{%
%TCIMACRO{\U{3a9}}%
%BeginExpansion
\Omega%
%EndExpansion
}$

\subsection{New Note Types}

Margin Hint with icon: 
\CustomNote{Margin Hint}{%
Margin Hint}

\vspace{1pt}Margin Hint with text: 
\CustomNote[Hint!]{Margin Hint}{%
Margin Hint}

Margin Hint with picture: 
\CustomNote[{\FRAME{itbpF}{0.3528in}{0.3105in}{0in}{}{}{swplogo.wmf}{\special%
{language "Scientific Word";type "GRAPHIC";maintain-aspect-ratio
TRUE;display "PICT";valid_file "F";width 0.3528in;height 0.3105in;depth
0in;original-width 0.3226in;original-height 0.2811in;cropleft "0";croptop
"1";cropright "1";cropbottom "0";filename
'../Graphics/swplogo.wmf';file-properties "XNPEU";}}}]{Margin Hint}{%
Margin Hint}

Solution Note with icon: 
\CustomNote{Solution Note}{%
Solution Note}

Solution Note with text: 
\CustomNote[Solution Note!]{Solution Note}{%
Solution Note} \ 

\vspace{1pt}Solution Note with picture: 
\CustomNote[{\FRAME{itbpF}{0.5405in}{0.2447in}{0in}{}{}{nb2.gif}{\special%
{language "Scientific Word";type "GRAPHIC";maintain-aspect-ratio
TRUE;display "PICT";valid_file "F";width 0.5405in;height 0.2447in;depth
0in;original-width 1.8749in;original-height 0.8233in;cropleft "0";croptop
"1";cropright "1";cropbottom "0";filename
'../Graphics/Nb2.gif';file-properties "XNPEU";}}}]{Solution Note}{%
Solution Note} \ 

Problem Solving Hint with icon: 
\CustomNote{Problem Solving Hint}{%
Problem Solving Hint}

Problem Solving Hint with text: 
\CustomNote[Problem Solving Hint!]{Problem Solving Hint}{%
Problem Solving Hint}

Problem Solving Hint with picture: 
\CustomNote[{\FRAME{itbpF}{0.5682in}{0.3822in}{0in}{}{}{triangle.wmf}{\special%
{language "Scientific Word";type "GRAPHIC";maintain-aspect-ratio
TRUE;display "PICT";valid_file "F";width 0.5682in;height 0.3822in;depth
0in;original-width 2.3099in;original-height 1.5264in;cropleft "0";croptop
"1";cropright "1";cropbottom "0";filename
'../Graphics/TRIANGLE.WMF';file-properties "XNPEU";}}}]{Problem Solving Hint}
{%
Problem Solving Hint}

Note with icon: 
\CustomNote{Note}{%
Note}

Note with text: 
\CustomNote[Note!]{Note}{%
Note} \ 

Note with picture: 
\CustomNote[{\FRAME{itbpF}{0.3892in}{0.5362in}{0in}{}{}{barry.jpg}{\special%
{language "Scientific Word";type "GRAPHIC";maintain-aspect-ratio
TRUE;display "PICT";valid_file "F";width 0.3892in;height 0.5362in;depth
0in;original-width 0.4566in;original-height 0.6434in;cropleft "0";croptop
"1";cropright "1";cropbottom "0";filename
'../Graphics/barry.jpg';file-properties "XNPEU";}}}]{Note}{%
Note}

Answer Note with icon: 
\CustomNote{AnswerNote}{%
Answer Note}.

Answer Note with text: 
\CustomNote[Answer Note!]{AnswerNote}{%
Answer Note}.

Answer Note with picture: 
\CustomNote[{\FRAME{itbpF}{0.5353in}{0.5353in}{0in}{}{}{sticker.gif}{\special%
{language "Scientific Word";type "GRAPHIC";maintain-aspect-ratio
TRUE;display "PICT";valid_file "F";width 0.5353in;height 0.5353in;depth
0in;original-width 1.1459in;original-height 1.1459in;cropleft "0";croptop
"1";cropright "1";cropbottom "0";filename
'../Graphics/STICKER.GIF';file-properties "XNPEU";}}}]{AnswerNote}{%
Answer Note}.

Here is a note containing a picture: 
\CustomNote[{\FRAME{itbpF}{0.5353in}{0.5353in}{0in}{}{}{sticker.gif}{\special%
{language "Scientific Word";type "GRAPHIC";maintain-aspect-ratio
TRUE;display "PICT";valid_file "F";width 0.5353in;height 0.5353in;depth
0in;original-width 1.1459in;original-height 1.1459in;cropleft "0";croptop
"1";cropright "1";cropbottom "0";filename
'../Graphics/Sticker.gif';file-properties "XNPEU";}}}]{AnswerNote}{%
This answer note contains a picture:\FRAME{dtbpF}{0.3528in}{0.3105in}{0pt}{}{%
}{swplogo.wmf}{\special{language "Scientific Word";type
"GRAPHIC";maintain-aspect-ratio TRUE;display "PICT";valid_file "F";width
0.3528in;height 0.3105in;depth 0pt;original-width 0.3226in;original-height
0.2811in;cropleft "0";croptop "1";cropright "1";cropbottom "0";filename
'../Graphics/swplogo.wmf';file-properties "XNPEU";}}}.

\subsection{HTML Fields}

These are analogous to TeX Fields, but are active when you Export to HTML.

%TCIMACRO{\HTMLButton{HTML Field}{<hr title="from an HTML field"/>}}%

\section{Conclusion}

We are now done with our sample article.

The following references automatically generate a References section heading.

\begin{thebibliography}{99}
\bibitem{black} D. Black, \emph{The Theory of Committees and Elections},
Cambridge: Cambridge University Press (1958).

\bibitem{blyth} C. Blyth, \emph{Some probability paradoxes in choice from
among random alternatives}, J. Amer. Statist. Assoc. \textbf{67} (1972),
366--373.

\bibitem{Chang} Chang Li-chien, \emph{On the maximin probability of cyclic
random inequalities}, Scientia Sinica \textbf{10} (1961), 499--504.

\bibitem{condorcet} Marquis de Condorcet, \emph{Essai sur l'application de
l'analyse \`{a} la probabilit\'{e} des d\'{e}cisions rendues \`{a} la
pluralit\'{e} des voix}, Paris (1785).

\bibitem{dunford} N. Dunford and J. Schwartz, \emph{Functional Analysis}, v.
2, John Wiley and Sons, New York, 1963.

\bibitem{funkenbusch} W. W. Funkenbusch, \emph{A gaming wheel based on
cyclic advantage in symbol choice}, The Gambling Papers, Vol. XIII (1982),
68--83 University of Nevada, Reno.

\bibitem{gardner} M. Gardner, \emph{The paradox of the nontransitive dice
and the elusive principle of indifference}, Scientific American \textbf{223}
(1970), 110--114.

\bibitem{gardnerb} M. Gardner, \emph{On the paradoxical situations that
arise from nontransitive relations}, Scientific American \textbf{231}
(1974), 120--125.

\bibitem{funkensaari} W. W. Funkenbusch and Saari, D. G., \emph{Preferences
among preferences or nested cyclic stochastic inequalities}, Congr. Numer. 
\textbf{39} (1983), 419--432.

\bibitem{struwe} M. Struwe, \emph{Semilinear wave equations}, Bull. Amer.
Math. Soc. \textbf{26} (1992), 53-85.

\bibitem{thurston} W.P. Thurston, \emph{Geometry and topology of three
manifolds}, Lecture notes, Princeton Univ., NJ, 1979.
\end{thebibliography}

\end{document}
