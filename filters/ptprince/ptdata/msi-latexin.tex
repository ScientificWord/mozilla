
\chardef\active=13 


\catcode`\{=1 % left brace is begin-group character
\catcode`\}=2 % right brace is end-group character
\catcode`\$=3 % dollar sign is math shift
\catcode`\&=4 % ampersand is alignment tab
\catcode`\#=6 % hash mark is macro parameter character
\catcode`\^=7 % circumflex and uparrow are for superscripts
\catcode`\_=8 % underline and downarrow are for subscripts
\catcode`\^^I=10 % ascii tab is a blank space
\catcode`\~=\active % tilde is active
\catcode`\^^L=\active \outer\def^^L{\par}% ascii form-feed is \outer\par
\catcode`@=11

%\def~{\nobreakspace{}}

\def\@makeother#1{\catcode`#112\relax}

\catcode`\%=12
\def\@percentchar{%}
\catcode`\%=14



\countdef\count@=255
\let\bgroup={ \let\egroup=}
\def\makeatletter{\catcode`\@=11}
\def\makeatother{\catcode`\@=12}



\def\scantospace#1 {\def\@temp{#1}\@gobbletoend}%
\def\@gobbletoend#1>{}

\def\msi@open@element<#1>{\scantospace#1 >%
   \expandafter\msiopentag\expandafter{\@temp}{<#1>}}

\def\msi@close@element</#1>{\msiclosetag{#1}{</#1>}}

\def\msi@pass#1{\def#1{\msitag{#1}}}
\def\msi@pass@newline#1{\def#1{\br\msihmode\msitag{<texb>#1</texb>}}}


%% \documentclass

\def\documentclass{%
   \@ifnextchar[{\@docclassA}{\@docclassB}
}

   
\def\@docclassA[#1]#2{%
   \msi@output=1\relax
   \@@documentclass[#1]{#2}%
   \msi@output=0\relax
   \global\msi@dobraces=0%
   \out@documentclassA[#1]{#2}
}

\def\@docclassB#1{%
   \msi@output=1\relax
   \@@documentclass{#1}%
   \msi@output=0\relax
   \global\msi@dobraces=0%
   \out@documentclassB{#1}
}


\def\@@documentclass{%
  \@fileswithoptions\@clsextension
}


%% \usepackage and \RequirePackage

\def\usepackage{%
  \@ifnextchar[{\@usepackagea}{\@usepackageb}}

\def\@usepackagea[#1]#2{%
   \out@usepackageA[#1]{#2}
   \advance\msi@output by 1\relax
   \@@usepackage[#1]{#2}%
   \advance\msi@output by -1\relax
}

\def\@usepackageb#1{%
   \out@usepackageB{#1}%
   \advance\msi@output by 1\relax
   \@@usepackage{#1}%
   \advance\msi@output by -1\relax
}
\def\@@usepackage{\@fileswithoptions\@pkgextension}
\def\RequirePackage{\@fileswithoptions\@pkgextension}



%% \begin{document}

\def\msi@begin@document{%
   \global\msi@output=0\relax
   \global\let\input\@@input
   \out@begin@document
}

\def\msi@end@document{%
   \out@end@document
}



%% \begin and \end

%% \def\begin#1{%
%%   \ifmmode
%%     \def\dobegin{\string\begin{#1}}%
%%   \else
%%     \def\dobegin{%
%%       \begingroup
%%       \msitag{^^0a}\csname msi@begin@#1\endcsname}%
%%   \fi
%%   \dobegin
%% }

\def\begin#1{%
   \begingroup
   \msitag{^^0a}\csname msi@begin@#1\endcsname}

\def\end#1{%
   \csname msi@end@#1\endcsname
   \msitag{^^0a}%
   \endgroup}



%% \def\end#1{%
%%   \ifmmode
%%     \def\reserved@@a{#1}
%%     \def\reserved@@b{equation}
%%     \ifx\reserved@@a\reserved@@b
%%         $$
%%     \else
%%        \def\reserved@@b{align*}
%%        \ifx\reserved@@a\reserved@@b
%%           $$
%%        \else
%%           \string\end{#1}%
%%        \fi
%%     \fi
%%   \else
%%     \csname msi@end@#1\endcsname
%%     \msitag{^^0a}%
%%     \endgroup
%%   \fi
%% }
   




%%%%%%%%%%%%%%%%

\ifx\@@input\@undefined\let\@@input\input\fi
\ifx\@@end\@undefined\let\@@end\end\fi

\chardef\@inputcheck0
\chardef\sixt@@n=16
\newlinechar`\^^J

\def\typeout{\immediate\write17}
\def\dospecials{\do\ \do\\\do\{\do\}\do\$\do\&%
  \do\#\do\^\do\_\do\%\do\~}


\def\space{ }
\def\@tempswafalse{\let\if@tempswa\iffalse}
\def\@tempswatrue{\let\if@tempswa\iftrue}
\let\if@tempswa\iffalse
\def\loop#1\repeat{\def\iterate{#1\relax\expandafter\iterate\fi}%
  \iterate \let\iterate\relax}
\let\repeat\fi
\def\two@digits#1{\ifnum#1<10 0\fi\number#1}

\long\def\@firstoftwo#1#2{#1}
\long\def\@secondoftwo#1#2{#2}

\def\ProvidesFile#1{%
  \begingroup
    \catcode`\ 10 %
    \ifnum \endlinechar<256 %
      \ifnum \endlinechar>\m@ne
        \catcode\endlinechar 10 %
      \fi
    \fi
    \@makeother\/%
    \@ifnextchar[{\@providesfile{#1}}{\@providesfile{#1}[]}}

\def\@providesfile#1[#2]{%
    \wlog{File: #1 #2}%
    \@addtofilelist{ #2}%
    \endgroup}

\long\def\@addtofilelist#1{}
\def\@empty{}
\let\@currdir\@undefined
\let\input@path\@undefined
\let\filename@parse\@undefined
\def\strip@prefix#1>{}


\begingroup
    \count@\time

    \divide\count@ 60
    \count2=-\count@
    \multiply\count2 60
    \advance\count2 \time
    \edef\today{%
      \the\year/\two@digits{\the\month}/\two@digits{\the\day}:%
        \two@digits{\the\count@}:\two@digits{\the\count2}}

    \def\IfFileExists#1#2#3{%
      \openin\@inputcheck#1 %
      \ifeof\@inputcheck
         #3\relax
      \else
        \read\@inputcheck to \reserved@a
        \ifx\reserved@a\today
          \typeout{#1 found}#2\relax
        \else
          \typeout{BAD: old file \reserved@a (should be \today)}%
          #3\relax
        \fi
      \fi
      \closein\@inputcheck}

    \endlinechar=-1
    \gdef\@currdir{./}
    \immediate\closeout15 %
\endgroup

\ifx\filename@parse\@undefined
  \def\reserved@a{./}
  \ifx\@currdir\reserved@a
    \typeout{^^JDefining UNIX/DOS style filename parser.^^J}
    \def\filename@parse#1{%
      \let\filename@area\@empty
      \expandafter\filename@path#1/\\}
    \def\filename@path#1/#2\\{%
      \ifx\\#2\\%
         \def\reserved@a{\filename@simple#1.\\}%
      \else
         \edef\filename@area{\filename@area#1/}%
         \def\reserved@a{\filename@path#2\\}%
      \fi
      \reserved@a}
  \else\def\reserved@a{[]}\ifx\@currdir\reserved@a
    \typeout{^^JDefining VMS style filename parser.^^J}
    \def\filename@parse#1{%
      \let\filename@area\@empty
      \expandafter\filename@path#1]\\}
    \def\filename@path#1]#2\\{%
      \ifx\\#2\\%
         \def\reserved@a{\filename@simple#1.\\}%
      \else
         \edef\filename@area{\filename@area#1]}%
         \def\reserved@a{\filename@path#2\\}%
      \fi
      \reserved@a}
  \else\def\reserved@a{:}\ifx\@currdir\reserved@a
    \typeout{^^JDefining Mac style filename parser.^^J}
    \def\filename@parse#1{%
      \let\filename@area\@empty
      \expandafter\filename@path#1:\\}
    \def\filename@path#1:#2\\{%
      \ifx\\#2\\%
         \def\reserved@a{\filename@simple#1.\\}%
      \else
         \edef\filename@area{\filename@area#1:}%
         \def\reserved@a{\filename@path#2\\}%
      \fi
      \reserved@a}
  \else
    \typeout{^^JDefining generic filename parser.^^J}
    \def\filename@parse#1{%
      \let\filename@area\@empty
      \expandafter\filename@simple#1.\\}
  \fi\fi\fi
  \def\filename@simple#1.#2\\{%
    \ifx\\#2\\%
       \let\filename@ext\relax
    \else
       \edef\filename@ext{\filename@dot#2\\}%
    \fi
    \edef\filename@base{#1}}
  \def\filename@dot#1.\\{#1}
\else
  \typeout{^^J^^J%
    \noexpand\filename@parse was defined in texsys.cfg:^^J%
    \expandafter\strip@prefix\meaning\filename@parse.^^J%
    }
\fi
\ifx\@TeXversion\@undefined
  \ifx\@undefined\inputlineno
    \def\@TeXversion{2}
  \else
   {\catcode`\^^J=\active
     \def\reserved@a#1#2\@@{\if#1\string^3\fi}
     \edef\reserved@a{\expandafter\reserved@a\string^^J\@@}
     \ifx\reserved@a\@empty\else\gdef\@TeXversion{3}\fi}
  \fi
\fi



\def\dospecials{\do\ \do\\\do\{\do\}\do\$\do\&%
  \do\#\do\^\do\_\do\%\do\~}

\chardef\@ne=1
\chardef\tw@=2
\chardef\thr@@=3
\chardef\sixt@@n=16
\chardef\@cclv=255
\mathchardef\@cclvi=256
\mathchardef\@m=1000
\mathchardef\@M=10000
\mathchardef\@MM=20000
\message{registers,}
\count10=22 % allocates \count registers 23, 24, ...
\count11=9 % allocates \dimen registers 10, 11, ...
\count12=9 % allocates \skip registers 10, 11, ...
\count13=9 % allocates \muskip registers 10, 11, ...
\count14=9 % allocates \box registers 10, 11, ...
\count15=9 % allocates \toks registers 10, 11, ...
\count16=-1 % allocates input streams 0, 1, ...
\count17=-1 % allocates output streams 0, 1, ...
\count18=3 % allocates math families 4, 5, ...
\count19=0 % allocates \language codes 1, 2, ...
\count20=255 % allocates insertions 254, 253, ...
\countdef\insc@unt=20
\countdef\allocationnumber=21
\countdef\m@ne=22 \m@ne=-1
\def\wlog{\immediate\write\m@ne}
\countdef\count@=255
\dimendef\dimen@=0
\dimendef\dimen@i=1 % global only
\dimendef\dimen@ii=2
\skipdef\skip@=0
\toksdef\toks@=0
\def\newcount{\alloc@0\count\countdef\insc@unt}
\def\newdimen{\alloc@1\dimen\dimendef\insc@unt}
\def\newskip{\alloc@2\skip\skipdef\insc@unt}
\def\newmuskip{\alloc@3\muskip\muskipdef\@cclvi}
\def\newbox{\alloc@4\box\chardef\insc@unt}
\def\newhelp#1#2{\newtoks#1#1\expandafter{\csname#2\endcsname}}
\def\newtoks{\alloc@5\toks\toksdef\@cclvi}
\def\newread{\alloc@6\read\chardef\sixt@@n}
\def\newwrite{\alloc@7\write\chardef\sixt@@n}
\def\newlanguage{\alloc@9\language\chardef\@cclvi}
\def\alloc@#1#2#3#4#5{\global\advance\count1#1\@ne
  \ch@ck#1#4#2% make sure there's still room
  \allocationnumber\count1#1%
  \global#3#5\allocationnumber
  \wlog{\string#5=\string#2\the\allocationnumber}}
\def\newinsert#1{\global\advance\insc@unt \m@ne
  \ch@ck0\insc@unt\count
  \ch@ck1\insc@unt\dimen
  \ch@ck2\insc@unt\skip
  \ch@ck4\insc@unt\box
  \allocationnumber\insc@unt
  \global\chardef#1\allocationnumber
  \wlog{\string#1=\string\insert\the\allocationnumber}}
\gdef\ch@ck#1#2#3{%
  \ifnum\count1#1<#2\else
   \errmessage{No room for a new #3}%
  \fi}
\newdimen\maxdimen \maxdimen=16383.99999pt % the largest legal <dimen>
\newskip\hideskip \hideskip=-1000pt plus 1fill % negative but can grow
\newdimen\p@ \p@=1pt % this saves macro space and time
\newdimen\z@ \z@=0pt % can be used both for 0pt and 0
\newskip\z@skip \z@skip=0pt plus0pt minus0pt
\newbox\voidb@x % permanently void box register
\message{compatibility for TeX 2, }
\ifx\@undefined\inputlineno
  \newcount\inputlineno
  \inputlineno-1
  \ifx\@undefined\language
    \newcount\language
  \fi
  \newcount\lefthyphenmin
  \newcount\righthyphenmin
  \newcount\errorcontextlines
  \newcount\holdinginserts
  \newdimen\emergencystretch
  \newcount\badness
  \let\noboundary\relax
  \newcount\setlanguage
\fi

\pretolerance=100
\tolerance=200 % INITEX sets this to 10000
\hbadness=1000
\vbadness=1000
\linepenalty=10
\hyphenpenalty=50
\exhyphenpenalty=50
\binoppenalty=700
\relpenalty=500
\clubpenalty=150
\widowpenalty=150
\displaywidowpenalty=50
\brokenpenalty=100
\predisplaypenalty=10000
\doublehyphendemerits=10000
\finalhyphendemerits=5000
\adjdemerits=10000
\tracinglostchars=1
\uchyph=1
\defaulthyphenchar=`\-
\defaultskewchar=-1
\delimiterfactor=901
\showboxbreadth=-1
\showboxdepth=-1
\errorcontextlines=-1
\hfuzz=0.1pt
\vfuzz=0.1pt
\overfullrule=5pt
\maxdepth=4pt
\splitmaxdepth=\maxdimen
\boxmaxdepth=\maxdimen
\delimitershortfall=5pt
\nulldelimiterspace=1.2pt
\scriptspace=0.5pt
\parindent=20pt
\parskip=0pt plus 1pt
\abovedisplayskip=12pt plus 3pt minus 9pt
\abovedisplayshortskip=0pt plus 3pt
\belowdisplayskip=12pt plus 3pt minus 9pt
\belowdisplayshortskip=7pt plus 3pt minus 4pt
\topskip=10pt
\splittopskip=10pt
\parfillskip=0pt plus 1fil
\newskip\normalbaselineskip \normalbaselineskip=12pt
\newskip\normallineskip \normallineskip=1pt
\newdimen\normallineskiplimit \normallineskiplimit=0pt
\newcount\interfootnotelinepenalty \interfootnotelinepenalty=100
\def\magstephalf{1095 }
\def\magstep#1{\ifcase#1 \@m\or 1200\or 1440\or 1728\or
               2074\or 2488\fi\relax}
\def\^^M{\ } % control <return> = control <space>
\let\^^I\^^M % same for <tab>
\def\lq{`}
\def\rq{'}
\def\lbrack{[}
\def\rbrack{]}
\def \aa {\r a}
\def \AA {\r A}
\let\endgraf=\par
\let\endline=\cr
\def\space{ }
\let\empty\@empty
\def\null{\hbox{}}
\let\bgroup={
\let\egroup=}
{\catcode`\^^M=\active % these lines must end with %
  \gdef\obeylines{\catcode`\^^M\active \let^^M\par}%
  \global\let^^M\par} % this is in case ^^M appears in a \write
\def\obeyspaces{\catcode`\ \active}
{\obeyspaces\global\let =\space}
\long\def \loop #1\repeat{%
  \def\iterate{#1\relax  % Extra \relax
               \expandafter\iterate\fi
               }%
  \iterate
  \let\iterate\relax
}
\let\repeat=\fi
\chardef\%=`\%
\chardef\&=`\&
%\chardef\#=`\#
\def\#{%
  \ifmmode
    \string\#
  \else
    \char`\#
  \fi}

\def\leavevmode{\unhbox\voidb@x}
\def\mathhexbox#1#2#3{\mbox{$\m@th \mathchar"#1#2#3$}}
\def\ialign{\everycr{}\tabskip\z@skip\halign} % initialized \halign
\def\oalign#1{\leavevmode\vtop{\baselineskip\z@skip \lineskip.25ex%
  \ialign{##\crcr#1\crcr}}}
\def\o@lign{\lineskiplimit\z@ \oalign}
\def\ooalign{\lineskiplimit-\maxdimen \oalign}
\def\sh@ft#1{\dimen@.00#1ex\multiply\dimen@\fontdimen1\font
  \kern-.0156\dimen@} % compensate for slant in lowered accents
%%\def\hrulefill{\leavevmode\leaders\hrule\hfill\kern\z@}
\def\hrulefill{\out@hrulefill}
\def\dotfill{%
  \leavevmode
  \cleaders \hb@xt@ .44em{\hss.\hss}\hfill
  \kern\z@}
\sfcode`\)=0 \sfcode`\'=0 \sfcode`\]=0
\def\showoverfull{\tracingonline\@ne}
\gdef\showoutput{\tracingoutput\@ne
    \showboxbreadth\maxdimen\showboxdepth\maxdimen\errorstopmode
    \showoverfull}
\gdef\tracingall{\tracingcommands\tw@\tracingstats\tw@
  \tracingpages\@ne\tracinglostchars\@ne
  \tracingmacros\tw@\tracingparagraphs\@ne\tracingrestores\@ne
  \errorcontextlines\maxdimen\showoutput}
%%%\nonfrenchspacing
%%% From File: ltvers.dtx
\def\fmtname{LaTeX2e}
\edef\fmtversion{2001/06/01}

\def\two@digits#1{\ifnum#1<10 0\fi\number#1}
\def\typeout#1{\begingroup\set@display@protect
    \immediate\write\@unused{#1}\endgroup}
\newlinechar`\^^J
\let\@@par=\par
\let\@@hyph=\-        % Save original primitive definition
\def\-{\discretionary{-}{}{}}
\let\@dischyph=\-
\let\@@italiccorr=\/
\def\@height{height} \def\@depth{depth} \def\@width{width}
\def\@minus{minus}
\def\@plus{plus}
\def\hb@xt@{\hbox to}
\message{hacks,}
\def\typein{%
  \let\@typein\relax
  \@testopt\@xtypein\@typein}
\def\@xtypein[#1]#2{%
  \typeout{#2}%
  \advance\endlinechar\@M
  \read\@inputcheck to#1%
  \advance\endlinechar-\@M
  \@typein}
\def\@namedef#1{\expandafter\def\csname #1\endcsname}
\def\@nameuse#1{\csname #1\endcsname}
\def\@cons#1#2{\begingroup\let\@elt\relax\xdef#1{#1\@elt #2}\endgroup}
\def\@car#1#2\@nil{#1}
\def\@cdr#1#2\@nil{#2}
\def\@carcube#1#2#3#4\@nil{#1#2#3}
\def\@preamblecmds{}
\def\@onlypreamble#1{%
  \expandafter\gdef\expandafter\@preamblecmds\expandafter{%
       \@preamblecmds\do#1}}
\@onlypreamble\@onlypreamble
\@onlypreamble\@preamblecmds
\def\@star@or@long#1{%
  \@ifstar
   {\let\l@ngrel@x\relax#1}%
   {\let\l@ngrel@x\long#1}}

\let\l@ngrel@x\relax
\def\newcommand{\@star@or@long\new@command}
\def\new@command#1{%
  \@testopt{\@newcommand#1}0}
\def\@newcommand#1[#2]{%
  \@ifnextchar [{\@xargdef#1[#2]}%
                {\@argdef#1[#2]}}
\long\def\@argdef#1[#2]#3{%
   \@yargdef#1\@ne{#2}{#3}}
\long\def\@xargdef#1[#2][#3]#4{%
     \expandafter\def\expandafter#1\expandafter{%
          \expandafter
          \@protected@testopt
          \expandafter
          #1%
          \csname\string#1\endcsname
          {#3}}%
       \expandafter\@yargdef
          \csname\string#1\endcsname
           \tw@
           {#2}%
           {#4}}
\long\def\@testopt#1#2{%
  \@ifnextchar[{#1}{#1[{#2}]}}
\def\@protected@testopt#1{%%
  \ifx\protect\@typeset@protect
    \expandafter\@testopt
  \else
    \@x@protect#1%
  \fi}
\long \def \@yargdef #1#2#3{%
  \ifx#2\tw@
    \def\reserved@b##11{[####1]}%
  \else
    \let\reserved@b\@gobble
  \fi
  \expandafter
    \@yargd@f \expandafter{\number #3}#1%
}
\long \def \@yargd@f#1#2{%
  \def \reserved@a ##1#1##2##{%
    \expandafter\def\expandafter#2\reserved@b ##1#1%
    }%
  \l@ngrel@x \reserved@a 0##1##2##3##4##5##6##7##8##9###1%
}
\long\def\@reargdef#1[#2]{%
  \@yargdef#1\@ne{#2}}
\def\renewcommand{\@star@or@long\renew@command}
\def\renew@command#1{%
  \begingroup \escapechar\m@ne\xdef\@gtempa{{\string#1}}\endgroup
  \new@command#1}
\long\def\@ifdefinable #1#2{%
      \edef\reserved@a{\expandafter\@gobble\string #1}%
     \@ifundefined\reserved@a
         {\edef\reserved@b{\expandafter\@carcube \reserved@a xxx\@nil}%
          \ifx \reserved@b\@qend \@notdefinable\else
            \ifx \reserved@a\@qrelax \@notdefinable\else
              #2%
            \fi
          \fi}%
         \@notdefinable}
\let\@@ifdefinable\@ifdefinable
\long\def\@rc@ifdefinable#1#2{%
  \let\@ifdefinable\@@ifdefinable
  #2}

\def\renewenvironment{\@star@or@long\renew@environment}
\def\renew@environment#1{%
  \@ifundefined{#1}%
     {\@latex@error{Environment #1 undefined}\@ehc
     }\relax
  \expandafter\let\csname#1\endcsname\relax
  \expandafter\let\csname end#1\endcsname\relax
  \new@environment{#1}}
\long\def\@newenv#1#2#3#4{%
  \@ifundefined{#1}%
    {\expandafter\let\csname#1\expandafter\endcsname
                         \csname end#1\endcsname}%
    \relax
  \expandafter\new@command
     \csname #1\endcsname#2{#3}%
     \l@ngrel@x\expandafter\def\csname end#1\endcsname{#4}}
\def\newif#1{%
  \count@\escapechar \escapechar\m@ne
    \let#1\iffalse
    \@if#1\iftrue
    \@if#1\iffalse
  \escapechar\count@}
\def\@if#1#2{%
  \expandafter\def\csname\expandafter\@gobbletwo\string#1%
                    \expandafter\@gobbletwo\string#2\endcsname
                       {\let#1#2}}
\def\providecommand{\@star@or@long\provide@command}
\def\provide@command#1{%
  \begingroup
    \escapechar\m@ne\xdef\@gtempa{{\string#1}}%
  \endgroup
  \expandafter\@ifundefined\@gtempa
    {\def\reserved@a{\new@command#1}}%
    {\def\reserved@a{\renew@command\reserved@a}}%
   \reserved@a}%
\def\CheckCommand{\@star@or@long\check@command}
\@onlypreamble\CheckCommand
\def\check@command#1#2#{\@check@c#1{#2}}
\@onlypreamble\check@command
\long\def\@check@c#1#2#3{%
  \expandafter\let\csname\string\reserved@a\endcsname\relax
  \renew@command\reserved@a#2{#3}%
  \@ifundefined{\string\reserved@a}%
   {\@check@eq#1\reserved@a}%
   {\expandafter\@check@eq
           \csname\string#1\expandafter\endcsname
           \csname\string\reserved@a\endcsname}}
\@onlypreamble\@check@c
\def\@check@eq#1#2{%
  \ifx#1#2\else
     \@latex@warning@no@line
               {Command \noexpand#1 has
                changed.\MessageBreak
                Check if current package is valid}%
  \fi}
\@onlypreamble\@check@eq
\long\def \@gobble #1{}
\long\def \@gobbletwo #1#2{}
\long\def \@gobblefour #1#2#3#4{}
\long\def\@firstofone#1{#1}
\let\@iden\@firstofone
\long\def\@thirdofthree#1#2#3{#3}
\def\@expandtwoargs#1#2#3{%
\edef\reserved@a{\noexpand#1{#2}{#3}}\reserved@a}
\edef\@backslashchar{\expandafter\@gobble\string\\}
\def\@unexpandable@protect{\noexpand\protect\noexpand}
\def\DeclareRobustCommand{\@star@or@long\declare@robustcommand}
\def\declare@robustcommand#1{%
   \ifx#1\@undefined\else\ifx#1\relax\else
      \@latex@info{Redefining \string#1}%
   \fi\fi
   \edef\reserved@a{\string#1}%
   \def\reserved@b{#1}%
   \edef\reserved@b{\expandafter\strip@prefix\meaning\reserved@b}%
   \edef#1{%
      \ifx\reserved@a\reserved@b
         \noexpand\x@protect
         \noexpand#1%
      \fi
      \noexpand\protect
      \expandafter\noexpand\csname
         \expandafter\@gobble\string#1 \endcsname
   }%
   \let\@ifdefinable\@rc@ifdefinable
   \expandafter\new@command\csname
      \expandafter\@gobble\string#1 \endcsname
}
\def\x@protect#1{%
   \ifx\protect\@typeset@protect\else
      \@x@protect#1%
   \fi
}
\def\@x@protect#1\fi#2#3{%
   \fi\protect#1%
}
\let\@typeset@protect\relax
\def\set@display@protect{\let\protect\string}
\def\set@typeset@protect{\let\protect\@typeset@protect}
\def\protected@edef{%
   \let\@@protect\protect
   \let\protect\@unexpandable@protect
   \afterassignment\restore@protect
   \edef
}
\def\protected@xdef{%
   \let\@@protect\protect
   \let\protect\@unexpandable@protect
   \afterassignment\restore@protect
   \xdef
}
\def\unrestored@protected@xdef{%
   \let\protect\@unexpandable@protect
   \xdef
}
\def\restore@protect{\let\protect\@@protect}
\set@typeset@protect

\def\@ifundefined#1{%
  \expandafter\ifx\csname#1\endcsname\relax
    \let\reserved@@a\@firstoftwo
  \else
    \let\reserved@@a\@secondoftwo
  \fi
  \reserved@@a}

\edef\@qend{\expandafter\@cdr\string\end\@nil}
\edef\@qrelax{\expandafter\@cdr\string\relax\@nil}
\long\def\@ifnextchar#1#2#3{%
  \let\reserved@d=#1%
  \def\reserved@a{#2}%
  \def\reserved@b{#3}%
  \futurelet\@let@token\@ifnch}
\def\@ifnch{%
  \ifx\@let@token\@sptoken
    \let\reserved@c\@xifnch
  \else
    \ifx\@let@token\reserved@d
      \let\reserved@c\reserved@a
    \else
      \let\reserved@c\reserved@b
    \fi
  \fi
  \reserved@c}
\def\:{\let\@sptoken= } \:  % this makes \@sptoken a space token
\def\:{\@xifnch} \expandafter\def\: {\futurelet\@let@token\@ifnch}

\def\@ifstar#1{\@ifnextchar *{\@firstoftwo{#1}}}
\long\def\@dblarg#1{\@ifnextchar[{#1}{\@xdblarg{#1}}}
\long\def\@xdblarg#1#2{#1[{#2}]{#2}}
\def\@sanitize{\@makeother\ \@makeother\\\@makeother\$\@makeother\&%
\@makeother\#\@makeother\^\@makeother\_\@makeother\%\@makeother\~}
\def \@onelevel@sanitize #1{%
  \edef #1{\expandafter\strip@prefix
           \meaning #1}%
}
%%% From File: ltalloc.dtx
\chardef\@xxxii=32
\mathchardef\@Mi=10001
\mathchardef\@Mii=10002
\mathchardef\@Miii=10003
\mathchardef\@Miv=10004
\newcount\@tempcnta
\newcount\@tempcntb
\newif\if@tempswa
\newdimen\@tempdima
\newdimen\@tempdimb
\newdimen\@tempdimc
\newbox\@tempboxa
\newskip\@tempskipa
\newskip\@tempskipb
\newtoks\@temptokena
\newtoks\@temptokenb
\newskip\@flushglue \@flushglue = 0pt plus 1fil
%%% From File: ltcntrl.dtx
\message{control,}
\long\def\@whilenum#1\do #2{\ifnum #1\relax #2\relax\@iwhilenum{#1\relax
     #2\relax}\fi}
\long\def\@iwhilenum#1{\ifnum #1\expandafter\@iwhilenum
         \else\expandafter\@gobble\fi{#1}}
\long\def\@whiledim#1\do #2{\ifdim #1\relax#2\@iwhiledim{#1\relax#2}\fi}
\long\def\@iwhiledim#1{\ifdim #1\expandafter\@iwhiledim
        \else\expandafter\@gobble\fi{#1}}
\long\def\@whilesw#1\fi#2{#1#2\@iwhilesw{#1#2}\fi\fi}
\long\def\@iwhilesw#1\fi{#1\expandafter\@iwhilesw
         \else\@gobbletwo\fi{#1}\fi}
\def\@nnil{\@nil}
\def\@empty{}
\def\@fornoop#1\@@#2#3{}
\long\def\@for#1:=#2\do#3{%
  \expandafter\def\expandafter\@fortmp\expandafter{#2}%
  \ifx\@fortmp\@empty \else
    \expandafter\@forloop#2,\@nil,\@nil\@@#1{#3}\fi}
\long\def\@forloop#1,#2,#3\@@#4#5{\def#4{#1}\ifx #4\@nnil \else
       #5\def#4{#2}\ifx #4\@nnil \else#5\@iforloop #3\@@#4{#5}\fi\fi}
\long\def\@iforloop#1,#2\@@#3#4{\def#3{#1}\ifx #3\@nnil
       \expandafter\@fornoop \else
      #4\relax\expandafter\@iforloop\fi#2\@@#3{#4}}
\def\@tfor#1:={\@tf@r#1 }
\long\def\@tf@r#1#2\do#3{\def\@fortmp{#2}\ifx\@fortmp\space\else
    \@tforloop#2\@nil\@nil\@@#1{#3}\fi}
\long\def\@tforloop#1#2\@@#3#4{\def#3{#1}\ifx #3\@nnil
       \expandafter\@fornoop \else
      #4\relax\expandafter\@tforloop\fi#2\@@#3{#4}}
\long\def\@break@tfor#1\@@#2#3{\fi\fi}
\def\@removeelement#1#2#3{%
  \def\reserved@a##1,#1,##2\reserved@a{##1,##2\reserved@b}%
  \def\reserved@b##1,\reserved@b##2\reserved@b{%
    \ifx,##1\@empty\else##1\fi}%
  \edef#3{%
    \expandafter\reserved@b\reserved@a,#2,\reserved@b,#1,\reserved@a}}
%%% From File: lterror.dtx
\expandafter\let\csname ver@autoerr.sty\endcsname\fmtversion
\let\MessageBreak\relax
\def\GenericInfo#1#2{\relax}
\def\GenericError#1#2#3#4{\relax}

\gdef\PackageError#1#2#3{%
   \GenericError{%
      (#1)\@spaces\@spaces\@spaces\@spaces
   }{%
      Package #1 Error: #2%
   }{%
      See the #1 package documentation for explanation.%
   }{#3}%
}
\def\PackageWarning#1#2{}
\def\PackageWarningNoLine#1#2{}
\def\PackageInfo#1#2{}
\gdef\ClassError#1#2#3{}
\def\ClassWarning#1#2{}
\def\ClassWarningNoLine#1#2{}
\def\ClassInfo#1#2{}
\gdef\@latex@error#1#2{}
\def\@latex@warning#1{}
\def\@latex@warning@no@line#1{}
\def\@latex@info#1{}
\def\@latex@info@no@line#1{}
\let\c@errorcontextlines\errorcontextlines
\c@errorcontextlines=-1
\ifnum\inputlineno=\m@ne
  \let\on@line\@empty
\else
  \def\on@line{ on input line \the\inputlineno}
\fi
\let\@warning\@latex@warning
\let\@@warning\@latex@warning@no@line
\global\let\@latexerr\@latex@error
\def\@spaces{\space\space\space\space}
\gdef\@eha{%
}
\gdef\@ehb{%
  You've lost some text. \space \@ehc}
\gdef\@ehc{%
  Try typing \space <return> %
  \space to proceed.\MessageBreak
  If that doesn't work, type \space X <return> \space to quit.}
\gdef\@ehd{%
  You're in trouble here.  \space\@ehc}
\gdef\@notdefinable{%
 \@latex@error{%
   Command \@backslashchar\reserved@a\space
   already defined.\MessageBreak
   Or name \@backslashchar\@qend... illegal,
   see p.192 of the manual}\@eha}
\gdef\@nolnerr{%
  \@latex@error{There's no line here to end}\@eha}
\gdef\@nocounterr#1{%
  \@latex@error{No counter '#1' defined}\@eha}
\gdef\@nocnterr{\@nocounterr?}
\gdef\@ctrerr{%
  \@latex@error{Counter too large}\@ehb}
\gdef\@nodocument{%
  \@latex@error{Missing \protect\begin{document}}\@ehd}
\gdef\@badend#1{%
  \@latex@error{\protect\begin{\@currenvir}\@currenvline
                     \space ended by \protect\end{#1}}\@eha}
\gdef\@badmath{%
  \@latex@error{Bad math environment delimiter}\@eha}
\gdef\@toodeep{%
  \@latex@error{Too deeply nested}\@ehd}
\gdef\@badpoptabs{%
  \@latex@error{\protect\pushtabs\space and \protect\poptabs
      \space don't match}\@ehd}
\gdef\@badtab{%
 \@latex@error{Undefined tab position}\@ehd}
\gdef\@preamerr#1{%
  \begingroup
    \let\protect\relax
    \@latex@error{\ifcase #1 Illegal character\or
     Missing @-exp\or Missing p-arg\fi\space
     in array arg}\@ehd
  \endgroup}
\gdef\@badlinearg{%
  \@latex@error{%
       Bad \protect\line\space or \protect\vector
       \space argument}\@ehb}
\gdef\@parmoderr{%
  \@latex@error{Not in outer par mode}\@ehb}
\gdef\@fltovf{%
  \@latex@error{Too many unprocessed floats}\@ehb}
\gdef\@latexbug{%
  \@latex@error{This may be a LaTeX bug}{Call for help}}
\gdef\@noitemerr{%
  \@latex@error{Something's wrong--perhaps a missing %
      \protect\item}\@ehc}
\gdef\@notprerr{%
  \@latex@error{Can be used only in preamble}\@eha}
\gdef\@inmatherr#1{%
   \relax
   \ifmmode
 \@latex@error{Command \protect#1 invalid in math mode}\@ehc
   \fi}
%%% From File: ltpar.dtx
\message{par,}
\def\@setpar#1{\def\par{#1}\def\@par{#1}}
\def\@par{\let\par\@@par\par}
\def\@restorepar{\def\par{\@par}}
%%% From File: ltspace.dtx
\message{spacing,}
\expandafter\let\expandafter\@normalcr
     \csname\expandafter\@gobble\string\\ \endcsname
\DeclareRobustCommand\newline{\@normalcr\relax}
\def\@xnewline{\@ifnextchar[% ] bracket matching
                  \@newline
                 {\@gnewline\relax}}
\def\@newline[#1]{\let \reserved@e \vadjust
                   \@gnewline {\vskip #1}}
\def\@gnewline #1{%
  \ifvmode
    \@nolnerr
  \else
    \unskip \reserved@e {\reserved@f#1}\nobreak \hfil \break
  \fi}
\def\@getpen#1{\ifcase #1 \z@ \or \@lowpenalty\or
         \@medpenalty \or \@highpenalty
         \else \@M \fi}
\def\@nobreakfalse{\global\let\if@nobreak\iffalse}
\def\@nobreaktrue {\global\let\if@nobreak\iftrue}
\@nobreakfalse
\newdimen\@savsk
\newcount\@savsf
\def\@bsphack{%
  \relax
  \ifhmode
    \@savsk\lastskip
    \@savsf\spacefactor
  \fi}
\def\@esphack{%
  \relax
  \ifhmode
    \spacefactor\@savsf
    \ifdim\@savsk>\z@
      \ignorespaces
    \fi
  \fi}
\def\@Esphack{%
  \relax
  \ifhmode
    \spacefactor\@savsf
    \ifdim\@savsk>\z@
      \@ignoretrue
      \ignorespaces
    \fi
   \fi}
\def\@xaddvskip{%
  \ifdim\lastskip<\@tempskipb
    \vskip-\lastskip
    \vskip\@tempskipb
  \else
    \ifdim\@tempskipb<\z@
      \ifdim\lastskip<\z@
      \else
        \advance\@tempskipb\lastskip
        \vskip-\lastskip
        \vskip \@tempskipb
      \fi
    \fi
  \fi}
\def\addvspace#1{%
  \ifvmode
     \if@minipage\else
       \ifdim \lastskip =\z@
         \vskip #1\relax
       \else
       \@tempskipb#1\relax
         \@xaddvskip
       \fi
     \fi
  \else
    \@noitemerr
  \fi}
\def\addpenalty#1{%
  \ifvmode
    \if@minipage
    \else
      \if@nobreak
      \else
        \ifdim\lastskip=\z@
          \penalty#1\relax
        \else
          \@tempskipb\lastskip
          \vskip -\lastskip
          \penalty#1%
          \vskip\@tempskipb
        \fi
      \fi
    \fi
  \else
    \@noitemerr
  \fi}


\def\smallskip{
   \ifmmode\string\smallskip
   \else
      \vspace\smallskipamount
   \fi
}

\def\medskip{
   \ifmmode\string\medskip
   \else
      \vspace\medskipamount
   \fi
}

\def\bigskip{
   \ifmmode\string\bigskip
   \else
      \vspace\bigskipamount
   \fi
}


\newskip\smallskipamount \smallskipamount=3pt plus 1pt minus 1pt
\newskip\medskipamount   \medskipamount  =6pt plus 2pt minus 2pt
\newskip\bigskipamount   \bigskipamount =12pt plus 4pt minus 4pt

\DeclareRobustCommand{\nobreakspace}{%
   \leavevmode\nobreak\ }

\expandafter\let\expandafter\@xobeysp\csname nobreakspace \endcsname
\DeclareRobustCommand{\,}{%
   \relax\ifmmode\mskip\thinmuskip\else\thinspace\fi
}
\def\@{\spacefactor\@m}
\DeclareRobustCommand\hspace{\@ifstar\@hspacer\@hspace}
\def\@hspace#1{\hskip #1\relax}
\def\@hspacer#1{\vrule \@width\z@\nobreak
                \hskip #1\hskip \z@skip}
\newskip\fill
\fill = 0pt plus 1fill
\def\stretch#1{\z@ \@plus #1fill\relax}
{\catcode`\^^M=13 \gdef\obeycr{\catcode`\^^M13 \def^^M{\\\relax}%
    \@gobblecr}%
{\catcode`\^^M=13 \gdef\@gobblecr{\@ifnextchar
\@gobble\ignorespaces}}
\gdef\restorecr{\catcode`\^^M5 }}



\newread\@inputcheck
\newwrite\@unused
\newwrite\@mainaux
\newwrite\@partaux
\newif\if@filesw \@fileswtrue
\newif\if@partsw \@partswfalse
\newcount\@clubpenalty
\@clubpenalty \clubpenalty
\def\document{}%
\let\normalsfcodes\@empty
\def\nofiles{%
  \@fileswfalse
  \typeout{No auxiliary output files.^^J}%
  \long\def\protected@write##1##2##3%
    {\write\m@ne{}\if@nobreak\ifvmode\nobreak\fi\fi}%
  \let\makeindex\relax
  \let\makeglossary\relax}
\@onlypreamble\nofiles
\long\def \protected@write#1#2#3{%
      \begingroup
       \let\thepage\relax
       #2%
       \let\protect\@unexpandable@protect
       \edef\reserved@a{\write#1{#3}}%
       \reserved@a
      \endgroup
      \if@nobreak\ifvmode\nobreak\fi\fi
}
\let\@auxout=\@mainaux
\def\includeonly#1{%
  \@partswtrue
  \edef\@partlist{\zap@space#1 \@empty}}
\@onlypreamble\includeonly
\def\include#1{\relax
  \ifnum\@auxout=\@partaux
    \@latex@error{\string\include\space cannot be nested}\@eha
  \else \@include#1 \fi}
\def\@include#1 {%
  \clearpage
  \if@filesw
    \immediate\write\@mainaux{\string\@input{#1.aux}}%
  \fi
  \@tempswatrue
  \if@partsw
    \@tempswafalse
    \edef\reserved@b{#1}%
    \@for\reserved@a:=\@partlist\do
      {\ifx\reserved@a\reserved@b\@tempswatrue\fi}%
  \fi
  \if@tempswa
    \let\@auxout\@partaux
    \if@filesw
      \immediate\openout\@partaux #1.aux
      \immediate\write\@partaux{\relax}%
    \fi
    \@input@{#1.tex}%
    \clearpage
    \@writeckpt{#1}%
    \if@filesw
      \immediate\closeout\@partaux
    \fi
  \else
    \deadcycles\z@
    \@nameuse{cp@#1}%
  \fi
  \let\@auxout\@mainaux}
\def\@writeckpt#1{%
  \if@filesw
    \immediate\write\@partaux{\string\@setckpt{#1}\@charlb}%
    {\let\@elt\@wckptelt \cl@@ckpt}%
    \immediate\write\@partaux{\@charrb}%
  \fi}
\def\@wckptelt#1{%
  \immediate\write\@partaux{%
    \string\setcounter{#1}{\the\@nameuse{c@#1}}}}
\def\@setckpt#1{\global\@namedef{cp@#1}}
{\catcode`[=1 \catcode`]=2
\catcode`{=11 \catcode`}=11
\gdef\@charlb[{]
\gdef\@charrb[}]
]% }brace matching
\long\def \IfFileExists#1#2#3{%
  \openin\@inputcheck#1 %
  \ifeof\@inputcheck
    \ifx\input@path\@undefined
      \def\reserved@a{#3}%
    \else
      \def\reserved@a{\@iffileonpath{#1}{#2}{#3}}%
    \fi
  \else
    \closein\@inputcheck
    \edef\@filef@und{#1 }%
    \def\reserved@a{#2}%
  \fi
  \reserved@a}
\long\def\@iffileonpath#1{%
  \let\reserved@a\@secondoftwo
  \expandafter\@tfor\expandafter\reserved@b\expandafter
             :\expandafter=\input@path\do{%
    \openin\@inputcheck\reserved@b#1 %
    \ifeof\@inputcheck\else
      \edef\@filef@und{\reserved@b#1 }%
      \let\reserved@a\@firstoftwo%
      \closein\@inputcheck
      \@break@tfor
    \fi}%
  \reserved@a}
\long\def \InputIfFileExists#1#2{%
  \IfFileExists{#1}%
    {#2\@addtofilelist{#1}\@@input \@filef@und}}

\def\@obsoletefile#1#2{%
   \@latex@warning@no@line{inputting `#1' instead of obsolete `#2'}}
\@onlypreamble\@obsoletefile
\let\@filelist\@gobble
\def\listfiles{%
  \let\listfiles\relax
  \def\@listfiles##1##2##3##4##5##6##7##8##9\@@{%
     \def\reserved@d{\\}%
     \@tfor\reserved@c:=##1##2##3##4##5##6##7##8\do{%
       \ifx\reserved@c\reserved@d
         \edef\filename@area{ \filename@area}%
       \fi}}%
  \def\@dofilelist{%
     \typeout{^^J *File List*}%
     \@for\@currname:=\@filelist\do{%
       \filename@parse\@currname
       \edef\reserved@a{%
          \filename@base.%
          \ifx\filename@ext\relax tex\else\filename@ext\fi}%
       \expandafter\let\expandafter\reserved@b
                              \csname ver@\reserved@a\endcsname
       \expandafter\expandafter\expandafter\@listfiles\expandafter
             \filename@area\filename@base\\\\\\\\\\\\\\\\\\\@@
       \typeout{%
         \filename@area\reserved@a
         \ifx\reserved@b\relax\else\@spaces\reserved@b\fi}}%
     \typeout{ ***********^^J}}}
\@onlypreamble\listfiles
\let\@dofilelist\relax
%%% From File: ltoutenc.dtx
\message{font encodings,}
\def\DeclareTextCommand{%
   \@dec@text@cmd\newcommand}
\def\ProvideTextCommand{%
   \@dec@text@cmd\providecommand}
\def\DeclareTextSymbol#1#2#3{%
   \@dec@text@cmd\chardef#1{#2}#3\relax}
\def\@dec@text@cmd#1#2#3{%
   \expandafter\def\expandafter#2%
      \expandafter{%
         \csname#3-cmd\expandafter\endcsname
         \expandafter#2%
         \csname#3\string#2\endcsname
      }%
   \let\@ifdefinable\@rc@ifdefinable
   \expandafter#1\csname#3\string#2\endcsname}
\@onlypreamble\DeclareTextCommand
\@onlypreamble\DeclareTextSymbol
\def\@current@cmd#1{%
   \ifx\protect\@typeset@protect
      \@inmathwarn#1%
   \else
      \noexpand#1\expandafter\@gobble
   \fi}
\def\@changed@cmd#1#2{%
   \ifx\protect\@typeset@protect
      \@inmathwarn#1%
      \expandafter\ifx\csname\cf@encoding\string#1\endcsname\relax
         \expandafter\ifx\csname ?\string#1\endcsname\relax
            \expandafter\def\csname ?\string#1\endcsname{%
               \TextSymbolUnavailable#1%
            }%
         \fi
         \global\expandafter\let
               \csname\cf@encoding \string#1\expandafter\endcsname
               \csname ?\string#1\endcsname
      \fi
      \csname\cf@encoding\string#1%
         \expandafter\endcsname
   \else
      \noexpand#1%
   \fi}
\gdef\TextSymbolUnavailable#1{%
   \@latex@error{%
      Command \protect#1 unavailable in encoding \cf@encoding%
   }\@eha}
\def\@inmathwarn#1{%
   \ifmmode
      \@latex@warning{Command \protect#1 invalid in math mode}%
   \fi}
\def\DeclareTextCommandDefault#1{%
   \DeclareTextCommand#1?}
\def\ProvideTextCommandDefault#1{%
   \ProvideTextCommand#1?}
\@onlypreamble\DeclareTextCommandDefault
\expandafter\let\csname?-cmd\endcsname\@changed@cmd
\def\DeclareTextAccent#1#2#3{%
   \DeclareTextCommand#1{#2}{\add@accent{#3}}}
\@onlypreamble\DeclareTextAccent
\def\add@accent#1#2{\hmode@bgroup
   \let\hmode@start@before@group\@firstofone
   \setbox\@tempboxa\hbox{#2%
      \global\mathchardef\accent@spacefactor\spacefactor}%
   \accent#1 #2\egroup\spacefactor\accent@spacefactor}
\let\accent@spacefactor\relax
\def\hmode@bgroup{\leavevmode\bgroup}
\def\DeclareTextCompositeCommand#1#2#3#4{%
  \expandafter\let\expandafter\reserved@a\csname#2\string#1\endcsname
  \expandafter\expandafter\expandafter\ifx
  \expandafter\@car\reserved@a\relax\relax\@nil \@text@composite \else
      \edef\reserved@b##1{%
         \def\expandafter\noexpand
            \csname#2\string#1\endcsname####1{%
            \noexpand\@text@composite
               \expandafter\noexpand\csname#2\string#1\endcsname
               ####1\noexpand\@empty\noexpand\@text@composite
               {##1}}}%
      \expandafter\reserved@b\expandafter{\reserved@a{##1}}%
   \fi
   \expandafter\def\csname\expandafter\string\csname
      #2\endcsname\string#1-\string#3\endcsname{#4}}
\@onlypreamble\DeclareTextCompositeCommand
\def\@text@composite#1#2#3\@text@composite{%
   \expandafter\@text@composite@x
      \csname\string#1-\string#2\endcsname}
\def\@text@composite@x#1{%
   \ifx#1\relax
      \expandafter\@secondoftwo
   \else
      \expandafter\@firstoftwo
   \fi
   #1}
\catcode\z@=11\relax
\def\DeclareTextComposite#1#2#3#4{%
   \def\reserved@a{\DeclareTextCompositeCommand#1{#2}{#3}}%
   \bgroup
      \lccode\z@#4%
      \lowercase{%
   \egroup
      \reserved@a ^^@}}
\catcode\z@=15\relax
\@onlypreamble\DeclareTextComposite
\def\UseTextAccent#1#2#3{%
  \hmode@start@before@group
   {%
    \let\hmode@start@before@group\@firstofone
    \let\@curr@enc\cf@encoding
    \@use@text@encoding{#1}%
    #2{\@use@text@encoding\@curr@enc#3}%
   }}
\def\UseTextSymbol#1#2{%
       \hmode@start@before@group
       {%
          \def\@wrong@font@char{\MessageBreak
             for \noexpand\symbol`\string#2'}%
          \@use@text@encoding{#1}%
          #2%
       }%
    }
\def\@use@text@encoding#1{%
   \edef\f@encoding{#1}%
   \xdef\font@name{%
      \csname\curr@fontshape/\f@size\endcsname}%
   \pickup@font
   \font@name
   \@@enc@update}
\let\hmode@start@before@group\leavevmode
\def\DeclareTextSymbolDefault#1#2{%
   \DeclareTextCommandDefault#1{\UseTextSymbol{#2}#1}}
\def\DeclareTextAccentDefault#1#2{%
   \DeclareTextCommandDefault#1{\UseTextAccent{#2}#1}}
\@onlypreamble\DeclareTextSymbolDefault
\@onlypreamble\DeclareTextAccentDefault
\def\UndeclareTextCommand#1#2{%
  \expandafter\ifx\csname#2\string#1\endcsname\relax
  \else
     \global\expandafter\let\csname#2\string#1\endcsname
            \@undefined
     \expandafter\expandafter\expandafter
     \ifx\expandafter\@thirdofthree#1\@undefined
       \expandafter\gdef\expandafter#1\expandafter
          {\csname ?-cmd\expandafter\endcsname\expandafter
           #1\csname?\string#1\endcsname}%
     \fi
 \fi
}
\@onlypreamble\UndeclareTextCommand
\def\@tabacckludge#1{\expandafter\@changed@cmd
                                 \csname\string#1\endcsname\relax}

\def\setcounter#1#2{%
    \global\csname c@#1\endcsname#2\relax}

\def\addtocounter#1#2{%
    \global\advance\csname c@#1\endcsname #2\relax}


\def\newcounter#1{%
  \expandafter\@ifdefinable \csname c@#1\endcsname
    {\@definecounter{#1}}%
  \@ifnextchar[{\@newctrA{#1}}{\out@newcounterA{#1}}%
}

\def\@newctrA#1[#2]{%
  \@ifundefined{c@#2}{\@nocounterr{#2}}{\@addtoreset{#1}{#2}}%
  \out@newcounterB{#1}[#2]%
}


\def\value#1{\csname c@#1\endcsname}
\def\stepcounter#1{%
  \addtocounter{#1}\@ne
  \begingroup
    \let\@elt\@stpelt
    \csname cl@#1\endcsname
  \endgroup}
\def\@stpelt#1{\global\csname c@#1\endcsname \z@}
\def\cl@@ckpt{\@elt{page}}
\def\@definecounter#1{\expandafter\newcount\csname c@#1\endcsname
     \setcounter{#1}\z@
     \global\expandafter\let\csname cl@#1\endcsname\@empty
     \@addtoreset{#1}{@ckpt}%
     \global\expandafter\let\csname p@#1\endcsname\@empty
     \expandafter
     \gdef\csname the#1\expandafter\endcsname\expandafter
          {\expandafter\@arabic\csname c@#1\endcsname}}
\def\@addtoreset#1#2{\expandafter\@cons\csname cl@#2\endcsname {{#1}}}
\def\arabic#1{\expandafter\@arabic\csname c@#1\endcsname}
\def\roman#1{\expandafter\@roman\csname c@#1\endcsname}
\def\Roman#1{\expandafter\@Roman\csname c@#1\endcsname}
\def\alph#1{\expandafter\@alph\csname c@#1\endcsname}
\def\Alph#1{\expandafter\@Alph\csname c@#1\endcsname}
\def\fnsymbol#1{\expandafter\@fnsymbol\csname c@#1\endcsname}
\def\@arabic#1{\number #1}  %% changed 29 Apr 86
\def\@roman#1{\romannumeral #1}

\def\@Roman#1{\expandafter\@slowromancap\romannumeral #1@}
\def\@slowromancap#1{\ifx @#1% then terminate
     \else
       \if i#1I\else\if v#1V\else\if x#1X\else\if l#1L\else\if
       c#1C\else\if d#1D\else \if m#1M\else#1\fi\fi\fi\fi\fi\fi\fi
       \expandafter\@slowromancap
     \fi
}
\def\@alph#1{%
  \ifcase#1\or a\or b\or c\or d\or e\or f\or g\or h\or i\or j\or
   k\or l\or m\or n\or o\or p\or q\or r\or s\or t\or u\or v\or w\or x\or
    y\or z\else\@ctrerr\fi}
\def\@Alph#1{%
  \ifcase#1\or A\or B\or C\or D\or E\or F\or G\or H\or I\or J\or
   K\or L\or M\or N\or O\or P\or Q\or R\or S\or T\or U\or V\or W\or X\or
    Y\or Z\else\@ctrerr\fi}
\def\@fnsymbol#1{\ensuremath{\ifcase#1\or *\or \dagger\or \ddagger\or
   \mathsection\or \mathparagraph\or \|\or **\or \dagger\dagger
   \or \ddagger\ddagger \else\@ctrerr\fi}}



\def\newlength#1{\newskip#1}
\def\setlength#1#2{}
\def\addtolength#1#2{}

\msi@pass@newline{\@settodim}
\msi@pass@newline{\settoheight}
\msi@pass@newline{\settodepth}
\msi@pass@newline{\settowidth}
\def\@settopoint#1{\msitag{\@settoppoint{#1}}}

\expandafter\let\csname ver@autofss1.sty\endcsname\fmtversion
\def\@nomath#1{\relax\ifmmode
   \@font@warning{Command \noexpand#1invalid in math mode}\fi}
\gdef\no@alphabet@error#1{\relax \ifmmode
    \@latex@error{Math\space alphabet\space identifier\space
          \noexpand#1is\space undefined\space in\space math\space
           version\space `\math@version'}%
        {Your\space requested\space math\space alphabet\space
         is\space undefined\space in\space the\space current\space
          math\space version.^^JCheck\space the\space spelling\space
          or\space use\space the\space \noexpand\SetMathAlphabet\space
          command.}
     \fi}
\def\new@mathgroup{\alloc@8\mathgroup\chardef\sixt@@n}
\let\mathgroup\fam
\let\newfam\new@mathgroup
\@onlypreamble\new@mathgroup
\def\DeclareFontShape{\begingroup
   \nfss@catcodes
   \expandafter\endgroup
   \DeclareFontShape@}
\def\DeclareFontShape@#1#2#3#4#5#6{%
   \expandafter\ifx\csname #1+#2\endcsname\relax
     \@latex@error{Font family `#1+#2' unknown}\@eha
   \else
     \expandafter
       \xdef\csname#1/#2/#3/#4\endcsname{\expandafter\noexpand
                                   \csname #5\endcsname}%
     \def\reserved@a{#6}%
     \global
     \expandafter\let\csname#5\expandafter\endcsname
        \ifx\reserved@a\@empty
          \@empty
        \else
          \reserved@a
        \fi
   \fi
  }
\def\DeclareFixedFont#1#2#3#4#5#6{%
   \begingroup
      \math@fontsfalse
      \every@math@size{}%
      \fontsize{#6}\z@
      \usefont{#2}{#3}{#4}{#5}%
      \global\expandafter\let\expandafter#1\the\font
   \endgroup
  }
\def\do@subst@correction{%
       \xdef\subst@correction{%
          \font@name
          \global\expandafter\font
            \csname \curr@fontshape/\f@size\endcsname
            \noexpand\fontname\font
           \relax}%
       \aftergroup\subst@correction
}
\def\DeclareFontFamily#1#2#3{%
 \@ifundefined{T@#1}%
    {%
     \@latex@error{Encoding scheme  `#1' unknown}\@eha
    }%
    {%
     \def\reserved@a{#3}%
     \global
     \expandafter\let\csname #1+#2\expandafter\endcsname
            \ifx \reserved@a\@empty
              \@empty
            \else \reserved@a
            \fi
    }%
}
\let\cdp@list\@empty
\@onlypreamble\cdp@list
\let\cdp@elt\relax
\@onlypreamble\cdp@elt
\def\DeclareFontEncoding{%
   \begingroup
   \nfss@catcodes
   \expandafter\endgroup
   \DeclareFontEncoding@}
\@onlypreamble\DeclareFontEncoding
\def\DeclareFontEncoding@#1#2#3{%
  \expandafter
  \ifx\csname T@#1\endcsname\relax
     \def\cdp@elt{\noexpand\cdp@elt}%
     \xdef\cdp@list{\cdp@list\cdp@elt{#1}%
                    {\default@family}{\default@series}%
                    {\default@shape}}%
     \expandafter\let\csname#1-cmd\endcsname\@changed@cmd
  \else
     \@font@info{Redeclaring font encoding #1}%
  \fi
  \global\@namedef{T@#1}{#2}%
  \global\@namedef{M@#1}{\default@M#3}%
  \xdef\LastDeclaredEncoding{#1}%
  }
\@onlypreamble\DeclareFontEncoding@
\def\LastDeclaredEncoding{}
\def\DeclareFontSubstitution#1#2#3#4{%
  \expandafter
  \ifx\csname T@#1\endcsname\relax
    \@latex@error{Encoding scheme  `#1' unknown}\@eha
  \else
    \begingroup
       \def\reserved@a{#1}%
       \toks@{}%
       \def\cdp@elt##1##2##3##4{%
          \def\reserved@b{##1}%
          \ifx\reserved@a\reserved@b
             \addto@hook\toks@{\cdp@elt{#1}{#2}{#3}{#4}}%
          \else
             \addto@hook\toks@{\cdp@elt{##1}{##2}{##3}{##4}}%
          \fi}%
        \cdp@list
        \xdef\cdp@list{\the\toks@}%
    \endgroup
    \global
    \@namedef{D@#1}{%
           \def\default@family{#2}%
           \def\default@series{#3}%
           \def\default@shape{#4}%
           }%
  \fi
 }
\@onlypreamble\DeclareFontSubstitution
\def\DeclareFontEncodingDefaults#1#2{%
  \ifx\relax#1\else
    \ifx\default@T\@empty\else
      \@font@info{Overwriting encoding scheme text defaults}%
    \fi
    \gdef\default@T{#1}%
  \fi
  \ifx\relax#2\else
    \ifx\default@M\@empty\else
      \@font@info{Overwriting encoding scheme math defaults}%
    \fi
    \gdef\default@M{#2}%
  \fi
}
\@onlypreamble\DeclareFontEncodingDefaults
\let\default@T\@empty
\let\default@M\@empty
\def\DeclarePreloadSizes#1#2#3#4#5{%
 \@ifundefined{T@#1}%
   {\@latex@error{Encoding scheme  `#1' unknown}\@eha}%
   {%
   \begingroup
    \def\reserved@f##1,{%
        \if>##1>%
          \let\reserved@f\relax
        \else
          \xdef\font@name{\csname#1/#2/#3/#4/##1\endcsname}%
          \pickup@font
          \global\expandafter\let\font@name\relax
        \fi
        \reserved@f}%
     \reserved@f#5,,%
   \endgroup
   }%
}
\@onlypreamble\DeclarePreloadSizes
\newif\ifmath@fonts \math@fontstrue
\def\DeclareMathSizes{%
  \@ifstar{\@DeclareMathSizes\math@fontsfalse}%
          {\@DeclareMathSizes{}}}
\@onlypreamble\DeclareMathSizes
\def\@DeclareMathSizes#1#2#3#4#5{%
    \@defaultunits\dimen@#2pt\relax\@nnil
    \if$#3$%
      \expandafter \let
        \csname S@\strip@pt\dimen@\endcsname
        \math@fontsfalse
    \else
      \expandafter \gdef
      \csname S@\strip@pt\dimen@\endcsname
            {\gdef\tf@size{#3}\gdef\sf@size{#4}%
                             \gdef\ssf@size{#5}%
             #1%
                             }%
    \fi}
\@onlypreamble\@DeclareMathSizes
\DeclareRobustCommand\fontencoding[1]{%
    \expandafter\ifx\csname T@#1\endcsname\relax
      \@latex@error{Encoding scheme `#1' unknown}\@eha
    \else
      \edef\f@encoding{#1}%
      \ifx\cf@encoding\f@encoding
        \let\enc@update\relax
      \else
        \let\enc@update\@@enc@update
      \fi
    \fi
}
\def\@@enc@update{%
          \expandafter
          \let
            \csname\cf@encoding -cmd\endcsname
            \@changed@cmd
          \expandafter
          \let
            \csname\f@encoding-cmd\endcsname
            \@current@cmd
        \default@T
        \csname T@\f@encoding\endcsname
        \csname D@\f@encoding\endcsname
        \let\enc@update\relax
        \let\cf@encoding\f@encoding
}
\let\enc@update\relax
\DeclareRobustCommand\fontfamily[1]{\edef\f@family{#1}}
\DeclareRobustCommand\fontseries[1]{\edef\f@series{#1}}
\DeclareRobustCommand\fontshape [1]{\edef\f@shape{#1}}
\def\usefont#1#2#3#4{\fontencoding{#1}\fontfamily{#2}%
             \fontseries{#3}\fontshape{#4}\selectfont
             \ignorespaces}
\DeclareRobustCommand\linespread[1]
   {\set@fontsize{#1}\f@size\f@baselineskip}
%\DeclareRobustCommand\fontsize[2]
%   {\set@fontsize\baselinestretch{#1}{#2}}
\let\f@family\@empty
\let\f@series\@empty
\let\f@shape\@empty
\let\f@size\@empty
\let\f@baselineskip\@empty
\let\f@linespread\@empty
\let\f@encoding\@empty
\let\cf@encoding\@empty
\def\@defaultunits{\afterassignment\remove@to@nnil}
\begingroup
  \catcode`P=12
  \catcode`T=12
  \lowercase{
    \def\x{\def\rem@pt##1.##2PT{##1\ifnum##2>\z@.##2\fi}}}
  \expandafter\endgroup\x
\def\strip@pt{\expandafter\rem@pt\the}
\DeclareRobustCommand\mathversion[1]
         {\@nomath\mathversion
          \expandafter\ifx\csname mv@#1\endcsname\relax
          \@latex@error{Math version `#1' is not defined}\@eha\else
          \edef\math@version{#1}%
          \gdef\glb@currsize{}%
          \aftergroup\glb@settings
          \fi}
\let\frozen@everymath\everymath
\let\frozen@everydisplay\everydisplay
\newtoks\everymath
\newtoks\everydisplay
\frozen@everymath = {\the\everymath}
\frozen@everydisplay = {\the\everydisplay}
\let\curr@math@size\@empty
\def\pickup@font{%
    \expandafter \ifx \font@name \relax
       \define@newfont
    \fi}
{\catcode`\/=12
\gdef\split@name#1/#2/#3/#4/#5\@nil{\def\f@encoding{#1}%
                                    \def\f@family{#2}%
                                    \def\f@series{#3}%
                                    \def\f@shape{#4}%
                                    \def\f@size{#5}}}
\def\curr@fontshape{\f@encoding/\f@family/\f@series/\f@shape}
\def\define@newfont{%
  \begingroup
    \let\typeout\@font@info
    \escapechar\m@ne
    \expandafter\expandafter\expandafter
       \split@name\expandafter\string\font@name\@nil
      \try@load@fontshape % try always
    \expandafter\ifx
       \csname\curr@fontshape\endcsname \relax
      \wrong@fontshape\else
      \extract@font\fi
  \endgroup}
\def\try@load@fontshape{%
}
\def\nfss@catcodes{%
     \makeatletter
     \catcode`\ 9%
     \catcode`\^^I9%
     \catcode`\^^M9%
     \catcode`\\\z@
     \catcode`\{\@ne
     \catcode`\}\tw@
     \catcode`\#6%
     \catcode`\^7%
     \catcode`\%14%
   \@makeother\<%
   \@makeother\>%
   \@makeother\*%
   \@makeother\.%
   \@makeother\-%
   \@makeother\/%
   \@makeother\[%
   \@makeother\]%
   \@makeother\`%
   \@makeother\'%
   \@makeother\"%
}
\def\DeclareErrorFont#1#2#3#4#5{}

\@onlypreamble\DeclareErrorFont
\def\wrong@fontshape{%
    \csname D@\f@encoding\endcsname      % install defaults if in math
    \edef\reserved@a{\csname\curr@fontshape\endcsname}%
  \ifx\last@fontshape\reserved@a
     \errmessage{Corrupted NFSS tables}%
     \error@fontshape
  \else
    \let\f@shape\default@shape
    \expandafter\ifx\csname\curr@fontshape\endcsname\relax
       \let\f@series\default@series
        \expandafter
          \ifx\csname\curr@fontshape\endcsname\relax
           \let\f@family\default@family
        \fi \fi
  \fi
     \@font@warning{Font shape `\expandafter\string\reserved@a'
                     \expandafter\@gobble\string\@undefined\MessageBreak
                   using `\curr@fontshape' instead\@wrong@font@char}%
    \global\let\last@fontshape\reserved@a
    \gdef\@defaultsubs{%
      \@font@warning{Some font shapes were not available, defaults
                      substituted.\@gobbletwo}}%
    \global\expandafter\expandafter\expandafter\let
       \expandafter\reserved@a
           \csname\curr@fontshape\endcsname
    \xdef\font@name{%
      \csname\curr@fontshape/\f@size\endcsname}%
    \pickup@font}
\let\@wrong@font@char\@empty
\let\@defaultsubs\relax
\def\strip@prefix#1>{}
\let\install@mathalphabet\gdef
\let\math@fonts\@empty
\def\extract@alph@from@version#1#2#3{%
     \def\reserved@a##1\install@mathalphabet#3##2##3\@nil{%
         \def\reserved@b{##2}%
         \def\reserved@c####1{\gdef#1{##1####1##3}}}%
     \expandafter\reserved@a#1\@nil
     \def\reserved@a\select@group#3##1##2\@nil{%
        \reserved@c{%
           \getanddefine@fonts{#2}##2%
           \install@mathalphabet#3{%
              \relax\ifmmode \else \non@alpherr#3\fi
              \use@mathgroup##1{#2}}}%
       \gdef#3{\relax\ifmmode \else \non@alpherr#3\fi
               \use@mathgroup##1{#2}}}%
     \expandafter\reserved@a\reserved@b\@nil
     }
\let\math@bgroup\bgroup
\def\math@egroup#1{#1\egroup}
\gdef\calculate@math@sizes{%
  \@font@info{Calculating\space math\space sizes\space for\space
              size\space <\f@size>}%
  \dimen@\f@size \p@
  \@tempdimb \defaultscriptratio \dimen@
  \dimen@ \defaultscriptscriptratio \dimen@
  \expandafter\xdef\csname S@\f@size\endcsname{%
    \gdef\noexpand\tf@size{\f@size}%
    \gdef\noexpand\sf@size{\strip@pt\@tempdimb}%
    \gdef\noexpand\ssf@size{\strip@pt\dimen@}%
    \noexpand\math@fontstrue}}
\def\defaultscriptratio{.7}
\def\defaultscriptscriptratio{.5}
\ifx\noaccents@\@undefined
  \let\noaccents@\@empty
\fi
\gdef\showhyphens#1{%
  \setbox0\vbox{%
    \color@begingroup
    \everypar{}%
    \parfillskip\z@skip\hsize\maxdimen
    \normalfont
    \pretolerance\m@ne\tolerance\m@ne\hbadness\z@\showboxdepth\z@\ #1%
    \color@endgroup}}
\long\def\addto@hook#1#2{#1\expandafter{\the#1#2}}
 \def\@vpt{5}
 \def\@vipt{6}
 \def\@viipt{7}
 \def\@viiipt{8}
 \def\@ixpt{9}
 \def\@xpt{10}
 \def\@xipt{10.95}
 \def\@xiipt{12}
 \def\@xivpt{14.4}
 \def\@xviipt{17.28}
 \def\@xxpt{20.74}
 \def\@xxvpt{24.88}


\def\tracingfonts{%
  \@font@warning{Command \noexpand\tracingfonts
           not provided.\MessageBreak
           Use the `tracefnt' package.\MessageBreak Command found:}%
       \count@}
\def\@font@info#1{%
         \GenericInfo{(Font)\@spaces\@spaces\@spaces\space\space}%
                     {LaTeX Font Info: \space\space\space#1}}%
\def\@font@warning#1{%
         \GenericWarning{(Font)\@spaces\@spaces\@spaces\space\space}%
                        {LaTeX Font Warning: #1}}%
\def\extract@font{%
   \get@external@font
   \global\expandafter\font\font@name\external@font\relax
    \font@name \relax
    \csname \f@encoding+\f@family\endcsname
    \csname\curr@fontshape\endcsname
    \relax
       }
\def\get@external@font{%
   \let\external@font\@empty
   \edef\font@info{\expandafter\expandafter\expandafter\string
        \csname \curr@fontshape \endcsname}%
   \try@size@range
   \ifx\external@font\@empty
      \try@size@substitution
      \ifx\external@font\@empty
         \@latex@error{Font \expandafter \string\font@name\space
                     not found}\@eha
         \error@fontshape
         \get@external@font
   \fi\fi
}
\def\set@fontsize#1#2#3{%
    \@defaultunits\@tempdimb#2pt\relax\@nnil
    \edef\f@size{\strip@pt\@tempdimb}%
    \@defaultunits\@tempskipa#3pt\relax\@nnil
    \edef\f@baselineskip{\the\@tempskipa}%
    \edef\f@linespread{#1}%
    \let\baselinestretch\f@linespread
      \def\size@update{%
        \baselineskip\f@baselineskip\relax
        \baselineskip\f@linespread\baselineskip
        \normalbaselineskip\baselineskip
        \setbox\strutbox\hbox{%
          \vrule\@height.7\baselineskip
                \@depth.3\baselineskip
                \@width\z@}%
        \let\size@update\relax}%
  }
\let\size@update\relax
\def\glb@settings{%
     \expandafter\ifx\csname S@\f@size\endcsname\relax
       \calculate@math@sizes
     \fi
     \csname S@\f@size\endcsname
     \ifmath@fonts
       \begingroup
         \escapechar\m@ne
         \csname mv@\math@version \endcsname
         \globaldefs\@ne
         \math@fonts
         \let \glb@currsize \f@size
       \endgroup
        \the\every@math@size
     \fi
}
\def\baselinestretch{1}
\newtoks\every@math@size
\every@math@size={}
\let\init@restore@glb@settings\relax
\def\restglb@settings{%
      \begingroup
        \let\f@size\curr@math@size
        \ifx\glb@currsize \f@size
        \else
          \glb@settings
        \fi
      \endgroup
}
\def\use@mathgroup#1#2{\relax\ifmmode
     \math@bgroup
         \expandafter\ifx\csname M@\f@encoding\endcsname#1\else
         #1\fi
         \mathgroup#2\relax
     \expandafter\math@egroup\fi}%
\def\getanddefine@fonts#1#2{%
  \xdef\font@name{\csname \string#2/\tf@size\endcsname}%
  \pickup@font \let\textfont@name\font@name
  \xdef\font@name{\csname \string#2/\sf@size\endcsname}%
  \pickup@font \let\scriptfont@name\font@name
  \xdef\font@name{\csname \string#2/\ssf@size\endcsname}%
  \pickup@font
  \edef\math@fonts{\math@fonts
               \textfont#1\textfont@name
               \scriptfont#1\scriptfont@name
               \scriptscriptfont#1\font@name}%
    }
\def\ifnot@nil#1{\def\reserved@a{#1}%
  \ifx\reserved@a\@nnil \expandafter\@gobble
  \else \expandafter\@firstofone\fi}
\def\remove@to@nnil#1\@nnil{}
\def\remove@angles#1>{\set@simple@size@args}
\def\remove@star#1*{#1}
\def\extract@sizefn#1*#2\@nil{%
  \if>#2>\set@size@funct@args#1\@nil
         \let\sizefn@info\@empty
  \else\expandafter\set@size@funct@args\remove@star#2\@nil
       \def\sizefn@info{#1}\fi
  }
\def\try@simple@size{%
    \def\reserved@a{\def\extract@fontinfo####1}%
    \expandafter\reserved@a\expandafter<\f@size>##2<##3\@nnil{%
          \ifnot@nil{##2}%
            {\set@simple@size@args##2<##3\@nnil
             \execute@size@function\sizefn@info
            }}%
    \expandafter\expandafter
    \expandafter\extract@fontinfo\expandafter\font@info
    \expandafter<\f@size>\@nil<\@nnil
}
\def\set@simple@size@args#1<{%
          \if<#1<%
            \expandafter\remove@angles
          \else
            \extract@sizefn#1*\@nil
            \expandafter\remove@to@nnil
          \fi}
\def\extract@rangefontinfo#1<#2>{%
       \is@range#2->\@nil#2>}
\def\is@range#1-#2\@nil{%
   \if>#2\expandafter\check@single\else
      \expandafter\check@range\fi}
\def\check@range#1-#2>#3<#4\@nnil{%
  \ifnot@nil{#3}{%
     \def\reserved@f{\extract@rangefontinfo<#4\@nnil}%
     \upper@bound0#2\p@
     \ifdim\upper@bound=\z@ \upper@bound\maxdimen\fi
     \ifdim \f@size \p@<\upper@bound
       \lower@bound0#1\p@
       \ifdim \f@size \p@<\lower@bound
        \else
          \set@simple@size@args#3<#4\@nnil
          \execute@size@function\sizefn@info
         \ifx\external@font\@empty
         \else
           \let\reserved@f\@empty
         \fi
       \fi
     \fi
     \reserved@f}}
\newdimen\lower@bound
\newdimen\upper@bound
\def\check@single#1>#2<#3\@nnil{%
     \def\reserved@f{\extract@rangefontinfo<#3\@nnil}%
     \ifdim \f@size \p@=#1\p@
        \set@simple@size@args#2<#3\@nnil
        \execute@size@function\sizefn@info
        \ifx\external@font\@empty
        \else
          \let\reserved@f\@empty
        \fi
     \fi
     \reserved@f}
\def\set@size@funct@args{\@ifnextchar[%
  \set@size@funct@args@{\set@size@funct@args@[]}}
\def\set@size@funct@args@[#1]#2\@nil{%
  \def\mandatory@arg{#2}%
  \def\optional@arg{#1}}
\def\DeclareSizeFunction#1#2{\@namedef{s@fct@#1}{#2}}
\@onlypreamble\DeclareSizeFunction
\def\execute@size@function#1{%    %%% could be added to autoload as well
      \csname s@fct@#1\endcsname
}
\def\try@size@range{%
    \expandafter\extract@rangefontinfo\font@info <-*>\@nil<\@nnil
}
\gdef\try@size@substitution{%
  \@tempdimb \maxdimen
  \let \best@size \@empty
  \expandafter \try@simples \font@info <\number\@M>\@nil<\@nnil

}
\def\font@submax{0pt}
\def\fontsubfuzz{.4pt}
\gdef\try@simples#1<#2>{%
  \tryif@simple#2->\tryif@simple}
\gdef\tryif@simple#1-#2\tryif@simple{%
  \let \reserved@f \try@simples
  \if>#2%
    \dimen@ #1\p@
    \ifdim \dimen@<\@M\p@
      \ifdim \f@size\p@<\dimen@
        \@tempdimc \dimen@
        \advance\@tempdimc -\f@size\p@
      \else
        \@tempdimc \f@size\p@
        \advance\@tempdimc -\dimen@
      \fi
      \ifdim \@tempdimc<\@tempdimb
        \@tempdimb \@tempdimc
        \def \best@size{#1}%
      \fi
    \else
  \ifx \external@font\@empty
    \ifx \best@size\@empty
    \else
      \ifdim \@tempdimb>\font@submax \relax
        \xdef \font@submax {\the\@tempdimb}%
      \fi
      \let \f@user@size \f@size
      \let \f@size \best@size
      \ifdim \@tempdimb>\fontsubfuzz\relax
        \@font@warning{Font\space shape\space
            `\curr@fontshape'\space in\space size\space
             <\f@user@size>\space not\space available\MessageBreak
             size\space <\f@size>\space substituted}%
      \fi
      \try@simple@size
      \do@subst@correction
    \fi
 \fi
      \let \reserved@f \remove@to@nnil
    \fi
  \fi
  \reserved@f}
\DeclareSizeFunction{}{\empty@sfcnt\@font@warning}
\DeclareSizeFunction{s}{\empty@sfcnt\@font@info}
\def\empty@sfcnt#1{%
      \@tempdimb \f@size\p@
      \ifx\optional@arg\@empty
      \else
        \@tempdimb \optional@arg\@tempdimb
        #1{Font\space shape\space `\curr@fontshape'\space
           will\space be\MessageBreak
           scaled\space to\space size\space \the\@tempdimb}%
      \fi
      \edef\external@font{\mandatory@arg\space at\the\@tempdimb}}
\DeclareSizeFunction{gen}{\gen@sfcnt\@font@warning}
\DeclareSizeFunction{sgen}{\gen@sfcnt\@font@info}
\def\gen@sfcnt{%
      \edef\mandatory@arg{\mandatory@arg\f@size}%
      \empty@sfcnt}
\DeclareSizeFunction{genb}{\genb@sfcnt\@font@warning}
\DeclareSizeFunction{sgenb}{\genb@sfcnt\@font@info}
\def\genb@sfcnt{%
    \edef\mandatory@arg{\mandatory@arg\expandafter\genb@x\f@size..\@@}%
    \empty@sfcnt}
\def\genb@x#1.#2.#3\@@{\two@digits{#1}\genb@y#200\@@}
\def\genb@y#1#2#3\@@{#1#2}
\DeclareSizeFunction{sub}{\sub@sfcnt\@font@warning}
\DeclareSizeFunction{ssub}{\sub@sfcnt\@font@info}
\def\sub@sfcnt#1{%
    \edef\mandatory@arg{\f@encoding/\mandatory@arg}%
    \begingroup
     \expandafter\split@name\mandatory@arg/\@nil
     \try@load@fontshape
    \endgroup
    \let\f@user@size\f@size
    \expandafter
    \ifx\csname\mandatory@arg\endcsname\relax
      \errmessage{No\space declaration\space for\space
                  shape\space \mandatory@arg}%
      \error@fontshape
    \else
      #1{Font\space shape\space `\curr@fontshape'\space in\space
         size\space <\f@size>\space not\space available\MessageBreak
         Font\space shape\space `\mandatory@arg'\space tried\space
         instead}%
      \expandafter\split@name\mandatory@arg/\@nil
    \fi
    \edef\f@size{\f@user@size}%
    \get@external@font
    \do@subst@correction
}
\DeclareSizeFunction{subf}{\subf@sfcnt\@font@warning}
\DeclareSizeFunction{ssubf}{\subf@sfcnt\@font@info}
\def\subf@sfcnt#1{%
      #1{Font\space shape\space `\curr@fontshape'\space in\space
         size\space \f@size\space not\space available\MessageBreak
        external\space font\space `\mandatory@arg'\space used}%
      \empty@sfcnt#1%
      }
\DeclareSizeFunction{fixed}{\fixed@sfcnt\@font@warning}
\DeclareSizeFunction{sfixed}{\fixed@sfcnt\@font@info}
\def\fixed@sfcnt#1{%
  \ifx\optional@arg\@empty
    \let\external@font\mandatory@arg
  \else
    \edef\external@font{\mandatory@arg\space at\optional@arg pt}%
  \fi
  #1{External\space font\space `\external@font'\space loaded\space
     for\space size\MessageBreak
     <\f@size>}%
}

\gdef\new@fontshape#1#2#3#4{%
     \warn@rel@i\new@fontshape\DeclareFontShape
     \expandafter\scan@fontshape\@gobble#4<\@nil><<%
     \DeclareFontShape U{#1}{#2}{#3}\reserved@f}
\@onlypreamble\new@fontshape
\gdef\warn@rel@i#1#2{%
 \@font@warning{***  NFSS release 1 command
               \noexpand#1found\MessageBreak
   ***  Update by using release 2 command
        \string#2.\MessageBreak
   ***  Recovery is probably possible}%
}
\@onlypreamble\warn@rel@i
\gdef\scan@fontshape{%
  \let\reserved@f\@empty
  \let\reserved@e\@empty %        holds last info
  \scan@@fontshape
}
\@onlypreamble\scan@fontshape
\gdef\scan@@fontshape#1>#2#3<{%
  \ifx\@nil#1%
    \edef\reserved@f{\reserved@f\reserved@e}%
  \else
    \def\reserved@b{#1}%       nick names
    \def\reserved@c{#3}%
    \in@{ at}{#3}%
    \ifin@
      \in@{pt}{#3}%  not a proof but a good chance
      \ifin@
        \def\reserved@a##1 at##2pt##3\@nil{%
           \def\reserved@b{##2}%
           \def\reserved@c{##1}%
           }%
        \reserved@a#3\@nil
      \fi
    \fi
    \ifnum 0<0#2
      \edef\reserved@d{subf*\reserved@c}%
      \ifcase #2\or
      \or
      \else
        \errmessage{*** What's this? NFSS release 0? ***}%
      \fi
    \else
      \edef\reserved@d{#2\reserved@c}%
    \fi
    \ifx\reserved@d\reserved@e
      \edef\reserved@f{\reserved@f<\reserved@b>}%
    \else
      \edef\reserved@f{\reserved@f\reserved@e<\reserved@b>}%add old info
      \let\reserved@e\reserved@d
    \fi
    \expandafter\scan@@fontshape
  \fi
}
\@onlypreamble\scan@@fontshape
\gdef\subst@fontshape#1#2#3#4#5#6{%
     \warn@rel@i\subst@fontshape\DeclareFontShape
     \DeclareFontShape{U}{#1}{#2}{#3}{<->sub*#4/#5/#6}{}}
\@onlypreamble\subst@fontshape
\gdef\extra@def#1#2#3{%
     \warn@rel@i\extra@def\DeclareFontFamily
     \DeclareFontFamily{U}{#1}{}%
}
\@onlypreamble\extra@def
\gdef\default@mextra{%
  \warn@rel@i\default@mextra\DeclareFontEncodingDefaults
  \DeclareFontEncodingDefaults\relax
}
\@onlypreamble\default@mextra
\gdef\preload@sizes{%
     \warn@rel@i\preload@sizes\DeclarePreloadSizes
     \DeclarePreloadSizes U%
}
\@onlypreamble\preload@sizes
\gdef\err@rel@i#1#2{%
  \@latex@error{***  NFSS release 1 command \noexpand#1found%
          ^^J***  Recovery not possible. Use \string#2}%
       {The new release of NFSS doesn't support the
        \noexpand#1command^^Jany longer.
        Please upgrade your file to the syntax of NFSS
        release 2^^Jusing the \noexpand#2command.}%
  \batchmode\input.\relax
}
\@onlypreamble\err@rel@i
\gdef\newmathalphabet{%
  \if@no@font@opt
    \@latex@error{*** NFSS release 1 command
                    \noexpand\newmathalphabet found%
     ^^J \space*** Automatic recovery not possible.%
     ^^J \space*** TYPE H for Help%
              }%
       {Please look at the file usrguide.tex for hints on
        how to resolve this problem.}%
  \else
     \warn@rel@i\newmathalphabet\DeclareMathAlphabet
  \fi
  \@ifstar\newmathalphabet@@@
          \newmathalphabet@@}
\gdef\newmathalphabet@@#1{\DeclareMathAlphabet#1{U}{}{}{}}
\gdef\newmathalphabet@@@#1#2#3#4{%
       \DeclareMathAlphabet{#1}{U}{#2}{#3}{#4}}
\@onlypreamble\newmathalphabet
\@onlypreamble\newmathalphabet@@
\@onlypreamble\newmathalphabet@@@
\global\let\if@no@font@opt\iftrue
\gdef\@no@font@optfalse{\let\if@no@font@opt\iffalse}
\gdef\define@mathalphabet{%
      \err@rel@i\define@mathalphabet\DeclareMathAlphabet
}
\@onlypreamble\define@mathalphabet
\gdef\define@mathgroup{%
      \err@rel@i\define@mathgroup\DeclareSymbolFont
}
\@onlypreamble\define@mathgroup
\def\addtoversion#1#2{%
  \warn@rel@i\addtoversion\SetMathAlphabet
  \SetMathAlphabet#2{#1}{U}}
\@onlypreamble\addtoversion


\def\in@#1#2{%
 \def\in@@##1#1##2##3\in@@{%
  \ifx\in@##2\in@false\else\in@true\fi}%
 \in@@#2#1\in@\in@@}
\newif\ifin@
\let\version@list=\@empty
\@onlypreamble\version@list
\let\version@elt\relax
\@onlypreamble\version@elt
\let\alpha@list\@empty
\@onlypreamble\alpha@list
\let\alpha@elt\relax
\@onlypreamble\alpha@elt
\count18=-1
\def\select@group#1#2#3#4{%
 \ifx\math@bgroup\bgroup\else\relax\expandafter\@firstofone\fi
 {%
 \ifmmode
  \ifnum\csname c@mv@\math@version\endcsname<\sixt@@n
     \begingroup
       \escapechar\m@ne
       \getanddefine@fonts{\csname c@mv@\math@version\endcsname}#3%
       \globaldefs\@ne  \math@fonts
     \endgroup
     \init@restore@version
     \xdef#1{\noexpand\use@mathgroup\noexpand#2%
             {\number\csname c@mv@\math@version\endcsname}}%
     \global\advance\csname c@mv@\math@version\endcsname\@ne
   \else
     \let#1\relax
     \@latex@error{Too many math alphabets used in
                   version \math@version}%
        \@eha
   \fi
 \else \expandafter\non@alpherr\fi
 #1{#4}%
 }%
}
\@onlypreamble\restore@mathversion
\def\init@restore@version{%
        \global\let\init@restore@version\relax
        \xdef\restore@mathversion
             {\expandafter\noexpand\csname mv@\math@version\endcsname
              \global\csname c@mv@\math@version\endcsname
              \number\csname c@mv@\math@version\endcsname\relax}%
        \aftergroup\dorestore@version
}
\@onlypreamble\init@restore@version
\gdef\non@alpherr#1{\@latex@error{%
    \expandafter\@firstoftwo\string#1 allowed only in math mode}\@ehd}
\def\dorestore@version
 {\ifmmode
    \aftergroup\dorestore@version
  \else
    \gdef\init@restore@version{%
        \global\let\init@restore@version\relax
        \xdef\restore@mathversion
             {\expandafter\noexpand\csname mv@\math@version\endcsname
              \global\csname c@mv@\math@version\endcsname
              \number\csname c@mv@\math@version\endcsname\relax}%
        \aftergroup\dorestore@version
    }%
    \begingroup
      \let\getanddefine@fonts\@gobbletwo
      \restore@mathversion
    \endgroup
  \fi}%
\@onlypreamble\dorestore@version
\def\document@select@group#1#2#3#4{%
 \ifx\math@bgroup\bgroup\else\relax\expandafter\@firstofone\fi
 {%
 \ifmmode
   \ifnum\csname c@mv@\math@version\endcsname<\sixt@@n
     \begingroup
       \escapechar\m@ne
       \getanddefine@fonts{\csname c@mv@\math@version\endcsname}#3%
       \globaldefs\@ne  \math@fonts
     \endgroup
     \expandafter\extract@alph@from@version
         \csname mv@\math@version\expandafter\endcsname
         \expandafter{\number\csname
                       c@mv@\math@version\endcsname}%
          #1%
     \global\advance\csname c@mv@\math@version\endcsname\@ne
   \else
     \let#1\relax
     \@latex@error{Too many math alphabets used
                   in version \math@version}%
        \@eha
  \fi
 \else \expandafter\non@alpherr\fi
 #1{#4}%
 }%
}
\def\process@table{%
    \def\cdp@elt##1##2##3##4{%
        \@font@info{Checking defaults for
                  ##1/##2/##3/##4}%
        \expandafter
        \ifx\csname##1/##2/##3/##4\endcsname\relax
          \begingroup
           \def\f@encoding{##1}\def\f@family{##2}%
           \try@load@fontshape
          \endgroup
        \fi
        \expandafter
        \ifx\csname##1/##2/##3/##4\endcsname\relax
             \@latex@error{This NFSS system isn't set up properly}%
                       {For encoding scheme ##1 the defaults
                        ##2/##3/##4 do not form a valid font shape}%
        \else
             \@font@info{... okay}%
        \fi}%
    \cdp@list
    \begingroup
       \escapechar\m@ne
       \error@fontshape
       \expandafter\ifx\csname \curr@fontshape\endcsname\relax
          \begingroup
            \try@load@fontshape
           \endgroup
       \fi
       \expandafter\ifx\csname \curr@fontshape\endcsname\relax
         \@latex@error{This NFSS system isn't set up properly}%
            {The system maintainer forgot to specify a suitable
             substitution
             font shape using the \noexpand\DeclareErrorFont
             command}%
       \fi
    \endgroup
    \let\select@group\document@select@group
    \fontencoding{\encodingdefault}%
    \fontfamily{\familydefault}%
    \fontseries{\seriesdefault}%
    \fontshape{\shapedefault}%
 \everyjob{}%
}
\@onlypreamble\process@table
\def\DeclareMathVersion#1{%
  \expandafter\new@mathversion\csname mv@#1\endcsname}
\@onlypreamble\DeclareMathVersion
\def\new@mathversion#1{%
  \expandafter\in@\expandafter#1\expandafter{\version@list}%
  \ifin@
    \@font@info{Redeclaring math version
               `\expandafter\@gobblefour\string#1'}%
  \else
    \global\expandafter\newcount\csname c@\expandafter
                                \@gobble\string#1\endcsname
    \def\version@elt{\noexpand\version@elt\noexpand}%
    \edef\version@list{\version@list\version@elt#1}%
  \fi
  \toks@{}%
  \count@\z@
  \def\group@elt##1##2{%
       \advance\count@\@ne
       \addto@hook\toks@{\getanddefine@fonts##1##2}%
       }%
  \group@list
  \global\csname c@\expandafter\@gobble\string#1\endcsname\count@
  \def\alpha@elt##1##2##3{%
       \ifx##2\no@alphabet@error
         \toks@\expandafter{\the\toks@\install@mathalphabet##1%
             {\no@alphabet@error##1}}%
       \else
         \toks@\expandafter{\the\toks@\install@mathalphabet##1%
             {\select@group##1##2##3}}%
       \fi
          }%
  \alpha@list
  \xdef#1{\the\toks@}%
}
\@onlypreamble\new@mathversion

\def\DeclareSymbolFont#1#2#3#4#5{%
   \expandafter\new@mathgroup\csname sym#1\endcsname
   \expandafter\new@symbolfont\csname sym#1\endcsname
                     {#2}{#3}{#4}{#5}%
}

\@onlypreamble\DeclareSymbolFont
\let\group@list\@empty
\@onlypreamble\group@list
\let\group@elt\relax
\@onlypreamble\group@elt
\def\new@symbolfont#1#2#3#4#5{\relax}
\@onlypreamble\new@symbolfont
\def\SetSymbolFont#1#2#3#4#5#6{%
 \@tempswafalse
 \edef\reserved@b{#3}%
 \def\cdp@elt##1##2##3##4{\def\reserved@c{##1}%
      \ifx\reserved@b\reserved@c \@tempswatrue\fi}%
 \cdp@list
 \if@tempswa
  \expandafter\SetSymbolFont@
    \csname mv@#2\expandafter\endcsname\csname#3/#4/#5/#6\expandafter
    \endcsname \csname sym#1\endcsname
 \else
  \@latex@error{Encoding scheme  `#3' unknown}\@eha
 \fi
}
\@onlypreamble\SetSymbolFont
\def\SetSymbolFont@#1#2#3{%
  \expandafter\in@\expandafter#1\expandafter{\version@list}%
  \ifin@
    \expandafter\in@\expandafter#3\expandafter{\group@list}%
    \ifin@

      \begingroup
        \expandafter\get@cdp\string#2\@nil\reserved@a
        \toks@{}%
        \def\install@mathalphabet##1##2{%
             \addto@hook\toks@{\install@mathalphabet##1{##2}}%
            }%
        \def\getanddefine@fonts##1##2{%
          \ifnum##1=#3%
             \addto@hook\toks@{\getanddefine@fonts#3#2}%
             \expandafter\get@cdp\string##2\@nil\reserved@b
             \ifx\reserved@a\reserved@b\else
                \@font@warning{Encoding `\reserved@b' has changed
                    to `\reserved@a' for symbol font\MessageBreak
                   `\expandafter\@gobblefour\string#3' in the
                    math version `\expandafter
                    \@gobblefour\string#1'}%
             \fi
             \@font@info{%
                Overwriting symbol font
                `\expandafter\@gobblefour\string#3' in
                 version `\expandafter
                \@gobblefour\string#1'\MessageBreak
                \@spaces \expandafter\@gobble\string##2 -->
                         \expandafter\@gobble\string#2}%
          \else
             \addto@hook\toks@{\getanddefine@fonts##1##2}%
          \fi}%
         #1%
         \xdef#1{\the\toks@}%
      \endgroup
    \else
       \@latex@error{Symbol font `\expandafter\@gobblefour\string#3'
                  not defined}\@eha
    \fi
  \else
    \@latex@error{Math version `\expandafter\@gobblefour\string#1'
       is not
       defined}{You probably mispelled the name of the math
       version.^^JOr you have to specify an additional package.}%
  \fi
}
\@onlypreamble\SetSymbolFont@
\def\get@cdp#1#2/#3\@nil#4{\def#4{#2}}
\@onlypreamble\get@cdp
\def\DeclareMathAlphabet#1#2#3#4#5{%
 \@tempswafalse
 \edef\reserved@b{#2}%
 \def\cdp@elt##1##2##3##4{\def\reserved@c{##1}%
      \ifx\reserved@b\reserved@c \@tempswatrue\fi}%
 \cdp@list
 \if@tempswa
   \expandafter\ifx
   \csname\expandafter\@gobble\string#1\endcsname
   \relax
      \new@mathalphabet#1{#2}{#3}{#4}{#5}%
   \else
     \edef\reserved@a{\noexpand\in@{\string\select@group}%
         {\expandafter\meaning\csname @\expandafter
          \@gobble\string#1\endcsname}}%
     \reserved@a
     \ifin@
       \@font@info{Redeclaring math alphabet \string#1}%
       \def\version@elt##1{%
         \expandafter\SetMathAlphabet@\expandafter
            ##1\csname#2/#3/#4/#5\expandafter\endcsname
            \csname M@#2\expandafter\endcsname
            \csname @\expandafter\@gobble\string#1\endcsname#1}%
       \version@list
     \else
       \edef\reserved@a{\noexpand\in@{\string\use@mathgroup}%
         {\expandafter\meaning\csname @\expandafter
          \@gobble\string#1\endcsname}}%
       \reserved@a
       \ifin@
         \@font@info{Redeclaring math alphabet \string#1}%
         \new@mathalphabet#1{#2}{#3}{#4}{#5}%
       \else
         \@latex@error{Command `\string#1' already defined}\@eha
       \fi
     \fi
  \fi
 \else
  \@latex@error{Encoding scheme  `#2' unknown}\@eha
 \fi
  }
\@onlypreamble\DeclareMathAlphabet
\def\new@mathalphabet#1#2#3#4#5{%
    \toks@\expandafter{\alpha@list}%
    \edef#1{\expandafter\noexpand\csname @\expandafter
            \@gobble\string#1\endcsname
            \if/#5/%
               \noexpand\no@alphabet@error
               \noexpand\no@alphabet@error
            \else
               \expandafter\noexpand\csname M@#2\endcsname
               \expandafter\noexpand\csname#2/#3/#4/#5\endcsname
            \fi
           }%
    \toks2\expandafter{#1}%
    \edef\alpha@list{\the\toks@\noexpand\alpha@elt\the\toks2}%
    \def\version@elt##1{\toks@\expandafter{##1}%
                   \edef##1{\the\toks@\install@mathalphabet
                            \expandafter\noexpand
                            \csname @\expandafter\@gobble
                               \string#1\endcsname
                           {\if/#5/%
                             \noexpand\no@alphabet@error
                             \noexpand#1%
                            \else
                             \noexpand\select@group\the\toks2
                            \fi}}%
                 }%
    \version@list
    \expandafter\edef\csname @\expandafter\@gobble
                \string#1\endcsname{\if/#5/%
              \noexpand\no@alphabet@error
              \noexpand#1%
            \else
              \noexpand\select@group\the\toks2
            \fi}%
    \edef#1{\noexpand\protect
            \expandafter\noexpand\csname @\expandafter
            \@gobble\string#1\endcsname}%
}
\@onlypreamble\new@mathalphabet
\def\SetMathAlphabet#1#2#3#4#5#6{%
 \@tempswafalse
 \edef\reserved@b{#3}%
 \def\cdp@elt##1##2##3##4{\def\reserved@c{##1}%
      \ifx\reserved@b\reserved@c \@tempswatrue\fi}%
 \cdp@list
 \if@tempswa
  \expandafter\SetMathAlphabet@
    \csname mv@#2\expandafter\endcsname\csname#3/#4/#5/#6\expandafter
    \endcsname \csname M@#3\expandafter\endcsname
    \csname @\expandafter\@gobble\string#1\endcsname#1%
 \else
  \@latex@error{Encoding scheme  `#3' unknown}\@eha
 \fi
}
\@onlypreamble\SetMathAlphabet
\def\SetMathAlphabet@#1#2#3#4#5{%
  \expandafter\in@\expandafter#1\expandafter{\version@list}%
  \ifin@
    \expandafter\in@\expandafter#4\expandafter{\alpha@list}%
    \ifin@
      \begingroup
        \toks@{}%
        \def\getanddefine@fonts##1##2{%
             \addto@hook\toks@{\getanddefine@fonts##1##2}%
            }%
        \def\reserved@c##1##2##3##4{%                % for message below
            \expandafter\@gobble\string##4}%
        \def\install@mathalphabet##1##2{%
          \ifx##1#4%
             \addto@hook\toks@
                {\install@mathalphabet#4{\select@group#4#3#2}}%
             \@font@info{Overwriting math alphabet
                `\string#5' in version `\expandafter
                 \@gobblefour\string#1'\MessageBreak
                 \@spaces \reserved@c##2 -->
                        \expandafter\@gobble\string#2}%
          \else
             \addto@hook\toks@{\install@mathalphabet##1{##2}}%
          \fi
          }%
        #1%
        \xdef#1{\the\toks@}%
      \endgroup
    \else
       \edef\reserved@a{%
         \noexpand\in@{\string\use@mathgroup}{\meaning#4}}%
       \reserved@a
       \ifin@
         \def\reserved@b##1\use@mathgroup##2##3{%
             \def\reserved@b{##3}\def\reserved@c{##2}}%
         \expandafter\reserved@b#4%
         \begingroup
           \def\install@mathalphabet##1##2{%
               \addto@hook\toks@{\install@mathalphabet##1{##2}}%
               }%
            \def\getanddefine@fonts##1##2{%
              \addto@hook\toks@{\getanddefine@fonts##1##2}%
              \ifnum##1=\reserved@b
                 \expandafter
                 \addto@hook\expandafter\toks@
                 \expandafter{\expandafter\install@mathalphabet
                 \expandafter#4\expandafter
                       {\expandafter\select@group\expandafter
                         #4\reserved@c##2}}%
              \fi
                      }%
           \def\version@elt##1{%
               \toks@{}%
               ##1%
               \xdef##1{\the\toks@}%
              }%
           \version@list
         \endgroup
         \expandafter\gdef\expandafter\alpha@list\expandafter
             {\alpha@list
              \alpha@elt #4\no@alphabet@error \no@alphabet@error}%
         \gdef#4{\no@alphabet@error #5}% fake things :-)
         \SetMathAlphabet@{#1}{#2}{#3}#4#5%
       \else
         \@latex@error{Command `\string#5' not defined as a
                       math alphabet}%
            {Use \noexpand\DeclareMathAlphabet to define it.}%
       \fi
    \fi
  \else
    \@latex@error{Math version `\expandafter\@gobblefour\string#1'
       is not
       defined}{You probably mispelled the name of the math
       version.^^JOr you have to specify an additional package.}%
  \fi
}
\@onlypreamble\SetMathAlphabet@
\def\DeclareMathAccent#1#2#3#4{%
  \expandafter\in@\csname sym#3\expandafter\endcsname
     \expandafter{\group@list}%
  \ifin@
    \begingroup
      \count\z@=#4\relax
      \count\tw@\count\z@
      \divide\count\z@\sixt@@n
      \count@\count\z@
      \multiply\count@\sixt@@n
      \advance\count\tw@-\count@
      \if\relax\noexpand#1% is command?
        \edef\reserved@a{\noexpand\in@{\string\mathaccent}{\meaning#1}}%
        \reserved@a
        \ifin@
          \expandafter\set@mathaccent
             \csname sym#3\endcsname#1#2%
             {\hexnumber@{\count\z@}\hexnumber@{\count\tw@}}%
          \@font@info{Redeclaring math accent \string#1}%
        \else
          \expandafter\ifx
          \csname\expandafter\@gobble\string#1\endcsname
          \relax
            \expandafter\set@mathaccent
               \csname sym#3\endcsname#1#2%
               {\hexnumber@{\count\z@}\hexnumber@{\count\tw@}}%
          \else
            \@latex@error{Command `\string#1' already defined}\@eha
          \fi
        \fi
      \else
       \@latex@error{Not a command name: `\noexpand#1'}\@eha
      \fi
    \endgroup
  \else
    \@latex@error{Symbol font `#3' is not defined}\@eha
  \fi
}
\@onlypreamble\DeclareMathAccent
\def\set@mathaccent#1#2#3#4{%
  \xdef#2{\mathaccent"\mathchar@type#3\hexnumber@#1#4\relax}}
\@onlypreamble\set@mathaccent
\def\DeclareMathSymbol#1#2#3#4{%
  \expandafter\in@\csname sym#3\expandafter\endcsname
     \expandafter{\group@list}%
  \ifin@
    \begingroup
      \count\z@=#4\relax
      \count\tw@\count\z@
      \divide\count\z@\sixt@@n
      \count@\count\z@
      \multiply\count@\sixt@@n
      \advance\count\tw@-\count@
      \if\relax\noexpand#1% is command?
        \edef\reserved@a{\noexpand\in@{\string\mathchar}{\meaning#1}}%
        \reserved@a
        \ifin@
          \expandafter\set@mathsymbol
             \csname sym#3\endcsname#1#2%
             {\hexnumber@{\count\z@}\hexnumber@{\count\tw@}}%
          \@font@info{Redeclaring math symbol \string#1}%
        \else
            \expandafter\ifx
            \csname\expandafter\@gobble\string#1\endcsname
            \relax
            \expandafter\set@mathsymbol
               \csname sym#3\endcsname#1#2%
               {\hexnumber@{\count\z@}\hexnumber@{\count\tw@}}%
          \else
            \@latex@error{Command `\string#1' already defined}\@eha
          \fi
        \fi
      \else
        \expandafter\set@mathchar
          \csname sym#3\endcsname#1#2
          {\hexnumber@{\count\z@}\hexnumber@{\count\tw@}}%
      \fi
    \endgroup
  \else
    \@latex@error{Symbol font `#3' is not defined}\@eha
  \fi
}
\@onlypreamble\DeclareMathSymbol
\def\set@mathchar#1#2#3#4{%
  \global\mathcode`#2="\mathchar@type#3\hexnumber@#1#4\relax}
\@onlypreamble\set@mathchar
\def\set@mathsymbol#1#2#3#4{%
  \global\mathchardef#2"\mathchar@type#3\hexnumber@#1#4\relax}
\@onlypreamble\set@mathsymbol
\def\DeclareMathDelimiter#1{%
  \if\relax\noexpand#1%
    \expandafter\@DeclareMathDelimiter
  \else
    \expandafter\@xxDeclareMathDelimiter
  \fi
  #1}
\@onlypreamble\DeclareMathDelimiter
\def\@xxDeclareMathDelimiter#1#2#3#4{%
   \begingroup
    \let\mathalpha\mathord
    \ifnum7=\mathchar@type{#2}%
      \endgroup
      \expandafter\@firstofone
    \else
      \endgroup
      \DeclareMathSymbol#1{#2}{#3}{#4}%
      \expandafter\@firstoftwo
    \fi
    {\@xDeclareMathDelimiter#1}{#2}{#3}{#4}}
\@onlypreamble\@xxDeclareMathDelimiter
\def\@DeclareMathDelimiter#1#2#3#4#5#6{%
  \expandafter\in@\csname sym#3\expandafter\endcsname
     \expandafter{\group@list}%
  \ifin@
    \expandafter\in@\csname sym#5\expandafter\endcsname
       \expandafter{\group@list}%
    \ifin@
      \begingroup
        \count\z@=#4\relax
        \count\tw@\count\z@
        \divide\count\z@\sixt@@n
        \count@\count\z@
        \multiply\count@\sixt@@n
        \advance\count\tw@-\count@
        \edef\reserved@c{\hexnumber@{\count\z@}\hexnumber@{\count\tw@}}%
      %
        \count\z@=#6\relax
        \count\tw@\count\z@
        \divide\count\z@\sixt@@n
        \count@\count\z@
        \multiply\count@\sixt@@n
        \advance\count\tw@-\count@
        \edef\reserved@d{\hexnumber@{\count\z@}\hexnumber@{\count\tw@}}%
      %
        \edef\reserved@a{\noexpand\in@{\string\delimiter}{\meaning#1}}%
        \reserved@a
        \ifin@
          \expandafter\set@mathdelimiter
             \csname sym#3\expandafter\endcsname
             \csname sym#5\endcsname#1#2%
             \reserved@c\reserved@d
          \@font@info{Redeclaring math delimiter \string#1}%
        \else
            \expandafter\ifx
            \csname\expandafter\@gobble\string#1\endcsname
            \relax
            \expandafter\set@mathdelimiter
              \csname sym#3\expandafter\endcsname
              \csname sym#5\endcsname#1#2%
              \reserved@c\reserved@d
          \else
            \@latex@error{Command `\string#1' already defined}\@eha
          \fi
        \fi
      \endgroup
    \else
      \@latex@error{Symbol font `#5' is not defined}\@eha
    \fi
  \else
    \@latex@error{Symbol font `#3' is not defined}\@eha
  \fi
}
\@onlypreamble\@DeclareMathDelimiter
\def\@xDeclareMathDelimiter#1#2#3#4#5{%
  \expandafter\in@\csname sym#2\expandafter\endcsname
     \expandafter{\group@list}%
  \ifin@
    \expandafter\in@\csname sym#4\expandafter\endcsname
       \expandafter{\group@list}%
    \ifin@
      \begingroup
        \count\z@=#3\relax
        \count\tw@\count\z@
        \divide\count\z@\sixt@@n
        \count@\count\z@
        \multiply\count@\sixt@@n
        \advance\count\tw@-\count@
        \edef\reserved@c{\hexnumber@{\count\z@}\hexnumber@{\count\tw@}}%
      %
        \count\z@=#5\relax
        \count\tw@\count\z@
        \divide\count\z@\sixt@@n
        \count@\count\z@
        \multiply\count@\sixt@@n
        \advance\count\tw@-\count@
        \edef\reserved@d{\hexnumber@{\count\z@}\hexnumber@{\count\tw@}}%
        \expandafter\set@@mathdelimiter
           \csname sym#2\expandafter\endcsname\csname sym#4\endcsname#1%
           \reserved@c\reserved@d
      \endgroup

    \else
      \@latex@error{Symbol font `#4' is not defined}\@eha
    \fi
  \else
    \@latex@error{Symbol font `#2' is not defined}\@eha
  \fi
}
\@onlypreamble\@xDeclareMathDelimiter
\def\set@mathdelimiter#1#2#3#4#5#6{%
  \xdef#3{\delimiter"\mathchar@type#4\hexnumber@#1#5%
                                     \hexnumber@#2#6 }}
\@onlypreamble\set@mathdelimiter
\def\set@@mathdelimiter#1#2#3#4#5{%
  \global\delcode`#3="\hexnumber@#1#4\hexnumber@#2#5\relax}
\@onlypreamble\set@@mathdelimiter
\def\DeclareMathRadical#1#2#3#4#5{%
  \expandafter\ifx
       \csname\expandafter\@gobble\string#1\endcsname
       \relax
     \let#1\radical
  \fi
  \edef\reserved@a{\noexpand\in@{\string\radical}{\meaning#1}}%
  \reserved@a
  \ifin@
    \expandafter\in@\csname sym#2\expandafter\endcsname
       \expandafter{\group@list}%
    \ifin@
      \expandafter\in@\csname sym#4\expandafter\endcsname
         \expandafter{\group@list}%
      \ifin@
        \begingroup
          \count\z@=#3\relax
          \count\tw@\count\z@
          \divide\count\z@\sixt@@n
          \count@\count\z@
          \multiply\count@\sixt@@n
          \advance\count\tw@-\count@
          \edef\reserved@c{%
            \hexnumber@{\count\z@}\hexnumber@{\count\tw@}}%
          \count\z@=#5\relax
          \count\tw@\count\z@
          \divide\count\z@\sixt@@n
          \count@\count\z@
          \multiply\count@\sixt@@n
          \advance\count\tw@-\count@
          \edef\reserved@d{%
            \hexnumber@{\count\z@}\hexnumber@{\count\tw@}}%
          \xdef#1{\radical"\expandafter\hexnumber@
                                \csname sym#2\endcsname\reserved@c
                             \expandafter\hexnumber@
                                \csname sym#4\endcsname\reserved@d
                  \relax}%
        \endgroup
      \else
        \@latex@error{Symbol font `#4' is not defined}\@eha
      \fi
    \else
      \@latex@error{Symbol font `#2' is not defined}\@eha
    \fi
  \else
    \@latex@error{Command `\string#1' already defined}\@eha
  \fi
}
\@onlypreamble\DeclareMathRadical
\let\mathalpha\relax
\def\mathchar@type#1{%
  \ifodd 2#11 #1\else             % is this non-negative number?
    \ifx#1\mathord 0\else
     \ifx#1\mathop   1\else
       \ifx#1\mathbin 2\else
         \ifx#1\mathrel 3\else
           \ifx#1\mathopen 4\else
             \ifx#1\mathclose 5\else
               \ifx#1\mathpunct 6\else
                   7%             % anything else is variable ord
               \fi
             \fi
           \fi
         \fi
       \fi
     \fi
    \fi
  \fi}
\@onlypreamble\mathchar@type
\def\DeclareSymbolFontAlphabet#1#2{%
   \expandafter\DeclareSymbolFontAlphabet@
     \csname @\expandafter\@gobble\string#1\endcsname{#2}#1}
\@onlypreamble\DeclareSymbolFontAlphabet
\def\DeclareSymbolFontAlphabet@#1#2#3{%
    \@tempswatrue
  \expandafter\in@\csname sym#2\expandafter\endcsname
     \expandafter{\group@list}%
  \ifin@
    \expandafter\in@\expandafter#1\expandafter{\alpha@list}%
    \ifin@
      \@font@info{Redeclaring math alphabet \string#3}%
      \toks@{}%
      \def\alpha@elt##1##2##3{%
          \ifx##1#1\else\addto@hook\toks@{\alpha@elt##1##2##3}\fi}%
      \alpha@list
      \xdef\alpha@list{\the\toks@}%
      \def\version@elt##1{%
          \begingroup
            \toks@{}%
            \def\getanddefine@fonts####1####2{%
               \addto@hook\toks@{\getanddefine@fonts####1####2}}%
            \def\install@mathalphabet####1####2{%
               \ifx####1#1\else
                 \addto@hook\toks@{\install@mathalphabet
                                    ####1{####2}}\fi}%
            ##1%
            \xdef##1{\the\toks@}%
          \endgroup
          }%
      \version@list
    \else
      \expandafter\ifx
      \csname\expandafter\@gobble\string#1\endcsname
      \relax
      \else
        \edef\reserved@a{%
          \noexpand\in@{\string\use@mathgroup}{\meaning#1}}%
        \reserved@a
        \ifin@
          \@font@info{Redeclaring math alphabet \string#3}%
        \else
          \@tempswafalse
          \@latex@error{Command `\string#3' already defined}\@eha
        \fi
      \fi
    \fi
   \else
     \@tempswafalse
     \@latex@error{Unknown symbol font `#2'}\@eha
   \fi
   \if@tempswa
     \def\group@elt##1##2{%
        \expandafter\ifx\csname sym#2\endcsname##1%
        \expandafter\reserved@a\string##2\@nil
        \fi}%
     \def\reserved@a##1##2/##3\@nil{%
        \def\reserved@a{##2}}%
     \group@list
     \toks@{\relax\ifmmode \else \non@alpherr#1\fi}%
     \edef#1{\the\toks@
             \noexpand\use@mathgroup
             \expandafter\noexpand\csname M@\reserved@a\endcsname
             \csname sym#2\endcsname}%
     \def#3{\protect#1}%
   \fi
}
\@onlypreamble\DeclareSymbolFontAlphabet@

\DeclareMathVersion{normal}
\DeclareMathVersion{bold}
\DeclareRobustCommand\rmfamily
        {\not@math@alphabet\rmfamily\mathrm
         \fontfamily\rmdefault\selectfont}
\DeclareRobustCommand\sffamily
        {\not@math@alphabet\sffamily\mathsf
         \fontfamily\sfdefault\selectfont}
\DeclareRobustCommand\ttfamily
        {\not@math@alphabet\ttfamily\mathtt
         \fontfamily\ttdefault\selectfont}
\DeclareRobustCommand\bfseries
        {\not@math@alphabet\bfseries\mathbf
         \fontseries\bfdefault\selectfont}
\DeclareRobustCommand\mdseries
        {\not@math@alphabet\mdseries\relax
         \fontseries\mddefault\selectfont}
\DeclareRobustCommand\upshape
        {\not@math@alphabet\upshape\relax
         \fontshape\updefault\selectfont}
\DeclareRobustCommand\slshape
        {\not@math@alphabet\slshape\relax
         \fontshape\sldefault\selectfont}
\DeclareRobustCommand\scshape
        {\not@math@alphabet\scshape\relax
         \fontshape\scdefault\selectfont}
\DeclareRobustCommand\itshape
        {\not@math@alphabet\itshape\mathit
         \fontshape\itdefault\selectfont}
\DeclareRobustCommand\em
        {\@nomath\em \ifdim \fontdimen\@ne\font >\z@
                       \upshape \else \itshape \fi}
\def\not@math@alphabet#1#2{%
   \relax
   \ifmmode
     \@latex@error{Command \noexpand#1invalid in math mode}%
        {%
         Please
         \ifx#2\relax
            define a new math alphabet^^J%
            if you want to use a special font in math mode%
          \else
            use the math alphabet \noexpand#2instead of
            the #1command%
         \fi
         .
        }%
   \fi}
\def\boldmath{\@nomath\boldmath
              \mathversion{bold}}
\def\unboldmath{\@nomath\unboldmath
              \mathversion{normal}}
\def\math@version{normal}
\def\newfont#1#2{\@ifdefinable#1{\font#1=#2\relax}}
\def\symbol#1{\char #1\relax}
\def\@setfontsize#1#2#3{\@nomath#1%
    \ifx\protect\@typeset@protect
      \let\@currsize#1%
    \fi
    \fontsize{#2}{#3}\selectfont}
\def\@setsize#1#2#3#4{\@setfontsize#1{#4}{#2}}
\def\oldstylenums#1{%
   \begingroup
    \spaceskip\fontdimen\tw@\font
    \usefont{OML}{\rmdefault}{\f@series}{it}%
    \mathgroup\symletters #1%
   \endgroup
}
\def\hexnumber@#1{\ifcase\number#1
 0\or 1\or 2\or 3\or 4\or 5\or 6\or 7\or 8\or
 9\or A\or B\or C\or D\or E\or F\fi}
\def\nfss@text#1{{\mbox{#1}}}
\DeclareRobustCommand\normalfont
                 {\usefont\encodingdefault
                          \familydefault
                          \seriesdefault
                          \shapedefault
                  \relax}
\let\reset@font\normalfont

\let\@addtofilelist\@gobble
\let\@addtofilelist\@gobble
\let\@acci\' \let\@accii\` \let\@acciii\=

\def \DeclareTextFontCommand #1#2{%
  \DeclareRobustCommand#1[1]{%
    \ifmmode
      \nfss@text{#2##1}%
    \else
      \hmode@bgroup
       \text@command{##1}%
       #2\check@icl ##1\check@icr
       \expandafter
      \egroup
    \fi
                       }%
}
\let \nocorr \relax
\let \check@icl \@empty
\let \check@icr \@empty
\def \text@command #1{%
  \def \reserved@a {#1}%
  \ifx \reserved@a \@empty
    \let \check@icl \@empty
    \let \check@icr \@empty
  \else
    \ifx \reserved@a \space
      \let \check@icl \@empty
      \let \check@icr \@empty
    \else
      \check@nocorr@ #1\nocorr\@nil
    \fi
  \fi
}
\def \check@nocorr@ #1#2\nocorr#3\@nil {%
  \let \check@icl \maybe@ic
  \def \check@icr {\ifvmode \else \aftergroup \maybe@ic \fi}%
  \def \reserved@a {\nocorr}%
  \def \reserved@b {#1}%
  \def \reserved@c {#3}%
  \ifx \reserved@a \reserved@b

    \ifx \reserved@c \@empty
      \let \check@icl \@empty
    \else
      \let \check@icl \@empty
      \let \check@icr \@empty
    \fi
  \else
    \ifx \reserved@c \@empty
    \else
      \let \check@icr \@empty
    \fi
  \fi
}
\def \maybe@ic {\futurelet\@let@token\maybe@ic@}
\def \maybe@ic@ {%
  \ifdim \fontdimen\@ne\font>\z@
  \else
    \@tempswatrue
    \expandafter\@tfor\expandafter\reserved@a\expandafter:\expandafter=%
        \nocorrlist
    \do \t@st@ic
    \if@tempswa \sw@slant \fi
  \fi
}
\def \t@st@ic {%
  \expandafter\let\expandafter\reserved@b\expandafter=\reserved@a\relax
  \ifx\reserved@b\@let@token
    \@tempswafalse
    \@break@tfor
  \fi
}
\def \sw@slant {%
  \ifdim \lastskip=\z@
    \fix@penalty
  \else
    \skip@ \lastskip
    \unskip
    \fix@penalty
    \hskip \skip@
  \fi
}
\def \fix@penalty {%
  \ifnum \lastpenalty=\z@
    \@@italiccorr
  \else
    \count@ \lastpenalty
    \unpenalty
    \@@italiccorr
    \penalty \count@
  \fi
}
\def \nocorrlist {,.}
\ifx \nfss@text\@undefined
  \def \nfss@text {\leavevmode\hbox}
\fi
\def \DeclareOldFontCommand #1#2#3{%
  \DeclareRobustCommand #1{\@fontswitch {#2}{#3}}%
}
\def \@fontswitch #1#2{%
  \ifmmode
     \let \math@bgroup \relax
     \def \math@egroup {\let \math@bgroup \@@math@bgroup
                        \let \math@egroup \@@math@egroup}%
     #2\relax
  \else
     #1%
  \fi
}

\message{x-ref,}
\def\G@refundefinedtrue{%
  \gdef\@refundefined{%
    \@latex@warning@no@line{There were undefined references}}}
\let\@refundefined\relax
\def\@setref#1#2#3{%
  \ifx#1\relax
   \protect\G@refundefinedtrue
   \nfss@text{\reset@font\bfseries ??}%
   \@latex@warning{Reference `#3' on page \thepage \space
             undefined}%
  \else
   \expandafter#2#1\null
  \fi}
\def\ref#1{\expandafter\@setref\csname r@#1\endcsname\@firstoftwo{#1}}
\def\pageref#1{\expandafter\@setref\csname r@#1\endcsname
                                   \@secondoftwo{#1}}
\def\@newl@bel#1#2#3{{%
  \@ifundefined{#1@#2}%
    \relax
    {\gdef \@multiplelabels {%
       \@latex@warning@no@line{There were multiply-defined labels}}%
     \@latex@warning@no@line{Label `#2' multiply defined}}%
  \global\@namedef{#1@#2}{#3}}}
\def\newlabel{\@newl@bel r}
\@onlypreamble\@newl@bel
\let \@multiplelabels \relax
\def\@currentlabel{}



\message{environments,}
\def\@currenvir{document}
\def\@ignorefalse{\global\let\if@ignore\iffalse}
\def\@ignoretrue {\global\let\if@ignore\iftrue}
\@ignorefalse
\let\ignorespacesafterend\@ignoretrue
\def\enddocument{%
   \let\AtEndDocument\@firstofone
   \@enddocumenthook
   \@checkend{document}%
   \clearpage
   \begingroup
     \if@filesw
       \immediate\closeout\@mainaux
       \let\@setckpt\@gobbletwo
       \let\@newl@bel\@testdef
       \@tempswafalse
       \makeatletter \input\jobname.aux
     \fi
     \@dofilelist
     \ifdim \font@submax >\fontsubfuzz\relax
       \@font@warning{Size substitutions with differences\MessageBreak
                  up to \font@submax\space have occurred.\@gobbletwo}%
     \fi
     \@defaultsubs
     \@refundefined
     \if@filesw
       \ifx \@multiplelabels \relax
         \if@tempswa
           \@latex@warning@no@line{Label(s) may have changed.
               Rerun to get cross-references right}%
         \fi
       \else
         \@multiplelabels
       \fi
     \fi
   \endgroup
   \deadcycles\z@\@@end}
\def\@testdef #1#2#3{%
  \def\reserved@a{#3}\expandafter \ifx \csname #1@#2\endcsname
 \reserved@a  \else \@tempswatrue \fi}
\long\def\@writefile#1#2{%
  \@ifundefined{tf@#1}\relax
    {\@temptokena{#2}%
     \immediate\write\csname tf@#1\endcsname{\the\@temptokena}%
    }%
}
\def\stop{\clearpage\deadcycles\z@\let\par\@@par\@@end}

%% Verbatim

\def\@make@braces@other{%
  \catcode`\{=12%
  \catcode`\}=12%
  \relax
}
\def\@make@braces@normal{%
  \catcode`\{=1%
  \catcode`\}=2%
  \relax
}


\bgroup
  \@make@braces@other
  \catcode `|=0
  \catcode`\[=1
  \catcode`\]=2
  \catcode`\\=12
  |gdef|@scan@verbatim#1\end{verbatim}[|out@verbatim[#1]%
     |end[verbatim]%
     |@make@braces@normal%
  ] 
|egroup

\bgroup
   \catcode`\&=\active
   \catcode`\<=\active
   \catcode`\>=\active
   \gdef&{\msitag{<amp/>}}
   \gdef<{\msitag{<langle/>}}
   \gdef>{\msitag{<rangle/>}}
\egroup


\def\msi@begin@verbatim{%
   \@make@braces@other
   \def\par{\char10}
   \obeylines
   \obeyspaces
   \catcode`\\=12
   \catcode`\%=12
   \catcode`\#=12
   \catcode`\$=12
   \catcode`\^=12
   \catcode`\_=12
   \catcode`\&=\active
   \catcode`\<=\active
   \catcode`\>=\active
   \relax\@scan@verbatim}
   
\def\msi@end@verbatim{}


\@namedef{verbatim*}{\@verbatim\@sxverbatim}
\expandafter\let\csname endverbatim*\endcsname =\endverbatim

\let\verb@balance@group\@empty
\def\verb@egroup{\global\let\verb@balance@group\@empty\egroup}
\begingroup
  \obeylines%
  \gdef\verb@eol@error{\obeylines%
    \def^^M{\verb@egroup\@latex@error{%
            \noexpand\verb ended by end of line}\@ehc}}%
\endgroup
\def\verb{\relax\ifmmode\hbox\else\leavevmode\null\fi
  \bgroup
    \verb@eol@error \let\do\@makeother \dospecials
    \verbatim@font\@noligs
    \@ifstar\@sverb\@verb}
\def\@sverb#1{%
  \catcode`#1\active
  \lccode`\~`#1%
  \gdef\verb@balance@group{\verb@egroup
     \@latex@error{\noexpand\verb illegal in command argument}\@ehc}%
  \aftergroup\verb@balance@group
  \lowercase{\let~\verb@egroup}}%
\def\@verb{\@vobeyspaces \frenchspacing \@sverb}
\def\verbatim@nolig@list{\do\`\do\<\do\>\do\,\do\'\do\-}
\def\do@noligs#1{%
  \catcode`#1\active
  \begingroup
     \lccode`\~`#1\relax
     \lowercase{\endgroup\def~{\leavevmode\kern\z@\char`#1}}}
\def\@noligs{\let\do\do@noligs \verbatim@nolig@list}




\def\pmod#1{\allowbreak\,({\rm mod})\,\,#1)}
\def\bmod{\mod}

\def\bigl{\mathopen\big}
\def\bigm{\mathrel\big}
\def\bigr{\mathclose\big}
\def\Bigl{\mathopen\Big}
\def\Bigm{\mathrel\Big}
\def\Bigr{\mathclose\Big}
\def\biggl{\mathopen\bigg}
\def\biggm{\mathrel\bigg}
\def\biggr{\mathclose\bigg}
\def\Biggl{\mathopen\Bigg}
\def\Biggm{\mathrel\Bigg}
\def\Biggr{\mathclose\Bigg}
\newdimen\jot
\jot=3pt
\newcount\interdisplaylinepenalty
\interdisplaylinepenalty=100
\def\choose{\atopwithdelims()}
\def\brack{\atopwithdelims[]}
\def\brace{\atopwithdelims\{\}}
\def\mathpalette#1#2{%
  \mathchoice
    {#1\displaystyle{#2}}%
    {#1\textstyle{#2}}%
    {#1\scriptstyle{#2}}%
    {#1\scriptscriptstyle{#2}}}
\newbox\rootbox
\def\root#1\of{%
  \setbox\rootbox\hbox{$\m@th\scriptscriptstyle{#1}$}%
  \mathpalette\r@@t}
\def\r@@t#1#2{%
  \setbox\z@\hbox{$\m@th#1\sqrtsign{#2}$}%
  \dimen@\ht\z@ \advance\dimen@-\dp\z@
  \mkern5mu\raise.6\dimen@\copy\rootbox
  \mkern-10mu\box\z@}

\def\buildrel#1\over#2{\mathrel{\mathop{\kern\z@#2}\limits^{#1}}}
\def\cases#1{\left\{\,\vcenter{\normalbaselines\m@th
    \ialign{$##\hfil$&\quad{##}\hfil\crcr#1\crcr}}\right.}

\def\bordermatrix#1{\begingroup \m@th
  \@tempdima 8.75\p@
  \setbox\z@\vbox{%
    \def\cr{\crcr\noalign{\kern2\p@\global\let\cr\endline}}%
    \ialign{$##$\hfil\kern2\p@\kern\@tempdima&\thinspace\hfil$##$\hfil
      &&\quad\hfil$##$\hfil\crcr
      \omit\strut\hfil\crcr\noalign{\kern-\baselineskip}%
      #1\crcr\omit\strut\cr}}%
  \setbox\tw@\vbox{\unvcopy\z@\global\setbox\@ne\lastbox}%
  \setbox\tw@\hbox{\unhbox\@ne\unskip\global\setbox\@ne\lastbox}%
  \setbox\tw@\hbox{$\kern\wd\@ne\kern-\@tempdima\left(\kern-\wd\@ne
    \global\setbox\@ne\vbox{\box\@ne\kern2\p@}%
    \vcenter{\kern-\ht\@ne\unvbox\z@\kern-\baselineskip}\,\right)$}%
  \null\;\vbox{\kern\ht\@ne\box\tw@}\endgroup}
\def\openup{\afterassignment\@penup\dimen@}
\def\@penup{\advance\lineskip\dimen@
  \advance\baselineskip\dimen@
  \advance\lineskiplimit\dimen@}
\newif\ifdt@p
\def\displ@y{\global\dt@ptrue\openup\jot\m@th
  \everycr{\noalign{\ifdt@p \global\dt@pfalse \ifdim\prevdepth>-1000\p@
      \vskip-\lineskiplimit \vskip\normallineskiplimit \fi
      \else \penalty\interdisplaylinepenalty \fi}}}
\def\@lign{\tabskip\z@skip\everycr{}} % restore inside \displ@y
\def\displaylines#1{\displ@y \tabskip\z@skip
  \halign{\hb@xt@\displaywidth{$\@lign\hfil\displaystyle##\hfil$}\crcr
    #1\crcr}}
\let\sp=^
\let\sb=_
\def\>{\mskip\medmuskip}
\def\;{\mskip\thickmuskip}
\def\!{\mskip-\thinmuskip}
\def\*{\discretionary{\thinspace\the\textfont2\char2}{}{}}
\let\:=\>
\def\active@math@prime{^\bgroup\prim@s}
{\catcode`\'=\active \global\let'\active@math@prime}
\def\prim@s{%
  \prime\futurelet\@let@token\pr@m@s}
\def\pr@m@s{%
  \ifx'\@let@token
    \expandafter\pr@@@s
  \else
    \ifx^\@let@token
      \expandafter\expandafter\expandafter\pr@@@t
    \else
      \egroup
    \fi
  \fi}
\def\pr@@@s#1{\prim@s}
\def\pr@@@t#1#2{#2\egroup}
{\catcode`\_=\active \gdef_{\_}} % _ in math is
                                 % either subscript or \_
\def\({$}
\def\){$}
\def\[{$$}
\def\]{$$}

\let\math=\(
\let\endmath=\)
\def\displaymath{\[}
\def\enddisplaymath{\]\@ignoretrue}
\@definecounter{equation}
%\def\equation{$$\refstepcounter{equation}}
%\def\endequation{\eqno \hbox{\@eqnnum}$$\@ignoretrue}
\def\@eqnnum{{\normalfont \normalcolor (\theequation)}}

\def\frac#1#2{{\begingroup#1\endgroup\over#2}}
\DeclareRobustCommand\sqrt{\@ifnextchar[\@sqrt\sqrtsign}
\def\@sqrt[#1]{\root #1\of}
\newcount\@eqcnt
\newcount\@eqpen
\newif\if@eqnsw\@eqnswtrue
\newskip\@centering
\@centering = 0pt plus 1000pt


\let\@eqnsel=\relax
%%\def\nonumber{\global\@eqnswfalse}
\msi@pass{\nonumber}
\def\@eqncr{%
   {\ifnum0=`}\fi
   \@ifstar{%
      \global\@eqpen\@M\@yeqncr
   }{%
      \global\@eqpen\interdisplaylinepenalty \@yeqncr
   }%
}
\def\@yeqncr{\@testopt\@xeqncr\z@skip}
\def\@xeqncr[#1]{%
   \ifnum0=`{\fi}%
   \@@eqncr
   \noalign{\penalty\@eqpen\vskip\jot\vskip #1\relax}%
}
\def\@@eqncr{\let\reserved@a\relax
    \ifcase\@eqcnt \def\reserved@a{& & &}\or \def\reserved@a{& &}%
     \or \def\reserved@a{&}\else
       \let\reserved@a\@empty
       \@latex@error{Too many columns in eqnarray environment}\@ehc\fi
     \reserved@a \if@eqnsw\@eqnnum\stepcounter{equation}\fi
     \global\@eqnswtrue\global\@eqcnt\z@\cr}
\let\@seqncr=\@eqncr
\@namedef{eqnarray*}{\def\@eqncr{\nonumber\@seqncr}\eqnarray}
\@namedef{endeqnarray*}{\nonumber\endeqnarray}

\DeclareRobustCommand{\ensuremath}{%
  \ifmmode
    \expandafter\@firstofone
  \else
    \expandafter\@ensuredmath
  \fi}
\long\def\@ensuredmath#1{$\relax#1$}
%%% From File: ltlists.dtx
\newskip\topsep
\newskip\partopsep
\newskip\itemsep
\newskip\parsep
\newskip\@topsep
\newskip\@topsepadd
\newskip\@outerparskip
\newdimen\leftmargin
\newdimen\rightmargin
\newdimen\listparindent
\newdimen\itemindent
\newdimen\labelwidth
\newdimen\labelsep
\newdimen\linewidth
\newdimen\@totalleftmargin \@totalleftmargin=\z@
\newdimen\leftmargini
\newdimen\leftmarginii
\newdimen\leftmarginiii
\newdimen\leftmarginiv
\newdimen\leftmarginv
\newdimen\leftmarginvi
\newcount\@listdepth \@listdepth=0
\newcount\@itempenalty
\newcount\@beginparpenalty
\newcount\@endparpenalty
\newbox\@labels
\newif\if@inlabel \@inlabelfalse
\newif\if@newlist   \@newlistfalse
\newif\if@noparitem \@noparitemfalse
\newif\if@noparlist \@noparlistfalse
\newif\if@noitemarg \@noitemargfalse
\newif\if@nmbrlist  \@nmbrlistfalse
\def\list#1#2{%
  \ifnum \@listdepth >5\relax
    \@toodeep
  \else
    \global\advance\@listdepth\@ne
  \fi
  \rightmargin\z@
  \listparindent\z@
  \itemindent\z@
  \csname @list\romannumeral\the\@listdepth\endcsname
  \def\@itemlabel{#1}%
  \let\makelabel\@mklab
  \@nmbrlistfalse
  #2\relax
  \@trivlist
  \parskip\parsep
  \parindent\listparindent
  \advance\linewidth -\rightmargin
  \advance\linewidth -\leftmargin
  \advance\@totalleftmargin \leftmargin
  \parshape \@ne \@totalleftmargin \linewidth
  \ignorespaces}
\newcount\par@deathcycles
\def\@trivlist{%
  \if@noskipsec \leavevmode \fi
  \@topsepadd \topsep
  \ifvmode
    \advance\@topsepadd \partopsep
  \else
    \unskip \par
  \fi
  \if@inlabel
    \@noparitemtrue
    \@noparlisttrue
  \else
    \if@newlist \@noitemerr \fi
    \@noparlistfalse
    \@topsep \@topsepadd
  \fi
  \advance\@topsep \parskip
  \leftskip \z@skip
  \rightskip \@rightskip
  \parfillskip \@flushglue
  \par@deathcycles \z@
  \@setpar{\if@newlist
             \advance\par@deathcycles \@ne
             \ifnum \par@deathcycles >\@m
               \@noitemerr
               {\@@par}%
             \fi
           \else
             {\@@par}%
           \fi}%
  \global \@newlisttrue
  \@outerparskip \parskip}
\def\trivlist{%
  \parsep\parskip
  \@nmbrlistfalse
  \@trivlist
  \labelwidth\z@
  \leftmargin\z@
  \itemindent\z@
  \let\@itemlabel\@empty
  \def\makelabel##1{##1}}
\def\endlist{%
  \global\advance\@listdepth\m@ne
  \endtrivlist}
\def\endtrivlist{%
  \if@inlabel
    \leavevmode
    \global \@inlabelfalse
  \fi
  \if@newlist
    \@noitemerr
    \global \@newlistfalse
  \fi
  \ifhmode\unskip \par\fi
  \if@noparlist \else
    \ifdim\lastskip >\z@
      \@tempskipa\lastskip \vskip -\lastskip
      \advance\@tempskipa\parskip \advance\@tempskipa -\@outerparskip
      \vskip\@tempskipa
    \fi
    \@endparenv
  \fi
}
\def\@endparenv{%
  \addpenalty\@endparpenalty\addvspace\@topsepadd\@endpetrue}

\def\@doendpe{\@endpetrue
     \def\par{\@restorepar\everypar{}\par\@endpefalse}\everypar
               {{\setbox\z@\lastbox}\everypar{}\@endpefalse}}

\newif\if@endpe
\@endpefalse
\def\@mklab#1{\hfil #1}

\def\item{%
   \@ifnextchar[{\@itema}{\@itemb}%
}
\def\@itema[#1]{%
   \out@itema[{#1}]%
}
\def\@itemb{%
   \out@itemb
}

\def\@donoparitem{%
  \@noparitemfalse
  \global\setbox\@labels\hbox{\hskip -\leftmargin
                               \unhbox\@labels
                                \hskip \leftmargin}%
  \if@minipage\else
    \@tempskipa\lastskip
    \vskip -\lastskip
    \advance\@tempskipa\@outerparskip
    \advance\@tempskipa -\parskip
    \vskip\@tempskipa
  \fi}
\def\@item[#1]{%
  \if@noparitem
    \@donoparitem
  \else
    \if@inlabel
      \indent \par
    \fi
    \ifhmode
      \unskip\unskip \par
    \fi
    \if@newlist
      \if@nobreak
        \@nbitem
      \else
        \addpenalty\@beginparpenalty
        \addvspace\@topsep
        \addvspace{-\parskip}%
      \fi
    \else
      \addpenalty\@itempenalty
      \addvspace\itemsep
    \fi
    \global\@inlabeltrue
  \fi
  \everypar{%
    \@minipagefalse
    \global\@newlistfalse
    \if@inlabel
      \global\@inlabelfalse
      {\setbox\z@\lastbox
       \ifvoid\z@
         \kern-\itemindent
       \fi}%
      \box\@labels
      \penalty\z@
    \fi
    \if@nobreak
      \@nobreakfalse
      \clubpenalty \@M
    \else
      \clubpenalty \@clubpenalty
      \everypar{}%
    \fi}%
  \if@noitemarg
    \@noitemargfalse
    \if@nmbrlist
      \refstepcounter\@listctr
    \fi
  \fi
  \sbox\@tempboxa{\makelabel{#1}}%
  \global\setbox\@labels\hbox{%
    \unhbox\@labels
    \hskip \itemindent
    \hskip -\labelwidth
    \hskip -\labelsep
    \ifdim \wd\@tempboxa >\labelwidth
      \box\@tempboxa
    \else
      \hbox to\labelwidth {\unhbox\@tempboxa}%
    \fi
    \hskip \labelsep}%
  \ignorespaces}
\def\makelabel#1{%
  \@latex@error{Lonely \string\item--perhaps a missing
        list environment}\@ehc}
\def\@nbitem{%
  \@tempskipa\@outerparskip
  \advance\@tempskipa -\parskip
  \addvspace\@tempskipa}
\def\usecounter#1{\@nmbrlisttrue\def\@listctr{#1}\setcounter{#1}\z@}
\newcount\@enumdepth \@enumdepth = 0
\@definecounter{enumi}
\@definecounter{enumii}
\@definecounter{enumiii}
\@definecounter{enumiv}
%%jcs \def\enumerate{%
%%jcs   \ifnum \@enumdepth >\thr@@\@toodeep\else
%%jcs     \advance\@enumdepth\@ne
%%jcs     \edef\@enumctr{enum\romannumeral\the\@enumdepth}%

%%jcs       \expandafter
%%jcs       \list
%%jcs         \csname label\@enumctr\endcsname
%%jcs         {\usecounter\@enumctr\def\makelabel##1{\hss\llap{##1}}}%
%%jcs   \fi}
%%jcs \let\endenumerate =\endlist
%%jcs \newcount\@itemdepth \@itemdepth = 0
%%jcs \def\itemize{%
%%jcs   \ifnum \@itemdepth >\thr@@\@toodeep\else
%%jcs     \advance\@itemdepth\@ne
%%jcs     \edef\@itemitem{labelitem\romannumeral\the\@itemdepth}%
%%jcs     \expandafter
%%jcs     \list
%%jcs       \csname\@itemitem\endcsname
%%jcs       {\def\makelabel##1{\hss\llap{##1}}}%
%%jcs   \fi}
%%jcs \let\enditemize =\endlist


%%% From File: ltboxes.dtx
\message{boxes,}
\def\makebox{%
  \leavevmode
  \@ifnextchar(%)
    \@makepicbox
    {\@ifnextchar[\@makebox\mbox}}

\long\def\mbox#1{%
  \ifmmode
     \msitag{\mbox{#1}}%
  \else 
     \leavevmode\hbox{#1}%
  \fi
}

\def\@makebox[#1]{%
  \@ifnextchar [{\@imakebox[#1]}{\@imakebox[#1][c]}}
\long\def\@begin@tempboxa#1#2{%
   \begingroup
     \setbox\@tempboxa#1{\color@begingroup#2\color@endgroup}%
     \def\width{\wd\@tempboxa}%
     \def\height{\ht\@tempboxa}%
     \def\depth{\dp\@tempboxa}%
     \let\totalheight\@ovri
     \totalheight\height
     \advance\totalheight\depth}
\let\@end@tempboxa\endgroup
\def\bm@c{\hss\unhbox\@tempboxa\hss}
\def\bm@l{\unhbox\@tempboxa\hss}\let\bm@t\bm@l
\def\bm@r{\hss\unhbox\@tempboxa}\let\bm@b\bm@r
\def\bm@s{\unhbox\@tempboxa}
\long\def\@imakebox[#1][#2]#3{%
  \@begin@tempboxa\hbox{#3}%
    \setlength\@tempdima{#1}%       support calc
    \hb@xt@\@tempdima{\csname bm@#2\endcsname}%
  \@end@tempboxa}
\def\@makepicbox(#1,#2){%
  \@ifnextchar[{\@imakepicbox(#1,#2)}{\@imakepicbox(#1,#2)[]}}
\long\def\@imakepicbox(#1,#2)[#3]#4{%
  \vbox to#2\unitlength
   {\let\mb@b\vss \let\mb@l\hss\let\mb@r\hss
    \let\mb@t\vss
    \@tfor\reserved@a :=#3\do{%
      \if s\reserved@a
        \let\mb@l\relax\let\mb@r\relax
      \else
        \expandafter\let\csname mb@\reserved@a\endcsname\relax
      \fi}%
    \mb@t
    \hb@xt@ #1\unitlength{\mb@l #4\mb@r}%
    \mb@b
    \kern\z@}}
\let\set@color\relax
\let\color@begingroup\relax
\let\color@endgroup\relax
\let\color@setgroup\relax
\let\normalcolor\relax
\let\color@hbox\relax
\let\color@vbox\relax
\let\color@endbox\relax

\def\@savebox#1[#2]{%
  \@ifnextchar [{\@isavebox#1[#2]}{\@isavebox#1[#2][c]}}
\long\def\@isavebox#1[#2][#3]#4{%
  \sbox#1{\@imakebox[#2][#3]{#4}}}
\def\@savepicbox#1(#2,#3){%
  \@ifnextchar[%]
    {\@isavepicbox#1(#2,#3)}{\@isavepicbox#1(#2,#3)[]}}
\long\def\@isavepicbox#1(#2,#3)[#4]#5{%
  \sbox#1{\@imakepicbox(#2,#3)[#4]{#5}}}
\def\lrbox#1{%
  \edef\reserved@a{%
    \endgroup
    \setbox#1\hbox{%
      \begingroup\aftergroup}%
        \def\noexpand\@currenvir{\@currenvir}%
        \def\noexpand\@currenvline{\on@line}}%
  \reserved@a
    \@endpefalse
    \color@setgroup
      \ignorespaces}
\def\endlrbox{\unskip\color@endgroup}
\def\usebox#1{\leavevmode\copy #1\relax}
\long\def\frame#1{%
  \leavevmode
  \hbox{%
    \hskip-\@wholewidth
    \vbox{%
      \vskip-\@wholewidth
      \hrule \@height\@wholewidth
      \hbox{%
        \vrule\@width\@wholewidth
        #1%
        \vrule\@width\@wholewidth}%
      \hrule\@height\@wholewidth
      \vskip-\@wholewidth}%
    \hskip-\@wholewidth}}
\newdimen\fboxrule
\newdimen\fboxsep

\def\@framebox[#1]{%
  \@ifnextchar[%]
    {\@iframebox[#1]}%
    {\@iframebox[#1][c]}}
\long\def\@iframebox[#1][#2]#3{%
  \leavevmode
  \@begin@tempboxa\hbox{#3}%
    \setlength\@tempdima{#1}%
    \setbox\@tempboxa\hb@xt@\@tempdima
         {\kern\fboxsep\csname bm@#2\endcsname\kern\fboxsep}%
    \@frameb@x{\kern-\fboxrule}%
  \@end@tempboxa}
\def\@frameb@x#1{%
  \@tempdima\fboxrule
  \advance\@tempdima\fboxsep
  \advance\@tempdima\dp\@tempboxa
  \hbox{%
    \lower\@tempdima\hbox{%
      \vbox{%
        \hrule\@height\fboxrule
        \hbox{%
          \vrule\@width\fboxrule
          #1%
          \vbox{%
            \vskip\fboxsep
            \box\@tempboxa
            \vskip\fboxsep}%
          #1%
          \vrule\@width\fboxrule}%
        \hrule\@height\fboxrule}%
                          }%
        }%
}
\def\@framepicbox(#1,#2){%
  \@ifnextchar[{\@iframepicbox(#1,#2)}{\@iframepicbox(#1,#2)[]}}
\long\def\@iframepicbox(#1,#2)[#3]#4{%
  \frame{\@imakepicbox(#1,#2)[#3]{#4}}}
\def\@arrayparboxrestore{%
  \let\if@nobreak\iffalse
  \let\if@noskipsec\iffalse
  \let\par\@@par
  \let\-\@dischyph
  \let\'\@acci\let\`\@accii\let\=\@acciii
  \parindent\z@ \parskip\z@skip
  \everypar{}%
  \linewidth\hsize
  \@totalleftmargin\z@
  \leftskip\z@skip \rightskip\z@skip \@rightskip\z@skip
  \parfillskip\@flushglue \lineskip\normallineskip
  \baselineskip\normalbaselineskip
  \sloppy}
\def\@parboxrestore{\@arrayparboxrestore\let\\\@normalcr}
\def\@minipagefalse{\global\let\if@minipage\iffalse}
\def\@minipagetrue {\global\let\if@minipage\iftrue}
\@minipagefalse
\def\minipage{%
  \@ifnextchar[%]
    \@iminipage
    {\@iiiminipage c\relax[s]}}
\def\@iminipage[#1]{%
  \@ifnextchar[%]
    {\@iiminipage{#1}}%
    {\@iiiminipage{#1}\relax[s]}}
\def\@iiminipage#1[#2]{%
  \@ifnextchar[%]
    {\@iiiminipage{#1}{#2}}%
    {\@iiiminipage{#1}{#2}[#1]}}
\def\@iiiminipage#1#2[#3]#4{%
  \leavevmode
  \@pboxswfalse
  \setlength\@tempdima{#4}%
  \def\@mpargs{{#1}{#2}[#3]{#4}}%
  \setbox\@tempboxa\vbox\bgroup
    \color@begingroup
      \hsize\@tempdima
      \textwidth\hsize \columnwidth\hsize
      \@parboxrestore
      \def\@mpfn{mpfootnote}\def\thempfn{\thempfootnote}\c@mpfootnote\z@
      \let\@footnotetext\@mpfootnotetext
      \let\@listdepth\@mplistdepth \@mplistdepth\z@
      \@minipagerestore
      \@setminipage}
 \let\@minipagerestore=\relax
\def\endminipage{%
    \par
    \unskip
    \ifvoid\@mpfootins\else
      \vskip\skip\@mpfootins
      \normalcolor
      \footnoterule
      \unvbox\@mpfootins
    \fi
    \@minipagefalse   %% added 24 May 89
  \color@endgroup
  \egroup
  \expandafter\@iiiparbox\@mpargs{\unvbox\@tempboxa}}
\newcount\@mplistdepth
\newinsert\@mpfootins
\long\def\@mpfootnotetext#1{%
  \global\setbox\@mpfootins\vbox{%
    \unvbox\@mpfootins
    \reset@font\footnotesize
    \hsize\columnwidth
    \@parboxrestore
    \protected@edef\@currentlabel
         {\csname p@mpfootnote\endcsname\@thefnmark}%
    \color@begingroup
      \@makefntext{%
        \rule\z@\footnotesep\ignorespaces#1\@finalstrut\strutbox}%
    \color@endgroup}}
\newif\if@pboxsw
\def\@finalstrut#1{%
  \unskip\ifhmode\nobreak\fi\vrule\@width\z@\@height\z@\@depth\dp#1}
\def\@@line{\hb@xt@\hsize}
\def\leftline#1{\@@line{#1\hss}}
\def\rightline#1{\@@line{\hss#1}}
\def\centerline#1{\@@line{\hss#1\hss}}

\expandafter\let\csname ver@autotabg.sty\endcsname\fmtversion
\newdimen\@gtempa
\chardef\@firsttab=\the\allocationnumber
\newdimen\@gtempa\newdimen\@gtempa\newdimen\@gtempa\newdimen\@gtempa
\newdimen\@gtempa\newdimen\@gtempa\newdimen\@gtempa\newdimen\@gtempa
\newdimen\@gtempa\newdimen\@gtempa\newdimen\@gtempa\newdimen\@gtempa
\newdimen\@gtempa
\chardef\@maxtab=\the\allocationnumber
\dimen\@firsttab=0pt
\newcount\@nxttabmar
\newcount\@curtabmar
\newcount\@curtab
\newcount\@hightab
\newcount\@tabpush
\newbox\@curline
\newbox\@curfield
\newbox\@tabfbox
\newif\if@rjfield
\gdef\@startline{%
     \ifnum \@nxttabmar >\@hightab
       \@badtab
       \global\@nxttabmar \@hightab
     \fi
     \global\@curtabmar \@nxttabmar
     \global\@curtab \@curtabmar
     \global\setbox\@curline \hbox {}%
     \@startfield
     \strut}
\gdef\@stopline{%
  \unskip
  \@stopfield
  \if@rjfield
    \global\@rjfieldfalse
    \@tempdima\@totalleftmargin
    \advance\@tempdima\linewidth
    \hb@xt@\@tempdima{%
      \@itemfudge\hskip\dimen\@curtabmar
      \box\@curline
      \hfil
      \box\@curfield}%
  \else
    \@addfield
    \hbox{\@itemfudge\hskip\dimen\@curtabmar\box\@curline}%
  \fi}
\gdef\@startfield{%
  \global\setbox\@curfield\hbox\bgroup\color@begingroup}
\gdef\@stopfield{%
  \color@endgroup\egroup}
\gdef\@contfield{%
  \global\setbox\@curfield\hbox\bgroup\color@begingroup
  \unhbox\@curfield}
\gdef\@addfield{\global\setbox\@curline\hbox{\unhbox
     \@curline\unhbox\@curfield}}
\gdef\@ifatmargin{\ifdim \wd\@curline =\z@}
\gdef\@tabcr{\@stopline \@ifstar{\penalty \@M \@xtabcr}\@xtabcr}
\gdef\@xtabcr{\@ifnextchar[\@itabcr{\@startline\ignorespaces}}
\gdef\@itabcr[#1]{\vskip #1\@startline\ignorespaces}
\gdef\kill{\@stopfield\@startline\ignorespaces}
\gdef\tabbing{\lineskip \z@skip\let\>\@rtab\let\<\@ltab\let\=\@settab
     \let\+\@tabplus\let\-\@tabminus\let\`\@tabrj\let\'\@tablab
     \let\\=\@tabcr
     \@hightab\@firsttab
     \global\@nxttabmar\@firsttab
     \dimen\@firsttab\@totalleftmargin
     \global\@tabpush\z@ \global\@rjfieldfalse
     \trivlist \item\relax
     \if@minipage\else\vskip\parskip\fi
     \setbox\@tabfbox\hbox{%
       \rlap{\hskip\@totalleftmargin\indent\the\everypar}}%
     \def\@itemfudge{\box\@tabfbox}%
     \@startline\ignorespaces}
\gdef\endtabbing{%
  \@stopline\ifnum\@tabpush >\z@ \@badpoptabs \fi\endtrivlist}
\gdef\@rtab{\@stopfield\@addfield\ifnum \@curtab<\@hightab
      \global\advance\@curtab \@ne \else\@badtab\fi
      \@tempdima\dimen\@curtab
      \advance\@tempdima -\dimen\@curtabmar
      \advance\@tempdima -\wd\@curline
      \global\setbox\@curline\hbox{\unhbox\@curline\hskip\@tempdima}%
      \@startfield\ignorespaces}
\gdef\@settab{\@stopfield\@addfield
  \ifnum \@curtab <\@maxtab
    \ifnum\@curtab =\@hightab
      \advance\@hightab \@ne
    \fi
    \global\advance\@curtab \@ne
  \else
    \@latex@error{Tab overflow}\@ehd
  \fi
  \dimen\@curtab \dimen\@curtabmar
  \advance\dimen\@curtab \wd\@curline
  \@startfield
  \ignorespaces}
\gdef\@ltab{\@ifatmargin\ifnum\@curtabmar >\@firsttab
      \global\advance\@curtab \m@ne \global\advance\@curtabmar\m@ne\else
      \@badtab\fi\else
      \@latex@error{\string\<\space in mid line}\@ehd\fi\ignorespaces}
\gdef\@tabplus{%
  \ifnum\@nxttabmar<\@hightab
    \global\advance\@nxttabmar\@ne
  \else
    \@badtab
  \fi
  \ignorespaces}
\gdef\@tabminus{%
  \ifnum\@nxttabmar>\@firsttab
    \global\advance\@nxttabmar\m@ne
  \else
    \@badtab
  \fi
  \ignorespaces}
\gdef\@tabrj{%
  \@stopfield\@addfield\global\@rjfieldtrue\@startfield\ignorespaces}
\gdef\@tablab{%
  \@stopfield
  \global\setbox\@curline\hbox{%
    \box\@curline
    \hskip-\wd\@curfield \hskip-\tabbingsep
    \box\@curfield
    \hskip\tabbingsep}%
  \@startfield
  \ignorespaces}
\gdef\pushtabs{%
  \@stopfield\@addfield\global\advance\@tabpush \@ne \begingroup
       \@contfield}
\gdef\poptabs{\@stopfield\@addfield
  \ifnum \@tabpush >\z@
    \endgroup
    \global\advance\@tabpush \m@ne
    \ifnum \@curtab >\@hightab
      \global \@curtab \@hightab
      \@badtab
    \fi
  \else
    \@badpoptabs
  \fi
  \@contfield}
\newdimen\tabbingsep
\def\extracolsep#1{\tabskip #1\relax}
\def\array{\let\@acol\@arrayacol \let\@classz\@arrayclassz
 \let\@classiv\@arrayclassiv
 \let\\\@arraycr\let\@halignto\@empty\@tabarray}

\long\def\multicolumn#1#2#3{\multispan{#1}\begingroup
  \@mkpream{#2}%
  \def\@sharp{#3}\set@typeset@protect
  \let\@startpbox\@@startpbox\let\@endpbox\@@endpbox
  \@arstrut \@preamble\hbox{}\endgroup\ignorespaces}
\def\@xexnoop #1\@@{}
\def\@expast#1{\@xexpast #1*0x\@@}
\def\@xexpast#1*#2#3#4\@@{%
  \edef\reserved@a{#1}%
  \@tempcnta#2\relax
  \ifnum\@tempcnta>\z@
    \@whilenum\@tempcnta>\z@\do
       {\edef\reserved@a{\reserved@a#3}\advance\@tempcnta \m@ne}%
    \let\reserved@b\@xexpast
  \else
    \let\reserved@b\@xexnoop
  \fi
  \expandafter\reserved@b\reserved@a #4\@@}
\newif\if@firstamp
\def\@addamp{%
  \if@firstamp
    \@firstampfalse
  \else
    \edef\@preamble{\@preamble &}%
  \fi}
\def\@arrayacol{\edef\@preamble{\@preamble \hskip \arraycolsep}}
\def\@tabacol{\edef\@preamble{\@preamble \hskip \tabcolsep}}
\def\@ampacol{\@addamp \@acol}
\def\@acolampacol{\@acol\@addamp\@acol}
\def\@mkpream#1{\@firstamptrue\@lastchclass6
  \let\@preamble\@empty
  \let\protect\@unexpandable@protect
  \let\@sharp\relax
  \let\@startpbox\relax\let\@endpbox\relax
  \@expast{#1}%
  \expandafter\@tfor \expandafter
    \@nextchar \expandafter:\expandafter=\reserved@a\do
       {\@testpach\@nextchar
    \ifcase \@chclass \@classz \or \@classi \or \@classii \or \@classiii
      \or \@classiv \or\@classv \fi\@lastchclass\@chclass}%
  \ifcase \@lastchclass \@acol
      \or \or \@preamerr \@ne\or \@preamerr \tw@\or \or \@acol \fi}
\def\@arrayclassz{\ifcase \@lastchclass \@acolampacol \or \@ampacol \or
   \or \or \@addamp \or
   \@acolampacol \or \@firstampfalse \@acol \fi
\edef\@preamble{\@preamble
  \ifcase \@chnum
     \hfil$\relax\@sharp$\hfil \or $\relax\@sharp$\hfil
    \or \hfil$\relax\@sharp$\fi}}
\def\@tabclassz{%
  \ifcase\@lastchclass
    \@acolampacol
  \or
    \@ampacol
  \or
  \or
  \or
    \@addamp
  \or
    \@acolampacol
  \or
    \@firstampfalse\@acol
  \fi
  \edef\@preamble{%
    \@preamble{%
      \ifcase\@chnum
        \hfil\ignorespaces\@sharp\unskip\hfil
      \or
        \hskip1sp\ignorespaces\@sharp\unskip\hfil
      \or
        \hfil\hskip1sp\ignorespaces\@sharp\unskip
      \fi}}}
\def\@classi{%
  \ifcase\@lastchclass
    \@acol\@arrayrule
  \or
    \@addtopreamble{\hskip \doublerulesep}\@arrayrule
  \or
  \or
  \or
    \@arrayrule
  \or
    \@acol\@arrayrule
  \or
    \@arrayrule
  \fi}
\def\@classii{%
  \ifcase\@lastchclass
  \or
    \@addtopreamble{\hskip .5\arrayrulewidth}%
  \fi}
\def\@classiii{\ifcase \@lastchclass \@acolampacol \or
   \@addamp\@acol \or
   \or \or \@addamp \or
   \@acolampacol \or \@ampacol \fi}
\def\@tabclassiv{\@addtopreamble\@nextchar}
\def\@arrayclassiv{\@addtopreamble{$\@nextchar$}}
\def\@classv{\@addtopreamble{\@startpbox{\@nextchar}\ignorespaces
\@sharp\@endpbox}}
\def\@addtopreamble#1{\edef\@preamble{\@preamble #1}}
\newcount\@chclass
\newcount\@lastchclass
\newcount\@chnum
\newdimen\arraycolsep
\newdimen\tabcolsep
\newdimen\arrayrulewidth
\newdimen\doublerulesep
\def\arraystretch{1}    % Default value.
\newbox\@arstrutbox
\def\@arstrut{%
  \relax\ifmmode\copy\@arstrutbox\else\unhcopy\@arstrutbox\fi}
\def\@arrayrule{\@addtopreamble{\hskip -.5\arrayrulewidth
   \vrule \@width \arrayrulewidth\hskip -.5\arrayrulewidth}}
\def\@testpach#1{\@chclass \ifnum \@lastchclass=\tw@ 4 \else
    \ifnum \@lastchclass=3 5 \else
     \z@ \if #1c\@chnum \z@ \else
                              \if #1l\@chnum \@ne \else
                              \if #1r\@chnum \tw@ \else
          \@chclass \if #1|\@ne \else
                    \if #1@\tw@ \else
                    \if #1p3 \else \z@ \@preamerr 0\fi
  \fi  \fi  \fi  \fi  \fi  \fi
\fi}
\def\hline{%
  \noalign{\ifnum0=`}\fi\hrule \@height \arrayrulewidth \futurelet
   \reserved@a\@xhline}
\def\@xhline{\ifx\reserved@a\hline
               \vskip\doublerulesep
               \vskip-\arrayrulewidth
             \fi
      \ifnum0=`{\fi}}
\def\vline{\vrule \@width \arrayrulewidth}
\def\cline#1{\@cline#1\@nil}
\def\@cline#1-#2\@nil{%
  \omit
  \@multicnt#1%
  \advance\@multispan\m@ne
  \ifnum\@multicnt=\@ne\@firstofone{&\omit}\fi
  \@multicnt#2%
  \advance\@multicnt-#1%
  \advance\@multispan\@ne
  \leaders\hrule\@height\arrayrulewidth\hfill
  \cr
  \noalign{\vskip-\arrayrulewidth}}
\def\multispan{\omit\@multispan}
\def\@multispan#1{%
  \@multicnt#1\relax
  \loop\ifnum\@multicnt>\@ne \sp@n\repeat}
\def\sp@n{\span\omit\advance\@multicnt\m@ne}

\def\@startpbox#1{\vtop\bgroup \setlength\hsize{#1}\@arrayparboxrestore}
\def\@endpbox{\@finalstrut\@arstrutbox\par\egroup\hfil}
\let\@@startpbox=\@startpbox
\let\@@endpbox=\@endpbox
%%% From File: ltpictur.dtx
\expandafter\let\csname ver@autopict.sty\endcsname\fmtversion
\newdimen\@wholewidth
\newdimen\@halfwidth
\newdimen\unitlength \unitlength =1pt
\newbox\@picbox
\newdimen\@picht
\long\gdef\picture#1{\pictur@#1}
\gdef\pictur@(#1){%
  \@ifnextchar({\@picture(#1)}{\@picture(#1)(0,0)}}
\gdef\@picture(#1,#2)(#3,#4){%
  \@picht#2\unitlength
  \setbox\@picbox\hb@xt@#1\unitlength\bgroup
    \hskip -#3\unitlength
    \lower #4\unitlength\hbox\bgroup
      \ignorespaces}
\gdef\endpicture{%
  \egroup\hss\egroup
    \ht\@picbox\@picht\dp\@picbox\z@
    \mbox{\box\@picbox}}

\gdef\@killglue{\unskip\@whiledim \lastskip >\z@\do{\unskip}}
\gdef\thinlines{\let\@linefnt\tenln \let\@circlefnt\tencirc
  \@wholewidth\fontdimen8\tenln \@halfwidth .5\@wholewidth}
\gdef\thicklines{\let\@linefnt\tenlnw \let\@circlefnt\tencircw
  \@wholewidth\fontdimen8\tenlnw \@halfwidth .5\@wholewidth}
\gdef\linethickness#1{\@wholewidth #1\relax \@halfwidth .5\@wholewidth}
\gdef\shortstack{\@ifnextchar[\@shortstack{\@shortstack[c]}}
\gdef\@shortstack[#1]{%
  \leavevmode
  \vbox\bgroup
    \baselineskip-\p@\lineskip 3\p@
    \let\mb@l\hss\let\mb@r\hss
    \expandafter\let\csname mb@#1\endcsname\relax
    \let\\\@stackcr
    \@ishortstack}
\gdef\@ishortstack#1{\ialign{\mb@l {##}\unskip\mb@r\cr #1\crcr}\egroup}
\gdef\@stackcr{\@ifstar\@ixstackcr\@ixstackcr}
\gdef\@ixstackcr{\@ifnextchar[\@istackcr{\cr\ignorespaces}}
\gdef\@istackcr[#1]{\cr\noalign{\vskip #1}\ignorespaces}
\newif\if@negarg

\gdef\@sline{%
  \ifnum\@xarg<\z@ \@negargtrue \@xarg -\@xarg \@yyarg -\@yarg
  \else \@negargfalse \@yyarg \@yarg \fi
\ifnum \@yyarg >\z@ \@tempcnta\@yyarg \else \@tempcnta -\@yyarg \fi
\ifnum\@tempcnta>6 \@badlinearg\@tempcnta\z@ \fi
\ifnum\@xarg>6 \@badlinearg\@xarg \@ne \fi
\setbox\@linechar\hbox{\@linefnt\@getlinechar(\@xarg,\@yyarg)}%
\ifnum \@yarg >\z@ \let\@upordown\raise \@clnht\z@
   \else\let\@upordown\lower \@clnht \ht\@linechar\fi
\@clnwd \wd\@linechar
\if@negarg
  \hskip -\wd\@linechar \def\reserved@a{\hskip -2\wd\@linechar}%
\else
     \let\reserved@a\relax
\fi
\@whiledim \@clnwd <\@linelen \do
  {\@upordown\@clnht\copy\@linechar
   \reserved@a
   \advance\@clnht \ht\@linechar
   \advance\@clnwd \wd\@linechar}%
\advance\@clnht -\ht\@linechar
\advance\@clnwd -\wd\@linechar
\@tempdima\@linelen\advance\@tempdima -\@clnwd
\@tempdimb\@tempdima\advance\@tempdimb -\wd\@linechar
\if@negarg \hskip -\@tempdimb \else \hskip \@tempdimb \fi
\multiply\@tempdima \@m
\@tempcnta \@tempdima
\@tempdima \wd\@linechar \divide\@tempcnta \@tempdima
\@tempdima \ht\@linechar \multiply\@tempdima \@tempcnta
\divide\@tempdima \@m
\advance\@clnht \@tempdima
\ifdim \@linelen <\wd\@linechar
   \hskip \wd\@linechar
   \ifdim \@linelen = \z@
   \else
     \@picture@warn
   \fi
   \else\@upordown\@clnht\copy\@linechar\fi}
\gdef\@hline{\ifnum \@xarg <\z@ \hskip -\@linelen \fi
\vrule \@height \@halfwidth \@depth \@halfwidth \@width \@linelen
\ifnum \@xarg <\z@ \hskip -\@linelen \fi}
\gdef\@getlinechar(#1,#2){\@tempcnta#1\relax\multiply\@tempcnta 8%
  \advance\@tempcnta -9\ifnum #2>\z@ \advance\@tempcnta #2\relax\else
  \advance\@tempcnta -#2\relax\advance\@tempcnta 64 \fi
  \char\@tempcnta}

\gdef\@hvector{\@hline\hb@xt@\z@{\@linefnt
 \ifnum \@xarg <\z@ \@getlarrow(1,0)\hss\else
    \hss\@getrarrow(1,0)\fi}}
\gdef\@vvector{\ifnum \@yarg <\z@ \@downvector \else \@upvector \fi}
\gdef\@svector{\@sline
  \@tempcnta\@yarg \ifnum\@tempcnta <\z@ \@tempcnta -\@tempcnta\fi
  \ifnum\@tempcnta <5%
    \hskip -\wd\@linechar
    \@upordown\@clnht \hbox{\@linefnt  \if@negarg
    \@getlarrow(\@xarg,\@yyarg)\else \@getrarrow(\@xarg,\@yyarg)\fi}%
  \else\@badlinearg\fi}
\gdef\@getlarrow(#1,#2){\ifnum #2=\z@ \@tempcnta 27 % '33
  \else
  \@tempcnta #1\relax\multiply\@tempcnta \sixt@@n
  \advance\@tempcnta -9 \@tempcntb #2\relax\multiply\@tempcntb \tw@
  \ifnum \@tempcntb >\z@ \advance\@tempcnta \@tempcntb
  \else\advance\@tempcnta -\@tempcntb\advance\@tempcnta 64
  \fi\fi\char\@tempcnta}
\gdef\@getrarrow(#1,#2){\@tempcntb #2\relax
\ifnum\@tempcntb <\z@ \@tempcntb -\@tempcntb\relax\fi
\ifcase \@tempcntb\relax \@tempcnta 45  % '55
\or
\ifnum #1<\thr@@ \@tempcnta #1\relax\multiply\@tempcnta
24\advance\@tempcnta -6 \else \ifnum #1=\thr@@ \@tempcnta 49
\else\@tempcnta 58 \fi\fi\or
\ifnum #1<\thr@@ \@tempcnta=#1\relax\multiply\@tempcnta
24\advance\@tempcnta -\thr@@ \else \@tempcnta 51 \fi\or
\@tempcnta #1\relax\multiply\@tempcnta
\sixt@@n \advance\@tempcnta -\tw@ \else
\@tempcnta #1\relax\multiply\@tempcnta
\sixt@@n \advance\@tempcnta 7 \fi\ifnum #2<\z@ \advance\@tempcnta 64 \fi
\char\@tempcnta}
\gdef\@vline{\ifnum \@yarg <\z@ \@downline \else \@upline\fi}
\gdef\@upline{%
  \hb@xt@\z@{\hskip -\@halfwidth \vrule \@width \@wholewidth
   \@height \@linelen \@depth \z@\hss}}
\gdef\@downline{%
  \hb@xt@\z@{\hskip -\@halfwidth \vrule \@width \@wholewidth
   \@height \z@ \@depth \@linelen \hss}}
\gdef\@upvector{\@upline\setbox\@tempboxa\hbox{\@linefnt\char 54}%  % '66
  \raise \@linelen \hb@xt@\z@{\lower \ht\@tempboxa\box\@tempboxa\hss}}
\gdef\@downvector{\@downline\lower \@linelen
      \hb@xt@\z@{\@linefnt\char 63  % '77
      \hss}}
\gdef\dashbox#1(#2,#3){\leavevmode\hb@xt@\z@{\baselineskip \z@skip
\lineskip \z@skip
\@dashdim #2\unitlength
\@dashcnt \@dashdim \advance\@dashcnt 200
\@dashdim #1\unitlength\divide\@dashcnt \@dashdim
\ifodd\@dashcnt\@dashdim \z@
\advance\@dashcnt \@ne \divide\@dashcnt \tw@
\else \divide\@dashdim \tw@ \divide\@dashcnt \tw@
\advance\@dashcnt \m@ne
\setbox\@dashbox \hbox{\vrule \@height \@halfwidth \@depth \@halfwidth
\@width \@dashdim}\put(0,0){\copy\@dashbox}%
\put(0,#3){\copy\@dashbox}%
\put(#2,0){\hskip-\@dashdim\copy\@dashbox}%
\put(#2,#3){\hskip-\@dashdim\box\@dashbox}%
\multiply\@dashdim \thr@@
\fi
\setbox\@dashbox \hbox{\vrule \@height \@halfwidth \@depth \@halfwidth
\@width #1\unitlength\hskip #1\unitlength}\@tempcnta\z@
\put(0,0){\hskip\@dashdim \@whilenum \@tempcnta <\@dashcnt
\do{\copy\@dashbox\advance\@tempcnta \@ne }}\@tempcnta\z@
\put(0,#3){\hskip\@dashdim \@whilenum \@tempcnta <\@dashcnt
\do{\copy\@dashbox\advance\@tempcnta \@ne }}%
\@dashdim #3\unitlength
\@dashcnt \@dashdim \advance\@dashcnt 200
\@dashdim #1\unitlength\divide\@dashcnt \@dashdim
\ifodd\@dashcnt \@dashdim \z@
\advance\@dashcnt \@ne \divide\@dashcnt \tw@
\else
\divide\@dashdim \tw@ \divide\@dashcnt \tw@
\advance\@dashcnt \m@ne
\setbox\@dashbox\hbox{\hskip -\@halfwidth
\vrule \@width \@wholewidth
\@height \@dashdim}\put(0,0){\copy\@dashbox}%
\put(#2,0){\copy\@dashbox}%
\put(0,#3){\lower\@dashdim\copy\@dashbox}%
\put(#2,#3){\lower\@dashdim\copy\@dashbox}%
\multiply\@dashdim \thr@@
\fi
\setbox\@dashbox\hbox{\vrule \@width \@wholewidth
\@height #1\unitlength}\@tempcnta\z@
\put(0,0){\hskip -\@halfwidth \vbox{\@whilenum \@tempcnta <\@dashcnt
\do{\vskip #1\unitlength\copy\@dashbox\advance\@tempcnta \@ne }%
\vskip\@dashdim}}\@tempcnta\z@
\put(#2,0){\hskip -\@halfwidth \vbox{\@whilenum \@tempcnta<\@dashcnt
\do{\vskip #1\unitlength\copy\@dashbox\advance\@tempcnta \@ne }%
\vskip\@dashdim}}}\@makepicbox(#2,#3)}
\newif\if@ovt
\newif\if@ovb
\newif\if@ovl
\newif\if@ovr
\newdimen\@ovxx
\newdimen\@ovyy
\newdimen\@ovdx
\newdimen\@ovdy
\newdimen\@ovro
\newdimen\@ovri
\gdef\@getcirc#1{\@tempdima #1\relax \advance\@tempdima 2\p@
  \@tempcnta\@tempdima
  \@tempdima 4\p@ \divide\@tempcnta\@tempdima
  \ifnum \@tempcnta >10\relax \@tempcnta 10\relax\fi
  \ifnum \@tempcnta >\z@ \advance\@tempcnta\m@ne
    \else \@picture@warn \fi
  \multiply\@tempcnta 4\relax
  \setbox \@tempboxa \hbox{\@circlefnt
  \char \@tempcnta}\@tempdima \wd \@tempboxa}
\def\@picture@warn{\@latex@warning{%
     \string\oval, \string\circle, or \string\line\space
     size unavailable}}
\gdef\@put#1#2#3{\raise #2\hb@xt@\z@{\hskip #1#3\hss}}
\gdef\oval(#1,#2){\@ifnextchar[{\@oval(#1,#2)}{\@oval(#1,#2)[]}}
\gdef\@oval(#1,#2)[#3]{\begingroup\boxmaxdepth \maxdimen
  \@ovttrue \@ovbtrue \@ovltrue \@ovrtrue
  \@tfor\reserved@a :=#3\do{\csname @ov\reserved@a false\endcsname}%
  \@ovxx
  #1\unitlength \@ovyy #2\unitlength
  \@tempdimb \ifdim \@ovyy >\@ovxx \@ovxx\else \@ovyy \fi
  \advance \@tempdimb -2\p@
  \@getcirc \@tempdimb
  \@ovro \ht\@tempboxa \@ovri \dp\@tempboxa
  \@ovdx\@ovxx \advance\@ovdx -\@tempdima \divide\@ovdx \tw@
  \@ovdy\@ovyy \advance\@ovdy -\@tempdima \divide\@ovdy \tw@
  \@circlefnt \setbox\@tempboxa
  \hbox{\if@ovr \@ovvert32\kern -\@tempdima \fi
  \if@ovl \kern \@ovxx \@ovvert01\kern -\@tempdima \kern -\@ovxx \fi
  \if@ovt \@ovhorz \kern -\@ovxx \fi
  \if@ovb \raise \@ovyy \@ovhorz \fi}\advance\@ovdx\@ovro
  \advance\@ovdy\@ovro \ht\@tempboxa\z@ \dp\@tempboxa\z@
  \@put{-\@ovdx}{-\@ovdy}{\box\@tempboxa}%
  \endgroup}
\gdef\@ovvert#1#2{\vbox to\@ovyy{%
    \if@ovb \@tempcntb \@tempcnta \advance \@tempcntb #1\relax
      \kern -\@ovro \hbox{\char \@tempcntb}\nointerlineskip
    \else \kern \@ovri \kern \@ovdy \fi
    \leaders\vrule \@width \@wholewidth\vfil \nointerlineskip
    \if@ovt \@tempcntb \@tempcnta \advance \@tempcntb #2\relax
      \hbox{\char \@tempcntb}%
    \else \kern \@ovdy \kern \@ovro \fi}}
\gdef\@ovhorz{\hb@xt@\@ovxx{\kern \@ovro
    \if@ovr \else \kern \@ovdx \fi
    \leaders \hrule \@height \@wholewidth \hfil
    \if@ovl \else \kern \@ovdx \fi
    \kern \@ovri}}

\gdef\circle{%
   \msitag{\circle}%
   \@ifnextchar*{\@circle@star}{\@circle}%
}

\gdef\@circle@star*{\msitag{*}\@circle}
\def\@circle#1{\msitag{{#1}}}%



\newcount\@xarg
\newcount\@yarg
\newcount\@yyarg
\newcount\@multicnt
\newdimen\@xdim
\newdimen\@ydim
\newbox\@linechar
\newdimen\@linelen
\newdimen\@clnwd
\newdimen\@clnht
\newdimen\@dashdim
\newbox\@dashbox
\newcount\@dashcnt
\let\@linefnt\tenln
\let\@circlefnt\tencirc
\@wholewidth\fontdimen8\tenln
\@halfwidth .5\@wholewidth




\gdef\qbeziermax{500}
\newcommand\qbezier[2][0]{\bezier{#1}#2}
\gdef\bezier#1)#2(#3)#4({\@bezier#1)(#3)(}
\gdef\@bezier#1(#2,#3)(#4,#5)(#6,#7){%
  \ifnum #1=\z@
      \@ovxx #4\unitlength
        \advance\@ovxx -#2\unitlength
        \ifdim \@ovxx<\z@ \@ovxx -\@ovxx \fi
      \@ovdx #6\unitlength
        \advance\@ovdx -#4\unitlength
        \ifdim \@ovdx<\z@ \@ovdx -\@ovdx \fi
        \ifdim \@ovxx<\@ovdx \@ovxx \@ovdx \fi
      \@ovyy #5\unitlength
        \advance\@ovyy -#3\unitlength
        \ifdim \@ovyy<\z@ \@ovyy -\@ovyy \fi
      \@ovdy #7\unitlength
        \advance\@ovdy -#5\unitlength
        \ifdim \@ovdy<\z@  \@ovdy -\@ovdy \fi
        \ifdim \@ovyy<\@ovdy \@ovyy  \@ovdy \fi
      \@multicnt
         \ifdim \@ovxx>\@ovyy \@ovxx \else \@ovyy \fi
      \@ovxx .5\@halfwidth \divide\@multicnt\@ovxx
      \ifnum \qbeziermax<\@multicnt \@multicnt\qbeziermax\relax \fi
  \else \@multicnt#1\relax \fi
  \@tempcnta\@multicnt \advance\@tempcnta\@ne
  \@ovdx #4\unitlength \advance\@ovdx -#2\unitlength
      \multiply\@ovdx \tw@
  \@ovxx #6\unitlength \advance\@ovxx -#2\unitlength
      \advance\@ovxx -\@ovdx \divide\@ovxx\@multicnt
  \@ovdy #5\unitlength \advance\@ovdy -#3\unitlength
       \multiply\@ovdy \tw@
  \@ovyy #7\unitlength \advance\@ovyy -#3\unitlength
      \advance\@ovyy -\@ovdy \divide\@ovyy\@multicnt
  \setbox\@tempboxa\hbox{\vrule \@height\@halfwidth
                               \@depth \@halfwidth
                               \@width \@wholewidth}%
   \put(#2,#3){%
     \count@\z@
     \@whilenum{\count@<\@tempcnta}\do
        {\@xdim\count@\@ovxx
           \advance\@xdim\@ovdx
           \divide\@xdim\@multicnt
           \multiply\@xdim\count@
         \@ydim\count@\@ovyy
            \advance\@ydim\@ovdy
            \divide\@ydim\@multicnt
            \multiply\@ydim\count@
         \raise \@ydim
            \hb@xt@\z@{\kern\@xdim
                        \unhcopy\@tempboxa\hss}%
         \advance\count@\@ne}}}


\def\qbezier{%
   \br\msitag{\qbezier}%
   \@ifnextchar[{\@qbezier@opt}{\@qbezier}%
}

\def\@qbezier@opt[#1]{%
   \msitag{[#1]}%
   \@qbezier
}

\def\@qbezier(#1)(#2)(#3){%
   \msitag{(#1)(#2)(#3)}%
}


\def\newtheorem#1{%
  \@ifnextchar[{\out@newtheorema{#1}}{\@nthm{#1}}}{}
  
\def\@nthm#1#2{%
  \@ifnextchar[{\@xnthm{#1}{#2}}{\out@newtheoremb{#1}{#2}}}


  
\def\@xnthm#1#2[#3]{%
  \expandafter\@ifdefinable\csname #1\endcsname
    {\@definecounter{#1}\@newctr{#1}[#3]%
     \expandafter\xdef\csname the#1\endcsname{%
       \expandafter\noexpand\csname the#3\endcsname \@thmcountersep
          \@thmcounter{#1}}%
     \global\@namedef{#1}{\@thm{#1}{#2}}%
     \global\@namedef{end#1}{\@endtheorem}}}



\def\@thm#1#2{%
  \refstepcounter{#1}%
  \@ifnextchar[{\@ythm{#1}{#2}}{\@xthm{#1}{#2}}}
\def\@xthm#1#2{%
  \@begintheorem{#2}{\csname the#1\endcsname}\ignorespaces}
\def\@ythm#1#2[#3]{%
  \@opargbegintheorem{#2}{\csname the#1\endcsname}{#3}\ignorespaces}
\def\@thmcounter#1{\noexpand\arabic{#1}}
\def\@thmcountersep{.}
\def\@begintheorem#1#2{\trivlist
   \item[\hskip \labelsep{\bfseries #1\ #2}]\itshape}
\def\@opargbegintheorem#1#2#3{\trivlist
      \item[\hskip \labelsep{\bfseries #1\ #2\ (#3)}]\itshape}
\def\@endtheorem{\endtrivlist}


\message{sectioning,}
\newcount\@secpenalty
\@secpenalty = -300
\newif\if@noskipsec \@noskipsectrue
%%% jcs\def\@startsection#1#2#3#4#5#6{%
%%% jcs  \if@noskipsec \leavevmode \fi
%%% jcs  \par
%%% jcs  \@tempskipa #4\relax
%%% jcs  \@afterindenttrue
%%% jcs  \ifdim \@tempskipa <\z@
%%% jcs    \@tempskipa -\@tempskipa \@afterindentfalse
%%% jcs  \fi
%%% jcs  \if@nobreak
%%% jcs    \everypar{}%
%%% jcs  \else
%%% jcs    \addpenalty\@secpenalty\addvspace\@tempskipa
%%% jcs  \fi
%%% jcs  \@ifstar
%%% jcs    {\@ssect{#3}{#4}{#5}{#6}}%
%%% jcs    {\@dblarg{\@sect{#1}{#2}{#3}{#4}{#5}{#6}}}}
%%% jcs\def\@sect#1#2#3#4#5#6[#7]#8{%
%%% jcs  \ifnum #2>\c@secnumdepth
%%% jcs    \let\@svsec\@empty
%%% jcs  \else
%%% jcs    \refstepcounter{#1}%
%%% jcs    \protected@edef\@svsec{\@seccntformat{#1}\relax}%
%%% jcs  \fi
%%% jcs  \@tempskipa #5\relax
%%% jcs  \ifdim \@tempskipa>\z@
%%% jcs    \begingroup
%%% jcs      #6{%
%%% jcs        \@hangfrom{\hskip #3\relax\@svsec}%
%%% jcs          \interlinepenalty \@M #8\@@par}%
%%% jcs    \endgroup
%%% jcs    \csname #1mark\endcsname{#7}%
%%% jcs    \addcontentsline{toc}{#1}{%
%%% jcs      \ifnum #2>\c@secnumdepth \else
%%% jcs        \protect\numberline{\csname the#1\endcsname}%
%%% jcs      \fi
%%% jcs      #7}%
%%% jcs  \else
%%% jcs    \def\@svsechd{%
%%% jcs      #6{\hskip #3\relax
%%% jcs      \@svsec #8}%
%%% jcs      \csname #1mark\endcsname{#7}%
%%% jcs      \addcontentsline{toc}{#1}{%
%%% jcs        \ifnum #2>\c@secnumdepth \else
%%% jcs          \protect\numberline{\csname the#1\endcsname}%
%%% jcs        \fi
%%% jcs        #7}}%
%%% jcs  \fi
%%% jcs  \@xsect{#5}}
%%% jcs\def\@xsect#1{%
%%% jcs  \@tempskipa #1\relax
%%% jcs  \ifdim \@tempskipa>\z@
%%% jcs    \par \nobreak
%%% jcs    \vskip \@tempskipa
%%% jcs    \@afterheading
%%% jcs  \else
%%% jcs    \@nobreakfalse
%%% jcs    \global\@noskipsectrue
%%% jcs    \everypar{%
%%% jcs      \if@noskipsec
%%% jcs        \global\@noskipsecfalse
%%% jcs       {\setbox\z@\lastbox}%
%%% jcs        \clubpenalty\@M
%%% jcs        \begingroup \@svsechd \endgroup
%%% jcs        \unskip
%%% jcs        \@tempskipa #1\relax
%%% jcs        \hskip -\@tempskipa
%%% jcs      \else
%%% jcs        \clubpenalty \@clubpenalty
%%% jcs        \everypar{}%
%%% jcs      \fi}%
%%% jcs  \fi
%%% jcs  \ignorespaces}
\def\@seccntformat#1{\csname the#1\endcsname\quad}
\def\@ssect#1#2#3#4#5{%
  \@tempskipa #3\relax
  \ifdim \@tempskipa>\z@
    \begingroup
      #4{%
        \@hangfrom{\hskip #1}%
          \interlinepenalty \@M #5\@@par}%
    \endgroup
  \else
    \def\@svsechd{#4{\hskip #1\relax #5}}%
  \fi
  \@xsect{#3}}
\newif\if@afterindent \@afterindenttrue
\def\@afterheading{%
  \@nobreaktrue
  \everypar{%
    \if@nobreak
      \@nobreakfalse
      \clubpenalty \@M
      \if@afterindent \else
        {\setbox\z@\lastbox}%
      \fi
    \else
      \clubpenalty \@clubpenalty
      \everypar{}%
    \fi}}
\def\@hangfrom#1{\setbox\@tempboxa\hbox{{#1}}%
      \hangindent \wd\@tempboxa\noindent\box\@tempboxa}
\newcount\c@secnumdepth
\newcount\c@tocdepth
\def\secdef#1#2{\@ifstar{#2}{\@dblarg{#1}}}
\let\sectionmark\@gobble
\let\subsectionmark\@gobble
\let\subsubsectionmark\@gobble
\let\paragraphmark\@gobble
\let\subparagraphmark\@gobble
\message{contents,}
\def\@starttoc#1{%
  \begingroup
    \makeatletter
    \@input{\jobname.#1}%
    \if@filesw
      \expandafter\newwrite\csname tf@#1\endcsname
      \immediate\openout \csname tf@#1\endcsname \jobname.#1\relax
    \fi
    \@nobreakfalse
  \endgroup}
\def\addcontentsline#1#2#3{%
  \addtocontents{#1}{\protect\contentsline{#2}{#3}{\thepage}}}
\long\def\addtocontents#1#2{%
  \protected@write\@auxout
      {\let\label\@gobble \let\index\@gobble \let\glossary\@gobble}%
      {\string\@writefile{#1}{#2}}}
\def\contentsline#1{\csname l@#1\endcsname}
\def\@dottedtocline#1#2#3#4#5{%
  \ifnum #1>\c@tocdepth \else
    \vskip \z@ \@plus.2\p@
    {\leftskip #2\relax \rightskip \@tocrmarg \parfillskip -\rightskip
     \parindent #2\relax\@afterindenttrue
     \interlinepenalty\@M
     \leavevmode
     \@tempdima #3\relax
     \advance\leftskip \@tempdima \null\nobreak\hskip -\leftskip
     {#4}\nobreak
     \leaders\hbox{$\m@th
        \mkern \@dotsep mu\hbox{.}\mkern \@dotsep
        mu$}\hfill
     \nobreak
     \hb@xt@\@pnumwidth{\hfil\normalfont \normalcolor #5}%
     \par}%
  \fi}
\def\numberline#1{\hb@xt@\@tempdima{#1\hfil}}

\newcount\@floatpenalty
%%\def\caption{%
%%   \ifx\@captype\@undefined
%%     \@latex@error{\noexpand\caption outside float}\@ehd
%%     \expandafter\@gobble
%%   \else
%%     \refstepcounter\@captype
%%     \expandafter\@firstofone
%%   \fi
%%   {\@dblarg{\@caption\@captype}}%
%%}
\msi@pass{\caption}
\long\def\@caption#1[#2]#3{%
  \par
  \addcontentsline{\csname ext@#1\endcsname}{#1}%
    {\protect\numberline{\csname the#1\endcsname}{\ignorespaces #2}}%
  \begingroup
    \@parboxrestore
    \if@minipage
      \@setminipage
    \fi
    \normalsize
    \@makecaption{\csname fnum@#1\endcsname}{\ignorespaces #3}\par
  \endgroup}
\def\@float#1{%
  \@ifnextchar[%
    {\@xfloat{#1}}%
    {\edef\reserved@a{\noexpand\@xfloat{#1}[\csname fps@#1\endcsname]}%
     \reserved@a}}
\def\@dblfloat{%
  \if@twocolumn\let\reserved@a\@dbflt\else\let\reserved@a\@float\fi
  \reserved@a}
\def\@xfloat #1[#2]{%
  \@nodocument
  \def \@captype {#1}%
   \def \@fps {#2}%
   \@onelevel@sanitize \@fps
   \def \reserved@b {!}%
   \ifx \reserved@b \@fps
     \@fpsadddefault
   \else
     \ifx \@fps \@empty
       \@fpsadddefault
     \fi
   \fi
   \ifhmode
     \@bsphack
     \@floatpenalty -\@Mii
   \else
     \@floatpenalty-\@Miii
   \fi
  \ifinner
     \@parmoderr\@floatpenalty\z@
  \else
    \@next\@currbox\@freelist
      {%
       \@tempcnta \sixt@@n
       \expandafter \@tfor \expandafter \reserved@a
         \expandafter :\expandafter =\@fps
         \do
          {%
           \if \reserved@a h%
             \ifodd \@tempcnta
             \else
               \advance \@tempcnta \@ne
             \fi
           \fi
           \if \reserved@a t%
             \@setfpsbit \tw@
           \fi
           \if \reserved@a b%

             \@setfpsbit 4%
           \fi
           \if \reserved@a p%
             \@setfpsbit 8%
           \fi
           \if \reserved@a !%
             \ifnum \@tempcnta>15
               \advance\@tempcnta -\sixt@@n\relax
             \fi
           \fi
           }%
       \@tempcntb \csname ftype@\@captype \endcsname
       \multiply \@tempcntb \@xxxii
       \advance \@tempcnta \@tempcntb
       \global \count\@currbox \@tempcnta
       }%
    \@fltovf
  \fi
  \global \setbox\@currbox
    \color@vbox
      \normalcolor
      \vbox \bgroup
        \hsize\columnwidth
        \@parboxrestore
        \@floatboxreset
}
\def \@floatboxreset {%
        \reset@font
        \normalsize
        \@setminipage
}
\def \@setnobreak{%
  \if@nobreak
    \let\outer@nobreak\@nobreaktrue
    \@nobreakfalse
  \fi
}
\def \@setminipage{%
  \@minipagetrue
  \everypar{\@minipagefalse\everypar{}}%
}
\def\end@float{%
  \@endfloatbox
  \ifnum\@floatpenalty <\z@
    \@largefloatcheck
    \@cons\@currlist\@currbox
    \ifnum\@floatpenalty <-\@Mii
      \penalty -\@Miv
      \@tempdima\prevdepth
      \vbox{}%
      \prevdepth\@tempdima
      \penalty\@floatpenalty
    \else
      \vadjust{\penalty -\@Miv \vbox{}\penalty\@floatpenalty}\@Esphack
    \fi
  \fi
}
\def\end@dblfloat{%
\if@twocolumn
  \@endfloatbox
  \ifnum\@floatpenalty <\z@
    \@largefloatcheck
    \@cons\@dbldeferlist\@currbox
  \fi
    \ifnum \@floatpenalty =-\@Mii \@Esphack\fi
\else
  \end@float
\fi
}
\def \@endfloatbox{%
      \par\vskip\z@skip      %% \par\vskip\z@ added 15 Dec 87
      \@minipagefalse
      \outer@nobreak
    \egroup                  %% end of vbox
  \color@endbox
}
\let\outer@nobreak\@empty
\def \@largefloatcheck{%
  \ifdim \ht\@currbox>\textheight
    \@tempdima -\textheight
    \advance \@tempdima \ht\@currbox
    \@latex@warning {Float too large for page by \the\@tempdima}%
    \ht\@currbox \textheight
  \fi
}
\def\@dbflt#1{\@ifnextchar[{\@xdblfloat{#1}}{\@xdblfloat{#1}[tp]}}
\def\@xdblfloat#1[#2]{%
  \@xfloat{#1}[#2]\hsize\textwidth\linewidth\textwidth}
\def\@floatplacement{\global\@topnum\c@topnumber
    % Textpage bit, global:
   \global\@toproom \topfraction\@colht
   \global\@botnum  \c@bottomnumber
   \global\@botroom \bottomfraction\@colht
   \global\@colnum  \c@totalnumber
    % Floatpage bit, local:
   \@fpmin   \floatpagefraction\@colht}
\def \@dblfloatplacement {%
  \global \@dbltopnum \c@dbltopnumber
  \global \@dbltoproom \dbltopfraction\@colht
  \@textmin \@colht
  \advance \@textmin -\@dbltoproom
  \@fpmin \dblfloatpagefraction\textheight
  \@fptop \@dblfptop
  \@fpsep \@dblfpsep
  \@fpbot \@dblfpbot
}
%% \def\marginpar{%
%%   \ifhmode
%%     \@bsphack
%%     \@floatpenalty -\@Mii
%%   \else
%%     \@floatpenalty-\@Miii
%%   \fi
%%   \ifinner
%%     \@parmoderr
%%     \@floatpenalty\z@
%%   \else
%%     \@next\@currbox\@freelist{}{}%
%%     \@next\@marbox\@freelist{\global\count\@marbox\m@ne}%
%%        {\@floatpenalty\z@
%%         \@fltovf\def\@currbox{\@tempboxa}\def\@marbox{\@tempboxa}}%
%%   \fi
%%   \@ifnextchar [\@xmpar\@ympar}

\def\marginpar{\@ifnextchar[\@marginparA\@marginparB}

\def\@marginparA[#1]#2{%
   \out@marginparA{#1}{#2}%
}

\def\@marginparB#1{%
   \out@marginparB{#1}%
}

\long\def\@xmpar[#1]#2{%
  \@savemarbox\@marbox{#1}%
  \@savemarbox\@currbox{#2}%
  \@xympar}
\long\def\@ympar#1{%
  \@savemarbox\@marbox{#1}%
  \global\setbox\@currbox\copy\@marbox
  \@xympar}
\long\def \@savemarbox #1#2{%
  \global\setbox #1%
    \color@vbox
      \vtop{%
        \hsize\marginparwidth
        \@parboxrestore
        \@marginparreset
        #2%
        \@minipagefalse
        \outer@nobreak
        }%
    \color@endbox
}
\def \@marginparreset {%
        \reset@font
        \normalsize
        \@setminipage
}
\def \@xympar{%
  \ifnum\@floatpenalty <\z@\@cons\@currlist\@marbox\fi
  \setbox\@tempboxa
    \color@vbox
      \vbox \bgroup
  \end@float
  \@ignorefalse
  \@esphack
}
\def\reversemarginpar{\global\@mparbottom\z@ \@reversemargintrue}
\def\normalmarginpar{\global\@mparbottom\z@ \@reversemarginfalse}
\message{footnotes,}
\newinsert\footins
\skip\footins=\bigskipamount % space added when footnote is present
\count\footins=1000 % footnote magnification factor (1 to 1)
\dimen\footins=8in % maximum footnotes per page
%%%\def\footnoterule{\kern-3\p@
%%%  \hrule \@width 2in \kern 2.6\p@} % the \hrule is .4pt high
\def\footnoterule{\msitag{\footnoterule}}
\@definecounter{footnote}
\def\thefootnote{\@arabic\c@footnote}
\@definecounter{mpfootnote}
\def\thempfootnote{\itshape\@alph\c@mpfootnote}
\def\@makefnmark{\hbox{\@textsuperscript{\normalfont\@thefnmark}}}
\DeclareRobustCommand*\textsuperscript[1]{%
  \@textsuperscript{\selectfont#1}}
\def\@textsuperscript#1{%
  {\m@th\ensuremath{^{\mbox{\fontsize\sf@size\z@#1}}}}}
\newdimen\footnotesep
%%%\def\footnote{\@ifnextchar[\@xfootnote{\stepcounter\@mpfn
%%%     \protected@xdef\@thefnmark{\thempfn}%
%%%     \@footnotemark\@footnotetext}}
%%%\def\@xfootnote[#1]{%
%%%   \begingroup
%%%     \csname c@\@mpfn\endcsname #1\relax
%%%     \unrestored@protected@xdef\@thefnmark{\thempfn}%
%%%   \endgroup
%%%   \@footnotemark\@footnotetext}
%%%\long\def\@footnotetext#1{\insert\footins{%
%%%    \reset@font\footnotesize
%%%    \interlinepenalty\interfootnotelinepenalty
%%%    \splittopskip\footnotesep
%%%    \splitmaxdepth \dp\strutbox \floatingpenalty \@MM
%%%    \hsize\columnwidth \@parboxrestore
%%%    \protected@edef\@currentlabel{%
%%%       \csname p@footnote\endcsname\@thefnmark
%%%    }%
%%%    \color@begingroup
%%%      \@makefntext{%
%%%        \rule\z@\footnotesep\ignorespaces#1\@finalstrut\strutbox}%
%%%    \color@endgroup}}%
%%%\def\footnotemark{%
%%%   \@ifnextchar[\@xfootnotemark
%%%     {\stepcounter{footnote}%
%%%      \protected@xdef\@thefnmark{\thefootnote}%
%%%      \@footnotemark}}
%%%\def\@xfootnotemark[#1]{%
%%%   \begingroup
%%%      \c@footnote #1\relax
%%%      \unrestored@protected@xdef\@thefnmark{\thefootnote}%
%%%   \endgroup
%%%   \@footnotemark}
%%%\def\@footnotemark{%
%%%  \leavevmode
%%%  \ifhmode\edef\@x@sf{\the\spacefactor}\nobreak\fi
%%%  \@makefnmark
%%%  \ifhmode\spacefactor\@x@sf\fi
%%%  \relax}
%%%\def\footnotetext{%
%%%     \@ifnextchar [\@xfootnotenext
%%%       {\protected@xdef\@thefnmark{\thempfn}%
%%%    \@footnotetext}}
%%%\def\@xfootnotenext[#1]{%
%%%  \begingroup
%%%     \csname c@\@mpfn\endcsname #1\relax
%%%     \unrestored@protected@xdef\@thefnmark{\thempfn}%
%%%  \endgroup
%%%  \@footnotetext}
\def\@mpfn{footnote}
\def\thempfn{\thefootnote}
%%% From File: ltidxglo.dtx
\message{index,}
\def\makeindex{%
  \newwrite\@indexfile
  \immediate\openout\@indexfile=\jobname.idx
  \def\index{\@bsphack\begingroup
             \@sanitize
             \@wrindex}\typeout
    {Writing index file \jobname.idx}%
  \let\makeindex\@empty
}
\@onlypreamble\makeindex
\def\@wrindex#1{%
   \protected@write\@indexfile{}%
      {\string\indexentry{#1}{\thepage}}%
 \endgroup
 \@esphack}

\def\index#1{\msihmode\msitag{<index>#1</index>}}


%%\def\@index#1{\endgroup\@esphack}
\def\makeglossary{%
  \newwrite\@glossaryfile
  \immediate\openout\@glossaryfile=\jobname.glo
  \def\glossary{\@bsphack\begingroup
                \@sanitize
                \@wrglossary}\typeout
    {Writing glossary file \jobname.glo }%
  \let\makeglossary\@empty
}
\@onlypreamble\makeglossary
\def\@wrglossary#1{%
   \protected@write\@glossaryfile{}%
      {\string\glossaryentry{#1}{\thepage}}%
 \endgroup
 \@esphack}
\def\glossary{\@bsphack\begingroup\@sanitize\@index}



\def\bibitem{\@ifnextchar[\@lbibitem\@bibitem}

\def\@lbibitem[#1]#2{%
   \toks200{#1}%
   \out@bibitemA[{\the\toks200}]{#2}}

\def\@bibitem#1{\out@bibitemB{#1}}


\def\bibcite{\@newl@bel b}
\let\citation\@gobble
\DeclareRobustCommand\cite{%
  \@ifnextchar [{\@tempswatrue\out@citex}{\@tempswafalse\out@citey}}


\let\bibdata=\@gobble
\let\bibstyle=\@gobble

\def\nocite#1{\@bsphack
  \@for\@citeb:=#1\do{%
    \edef\@citeb{\expandafter\@firstofone\@citeb}%
    \if@filesw\immediate\write\@auxout{\string\citation{\@citeb}}\fi
    \@ifundefined{b@\@citeb}{\G@refundefinedtrue
        \@latex@warning{Citation `\@citeb' undefined}}{}}%
  \@esphack}
\expandafter\let\csname b@*\endcsname\@empty
\def\@cite#1#2{[{#1\if@tempswa , #2\fi}]}
\def\@biblabel#1{[#1]}

\expandafter\let\csname ver@autoout1.sty\endcsname\fmtversion
\message{output,}
\maxdeadcycles = 100
\let\@elt\relax
\def\@next#1#2#3#4{\ifx#2\@empty #4\else
   \expandafter\@xnext #2\@@#1#2#3\fi}
\def\@xnext \@elt #1#2\@@#3#4{\def#3{#1}\gdef#4{#2}}
\def\@testfalse{\global\let\if@test\iffalse}
\def\@testtrue {\global\let\if@test\iftrue}
\@testfalse
\def\@bitor#1#2{\@testfalse {\let\@elt\@xbitor
   \@tempcnta #1\relax #2}}
\def\@xbitor #1{\@tempcntb \count#1
   \ifnum \@tempcnta =\z@
   \else
     \divide\@tempcntb\@tempcnta
     \ifodd\@tempcntb \@testtrue\fi
   \fi}
\newinsert\bx@A
\newinsert\bx@B
\newinsert\bx@C
\newinsert\bx@D
\newinsert\bx@E
\newinsert\bx@F
\newinsert\bx@G
\newinsert\bx@H
\newinsert\bx@I
\newinsert\bx@J
\newinsert\bx@K
\newinsert\bx@L
\newinsert\bx@M
\newinsert\bx@N
\newinsert\bx@O
\newinsert\bx@P
\newinsert\bx@Q
\newinsert\bx@R
\gdef\@freelist{\@elt\bx@A\@elt\bx@B\@elt\bx@C\@elt\bx@D\@elt\bx@E
               \@elt\bx@F\@elt\bx@G\@elt\bx@H\@elt\bx@I\@elt\bx@J
                \@elt\bx@K\@elt\bx@L\@elt\bx@M\@elt\bx@N
                \@elt\bx@O\@elt\bx@P\@elt\bx@Q\@elt\bx@R}
\gdef\@toplist{}
\gdef\@botlist{}
\gdef\@midlist{}
\gdef\@currlist{}
\gdef\@deferlist{}
\gdef\@dbltoplist{}
\gdef\@dbldeferlist{}
\newdimen\topmargin
\newdimen\oddsidemargin
\newdimen\evensidemargin
\let\@themargin=\oddsidemargin
\newdimen\headheight
\newdimen\headsep
\newdimen\footskip
\newdimen\textheight
\newdimen\textwidth
\newdimen\columnwidth
\newdimen\columnsep
\newdimen\columnseprule
\newdimen\marginparwidth
\newdimen\marginparsep
\newdimen\marginparpush
\newbox\@begindvibox
\def \AtBeginDvi #1{%
  \global \setbox \@begindvibox
    \vbox{\unvbox \@begindvibox #1}%
}
\newdimen\@maxdepth
\@maxdepth = \maxdepth
\newdimen\paperheight
\newdimen\paperwidth
\newif \if@insert
\newif \if@fcolmade
\newif \if@specialpage \@specialpagefalse
\newif \if@firstcolumn \@firstcolumntrue
\newif \if@twocolumn   \@twocolumnfalse
\newif \if@twoside     \@twosidefalse
\newif \if@reversemargin \@reversemarginfalse
\newif \if@mparswitch  \@mparswitchfalse
\newcount \col@number
\col@number \@ne
\newcount\@topnum
\newdimen\@toproom
\newcount\@dbltopnum
\newdimen\@dbltoproom
\newcount\@botnum
\newdimen\@botroom
\newcount\@colnum
\newdimen\@textmin
\newdimen\@fpmin
\newdimen\@colht
\newdimen\@colroom
\newdimen\@pageht
\newdimen\@pagedp
\newdimen\@mparbottom \@mparbottom\z@
\newcount\@currtype
\newbox\@outputbox
\newbox\@leftcolumn
\newbox\@holdpg
\def\@thehead{\@oddhead} % initialization
\def\@thefoot{\@oddfoot}
\def\clearpage{%
  \ifvmode
    \ifnum \@dbltopnum =\m@ne
      \ifdim \pagetotal <\topskip
        \hbox{}%
      \fi
    \fi
  \fi
  \newpage
  \write\m@ne{}%
  \vbox{}%
  \penalty -\@Mi
}
\def\cleardoublepage{\clearpage\if@twoside \ifodd\c@page\else
    \hbox{}\newpage\if@twocolumn\hbox{}\newpage\fi\fi\fi}
\def\onecolumn{%
  \clearpage
  \global\columnwidth\textwidth
  \global\hsize\columnwidth
  \global\linewidth\columnwidth
  \global\@twocolumnfalse
  \col@number \@ne
  \@floatplacement}
\def \@emptycol {\vbox{}\penalty -\@M}
\def \twocolumn {%
  \clearpage
  \global\columnwidth\textwidth
  \global\advance\columnwidth-\columnsep
  \global\divide\columnwidth\tw@
  \global\hsize\columnwidth
  \global\linewidth\columnwidth
  \global\@twocolumntrue
  \global\@firstcolumntrue
  \col@number \tw@
  \@ifnextchar [\@topnewpage\@floatplacement
}
\long\def \@topnewpage [#1]{%
  \@nodocument
  \@next\@currbox\@freelist{}{}%
  \global \setbox\@currbox
    \color@vbox
      \normalcolor
      \vbox {%
        \hsize\textwidth
        \@parboxrestore
        \col@number \@ne
        #1%
        \vskip -\dbltextfloatsep
             }%
    \color@endbox
  \ifdim \ht\@currbox>\textheight
    \ht\@currbox \textheight
  \fi
  \global \count\@currbox \tw@
  \@tempdima -\ht\@currbox
  \advance \@tempdima -\dbltextfloatsep
  \global \advance \@colht \@tempdima
  \ifx \@dbltoplist \@empty
  \else
    \@latexerr{Float(s) lost}\@ehb
    \let \@dbltoplist \@empty
  \fi
  \@cons \@dbltoplist \@currbox
  \global \@dbltopnum \m@ne
  \ifdim \@colht<2.5\baselineskip
    \@latex@warning@no@line {Optional argument of \noexpand\twocolumn
                too tall on page \thepage}%
    \@emptycol
    \if@firstcolumn
    \else
      \@emptycol
    \fi
  \else
    \global \vsize \@colht
    \global \@colroom \@colht
    \@floatplacement
  \fi
}
\output {%
  \let \par \@@par
  \ifnum \outputpenalty<-\@M
    \@specialoutput
  \else
    \@makecol
    \@opcol
    \@startcolumn
    \@whilesw \if@fcolmade \fi
      {%
       \@opcol\@startcolumn}%
  \fi
  \ifnum \outputpenalty>-\@Miv
    \ifdim \@colroom<1.5\baselineskip
      \ifdim \@colroom<\textheight
        \@latex@warning@no@line {Text page \thepage\space
                               contains only floats}%
        \@emptycol
      \else
        \global \vsize \@colroom
      \fi
    \else
      \global \vsize \@colroom
    \fi
  \else
    \global \vsize \maxdimen
  \fi
}
\gdef\@specialoutput{%
   \ifnum \outputpenalty>-\@Mii
     \@doclearpage
   \else
     \ifnum \outputpenalty<-\@Miii
       \ifnum \outputpenalty<-\@MM \deadcycles \z@ \fi
       \global \setbox\@holdpg \vbox {\unvbox\@cclv}%
     \else
       \global \setbox\@holdpg \vbox{%
                      \unvbox\@holdpg
                      \unvbox\@cclv
                      \setbox\@tempboxa \lastbox
                      \unskip
                                     }%
       \@pagedp \dp\@holdpg
       \@pageht \ht\@holdpg
       \unvbox \@holdpg
       \@next\@currbox\@currlist{%
         \ifnum \count\@currbox>\z@
           \advance \@pageht \@pagedp
           \ifvoid\footins \else
             \advance \@pageht \ht\footins
             \advance \@pageht \skip\footins
             \advance \@pageht \dp\footins
           \fi
           \ifvbox \@kludgeins
             \ifdim \wd\@kludgeins=\z@
               \advance \@pageht \ht\@kludgeins
             \fi
           \fi
           \@reinserts
           \@addtocurcol
         \else
           \@reinserts
           \@addmarginpar
         \fi
         }\@latexbug
       \ifnum \outputpenalty<\z@
         \if@nobreak
           \nobreak
         \else
           \addpenalty \interlinepenalty
         \fi
       \fi
     \fi
   \fi
}
\def \@doclearpage {%
     \ifvoid\footins
       \setbox\@tempboxa\vsplit\@cclv to\z@ \unvbox\@tempboxa
       \setbox\@tempboxa\box\@cclv
       \xdef\@deferlist{\@toplist\@botlist\@deferlist}%
       \global \let \@toplist \@empty
       \global \let \@botlist \@empty
       \global \@colroom \@colht
       \ifx \@currlist\@empty
       \else
          \@latexerr{Float(s) lost}\@ehb
          \global \let \@currlist \@empty
       \fi
       \@makefcolumn\@deferlist
       \@whilesw\if@fcolmade \fi{\@opcol\@makefcolumn\@deferlist}%
       \if@twocolumn
         \if@firstcolumn
           \xdef\@dbldeferlist{\@dbltoplist\@dbldeferlist}%
           \global \let \@dbltoplist \@empty
           \global \@colht \textheight
           \begingroup
              \@dblfloatplacement
              \@makefcolumn\@dbldeferlist
              \@whilesw\if@fcolmade \fi{\@outputpage
                                        \@makefcolumn\@dbldeferlist}%
           \endgroup
         \else
           \vbox{}\clearpage
         \fi
       \fi
     \else
       \setbox\@cclv\vbox{\box\@cclv\vfil}%
       \@makecol\@opcol
       \clearpage
     \fi
}
\def \@opcol {%
  \if@twocolumn
    \@outputdblcol
  \else
    \@outputpage
  \fi
  \global \@mparbottom \z@ \global \@textfloatsheight \z@
  \@floatplacement
}
\gdef \@makecol {%
   \ifvoid\footins
     \setbox\@outputbox \box\@cclv
   \else
     \setbox\@outputbox \vbox {%
       \boxmaxdepth \@maxdepth
       \@tempdima\dp\@cclv
       \unvbox \@cclv
       \vskip \skip\footins
       \color@begingroup
         \normalcolor
         \footnoterule
         \unvbox \footins
       \color@endgroup
       }%
   \fi
   \xdef\@freelist{\@freelist\@midlist}%
   \global \let \@midlist \@empty
   \@combinefloats
   \ifvbox\@kludgeins
     \@makespecialcolbox
   \else
     \setbox\@outputbox \vbox to\@colht {%
       \@texttop
       \dimen@ \dp\@outputbox
       \unvbox \@outputbox
       \vskip -\dimen@
       \@textbottom
       }%
   \fi
   \global \maxdepth \@maxdepth
}
\gdef \@reinserts{%
  \ifvoid\footins\else\insert\footins{\unvbox\footins}\fi
  \ifvbox\@kludgeins\insert\@kludgeins
                      {\unvbox\@kludgeins}\fi
}
\gdef \@makespecialcolbox {%
   \setbox\@outputbox \vbox {%
     \@texttop
     \dimen@ \dp\@outputbox
     \unvbox\@outputbox
     \vskip-\dimen@
     }%
   \@tempdima \@colht
   \ifdim \wd\@kludgeins>\z@
     \advance \@tempdima -\ht\@outputbox
     \advance \@tempdima \pageshrink
     \setbox\@outputbox \vbox to \@colht {%
       \unvbox\@outputbox
       \vskip \@tempdima
       \@textbottom
       }%
   \else
     \advance \@tempdima -\ht\@kludgeins
     \setbox \@outputbox \vbox to \@colht {%
       \vbox to \@tempdima {%
         \unvbox\@outputbox
         \@textbottom}%
       \vss}%
   \fi
   {\setbox \@tempboxa \box \@kludgeins}%
}
\let \@texttop \relax
\let \@textbottom \relax
\def\@activechar@info #1{%
      \@latex@info@no@line {Active #1 character found while
                            output routine is active
                            \MessageBreak
                            This may be a bug in a package file
                            you are using}%
}
\begingroup
\obeylines\obeyspaces%
\catcode`\'\active%
\gdef\@resetactivechars{%
\def^^M{\@activechar@info{EOL}\space}%
\def {\@activechar@info{space}\space}%
\let'\active@math@prime}%
\endgroup
\def\@outputpage{%
\begingroup           % the \endgroup is put in by \aftergroup
  \let \protect \noexpand
  \@resetactivechars
  \global\let\@@if@newlist\if@newlist
  \global\@newlistfalse
  \@parboxrestore
  \shipout \vbox{%
    \set@typeset@protect
    \aftergroup \endgroup
    \aftergroup \set@typeset@protect
                                % correct? or just restore by ending
                                % the group?
  \if@specialpage
    \global\@specialpagefalse\@nameuse{ps@\@specialstyle}%
  \fi
  \if@twoside
    \ifodd\count\z@ \let\@thehead\@oddhead \let\@thefoot\@oddfoot
         \let\@themargin\oddsidemargin
    \else \let\@thehead\@evenhead
       \let\@thefoot\@evenfoot \let\@themargin\evensidemargin
    \fi
  \fi
  \reset@font
  \normalsize
  \normalsfcodes
  \let\label\@gobble
  \let\index\@gobble
  \let\glossary\@gobble
  \baselineskip\z@skip \lineskip\z@skip \lineskiplimit\z@
    \@begindvi
    \vskip \topmargin
    \moveright\@themargin \vbox {%
      \setbox\@tempboxa \vbox to\headheight{%
        \vfil
        \color@hbox
          \normalcolor
          \hb@xt@\textwidth{\@thehead}%
        \color@endbox
        }%                        %% 22 Feb 87
      \dp\@tempboxa \z@
      \box\@tempboxa
      \vskip \headsep
      \box\@outputbox
      \baselineskip \footskip
      \color@hbox
        \normalcolor
        \hb@xt@\textwidth{\@thefoot}%
      \color@endbox
      }%
    }%
  \global\let\if@newlist\@@if@newlist
  \global \@colht \textheight
  \stepcounter{page}%
  \let\firstmark\botmark
}
\def \@begindvi{%
  \unvbox \@begindvibox
  \global\let \@begindvi \@empty
}
\def \@combinefloats {%
    \ifx \@toplist\@empty \else \@cflt \fi
    \ifx \@botlist\@empty \else \@cflb \fi
}
\def \@cflt{%
    \let \@elt \@comflelt
    \setbox\@tempboxa \vbox{}%
    \@toplist
    \setbox\@outputbox \vbox{%
                             \boxmaxdepth \maxdepth
                             \unvbox\@tempboxa
                             \vskip -\floatsep
                             \topfigrule
                             \vskip \textfloatsep
                             \unvbox\@outputbox
                             }%
    \let\@elt\relax
    \xdef\@freelist{\@freelist\@toplist}%
    \global\let\@toplist\@empty
}
\def \@cflb {%
    \let\@elt\@comflelt
    \setbox\@tempboxa \vbox{}%
    \@botlist
    \setbox\@outputbox \vbox{%
                             \unvbox\@outputbox
                             \vskip \textfloatsep
                             \botfigrule
                             \unvbox\@tempboxa
                             \vskip -\floatsep
                             }%
    \let\@elt\relax
    \xdef\@freelist{\@freelist\@botlist}%
    \global \let \@botlist\@empty
}
\def\@comflelt#1{\setbox\@tempboxa
      \vbox{\unvbox\@tempboxa\box #1\vskip\floatsep}}
\def\@comdblflelt#1{\setbox\@tempboxa
      \vbox{\unvbox\@tempboxa\box #1\vskip\dblfloatsep}}
\def \@combinedblfloats{%
  \ifx \@dbltoplist \@empty
  \else
    \setbox\@tempboxa \vbox{}%
    \let \@elt \@comdblflelt
    \@dbltoplist
    \let \@elt \relax
    \xdef \@freelist {\@freelist\@dbltoplist}%
    \global\let \@dbltoplist \@empty
    \setbox\@outputbox \vbox to\textheight
      {%\boxmaxdepth\maxdepth   %% probably not needed, CAR
       \unvbox\@tempboxa\vskip-\dblfloatsep
       \ifnum \@dbltopnum>\m@ne
         \dblfigrule
       \fi
       \vskip \dbltextfloatsep
       \box\@outputbox
       }%
  \fi
}
\def \@startcolumn {%
  \global \@colroom \@colht
  \@tryfcolumn \@deferlist
  \if@fcolmade
  \else
    \begingroup
      \let \reserved@b \@deferlist
      \global \let \@deferlist \@empty
      \let \@elt \@scolelt
      \reserved@b
    \endgroup
  \fi
}
\def \@startdblcolumn {%
  \@tryfcolumn \@dbldeferlist
  \if@fcolmade
  \else
    \begingroup
      \let \reserved@b \@dbldeferlist
      \global \let \@dbldeferlist \@empty
      \let \@elt \@sdblcolelt
      \reserved@b
    \endgroup
  \fi
}
\def \@tryfcolumn #1{%
  \global \@fcolmadefalse
  \ifx #1\@empty
  \else
    \xdef\@trylist{#1}%
    \global \let \@failedlist \@empty
    \begingroup
      \let \@elt \@xtryfc \@trylist
    \endgroup
    \if@fcolmade
      \@vtryfc #1%
    \fi
  \fi
}
\def\@scolelt#1{\def\@currbox{#1}\@addtonextcol}
\def\@sdblcolelt#1{\def\@currbox{#1}\@addtodblcol}
\def\@vtryfc #1{%
  \global\setbox\@outputbox\vbox{}%
  \let\@elt\@wtryfc
  \@flsucceed
  \global\setbox\@outputbox \vbox to\@colht{%
    \vskip \@fptop
    \vskip -\@fpsep
    \unvbox \@outputbox
    \vskip \@fpbot}%
  \let\@elt\relax
  \xdef #1{\@failedlist\@flfail}%
  \xdef\@freelist{\@freelist\@flsucceed}}
\def\@wtryfc #1{%
  \global\setbox\@outputbox\vbox{%
    \unvbox\@outputbox
    \vskip\@fpsep
    \box #1}}
\def\@xtryfc #1{%
  \@next\reserved@a\@trylist{}{}%
  \@currtype \count #1%
  \divide\@currtype\@xxxii
  \multiply\@currtype\@xxxii
  \@bitor \@currtype \@failedlist
  \@testfp #1%
  \ifdim \ht #1>\@colht
    \@testtrue
  \fi
  \if@test
    \@cons\@failedlist #1%
  \else
    \@ytryfc #1%
  \fi}
\def\@ytryfc #1{%
  \begingroup
    \gdef\@flsucceed{\@elt #1}%
    \global\let\@flfail\@empty
    \@tempdima\ht #1%
    \let\@elt\@ztryfc
    \@trylist
    \ifdim \@tempdima >\@fpmin
      \global\@fcolmadetrue
    \else
      \@cons\@failedlist #1%
    \fi
  \endgroup
  \if@fcolmade
    \let\@elt\@gobble
  \fi}
\def\@ztryfc #1{%
  \@tempcnta \count#1%
  \divide\@tempcnta\@xxxii
  \multiply\@tempcnta\@xxxii
  \@bitor \@tempcnta {\@failedlist \@flfail}%
  \@testfp #1%
  \@tempdimb\@tempdima
  \advance\@tempdimb \ht#1%
  \advance\@tempdimb\@fpsep
  \ifdim \@tempdimb >\@colht
    \@testtrue
  \fi
  \if@test
    \@cons\@flfail #1%
  \else
    \@cons\@flsucceed #1%
    \@tempdima\@tempdimb
  \fi}
\def \@addtobot {%
   \@getfpsbit 4\relax
   \ifodd \@tempcnta
     \@flsetnum \@botnum
     \ifnum \@botnum>\z@
       \@tempswafalse
       \@flcheckspace \@botroom \@botlist
       \if@tempswa
         \global \maxdepth \z@
         \@flupdates \@botnum \@botroom \@botlist
         \@inserttrue
       \fi
     \fi
   \fi
}
\def \@addtotoporbot {%
   \@getfpsbit \tw@
   \ifodd \@tempcnta
     \@flsetnum \@topnum
     \ifnum \@topnum>\z@
       \@tempswafalse
       \@flcheckspace \@toproom \@toplist
       \if@tempswa
         \@bitor\@currtype{\@midlist\@botlist}%
         \if@test
         \else
          \@flupdates \@topnum \@toproom \@toplist
          \@inserttrue
         \fi
       \fi
     \fi
   \fi
   \if@insert
   \else
     \@addtobot
   \fi
}
\def \@addtocurcol {%
   \@insertfalse
   \@setfloattypecounts
   \ifnum \@fpstype=8
   \else
     \ifnum \@fpstype=24
     \else
       \@flsettextmin
       \advance \@textmin \@textfloatsheight
       \@reqcolroom \@pageht
       \ifdim \@textmin>\@reqcolroom
         \@reqcolroom \@textmin
       \fi
       \advance \@reqcolroom \ht\@currbox
       \ifdim \@colroom>\@reqcolroom
         \@flsetnum \@colnum
         \ifnum \@colnum>\z@
           \@bitor\@currtype\@deferlist
           \if@test
           \else
             \@bitor\@currtype\@botlist
             \if@test
               \@addtobot
             \else
               \ifodd \count\@currbox
                 \advance \@reqcolroom \intextsep
                 \ifdim \@colroom>\@reqcolroom
                   \global \advance \@colnum \m@ne
                   \global \advance \@textfloatsheight \ht\@currbox
                   \global \advance \@textfloatsheight 2\intextsep
                   \@cons \@midlist \@currbox
                   \if@nobreak
                     \nobreak
                     \@nobreakfalse
                     \everypar{}%
                   \else
                     \addpenalty \interlinepenalty
                   \fi
                   \vskip \intextsep
                   \box\@currbox
                   \penalty\interlinepenalty
                   \vskip\intextsep
                   \ifnum\outputpenalty <-\@Mii \vskip -\parskip\fi
                   \outputpenalty \z@
                   \@inserttrue
                 \fi
               \fi
               \if@insert
               \else
                 \@addtotoporbot
               \fi
             \fi
           \fi
         \fi
       \fi
     \fi
   \fi
   \if@insert
   \else
     \@resethfps
     \@cons\@deferlist\@currbox
   \fi
}
\def\@addtonextcol{%
  \begingroup
   \@insertfalse
   \@setfloattypecounts
   \ifnum \@fpstype=8
   \else
     \ifnum \@fpstype=24
     \else
       \@flsettextmin
       \@reqcolroom \ht\@currbox
       \advance \@reqcolroom \@textmin
       \ifdim \@colroom>\@reqcolroom
         \@flsetnum \@colnum
         \ifnum\@colnum>\z@
            \@bitor\@currtype\@deferlist
            \if@test
            \else
              \@addtotoporbot
            \fi
         \fi
       \fi
     \fi
   \fi
   \if@insert
   \else
     \@cons\@deferlist\@currbox
   \fi
  \endgroup
}
\def\@addtodblcol{%
  \begingroup
   \@insertfalse
   \@setfloattypecounts
   \@getfpsbit \tw@
   \ifodd\@tempcnta
     \@flsetnum \@dbltopnum
     \ifnum \@dbltopnum>\z@
       \@tempswafalse
       \ifdim \@dbltoproom>\ht\@currbox
         \@tempswatrue
       \else
         \ifnum \@fpstype<\sixt@@n
           \advance \@dbltoproom \@textmin
           \ifdim \@dbltoproom>\ht\@currbox
             \@tempswatrue
           \fi
           \advance \@dbltoproom -\@textmin
         \fi
       \fi
       \if@tempswa
           \@bitor \@currtype \@dbldeferlist
           \if@test
           \else
              \@tempdima -\ht\@currbox
              \advance\@tempdima
                -\ifx \@dbltoplist\@empty \dbltextfloatsep \else
                                          \dblfloatsep \fi
              \global \advance \@dbltoproom \@tempdima
              \global \advance \@colht \@tempdima
              \global \advance \@dbltopnum \m@ne
              \@cons \@dbltoplist \@currbox
              \@inserttrue
           \fi
       \fi
     \fi
   \fi
   \if@insert
   \else
     \@cons\@dbldeferlist\@currbox
   \fi
  \endgroup
}
\def\@addmarginpar{\@next\@marbox\@currlist{\@cons\@freelist\@marbox
    \@cons\@freelist\@currbox}\@latexbug\@tempcnta\@ne
    \if@twocolumn
        \if@firstcolumn \@tempcnta\m@ne \fi
    \else
      \if@mparswitch
         \ifodd\c@page \else\@tempcnta\m@ne \fi
      \fi
      \if@reversemargin \@tempcnta -\@tempcnta \fi
    \fi
    \ifnum\@tempcnta <\z@  \global\setbox\@marbox\box\@currbox \fi
    \@tempdima\@mparbottom
    \advance\@tempdima -\@pageht
    \advance\@tempdima\ht\@marbox
    \ifdim\@tempdima >\z@
      \@latex@warning@no@line {Marginpar on page \thepage\space moved}%
    \else
      \@tempdima\z@
    \fi
    \global\@mparbottom\@pageht
    \global\advance\@mparbottom\@tempdima
    \global\advance\@mparbottom\dp\@marbox
    \global\advance\@mparbottom\marginparpush
    \advance\@tempdima -\ht\@marbox
    \global\setbox \@marbox
                   \vbox {\vskip \@tempdima
                          \box \@marbox}%
    \global \ht\@marbox \z@
    \global \dp\@marbox \z@
    \kern -\@pagedp
    \nointerlineskip
    \hb@xt@\columnwidth
      {\ifnum \@tempcnta >\z@
          \hskip\columnwidth \hskip\marginparsep
       \else
          \hskip -\marginparsep \hskip -\marginparwidth
       \fi
       \box\@marbox \hss}%
    \nointerlineskip
    \hbox{\vrule \@height\z@ \@width\z@ \@depth\@pagedp}}
\newinsert \@kludgeins
\global\dimen\@kludgeins \maxdimen
\global\count\@kludgeins 1000
\gdef \enlargethispage {%
   \@ifstar
     {%
      \@enlargepage{\hbox{\kern\p@}}}%
     {%
      \@enlargepage\@empty}%
}
\gdef\@enlargepage#1#2{%
   \@tempskipa#2\relax
   \ifdim \@tempskipa>.5\maxdimen
     \@latexerr{Suggested\space extra\space height\space
                (\the\@tempskipa)\space dangerously\space
                large}\@eha
   \else
     \ifdim \vsize<.5\maxdimen
       \@bsphack
         \insert\@kludgeins{#1\vskip-\@tempskipa}%
       \@esphack
     \else
       \@latexerr{Page\space height\space already\space
                  too\space large}\@eha
     \fi
   \fi
}
\def \suppressfloats {%
   \@ifnextchar [%
     \@flstop
    {\global \@colnum \z@}%
}
\def \@flstop [#1]{%
   \if t#1%
     \global \@topnum \z@
   \fi
   \if b#1%
     \global \@botnum \z@
   \fi
}
\newcount \@fpstype
\newdimen \@reqcolroom
\newdimen \@textfloatsheight
\def \@fpsadddefault {%
   \edef \@fps {\@fps\csname fps@\@captype \endcsname}%
   \@latex@warning {%
     No positions in optional float specifier.\MessageBreak
     Default added (so using `\@fps')}%
}
\def \@setfloattypecounts {%
  \@currtype \count\@currbox
  \@fpstype \count\@currbox
  \divide\@currtype\@xxxii \multiply\@currtype\@xxxii
  \advance \@fpstype -\@currtype
}
\def \@getfpsbit {%
   \@boxfpsbit \@currbox
}
\def \@boxfpsbit #1#2{%
   \@tempcnta \count#1%
   \divide \@tempcnta #2\relax
}
\def \@testfp #1{%
   \@boxfpsbit #18\relax % Really `#1 8' for human readers!
   \ifodd \@tempcnta
   \else
     \@testtrue
   \fi
}
\def \@setfpsbit #1{%
   \@tempcntb \@tempcnta
   \divide \@tempcntb #1\relax
   \ifodd \@tempcntb
   \else
     \advance \@tempcnta #1\relax
   \fi
}
\def \@resethfps {%
   \let\reserved@a\@empty
   \ifnum \@fpstype=\@ne
      \def \reserved@a {!}%
      \@fpstype 17
   \fi
   \ifnum \@fpstype=17
     \global \advance \count\@currbox \tw@
     \@latex@warning@no@line {%
       `\reserved@a h' float specifier changed to `\reserved@a ht'}%
   \fi
}
\def \@flsetnum #1{%
   \ifnum \@fpstype<\sixt@@n
     \ifnum #1=\z@
       #1\@ne
     \fi
   \fi
}
\def \@flsettextmin {%
   \ifnum \@fpstype<\sixt@@n
     \@textmin \z@
   \else
     \@textmin \textfraction\@colht
   \fi
}
\def \@flcheckspace #1#2{%
   \advance \@reqcolroom
     \ifx #2\@empty \textfloatsep \else \floatsep \fi
   \ifdim \@colroom>\@reqcolroom

     \ifdim #1>\ht\@currbox
       \@tempswatrue
     \else
       \ifnum \@fpstype<\sixt@@n
         \@tempswatrue
       \fi
     \fi
   \fi
}
\def \@flupdates #1#2#3{%
   \global \advance #1\m@ne
   \global \advance \@colnum \m@ne
   \@tempdima -\ht\@currbox
   \advance \@tempdima
     -\ifx #3\@empty \textfloatsep \else \floatsep \fi
   \global \advance #2\@tempdima
   \global \advance \@colroom \@tempdima
   \@cons #3\@currbox
}
\def\@makefcolumn #1{%
  \begingroup
    \@fpmin \z@
    \let \@testfp \@gobble
    \@tryfcolumn #1%
  \endgroup
}
\def\@outputdblcol{%
  \if@firstcolumn
    \global \@firstcolumnfalse
    \global \setbox\@leftcolumn \box\@outputbox
  \else
    \global \@firstcolumntrue
    \setbox\@outputbox \vbox {%
                         \hb@xt@\textwidth {%
                           \hb@xt@\columnwidth {%
                             \box\@leftcolumn \hss}%
                           \hfil
                           {\normalcolor\vrule \@width\columnseprule}%
                           \hfil
                           \hb@xt@\columnwidth {%
                             \box\@outputbox \hss}%
                                             }%
                              }%
    \@combinedblfloats
    \@outputpage
    \begingroup
      \@dblfloatplacement
      \@startdblcolumn
      \@whilesw\if@fcolmade \fi
        {\@outputpage
         \@startdblcolumn}%
    \endgroup
  \fi
}
\newcount\c@topnumber
\setcounter{topnumber}{2}
\newcommand\topfraction{.7}
\newcount\c@bottomnumber
\setcounter{bottomnumber}{1}
\newcommand\bottomfraction{.3}
\newcount\c@totalnumber
\setcounter{totalnumber}{3}
\newcommand\textfraction{.2}
\newcommand\floatpagefraction{.5}
\newcount\c@dbltopnumber
\setcounter{dbltopnumber}{2}
\newcommand\dbltopfraction{.7}
\newcommand\dblfloatpagefraction{.5}
\newskip\floatsep
\newskip\textfloatsep
\newskip\intextsep
\newskip\dblfloatsep
\newskip\dbltextfloatsep
\newskip\@fptop
\newskip\@fpsep
\newskip\@fpbot
\newskip\@dblfptop
\newskip\@dblfpsep
\newskip\@dblfpbot
\let\topfigrule=\relax
\let\botfigrule=\relax
\let\dblfigrule=\relax
%%% From File: ltclass.dtx
\newif\if@compatibility
\def\@documentclasshook{%
   \ifx\@normalsize\@undefined
      \let\@normalsize\normalsize
   \fi
}
\let\@declaredoptions\@empty
\let\@classoptionslist\relax
\@onlypreamble\@classoptionslist
\let\@unusedoptionlist\@empty
\@onlypreamble\@unusedoptionlist
\let\CurrentOption\@empty
\let\@currname\@empty
\global\let\@currext=\@empty
\def\@clsextension{cls}
\def\@pkgextension{sty}
\@onlypreamble\@clsextension
\@onlypreamble\@pkgextension
\@onlypreamble\@pushfilename
\@onlypreamble\@popfilename
\@onlypreamble\@p@pfilename
\gdef\@currnamestack{}
\@onlypreamble\@currnamestack

\def\@ptionlist#1{%
  \@ifundefined{opt@#1}\@empty{\csname opt@#1\endcsname}}

\@onlypreamble\@ptionlist
\def\@ifpackageloaded{\@ifl@aded\@pkgextension}
\def\@ifclassloaded{\@ifl@aded\@clsextension}
\@onlypreamble\@ifpackageloaded
\@onlypreamble\@ifclassloaded

\def\@ifl@aded#1#2{%
  \expandafter\ifx\csname ver@#2.#1\endcsname\relax
    \let\reserved@a\@secondoftwo
  \else
    \let\reserved@a\@firstoftwo
  \fi
  \reserved@a}

\@onlypreamble\@ifl@aded
\def\@ifpackagelater{\@ifl@ter\@pkgextension}
\def\@ifclasslater{\@ifl@ter\@clsextension}
\@onlypreamble\@ifpackagelater
\@onlypreamble\@ifclasslater
\def\@ifl@ter#1#2{%
  \expandafter\@ifl@t@r
    \csname ver@#2.#1\endcsname}
\@onlypreamble\@ifl@ter
\def\@ifl@t@r#1#2{%
  \ifnum\expandafter\@parse@version#1//00\@nil<%
        \expandafter\@parse@version#2//00\@nil
    \expandafter\@secondoftwo
  \else
    \expandafter\@firstoftwo
  \fi}
\@onlypreamble\@ifl@t@r
\def\@parse@version#1/#2/#3#4#5\@nil{#1#2#3#4 }
\@onlypreamble\@parse@version
\def\@ifpackagewith{\@if@ptions\@pkgextension}
\def\@ifclasswith{\@if@ptions\@clsextension}
\@onlypreamble\@ifpackagewith
\@onlypreamble\@ifclasswith
\def\@if@ptions#1#2{%
  \@expandtwoargs\@if@pti@ns{\@ptionlist{#2.#1}}}
\@onlypreamble\@if@ptions
\def\@if@pti@ns#1#2{%
  \let\reserved@a\@firstoftwo
  \@for\reserved@b:=#2\do{%
    \expandafter\in@\expandafter{\expandafter,\reserved@b,}{,#1,}%
    \ifin@\else\let\reserved@a\@secondoftwo\fi}%
  \reserved@a}
\@onlypreamble\@if@pti@ns
\def\ProvidesPackage#1{%
  \xdef\@gtempa{#1}%
  \ifx\@gtempa\@currname\else
    \@latex@warning@no@line{You have requested
      \@cls@pkg\space`\@currname',\MessageBreak
       but the \@cls@pkg\space provides `#1'}%
  \fi
  \@ifnextchar[\@pr@videpackage{\@pr@videpackage[]}}%]
\@onlypreamble\ProvidesPackage
\def\@pr@videpackage[#1]{%
  \expandafter\xdef\csname ver@\@currname.\@currext\endcsname{#1}%
  \ifx\@currext\@clsextension
    \typeout{Document Class: \@gtempa\space#1}%
  \else
    \wlog{Package: \@gtempa\space#1}%
  \fi}
\@onlypreamble\@pr@videpackage
\let\ProvidesClass\ProvidesPackage
\@onlypreamble\ProvidesClass
\def\ProvidesFile#1{%
  \begingroup
    \catcode`\ 10 %
    \ifnum \endlinechar<256 %
      \ifnum \endlinechar>\m@ne
        \catcode\endlinechar 10 %
      \fi
    \fi
    \@makeother\/%
    \@makeother\&%
    \@ifnextchar[{\@providesfile{#1}}{\@providesfile{#1}[]}}
\def\@pass@ptions#1#2#3{%
  \expandafter\xdef\csname opt@#3.#1\endcsname{%
    \@ifundefined{opt@#3.#1}\@empty
      {\csname opt@#3.#1\endcsname,}%
    \zap@space#2 \@empty}}
\@onlypreamble\@pass@ptions
\def\PassOptionsToPackage{\@pass@ptions\@pkgextension}
\def\PassOptionsToClass{\@pass@ptions\@clsextension}
\@onlypreamble\PassOptionsToPackage
\@onlypreamble\PassOptionsToClass
\def\DeclareOption{%
  %\let\@fileswith@pti@ns\@badrequireerror
  \@ifstar\@defdefault@ds\@declareoption}
\long\def\@declareoption#1#2{%
   \xdef\@declaredoptions{\@declaredoptions,#1}%
   \toks@{#2}%
   \expandafter\edef\csname ds@#1\endcsname{\the\toks@}}
\long\def\@defdefault@ds#1{%
  \toks@{#1}%
  \edef\default@ds{\the\toks@}}
\@onlypreamble\DeclareOption
\@onlypreamble\@declareoption
\@onlypreamble\@defdefault@ds
\def\OptionNotUsed{%
  \ifx\@currext\@clsextension
    \xdef\@unusedoptionlist{%
      \ifx\@unusedoptionlist\@empty\else\@unusedoptionlist,\fi
      \CurrentOption}%
  \fi}
\@onlypreamble\OptionNotUsed
\def\ProcessOptions{%
  \let\ds@\@empty
  \edef\@curroptions{\@ptionlist{\@currname.\@currext}}%
  \@ifstar\@xprocess@ptions\@process@ptions}
\@onlypreamble\ProcessOptions
\def\@process@ptions{%
  \@for\CurrentOption:=\@declaredoptions\do{%
    \ifx\CurrentOption\@empty\else
      \@expandtwoargs\in@{,\CurrentOption,}{%
         ,\ifx\@currext\@clsextension\else\@classoptionslist,\fi
         \@curroptions,}%
      \ifin@
        \@use@ption
        \expandafter\let\csname ds@\CurrentOption\endcsname\@empty
      \fi
    \fi}%
  \@process@pti@ns}
\@onlypreamble\@process@ptions
\def\@xprocess@ptions{%
  \ifx\@currext\@clsextension\else
    \@for\CurrentOption:=\@classoptionslist\do{%
      \ifx\CurrentOption\@empty\else
        \@expandtwoargs\in@{,\CurrentOption,}{,\@declaredoptions,}%
        \ifin@
          \@use@ption
          \expandafter\let\csname ds@\CurrentOption\endcsname\@empty
        \fi
      \fi}%
  \fi
  \@process@pti@ns}
\@onlypreamble\@xprocess@ptions
\def\@process@pti@ns{%
  \@for\CurrentOption:=\@curroptions\do{%
    \@ifundefined{ds@\CurrentOption}%
      {\@use@ption
       \default@ds}%
      \@use@ption}%
  \@for\CurrentOption:=\@declaredoptions\do{%

    \expandafter\let\csname ds@\CurrentOption\endcsname\relax}%
  \let\CurrentOption\@empty
  \let\@fileswith@pti@ns\@@fileswith@pti@ns
  \AtEndOfPackage{\let\@unprocessedoptions\relax}}
\@onlypreamble\@process@pti@ns
\def\@options{\ProcessOptions*}
\@onlypreamble\@options
\def\@use@ption{%
  \@expandtwoargs\@removeelement\CurrentOption
  \@unusedoptionlist\@unusedoptionlist
  \csname ds@\CurrentOption\endcsname}
\@onlypreamble\@use@ption
\def\ExecuteOptions#1{%
  \def\reserved@a##1\@nil{%
    \@for\CurrentOption:=#1\do{\csname ds@\CurrentOption\endcsname}%
    \edef\CurrentOption{##1}}%
  \expandafter\reserved@a\CurrentOption\@nil}
\@onlypreamble\ExecuteOptions

\def\LoadClass{%
  \ifx\@currext\@pkgextension
     \@latex@error
      {\noexpand\LoadClass in package file}%
      {You may only use \noexpand\LoadClass in a class file.}%
  \fi
  \@fileswithoptions\@clsextension}
\@onlypreamble\LoadClass
\def\@loadwithoptions#1#2#3{%
  \expandafter\let\csname opt@#3.#1\expandafter\endcsname
       \csname opt@\@currname.\@currext\endcsname
   #2{#3}}
\@onlypreamble\@loadwithoptions
\def\LoadClassWithOptions{%
  \@loadwithoptions\@clsextension\LoadClass}
\@onlypreamble\LoadClassWithOptions
\def\RequirePackageWithOptions{%
  \AtEndOfPackage{\let\@unprocessedoptions\relax}%
  \@loadwithoptions\@pkgextension\RequirePackage}
\@onlypreamble\RequirePackageWithOptions
%%jcs \let\usepackage\RequirePackage
\@onlypreamble\usepackage
\def\NeedsTeXFormat#1{%
  \def\reserved@a{#1}%
  \ifx\reserved@a\fmtname
    \expandafter\@needsformat
  \else
     \@latex@error{This file needs format `\reserved@a'%
       \MessageBreak but this is `\fmtname'}{%
       The current input file will not be processed
       further,\MessageBreak
       because it was written for some other flavor of
       TeX.\MessageBreak\@ehd}%
     \endinput \fi}
\@onlypreamble\NeedsTeXFormat
\def\@needsformat{%
  \@ifnextchar[%]
    \@needsf@rmat
    {}}
\@onlypreamble\@needsformat
\def\@needsf@rmat[#1]{%
    \@ifl@t@r\fmtversion{#1}{}%
    {\@latex@warning@no@line
        {You have requested release `#1' of LaTeX,\MessageBreak
         but only release `\fmtversion' is available}}}
\@onlypreamble\@needsf@rmat
\def\zap@space#1 #2{%
  #1%
  \ifx#2\@empty\else\expandafter\zap@space\fi
  #2}
\def\@fileswithoptions#1{%
  \@ifnextchar[%]
    {\@fileswith@ptions#1}%
    {\@fileswith@ptions#1[]}}
\@onlypreamble\@fileswithoptions
\def\@fileswith@ptions#1[#2]#3{%
  \@ifnextchar[%]
  {\@fileswith@pti@ns#1[#2]#3}%
  {\@fileswith@pti@ns#1[#2]#3[]}}
\@onlypreamble\@fileswith@ptions
\def\@fileswith@pti@ns#1[#2]#3[#4]{%
  \ifx#1\@clsextension
    \ifx\@classoptionslist\relax
      \xdef\@classoptionslist{\zap@space#2 \@empty}%
      \def\reserved@a{%
        \@onefilewithoptions#3[#2][#4]#1%
        \@documentclasshook}%
    \else
      \def\reserved@a{%
        \@onefilewithoptions#3[#2][#4]#1}%
    \fi
  \else
    \def\reserved@b##1,{%
      \ifx\@nil##1\relax\else
        \ifx\relax##1\relax\else
         \noexpand\@onefilewithoptions##1[#2][#4]\noexpand\@pkgextension
        \fi
        \expandafter\reserved@b
      \fi}%
      \edef\reserved@a{\zap@space#3 \@empty}%
      \edef\reserved@a{\expandafter\reserved@b\reserved@a,\@nil,}%
  \fi
  \reserved@a}
\@onlypreamble\@fileswith@pti@ns
\def\@onefilewithoptions#1[#2][#3]#4{%
  \xdef\@currname{#1}%
  \global\let\@currext#4%
  \expandafter\let\csname\@currname.\@currext-h@@k\endcsname\@empty
  \let\CurrentOption\@empty
  \@reset@ptions
  \makeatletter
  \def\reserved@a{%
    \@ifl@aded\@currext{#1}%
      {\@if@ptions\@currext{#1}{#2}{}{}}%
      {\@pass@ptions\@currext{#2}{#1}%
       \global\expandafter\let\csname ver@\@currname.\@currext\endcsname\@empty
       \InputIfFileExists
         {\@currname.\@currext}%
         {}%
         {}%
         \let\@unprocessedoptions\@@unprocessedoptions
         \csname\@currname.\@currext-h@@k\endcsname
         \expandafter\let\csname\@currname.\@currext-h@@k\endcsname\@undefined}
    \@reset@ptions}%
  \reserved@a}
\@onlypreamble\@onefilewithoptions
\let\@@fileswith@pti@ns\@fileswith@pti@ns
\@onlypreamble\@@fileswith@pti@ns
\def\@reset@ptions{%
  \global\ifx\@currext\@clsextension
    \let\default@ds\OptionNotUsed
   \else
    \let\default@ds\@unknownoptionerror
  \fi
  \global\let\ds@\@empty
  \global\let\@declaredoptions\@empty}
\@onlypreamble\@reset@ptions
\ifx\@begindocumenthook\@undefined
  \let\@begindocumenthook\@empty
\fi
\let\@enddocumenthook\@empty
\long\def\g@addto@macro#1#2{%
  \begingroup
    \toks@\expandafter{#1#2}%
    \xdef#1{\the\toks@}%
  \endgroup}
\def\AtEndOfPackage{%
  \expandafter\g@addto@macro\csname\@currname.\@currext-h@@k\endcsname}
\let\AtEndOfClass\AtEndOfPackage
\@onlypreamble\AtEndOfPackage
\@onlypreamble\AtEndOfClass
\def\AtBeginDocument{\g@addto@macro\@begindocumenthook}
\def\AtEndDocument{\g@addto@macro\@enddocumenthook}
\@onlypreamble\AtBeginDocument
\def\@cls@pkg{%
  \ifx\@currext\@clsextension
    document class%
  \else
    package%
  \fi}
\@onlypreamble\@cls@pkg

\def\@unknownoptionerror{%
  \@latex@error
    {Unknown option `\CurrentOption' for \@cls@pkg\space`\@currname'}%
    {The option `\CurrentOption' was not declared in
     \@cls@pkg\space`\@currname', perhaps you\MessageBreak
      misspelled its name.
     Try typing \space <return>
     \space to proceed.}}
\@onlypreamble\@unknownoptionerror

\def\@@unprocessedoptions{%
  \ifx\@currext\@pkgextension
    \edef\@curroptions{\@ptionlist{\@currname.\@currext}}%
    \@for\CurrentOption:=\@curroptions\do{%
        \ifx\CurrentOption\@empty\else\@unknownoptionerror\fi}%
  \fi}

\@onlypreamble\@unprocessedoptions
\@onlypreamble\@@unprocessedoptions
\def\@badrequireerror#1[#2]#3[#4]{%
  \@latex@error
    {\noexpand\RequirePackage or \noexpand\LoadClass
         in Options Section}%
    {The \@cls@pkg\space `\@currname' is defective.\MessageBreak
     It attempts to load `#3' in the options section, i.e.,\MessageBreak
     between \noexpand\DeclareOption and \string\ProcessOptions.}}
\@onlypreamble\@badrequireerror
\def\@twoloadclasserror{%
  \@latex@error
    {Two \noexpand\LoadClass commands}%
    {You may only use one \noexpand\LoadClass in a class file}}
\@onlypreamble\@twoloadclasserror
\def\@twoclasseserror#1#{%
  \@latex@error
    {Two \noexpand\documentclass or \noexpand\documentstyle commands}%
    {The document may only declare one class.}\@gobble}
\@onlypreamble\@twoclasseserror
\def\two@digits#1{\ifnum#1<10 0\fi\number#1}
\newcount\@lowpenalty
\newcount\@medpenalty
\newcount\@highpenalty
\unitlength = 1pt
\fboxsep = 3pt
\fboxrule = .4pt
\@maxdepth       = \maxdepth
\vsize = 1000pt
\@colroom = \vsize
\@colht = \vsize
\textheight=.5\maxdimen
\textwidth=\textheight


\def\reserved@a#1#2{%
   \@tempcnta#1\relax
   \@tempcntb#2\relax
   \reserved@b
}
\def\reserved@b{%
   \ifnum\@tempcnta>\@tempcntb\else
      \reserved@c\@tempcnta
      \advance\@tempcnta\@ne
      \expandafter\reserved@b
   \fi
}
\def\reserved@c#1{%
   \count@=#1\advance\count@ by -"20
   \uccode#1=\count@
   \lccode#1=#1
}
\reserved@a{`\a}{`\z}
\ifnum\inputlineno=\m@ne\else
  \reserved@a{"A0}{"BC}
  \reserved@a{"E0}{"FF}
\fi
\def\reserved@c#1{%
   \count@=#1\advance\count@ by "20
   \uccode#1=#1
   \lccode#1=\count@
   \sfcode#1=999
}
\reserved@a{`\A}{`\Z}
\ifnum\inputlineno=\m@ne\else
  \reserved@a{"80}{"9C}
  \reserved@a{"C0}{"DF}
\fi
\uccode`\^^Y=`\I     % dotless i
\lccode`\^^Y=`\^^Y   % dotless i
\uccode`\^^Z=`\J     % dotless j, ae in OT1
\lccode`\^^Z=`\^^Z   % dotless j, ae in OT1
\ifnum\inputlineno=\m@ne\else
  \lccode`\^^9d=`\i    % dotted I
  \uccode`\^^9d=`\^^9d % dotted I
  \lccode`\^^9e=`\^^9e % d-bar
  \uccode`\^^9e=`\^^d0 % d-bar
\fi
\lccode`\^^[=`\^^[   % oe in OT1
\InputIfFileExists{hyphen.cfg}
           {\typeout{===========================================^^J%
                      Local configuration file hyphen.cfg used^^J%
                     ===========================================}%
             \def\@addtofilelist##1{\xdef\@filelist{\@filelist,##1}}%
           }
           {\input{hyphen.ltx}}
\let\@addtofilelist\@gobble
\ifdim \font@submax >\z@
   \@font@warning{Size substitutions with differences\MessageBreak
                 up to \font@submax\space have occurred.\MessageBreak
                \MessageBreak
                Please check the transcript file
                carefully\MessageBreak
                and redo the format generation if necessary!
                \@gobbletwo}%
   \errhelp{Only stopped, to give you time to
            read the above message.}
   \errmessage{}
\def\font@submax{0pt}
\fi
\def\reserved@a#1#2{%
   \@tempcnta#1\relax
   \@tempcntb#2\relax
   \reserved@b
}
\def\reserved@b{%
   \ifnum\@tempcnta>\@tempcntb\else
      \reserved@c\@tempcnta
      \advance\@tempcnta\@ne
      \expandafter\reserved@b
   \fi
}
\catcode`\ =10
\catcode`\#=6
\catcode`\$=3
\catcode`\%=14
\catcode`\&=4
\catcode`\\=0
\catcode`\^=7
\catcode`\_=8
\catcode`\{=1
\catcode`\}=2
\catcode`\~=13
\catcode`\@=11
\catcode`\^^I=10
\catcode`\^^J=12
\catcode`\^^L=13
\catcode`\^^M=5
\def\reserved@c#1{\catcode#1=12\relax}
\reserved@c{`\!}
\reserved@c{`\"}
\reserved@a{`\'}{`\?}
\reserved@c{`\[}
\reserved@c{`\]}
\reserved@c{`\`}
\reserved@c{`\|}
\def\reserved@c#1{\catcode#1=11\relax}
\reserved@a{`\A}{`\Z}
\reserved@a{`\a}{`\z}
\def\reserved@c#1{\catcode#1=15\relax}
\reserved@a{0}{`\^^H}
\reserved@c{`\^^K}
\reserved@a{`\^^N}{31}
  \catcode"7F=15
\def\reserved@c#1{%
   \count@=#1\advance\count@ by -"20
   \uccode#1=\count@
   \lccode#1=#1
}
\reserved@a{`\a}{`\z}
\ifnum\inputlineno=\m@ne\else
  \reserved@a{"A0}{"BC}
  \reserved@a{"E0}{"FF}
\fi
\def\reserved@c#1{%
   \count@=#1\advance\count@ by "20
   \uccode#1=#1
   \lccode#1=\count@
   \sfcode#1=999
}
\reserved@a{`\A}{`\Z}
\ifnum\inputlineno=\m@ne\else
  \reserved@a{"80}{"9C}
  \reserved@a{"C0}{"DF}
\fi
\uccode`\^^Y=`\I     % dotless i
\lccode`\^^Y=`\^^Y   % dotless i
\uccode`\^^Z=`\J     % dotless j, ae in OT1
\lccode`\^^Z=`\^^Z   % dotless j, ae in OT1
\ifnum\inputlineno=\m@ne\else
  \lccode`\^^9d=`\i    % dotted I
  \uccode`\^^9d=`\^^9d % dotted I
  \lccode`\^^9e=`\^^9e % d-bar
  \uccode`\^^9e=`\^^d0 % d-bar
\fi
\lccode`\^^[=`\^^[   % oe in OT1
\DeclareRobustCommand{\MakeUppercase}[1]{{%
      \def\i{I}\def\j{J}%
      \def\reserved@a##1##2{\let##1##2\reserved@a}%
      \expandafter\reserved@a\@uclclist\reserved@b{\reserved@b\@gobble}%
      \protected@edef\reserved@a{\uppercase{#1}}%
      \reserved@a
   }}
\DeclareRobustCommand{\MakeLowercase}[1]{{%
      \def\reserved@a##1##2{\let##2##1\reserved@a}%
      \expandafter\reserved@a\@uclclist\reserved@b{\reserved@b\@gobble}%
      \protected@edef\reserved@a{\lowercase{#1}}%
      \reserved@a
   }}
\def\@uclclist{\oe\OE\o\O\ae\AE
      \dh\DH\dj\DJ\l\L\ng\NG\ss\SS\th\TH}
\protected@edef\MakeUppercase#1{\MakeUppercase{#1}}
\protected@edef\MakeLowercase#1{\MakeLowercase{#1}}

\let\reserved@a\@filelist
\let\reserved@b=\@undefined
\let\reserved@c=\@undefined
\let\reserved@d=\@undefined
\let\reserved@e=\@undefined
\let\reserved@f=\@undefined
\toks0{}
\toks2{}
\toks4{}
\toks6{}
\toks8{}
\errhelp{}
\def\@providesfile#1[#2]{%
    \wlog{File: #1 #2}%
    \expandafter\xdef\csname ver@#1\endcsname{#2}%
  \endgroup}
\let\@filelist\@gobble
\def\@addtofilelist#1{\xdef\@filelist{\@filelist,#1}}%


%
%    \FRAME{ framedata f|i tbph x F|T }              %#1
%          { contentswidth (scalar)  }               %#2
%          { contentsheight (scalar) }               %#3
%          { vertical shift when in-line (scalar) }  %#4
%          { caption }                               %#5
%          { label }                                 %#6
%          { name }                                  %#7
%          { body }                                  %#8
%
%    framedata is a string which can contain the following
%    characters: idftbphxFT
%    Their meaning is as follows:
%             i, d or f : in-line, display, or floating
%             t,b,p,h   : LaTeX floating placement options
%             x         : fit contents box to contents
%             F or T    : Figure or Table. 
%                         Later this can expand
%                         to a more general float class.
%
%
\newcount\dispkind%

\def\makeactives{
  \catcode`\"=\active
  \catcode`\;=\active
  \catcode`\:=\active
  \catcode`\'=\active
  \catcode`\~=\active
}
\bgroup
   \makeactives
   \gdef\activesoff{%
      \def"{\string"}%
      \def;{\string;}%
      \def:{\string:}%
      \def'{\string'}%
      \def~{\string~}%
      %\bbl@deactivate{"}%
      %\bbl@deactivate{;}%
      %\bbl@deactivate{:}%
      %\bbl@deactivate{'}%
    }
\egroup






\def\U#1{\ifmmode\string\U{#1}\else\char`\&\char`\#x#1;\fi}

\def\mathring#1{\if A#1\U{212B}\fi}

%% \def\msihyperref#1#2#3#4
%% Replace with contortion to convert & in parameter 4

\begingroup
   \catcode`\&=13%
   \gdef&{\ifmmode\char`\&\else\char`\&amp;\fi}
\endgroup
  
\def\msihyperref#1#2#3{%
     %\def\continue{\msihyperrefB{#1}{#2}{#3}}%
     \begingroup
     \catcode`\&=13%
     %\afterassignment\continue
     %\let\repl
     \msihyperrefB{#1}{#2}{#3}%
}

\def\msihyperrefB#1#2#3#4{%
     %\xdef\repl{#4}%
     %\catcode`\&=4\relax
     \endgroup
     \out@hyperref{#1}{#2}{#3}{#4}}%



%%\def\hyperref#1#2#3{%
%%  \begingroup
%%     \catcode`\&=13%
%%     \msihyperrefB{#1}{#2}{#3}%
%%}

\let\hyperref\msihyperref




\def\CustomNote{%
   \@ifnextchar[{\out@CustomNoteA}{\out@CustomNoteB}}%
   

\makeatother
