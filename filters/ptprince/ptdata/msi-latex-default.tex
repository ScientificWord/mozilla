

% msi-latex-default.tex
%
% This file should contain default definitions of all common LaTeX commands
% except for "tricky" ones that are covered elsewhere [1]. This file will always
% be read by Prince's input filter so anything in here will be defined for all
% imported TeX documents. Class and package files (.cls and .sty) may override
% any or all of this.
%
% In cases where the same command is commonly used in two different ways
% pick one of them to be the default and place its definition here. Place the other
% common definition in an appropriate cls file.


% Use MSIEmptyElement to convert
% From  \X
% To    <X/>
%
% For example, \MSIEmptyElement{maketitle} cause the input filter to translate
% \maketitle   to   <maketitle/>
%
% There is an optional parameter to set the XML name if it should be different
% fom the LaTeX name.

\MSIEmptyElement[italic-correction]{/}
\MSIEmptyElement{maketitle}
\MSIEmptyElement{tableofcontents}
\MSIEmptyElement{listoffigures}

\MSIEmptyElement{listoftables}
\MSIEmptyElement{frontmatter}
\MSIEmptyElement{backmatter}
\MSIEmptyElement{appendix}
\MSIEmptyElement{mainmatter}
\MSIEmptyElement[leftquote]{textquotedblleft}
\MSIEmptyElement[rightquote]{textquotedblright}
\MSIEmptyElement{textbackslash}





% Use MSISimpleEnvironment when you want to convert a
% from   \begin{X}...\end{X}
% to     <X>...</X>
%
% This macro takes an optional parameter which is an XML name
% in case you want it to be different from the LaTeX name.
% E.g. you could write \MSISimpleEnvironment[ABSTRACT]{abstract}
% and then you'd get <ABSTRACT> ... <ABSTRACT> in the output
 
\MSISimpleEnvironment{abstract}
\MSISimpleEnvironment{quote}
\MSISimpleEnvironment{quotation}
\MSISimpleEnvironment{center}
\MSISimpleEnvironment{flushleft}
\MSISimpleEnvironment{flushright}
\MSISimpleEnvironment{subequations}
\MSISimpleEnvironment{proof}




% Use \MSITextElement to create translations like
% From  \textbf{...}
% To    <bold> ... </bold>

\MSITextElement[bold]{textbf}
\MSITextElement[it]{textit}
\MSITextElement[sf]{textsf}
\MSITextElement[sl]{textsl}
\MSITextElement[tt]{texttt}
\MSITextElement[rm]{textrm}
\MSITextElement[sc]{textsc}
\MSITextElement[em]{emph}






% Use \MSIFontSwitch to create definitions for things like
% \it, \bf, \tiny. These are not so common in the LaTeX2e
% world. Most docs these days will use things like 
% \textbf{...} instead. A text tag has syntax like
%  .... \bf .....  and it is not always easy to see how to 
% translate into XML because LaTeX does not say when to end 
% the markup. The filter will generate <bold> ... </bold>.

\MSIFontSwitch{it}
\MSIFontSwitch[bold]{bf}
\MSIFontSwitch{rm}
\MSIFontSwitch{normalsize}
\MSIFontSwitch{tiny}
\MSIFontSwitch{small}
\MSIFontSwitch{Large}
\MSIFontSwitch{large}
\MSIFontSwitch{LARGE}
\MSIFontSwitch{huge}
\MSIFontSwitch{Huge}
\MSIFontSwitch{scriptsize}
\MSIFontSwitch{footnotesize}





