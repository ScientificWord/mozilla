\makeatletter

\newcounter{MaxMatrixCols}

\msi@inpreamble = 0
\msi@insubequations=0
\msi@preamble={}

\def\out@preamble#1{}
%   \msitag{^^0a<preambleTeX id="}\the\preambleid\msitag{"><![CDATA[^^0a%
%   \usepackage{bm}^^0a\usepackage{tabulary}}%  
%   \msitag{#1]]> </preambleTeX>}%
%   \msi@preamble={}}



\def\msi@rfi#1#2{\def#1{\msitag{<requestimplementation>#2</requestimplementation>}}}

\begingroup\catcode`\"=\active\relax
\gdef"{'}
\endgroup

\def\href{
    \begingroup
    \catcode`\~=\other
    \catcode`\"=\active
    \attrib@text
    \@href}


\def\@href#1#2{%
   \ifmmode
      \msitag{\msipassthru{<a href="#1">#2</a>}}%
   \else
      \msihmode\msiopentag{a}{<a href="}#1\msitag{">}#2\msiclosetag{a}{</a>}%
   \fi
   \endgroup}


% This writes the css file my.css. 
% Do this at the end because it happens that we don't know what css are required. 
 
\def\msimakecss{
   \def\css{2}	  % the css output file
   \openout\css={css/my.css} %% the CSS style file for this document
   \write\css{@import url("resource://app/res/css/latex_internal.css");}
   \write\css{@import url("resource://app/res/css/baselatex.css");}
   \write\css{@import url("resource://app/res/css/latex.css");}
   \ifx\classcss\@empty
     \write\css{@import url("resource://app/res/css/article.css");}
   \else
     \write\css{\classcss}
   \fi
   \write\css{\newtheorems}
   \write\css{body{}}

   \write\css{sectiontitle{}}
   \write\css{part>sectiontitle{}}
   \write\css{part>sectiontitle:before{}}
   \write\css{chapter>sectiontitle{}}
   \write\css{chapter>sectiontitle:before{}}
   \write\css{section>sectiontitle{}}  
   \write\css{section>sectiontitle:before{}}
   \write\css{subsection>sectiontitle{}}
   \write\css{subsection>sectiontitle:before{}}
   \write\css{subsubsection>sectiontitle{}}
   \write\css{subsubsection>sectiontitle:before{}}
   \write\css{paragraph>sectiontitle{}}
   \write\css{paragraph>sectiontitle:before{}}
   \write\css{bodyMath{}}
   \write\css{bodyText{}}
   \write\css{caption{}}
   \write\css{epigraph{}}
   \write\css{flushleft{}}
   \write\css{flushright{}}
   \write\css{frontmatter{}}
   \write\css{longQuotation{}}
   \write\css{plotcaption{}}
   \write\css{preface{}}
   \write\css{rtlBodyText{}}
   \write\css{shortQuote{}}
   \write\css{table{}}
   \write\css{verbatim{}}
   \write\css{blackboardBold{}}
   \write\css{bold{}}
   \write\css{calligraphic{}}
   \write\css{emphasized{}}
   \write\css{footnotesize{}}
   \write\css{fraktur{}}
   \write\css{Huge{}}
   \write\css{huge{}}
   \write\css{italics{}}
   \write\css{LARGE{}}
   \write\css{Large{}}
   \write\css{large{}}
   \write\css{phantom{}}
   \write\css{pre{}}
   \write\css{roman{}}
   \write\css{rtl{}}
   \write\css{sansSerif{}}
   \write\css{scriptsize{}}
   \write\css{slanted{}}
   \write\css{smallCaps{}}
   \write\css{small{}}
   \write\css{tiny{}}
   \write\css{typewriter{}}
   \write\css{bulletlist bulletlistItem{}}
   \write\css{bulletlist bulletlist bulletlistItem{}}
   \write\css{bulletlist bulletlist bulletlist bulletlistItem{}}
   \write\css{bulletlist bulletlist bulletlist bulletlist bulletlistItem{}}
   \write\css{numberedlist numberedlistItem{}}
   \write\css{numberedlist numberedlist numberedlistItem{}}
   \write\css{numberedlist numberedlist numberedlist numberedlistItem{}}
   \write\css{numberedlist numberedlist numberedlist numberedlist bulletlistItem{}}
   \write\css{numberedlist numberedlistItem:before{}}
   \write\css{numberedlist numberedlist numberedlistItem:before{}}
   \write\css{numberedlist numberedlist numberedlist numberedlistItem:before{}}
   \write\css{numberedlist numberedlist numberedlist numberedlist numberedlistItem:before{}}
   \write\css{address{}}
   \write\css{author{}}
   \write\css{date{}}
   \write\css{title{}}
   \write\css{acknowledgement{}}  
   \write\css{algorithm{}}
   \write\css{assertion{}}
   \write\css{assumption{}}
   \write\css{axiom{}}
   \write\css{case{}}
   \write\css{claim{}}
   \write\css{conclusion{}}
   \write\css{condition{}}
   \write\css{conjecture{}}
   \write\css{corollary{}}
   \write\css{criterion{}}
   \write\css{definition{}}
   \write\css{example{}}
   \write\css{exercise{}}
   \write\css{hypothesis{}}
   \write\css{lemma{}}
   \write\css{notation{}}
   \write\css{problem{}}
   \write\css{proof{}}
   \write\css{property{}}
   \write\css{proposition{}}
   \write\css{question{}}
   \write\css{remark{}}
   \write\css{summary{}}
   \write\css{theorem{}}
   \write\css{acknowledgement>*:first-child:before{}}  
   \write\css{algorithm>*:first-child:before{}}
   \write\css{assertion>*:first-child:before{}}
   \write\css{assumption>*:first-child:before{}}
   \write\css{axiom>*:first-child:before{}}
   \write\css{case>*:first-child:before{}}
   \write\css{claim>*:first-child:before{}}
   \write\css{conclusion>*:first-child:before{}}
   \write\css{condition>*:first-child:before{}}
   \write\css{conjecture>*:first-child:before{}}
   \write\css{corollary>*:first-child:before{}}
   \write\css{criterion>*:first-child:before{}}
   \write\css{definition>*:first-child:before{}}
   \write\css{example>*:first-child:before{}}
   \write\css{exercise>*:first-child:before{}}
   \write\css{hypothesis>*:first-child:before{}}
   \write\css{lemma>*:first-child:before{}}
   \write\css{notation>*:first-child:before{}}
   \write\css{problem>*:first-child:before{}}
   \write\css{proof>*:first-child:before{}}
   \write\css{property>*:first-child:before{}}
   \write\css{proposition>*:first-child:before{}}
   \write\css{question>*:first-child:before{}}
   \write\css{remark>*:first-child:before{}}
   \write\css{summary>*:first-child:before{}}
   \write\css{theorem>*:first-child:before{}}
   \closeout\css
}

\def\classcss{}
\def\newtheorems{}

\msi@dollar=`\$

\def\msi@pass#1{\def#1{\msitag{<texb xmlns="http://www.w3.org/1999/xhtml">#1</texb>}}}
\def\msi@pass@newline#1{\def#1{\msi@br\msihmode\msitag{<texb xmlns="http://www.w3.org/1999/xhtml">#1</texb>}}}

\msi@pass@newline{\@settodim}
\msi@pass@newline{\settoheight}
\msi@pass@newline{\settodepth}
\msi@pass@newline{\settowidth}

%\def\leftline#1{\@@line{#1\hss}}
%\def\rightline#1{\@@line{\hss#1}}
%\def\centerline#1{#1}


\def\MSIMathQuote#1{%
   \expandafter\def\csname#1\endcsname{\expandafter\string\csname#1\endcsname\@ifnextchar*{}{\char32\relax}}}


\def\left#1{\string\left#1}
\def\right#1{\string\right#1}


% \newcounter{#1}[#2]

%%\newtoks\preambleToks


%% redefine these to produce output

\def\newcounter#1{%
  \msihmode
  \msiopentag{texb}{<texb xmlns="http://www.w3.org/1999/xhtml" enc="1" name="newcounter">}
       \msitag{<![CDATA[\newcounter{#1}]]>}%
  \msiclosetag{texb}{</texb>}%
}

\let\setcounter\msidefaultb

%% \def\setcounter#1#2{%
%%     \ifnum\msi@inpreamble
%%        
%%     \else
%%        \expandafter\global\csname c@#1\endcsname#2\relax
%%        \msihmode
%%        \msiopentag{texb}{<texb xmlns="http://www.w3.org/1999/xhtml" enc="1" name="setcounter">}
%%           \msitag{<![CDATA[\setcounter{#1}{#2}]]>}%
%%        \msiclosetag{texb}{</texb>}%
%%     \fi}

%\def\addtocounter#1#2{%
%    %\expandafter\global\advance\csname c@#1\endcsname #2\relax
%    %\msihmode
%    \msiopentag{texb}{<texb xmlns="http://www.w3.org/1999/xhtml" enc="1" name="addtocounter">}
%       \msitag{<![CDATA[\addtocounter{#1}{#2}]]>}%
%    \msiclosetag{texb}{</texb>}}
\let\addtocounter\msidefaultb

\msi@inline@proc={jbm}
\msi@display@proc{jbm}


\msi@everyrowstart{\msiopentag{tr}{<tr>}}
\msi@everyrowend{\msiclosetag{tr}{</tr>}}


\msi@everycellstart{%
   %\msiclosetag{td}{</td>}%
   \msiopentag{td}{<td>}%
   \ifmmode\else\msiopentag{bodyText}{<bodyText>}\fi}


\msi@everycellend{%
   \ifmmode
   \else
      \msitag{<br type="_moz"/>}%
   \fi
   \msiclosetag{td}{</td>}}

\def\msi@stylesheets{\msitag{<?xml-stylesheet href="css/my.css" type="text/css"?>^^0a}}


\def\@brace#1{\char123\relax#1\char125\relax}

\def\msi@br{\ifvmode\msitag{^^0a}\fi}


%% \documentclass.
\newif\ifmsi@firstopt
\msi@firstopttrue

% ["pgorient", "papersize", "sides", "qual", "columns", "titlepage", "textsize", "eqnnopos", "eqnpos", "bibstyle"];

\expandafter\def\csname 10ptAttrib\endcsname{textsize="10pt"}
\expandafter\def\csname 11ptAttrib\endcsname{textsize="11pt"}
\expandafter\def\csname 12ptAttrib\endcsname{textsize="12pt"}

%\expandafter\def\csname portAttrib\endcsname{pgorient="port"}
\expandafter\def\csname landscapeAttrib\endcsname{pgorient="landscape"}

\expandafter\def\csname letterpaperAttrib\endcsname{papersize="letter"}
\expandafter\def\csname a4paperAttrib\endcsname{papersize="a4paper"}
\expandafter\def\csname a5paperAttrib\endcsname{papersize="a5paper"}
\expandafter\def\csname b5paperAttrib\endcsname{papersize="b5paper"}
\expandafter\def\csname legalpaperAttrib\endcsname{papersize="legalpaper"}
\expandafter\def\csname executivepaperAttrib\endcsname{papersize="executivepaper"}

\expandafter\def\csname twosideAttrib\endcsname{sides="twoside"}
\expandafter\def\csname onesideAttrib\endcsname{sides="oneside"}

\expandafter\def\csname finalAttrib\endcsname{qual="final"}
\expandafter\def\csname draftAttrib\endcsname{qual="draft"}

\expandafter\def\csname onecolumnAttrib\endcsname{columns="1"}
\expandafter\def\csname twocolumnAttrib\endcsname{columns="2"}

\expandafter\def\csname leqnoAttrib\endcsname{eqnnopos="leqno"}
\expandafter\def\csname reqnoAttrib\endcsname{eqnnopos="reqno"}

\expandafter\def\csname fleqnAttrib\endcsname{eqnpos="fleqn"}


\def\documentstyle{%
  \msitag{%
<?xml version="1.0" encoding="UTF-8"?>
<?sw-tagdefs href="resource://app/res/tagdefs/latexdefs.xml" type="text/xml" ?>
<!DOCTYPE html PUBLIC "-//W3C//DTD XHTML 1.1 plus MathML 2.0//EN" "http://www.w3.org/Math/DTD/mathml2/xhtml-math11-f.dtd">
<html xmlns="http://www.w3.org/1999/xhtml" xmlns:mml="http://www.w3.org/1998/Math/MathML" xmlns:sw="http://www.sciword.com/namespaces/sciword">
<head></head>
<body>
  Please convert your document to LaTeX2e.
</body></html>}
   \endinput}

\newtoks\temptoksb

\def\msiappend#1MSIXYZZY{%
   \expandafter\temptoksa\expandafter{\the\temptoksa #1}}

\def\msi@nextopt#1,{%
   \def\MSIXYZZY{MSIXYZZY}%
   \def\@Opt{#1}%
   \ifx\@Opt\MSIXYZZY%
      \let\@next\relax
   \else
      \ifmsi@firstopt
         \msi@firstoptfalse
      \else
         \expandafter\temptoksa\expandafter{\the\temptoksa\ }%
      \fi
      \edef\attrib{\csname #1Attrib\endcsname}%
      \expandafter\temptoksb\expandafter{\attrib}
      \expandafter\msiappend\the\temptoksb MSIXYZZY
      \let\@next\msi@nextopt
   \fi
   \@next}

\def\out@documentclassA[#1]#2{%
   %\msi@firstopttrue
   %\temptoksa={}%
   %\msi@nextopt#1,MSIXYZZY,%
   %\edef\colist{\the\temptoksa}%
   \msi@opendocument
   \temptoksa{#1}
   \ifx\temptoksa\relax
     \msi@br\msitag{<documentclass class="#2"/>}%
   \else
     \msi@br\msitag{<documentclass class="#2" options="#1"/>}%
   \fi}

\def\out@documentclassB#1{%
   \msi@opendocument
   \msi@br\msitag{<documentclass class="#1"/>}}

\newcount\preambleid

\def\out@preamble@tex#1{}
   %\advance\preambleid by 1
   %\msitag{^^0a<preambleTeX id="}\the\preambleid\msitag{"><![CDATA[^^0a%
   %  \usepackage{bm}^^0a\usepackage{tabulary}}%
   %\msipreambleextra   
   %\msitag{]]> </preambleTeX>}%
   %\msi@preamble={}}


%% \document and \enddocument

\def\out@begin@document{%
   %\expandafter\out@preamble@tex\expandafter{\the\msi@preamble}
   \msiclosetag{preamble}{</preamble>}%
   \msi@inpreamble = 0
   \msitag{</head>}\msitag{^^0a}%
   \msiopentag{body}{<body showexpanders="true" showshort="true">}
   \@paragraphs
   \msivmode
}

\def\out@end@document{%
   %%\br\msiclosetag{docbody}{</docbody>}%
   \msi@br\msiclosetag{body}{</body>}%
   \msi@br\msitag{</html>}%
   \msimakecss
   \endinput}


%% \usepackage


%% \usepackage and \RequirePackage

\def\usepackage{%
  %\begingroup\attrib@text
  [][]
  \@ifnextchar[{\@usepackagea}{\@usepackageb}}

\def\@usepackagea[#1]#2{%
   %\endgroup
   \out@usepackageA[#1]{#2}%
   \msi@inpreamble=0
   \advance\msi@output by 1\relax
   \@@usepackage[#1]{#2}%
   \advance\msi@output by -1\relax
   \msi@inpreamble=1
}

\def\@usepackageb#1{%
   %\endgroup
   \out@usepackageB{#1}%
   \msi@inpreamble=0
   \advance\msi@output by 1\relax
   \@@usepackage{#1}%
   \advance\msi@output by -1\relax
   \msi@inpreamble=1
}

\def\@@usepackage{\@fileswithoptions\@pkgextension}


\def\RequirePackage{%
  \@ifnextchar[{\@reqpackagea}{\@reqpackageb}}

\def\@reqpackagea[#1]#2{%
   \out@reqpackageA[#1]{#2}%
   \advance\msi@output by 1\relax
   \@@reqpackage[#1]{#2}%
   \advance\msi@output by -1\relax
}

\def\@reqpackageb#1{%
   \out@reqpackageB{#1}%
   \advance\msi@output by 1\relax
   \@@reqpackage{#1}%
   \advance\msi@output by -1\relax
}
\def\@@reqpackage{\@fileswithoptions\@pkgextension}


\def\out@usepackageA[#1]#2{%
   \msitag{<usepackage req="#2" opt="}#1\msitag{"/>}}

\def\out@usepackageB#1{%
   \msitag{^^0a}\msitag{<usepackage req="#1"/>}}

%% \RequirePackage
\def\out@reqpackageA[#1]#2{%
   \msitag{^^0a}\msitag{<usepackage requirepackage="true" req="#2" opt="}#1\msitag{"/>}}

\def\out@reqpackageB#1{%
   \msitag{^^0a}\msitag{<usepackage requirepackage="true"  req="#1"/>}}



%% Frontmatter

\def\title{%
  \msitag{^^0a}%
  \@ifnextchar[{\@titlea}{\@titleb}}

\def\@titlea[#1]#2{%
  \msiopentag{title}{<title>}%
    \msiopentag{short}{<shortTitle>}%
      #1%
    \msiclosetag{short}{</shortTitle>}%
      #2%
  \msiclosetag{title}{</title>}%
}

\def\@titleb#1{%
   \msiopentag{title}{<title>}#1\msiclosetag{title}{</title>}}

\def\author#1{\gdef\@author{#1}\out@author{#1}}

\def\date{\@ifnextchar[{\dateA}{\dateB}}
\def\dateA[#1]#2{%
   \gdef\@date{#2}\out@dateA{#1}{#2}}
\def\dateB#1{\gdef\@date{#1}\out@date{#1}}

\def\thanks#1{\out@thanks{#1}}
\def\and{\out@and}

%\def\frontmatter{\out@frontmatter}
%\def\mainmatter{\out@mainmatter}

\def\out@title#1{%
   {\everypar={}\msitag{^^0a}\msiopentag{title}{<title>}#1\msiclosetag{title}{</title>}}}


\def\out@thanks#1{%
   \msitag{%
      <notewrapper type="footnote" markOrText="undefined">
        <note style="padding: 10px;" type="footnote" hide="false">
          <bodyText>}%
   #1
   \msitag{%
          </bodyText>
        </note>
      </notewrapper>}}

\def\out@and{\msiclosetag{author}{</author>}\msitag{^^0a}\msiopentag{author}{<author>}}%

\def\out@author#1{%
  \begingroup
   \def\\{%%\msiclosetag{author}{</author>}%
          %%\msiclosetag{address}{</address>}
          %\msitag{^^0a^^0a}%
          \msitag{<msibr invisDisplay="&####x21b5;" type="newLine"/>}
          %\msiopentag{address}{<address>}%
          %\msiopentag{bodyText}{<bodyText>}%
          %\def\\{\par}%
          \ignorespaces}
          
   \def\and{%\msiclosetag{address}{</address>}%
            \msiclosetag{author}{</author>}\msitag{^^0a}\msiopentag{author}{<author>}\msiopentag{bodyText}{<bodyText>}}%
   \msitag{^^0a}%
   \msiopentag{author}{<author>}#1\msiclosetag{author}{</author>}%
  \endgroup
}

\def\out@date#1{%
   \ifnum\msi@inpreamble=1
      \msitopreambleextra{^^0a\date{#1}}%
   \else
      \msihmode\msi@br\msiopentag{date}{<date>}#1\msiclosetag{date}{</date>}%
   \fi}

\def\out@dateA#1#2{%
   \ifnum\msi@inpreamble=1
     \msitopreambleextra{^^0a\date[#1]{#2}}%
   \else
     \msihmode\msi@br
     \msiopentag{date}{<date>}%
      #2%
     \msiopentag{option}{<option>}%
         #1%
     \msiclosetag{option}{</option>}%
     \msiclosetag{date}{</date>}%
   \fi}

\def\MSIEmptyElement{%
  \msihmode\@ifnextchar[{\msi@empty@elmA}{\msi@empty@elmB}}

\def\msi@empty@elmA[#1]#2#3{%
     \expandafter\def\csname #2\endcsname{%
        \ifmmode
           %\msitag{\msipassthru{<#1/>}}%
           \expandafter\msitag\expandafter{\csname#2\endcsname}%
        \else
           \if#3T\relax%
             ^^0a%
           \else
             \msihmode              
           \fi           
           \msitag{<#1/>}%
        \fi}}

\def\msi@empty@elmB#1{\msi@empty@elmA[#1]{#1}}


\def\MSISimpleEnvironment{%
  \@ifnextchar[{\msi@simple@envA}{\msi@simple@envB}}

\def\msi@simple@envA[#1]#2{%
  \expandafter\def\csname msi@begin@#2\endcsname{\msiopentag{#1}{<#1>}\msihmode\ignorespaces}%
  \expandafter\def\csname msi@end@#2\endcsname{\msiclosetag{#1}{</#1>}}}

\def\msi@simple@envB#1{\msi@simple@envA[#1]{#1}}



%% newtheorem

\def\theoremstyle#1{%
  \def\msi@currentthmstyle{#1}%
  \msitag{^^0a<theoremstyle style="#1"/>}}
\theoremstyle{plain}

\def\msi@envWithLeadin#1[#2]{%
   \msiopentag{#1}{<#1 leadInType="custom">}\msitag{<envLeadIn><bodyText>#2</bodyText></envLeadIn>}\msihmode}

\def\msi@envWithOutLeadin#1{%
   \msiopentag{#1}{<#1>}\msihmode}


% should check for second optional
\def\out@newtheorema#1[#2]#3{%
  \endgroup% started in \newtheorem
  \msitopreambleextra{^^0a\newtheorem{#1}[#2]{#3}}
  \@ifundefined{msi@begin@#1}%
    {\@namedef{msi@begin@#1}{\msi@br\@ifnextchar[{\msi@envWithLeadin{#1}}{\msi@envWithOutLeadin{#1}}}%
     \@namedef{msi@end@#1}{\msi@br\msiclosetag{#1}{</#1>}}}%
    {}}

\def\out@newtheoremb#1#2{%
  \endgroup% started in \newtheorem
  \@ifundefined{msi@begin@#1}%
    {\msitopreambleextra{^^0a\newtheorem{#1}{#2}}
     \@namedef{msi@begin@#1}{\msi@br\msiopentag{#1}{<#1>}\msihmode}%
     \@namedef{msi@end@#1}{\msi@br\msiclosetag{#1}{</#1>}}}%
    {}}

%% recheck this one
\def\out@newtheoremc#1#2[#3]{%
  \endgroup% started in \newtheorem
  \msitopreambleextra{^^0a\newtheorem{#1}{#2}[#3]}
  \@ifundefined{msi@begin@#1}%
    {\@namedef{msi@begin@#1}{\msi@br\msiopentag{#1}{<#1>}\msihmode}%
     \@namedef{msi@end@#1}{\msi@br\msiclosetag{#1}{</#1>}}}%
    {}}



% Forms of the \newenvironment command

\newif\if@renew
\@renewfalse

\def\@hidden#1{%
   \advance\msi@output by 1\relax
   #1%
   \advance\msi@output by -1\relax
}

\def\newenvironment{%
   \@hidden{\@renewfalse}
   \@ifnextchar*{\@newenvironment{*}}{\@newenvironment{}}}

\def\renewenvironment{%
   \@renewtrue
   \@ifnextchar*{\@newenvironment{*}}{\@newenvironment{}}}

 
\def\@newenvironment#1#2{%
   \@ifnextchar[{\@newenv@a{#1}{#2}}{\@newenv@a{#1}{#2}[]}}


\def\@newenv@a#1#2[#3]{%
   \@ifnextchar[{\@newenv@b{#1}{#2}[#3]}{\@newenv@b{#1}{#2}[#3][]}}

\def\@newenv@b#1#2[#3][#4]#5#6{%
    \if@renew
       \msitopreambleextra{^^0a\renewenvironment#1{#2}}
    \else
       \msitopreambleextra{^^0a\newenvironment#1{#2}}
    \fi
    \def\msi@tmp{#3}%
    \ifx\msi@tmp\@empty
    \else
       %\expandafter\msi@preamble\expandafter{\the\msi@preamble[#3]}
       \msitopreambleextra{[#3]}%
    \fi
    \def\msi@tmp{#4}%
    \ifx\msi@tmp\@empty
    \else
       %\expandafter\msi@preamble\expandafter{\the\msi@preamble[#4]}
       \msitopreambleextra{[#4]}%
    \fi
    %\expandafter\msi@preamble\expandafter{\the\msi@preamble{#5}{#6}}
    \msitopreambleextra{{#5}{#6}}%
    \@ifundefined{msi@begin@#2}%
      {\@namedef{msi@begin@#2}{\msi@open@element<#2>}%
       \@namedef{msi@end@#2}{\msi@close@element</#2>}}
      {}%
}


%%\def\numberwithin#1#2{\msitag{\numberwithin{#1}{#2}}}
%% %% Newcommand and renewcommand
 
% \def\newcommand{\@ifnextchar*{\@newcommand}{\@newcommand{}}}
% \def\renewcommand{\@ifnextchar*{\@renewcommand}{\@renewcommand{}}}

 \let\providecommand\newcommand
 
 
 %%% #1 = optional star, #2 = command name
 \def\@newcommand#1#2{\@ifnextchar[{\@newcommandA{#1}{#2}}{\@newcommandA{#1}{#2}[]}}
 
 \def\@newcommandA#1#2[#3]{\@ifnextchar[{\@newcommandB{#1}{#2}[#3]}{\@newcommandB{#1}{#2}[#3][]}}

 \def\@newcommandB#1#2[#3][#4]#5{%
   \expandafter\msi@preamble\expandafter{\the\msi@preamble^^0a\newcommand#1{#2}}%
   \if x#3x
   \else
     \expandafter\msi@preamble\expandafter{\the\msi@preamble[#3]}%
   \fi
   \if x#4x
   \else
     \expandafter\msi@preamble\expandafter{\the\msi@preamble[#4]}\fi
   \expandafter\msi@preamble\expandafter{\the\msi@preamble{#5}}}%

 %%% #1 = optional star, #2 = command name
 \def\@renewcommand#1#2{\@ifnextchar[{\@renewcommandA{#1}{#2}}{\@renewcommandA{#1}{#2}[]}}
 
 \def\@renewcommandA#1#2[#3]{\@ifnextchar[{\@renewcommandB{#1}{#2}[#3]}{\@renewcommandB{#1}{#2}[#3][]}}

 \def\@renewcommandB#1#2[#3][#4]#5{%
   \expandafter\msi@preamble\expandafter{\the\msi@preamble^^0a\renewcommand#1{#2}}
   \if x#3x
   \else
     \expandafter\msi@preamble\expandafter{\the\msi@preamble[#3]}%
   \fi
   \if x#4x
   \else
     \expandafter\msi@preamble\expandafter{\the\msi@preamble[#4]}\fi
   \expandafter\msi@preamble\expandafter{\the\msi@preamble{#5}}}%
 

%   \MSILocalDef{#2}}


%% Citations and cross refs

{\catcode`\"=\active
 \gdef"{\char`\&##x22;}}


\def\attrib@text{%
  \catcode`\&=\active
  \catcode`\<=\active
  \catcode`\>=\active
  \catcode`\ =\other
  \catcode`\_=\other
  \catcode`\$=\other
  \catcode`\#=\other
  \catcode`\"=\active
  \def\ {\char`\&nbsp;}%
}

\def\cite{%
  \ifvmode\msihmode\fi
  \@ifnextchar[{\@tempswatrue\begingroup\attrib@text\out@citex}{\@tempswafalse\begingroup\attrib@text\out@citey}}


\def\out@citex[#1]#2{%
   \ifmmode
     \msitag{\msipassthru{<citation xmlns="http://www.w3.org/1999/xhtml" citekey="#2"  hasRemark="true">
                            <biblabel class="remark" xmlns="http://www.w3.org/1999/xhtml">#1</biblabel>
                           </citation>}}%
   \else
     \msiopentag{citation}{<citation xmlns="http://www.w3.org/1999/xhtml" 
                              citekey="}#2\msitag{" 
                              hasRemark="true">}%
          \msiopentag{biblabel}{<biblabel class="remark" xmlns="http://www.w3.org/1999/xhtml">}%
              #1%
          \msiclosetag{biblabel}{</biblabel>}%
     \msiclosetag{citation}{</citation>}%
   \fi
  \endgroup}

\def\out@citey#1{%
  \ifmmode
    \msitag{\msipassthru{<citation xmlns="http://www.w3.org/1999/xhtml" citekey="#1" hasRemark="false"></citation>}}%
  \else
    \msiopentag{citation}{<citation xmlns="http://www.w3.org/1999/xhtml" 
                                    citekey="}#1\msitag{" hasRemark="false">}%
    \msiclosetag{citation}{</citation>}%
  \fi
  \endgroup}


%\MSITextElement[a]{ref}
%%<xref key="mykey" reftype="page"/>
%\MSITextElement{pageref}

\def\ref{%
  \begingroup\attrib@text\@ref}
  
\def\@ref#1{%
  \ifmmode%
    \begingroup\msihmode
    \msitag{\msipassthru}
    \char`\{
    \msitag{<xref key="}#1\msitag{" href="}#1\msitag{" reftype="obj"/>}
    \char`\}
    \endgroup
  \else%
    \msihmode\msitag{<xref key="}#1\msitag{" href="}#1\msitag{" reftype="obj"/>}%
  \fi
  \endgroup}

\def\pageref{%
  \begingroup\attrib@text\@pageref}

\def\@pageref#1{%
  \ifmmode
    \msitag{\msipassthru{<xref key="#1" href="#1" reftype="page"/>}}%
  \else
    \msihmode\msitag{<xref key="}#1\msitag{" href="}#1\msitag{" reftype="page"/>}%
  \fi
  \endgroup}



\def\msi@pass#1{\def#1{\msitag{#1}}}

\def\@descr{descriptionListItem}
\def\@enumerate{numberedListItem}
\def\@bullet{bulletListItem}
\newif\if@first@item

\def\out@itema[#1]{
   \if@first@item
     \@first@itemfalse
   \else
     \def\enditemtag##1{\msiclosetag{##1}{</##1>}}%
     \expandafter\enditemtag\expandafter{\itemprefix}%
   \fi
   \def\startitemtag##1{\msi@br\msihmode\msiopentag{##1}{<##1>}\msiopentag{bodyText}{<bodyText>}}%
   \expandafter\startitemtag\expandafter{\itemprefix}%
   \ifx\itemprefix\@descr
     \msitag{<descriptionLabel>}#1\msitag{</descriptionLabel>}%
   \else
     \ifx\itemprefix\@enumerate
       \msitag{<numberedLabel>}#1\msitag{</numberedLabel>}%
	 \else
       \ifx\itemprefix\@bullet
         \msitag{<bulletLabel>}#1\msitag{</bulletLabel>}%
	   \else
         \msitag{<explicit-item>}#1\msitag{</explicit-item>}%
	   \fi
	 \fi
   \fi
}

\def\out@itemb{
   \if@first@item
     \@first@itemfalse
   \else
     \def\enditemtag##1{\msiclosetag{##1}{</##1>}}%
     \expandafter\enditemtag\expandafter{\itemprefix}%
   \fi
   \def\startitemtag##1{\msi@br\msihmode\msiopentag{##1}{<##1>}\msiopentag{bodyText}{<bodyText>}}%
   \expandafter\startitemtag\expandafter{\itemprefix}%
}


\def\out@bibitemA[#1]#2{%
   \if@first@item
     \global\@first@itemfalse
   \else
     \msiclosetag{bodyText}{</bodyText>}%
     \msiclosetag{bibitem}{</bibitem>}%
   \fi
   \msi@br\msihmode\msiopentag{bibitem}{}<bibitem hasLabel="true">%
   \msitag{<biblabel class="bibitemlabel">}#1\msitag{</biblabel>}%
   \msitag{<bibkey>}#2\msitag{</bibkey>}%
   \msi@br\msiopentag{bodyText}{<bodyText>}%
   \endgroup
}

\def\out@bibitemB#1{%
   \if@first@item
     \global\@first@itemfalse
   \else
     \msiclosetag{bodyText}{</bodyText>}%
     \msiclosetag{bibitem}{</bibitem>}%
   \fi
   \msi@br\msiopentag{bibitem}{<bibitem hasLabel="false">}%
   \msitag{<bibkey>}#1\msitag{</bibkey>}%
   \msi@br\msiopentag{bodyText}{<bodyText>}%
   \endgroup%
}


\def\@closeitem{\msiclosetag{li}{</li>}}
\def\@openitem{\msiopentag{li}{<li>}}


% A default item name
\def\itemprefix{item}

% #1=xml name, #2=latex env name, #3=xml item name
\def\MSIListEnvironment#1#2#3{%
   \expandafter\def\csname msi@begin@#2\endcsname{%
      \def\itemprefix{#3}%
      \msihmode\msitag{^^0a^^0a}\@first@itemtrue\msiopentag{#1}{<#1>}}

   \expandafter\def\csname msi@end@#2\endcsname{%
      \msi@br\msiclosetag{#1}{</#1>}}
}



\def\msi@begin@thebibliography#1{%
  \everypar={}%
  \msihmode\msitag{^^0a^^0a}%
  \@first@itemtrue
  \msiopentag{bibliography}{<bibliography>}}

\def\msi@end@thebibliography{%
  \msi@br\msiclosetag{bibliography}{</bibliography>}}


%\def\bibliographystyle#1{\def\the@style{#1}}
\def\bibliographystyle#1{\def\@bibstyle{#1}}
\def\bibliography#1{%
  \msihmode
  \msitag{<bibtexbibliography xmlns="http://www.w3.org/1999/xhtml"
  databaseFile="#1" styleFile="}\@bibstyle\msitag{"></bibtexbibliography>}}


\def\msi@thm@like#1{}



%\input msiframe.tex



%% \def\[#1\]{%
%%   \msitag{<mml:math>}
%%   \msiextproc{jbm}{$#1$}%
%%   \msitag{</mml:math>}
%% }
%% 


\newcount\seclevel
\seclevel=0

\newif\if@starred
\def\@gobble#1{}


%% Sectioning

% We always use <sectiontitle> rather than #4
\def\newsection#1#2#3#4{%
  \expandafter\newcount\csname#1number\endcsname
  \csname#1number\endcsname=0
  \expandafter\def\csname the#1\endcsname{#3}%
  \expandafter\def\csname #1\endcsname{%
     \@ifnextchar*{\@starredtrue\csname @#1\endcsname}%
                  {\@starredfalse\csname @#1\endcsname *}%
  }
  \expandafter\def\csname @#1\endcsname*{%
     \@ifnextchar[{\csname @@#1\endcsname}%
                  {\csname @@#1\endcsname[]}%
  }
  \expandafter\def\csname @@#1\endcsname[##1]##2{%
     \seclevel=#2%
     \expandafter\advance\csname #1number\endcsname by 1%
     \let\protect\relax
     \msiclosetag{#1}{</#1>}%
     \msitag{^^0a}%
     \if@starred
       \msiopentag{#1}{<#1 nonum="true">}%
     \else 
       \msiopentag{#1}{<#1>}%
     \fi
     {\everypar={}\msitag{^^0a}%
      \def\short{##1}%
      \msiopentag{sectiontitle}{<sectiontitle>}%
      \ifx\short\@empty
      \else
        \msiopentag{shortTitle}{<shortTitle>}%
          ##1%
        \msiclosetag{shortTitle}{</shortTitle>}%
      \fi
      ##2%
      \msiclosetag{sectiontitle}{</sectiontitle>}%
      \msitag{^^0a}%
     }%
     \par
  }%
}




\newsection{part}{1}{Part \the\sectionnumber}{sectiontitle}%
\newsection{chapter}{2}{Chapter \the\sectionnumber}{sectiontitle}%
\newsection{section}{2}{Section \the\sectionnumber}{sectiontitle}%
\newsection{subsection}{3}{\the\sectionnumber.\the\subsectionnumber}{sectiontitle}%
\newsection{subsubsection}{4}{\thesectionnumber.\thesubsectionnumber.\the\subsubsectionnumber}{sectiontitle}%
\newsection{paragraph}{5}{\thesectionnumber.\thesubsectionnumber.\the\subsubsectionnumber.\the\paragraphnumber}{sectiontitle}%
\newsection{subparagraph}{6}{\thesectionnumber.\thesubsectionnumber.\the\subsubsectionnumber.\the\paragraphnumber}{sectiontitle}%


\def\out@marginparA#1#2{%
  \@start@notewrapper
  \msiopentag{note}{<note type="marginpar" opt="#1">}%
    \msiopentag{bodyText}{<bodyText>}%
      #2%
    \msiclosetag{bodyText}{</bodyText>}%
  \msiclosetag{note}{</note>}%
  \@end@notewrapper
}

\def\out@marginparB#1{%
  \@start@notewrapper
  \msiopentag{note}{<note type="marginpar">}%
    \msiopentag{bodyText}{<bodyText>}%
      #1%
    \msiclosetag{bodyText}{</bodyText>}%
  \msiclosetag{note}{</note>}%
  \@end@notewrapper
}

\def\rule{%
  \@ifnextchar[{\@rulea}{\@ruleb}}%

\def\@rulea[#1]#2#3{%
  \ifmmode
     %\string\rule[#1]\@brace{#2}\@brace{#3}%
    \msitag{\msipassthru{<msirule lift="#1" width="#2" height="#3" %
                     style="vertical-align: #1; width: #2; height: #3; background-color: rgb(0, 0, 0);"/>}}%
  \else
    \msitag{<msirule lift="#1" width="#2" height="#3" %
                     style="vertical-align: #1; width: #2; height: #3; background-color: rgb(0, 0, 0);"/>}%
  \fi
}
\def\@ruleb#1#2{%
  \ifmmode
    \msitag{\msipassthru{<msirule lift="0pt" width="#1" height="#2" %
                     style="vertical-align: 0; width: #1; height: #2; background-color: rgb(0, 0, 0);"/>}}%
  \else
    \msitag{<msirule lift="0pt" width="#1" height="#2" %
                     style="vertical-align: 0; width: #1; height: #2; background-color: rgb(0, 0, 0);"/>}%
  \fi
}

\def\hspace{%
   \@ifnextchar*{\@hspacestar}{\@hspace}}

\def\@hspace#1{%
   \ifmmode
     \string\hspace\@brace{#1}%
   \else
     \msitag{<hspace dim="#1" type="customSpace" atEnd="false"></hspace>}%
   \fi
}

\def\@hspacestar#1#2{%
   \ifmmode
     \string\hspace\@brace{#2}%
   \else
     \msitag{<hspace dim="#2" type="customSpace" atEnd="true"></hspace>}%
   \fi
}

\def\@paragraphs{%
  \let\@par\par
  \def\par{\msiclosetag{bodyText}{</bodyText>}\@par}%
  \everypar{^^0a\msihmode\msiopentag{bodyText}{<bodyText>}}%
}


\everymath{%
   %\catcode`\&=13 %
   %\catcode`\'=\active
   \aftergroup\endmath
}
\def\endmath{}

\everydisplay{%
  %\catcode`\&=13 %
  %\catcode`\'=\active
  \aftergroup\enddisplay
}
\def\enddisplay{}


\def\@make@braces@other{%
  \catcode`\{=12%
  \catcode`\}=12%
  \relax
}
\def\@make@braces@normal{%
  \catcode`\{=1%
  \catcode`\}=2%
  \relax
}

\bgroup
  \catcode`\^^M=13\relax
  \gdef^^M{\char13\char10\relax}%
\egroup


%  \def\@make@linebreak@other{%
%    \catcode`\^^M=13\relax
%  }
%
%  \def\@make@linebreak@normal{%
%    \catcode`\^^M=5\relax
%  }


\def\linebreak{\msitag{<msibr type="lineBreak" invisDisplay="&##x21b5;"><br/></msibr>}}

\def\newline{%
  \@ifnextchar[{\newlinea}{\newlineb}}

\def\newlinea[#1]{\msitag{<msibr type="newLine" invisDisplay="&##x21b5;"><br/></msibr>}}

\def\newlineb{\msitag{<msibr type="newLine" invisDisplay="&##x21b5;"><br/></msibr>}}

\newtoks\temptoksa


\bgroup
  \@make@braces@other
  \catcode`\[=1
  \catcode`\]=2

  \gdef\@scan@align@star#1\end{align*}[%
     \egroup
     \@make@braces@normal
     \msiopentag[mml][<mml:math class="displayedmathml" mode="display" display="block">]%
     \msiextproc[jbm][\begin{align*}#1\end{align*}]%
     \msiclosetag[mml][</mml:math>]
  ]
  \gdef\@scan@align#1\end{align}[%
     \egroup
     \@make@braces@normal
     \msiopentag[mml][<mml:math class="displayedmathml" mode="display" display="block">]%
     \msiextproc[jbm][\begin{align}#1\end{align}]%
     \msiclosetag[mml][</mml:math>]
  ]

  \gdef\@scan@bmatrix#1\end{bmatrix}[%
     \egroup
     \@make@braces@normal
     \msitag[\begin{bmatrix}#1\end{bmatrix}]
     %\temptoksa=[\begin{bmatrix}#1\end{bmatrix}]%
     %\the\temptoksa
  ]
  \gdef\@scan@Bmatrix#1\end{Bmatrix}[%
     \egroup
     \@make@braces@normal
     \msitag[\begin{Bmatrix}#1\end{Bmatrix}]
     %\temptoksa=[\begin{Bmatrix}#1\end{Bmatrix}]%
     %\the\temptoksa
  ]
  \gdef\@scan@smallmatrix#1\end{smallmatrix}[%
     \egroup
     \@make@braces@normal
     \msitag[\begin{smallmatrix}#1\end{smallmatrix}]
  ]

  \gdef\@scan@vmatrix#1\end{vmatrix}[%
     \egroup
     \@make@braces@normal
     \msitag[\begin{vmatrix}#1\end{vmatrix}]
     %\temptoksa=[\begin{vmatrix}#1\end{vmatrix}]%
     %\the\temptoksa
  ]
  \gdef\@scan@Vmatrix#1\end{Vmatrix}[%
     \egroup
     \@make@braces@normal
     \msitag[\begin{Vmatrix}#1\end{Vmatrix}]
     %\temptoksa=[\begin{Vmatrix}#1\end{Vmatrix}]%
     %\the\temptoksa
  ]
  \gdef\@scan@pmatrix#1\end{pmatrix}[%
     \egroup
     \@make@braces@normal
     \msitag[\begin{pmatrix}#1\end{pmatrix}]
     %\temptoksa=[\begin{pmatrix}#1\end{pmatrix}]%
     %\the\temptoksa
  ]
\egroup


%\@namedef{msi@begin@multline*}{\ \@make@braces@other\relax\@scan@multline@star}



\@namedef{msi@begin@flalign}{\bgroup\msi@dollar=0\relax$$\string\begin{flalign}}
\@namedef{msi@end@flalign}{\string\end{flalign}$$\egroup}

\@namedef{msi@begin@flalign*}{\bgroup\msi@dollar=0\relax$$\string\begin{flalign*}}
\@namedef{msi@end@flalign*}{\string\end{flalign*}$$\egroup}

\@namedef{msi@begin@alignat}{\bgroup\msi@dollar=0\relax$$\string\begin{alignat}}
\@namedef{msi@end@alignat}{\string\end{alignat}$$\egroup}

\def\msi@begin@math{\bgroup\msi@dollar=0\relax$\string\begin{math}}
\def\msi@end@math{\string\end{math}$\egroup}

\def\msi@begin@subequations{\msi@insubequations=1}
\def\msi@end@subequations{\msi@insubequations=0}

\def\msi@begin@equation{\begingroup\msi@dollar=0$$\msi@dollar=`\$\string\begin{equation}}
\def\msi@end@equation{\string\end{equation}\msi@dollar=0$$\msi@dollar=`\$\endgroup}


\@namedef{msi@begin@equation*}{\bgroup\msi@dollar=0\relax$$\msi@dollar=`\$\string\begin{equation*}}
\@namedef{msi@end@equation*}{\string\end{equation*}\msi@dollar=0$$\msi@dollar=`\$\egroup}

\@namedef{msi@begin@displaymath}{\bgroup\msi@dollar=0\relax$$\msi@dollar=`\$\string\begin{equation*}}
\@namedef{msi@end@displaymath}{\string\end{equation*}\msi@dollar=0$$\msi@dollar=`\$\egroup}


\def\msi@begin@array{\begingroup\def\\{\char`\\\char`\\}\catcode`\&=12\string\begin{array}}
\def\msi@end@array{\string\end{array}\endgroup}


%\def\@label#1{\msikilllasttag{<tr>}{<tr marker="#1" style="">}}%

\def\MSIEqnarrayLikeEnvironment#1{%
  \@namedef{msi@begin@#1}{%
     \begingroup
        \def\\{\char`\\\char`\\}%
%        \let\label\@label
        \catcode`\&=12\msi@dollar=0\relax$$%
        \string\begin{#1}}%
  \@namedef{msi@end@#1}{\string\end{#1}$$\endgroup}}



\def\MSISplitLikeEnvironment#1{%
  \@namedef{msi@begin@#1}{
    \bgroup
      \def\\{\char`\\\char`\\}%
      \ifmmode
      \else
         \msi@dollar=0\relax$$
      \fi
      \string\begin{#1}\catcode`\&=12}%
  \@namedef{msi@end@#1}{%
      \string\end{#1}%
      \ifmmode
      \else
         $$
      \fi
    \egroup}}




\@namedef{msi@begin@align*}{\ \@make@braces@other\relax\@scan@align@star}
\@namedef{msi@begin@align}{\ \@make@braces@other\relax\@scan@align}



\def\msi@begin@bmatrix{\@make@braces@other\relax\@scan@bmatrix} 
\def\msi@begin@Bmatrix{\@make@braces@other\relax\@scan@Bmatrix} 
\def\msi@begin@smallmatrix{\@make@braces@other\relax\@scan@smallmatrix} 
\def\msi@begin@vmatrix{\@make@braces@other\relax\@scan@vmatrix} 
\def\msi@begin@Vmatrix{\@make@braces@other\relax\@scan@Vmatrix} 
\def\msi@begin@pmatrix{\@make@braces@other\relax\@scan@pmatrix}



{\catcode`\^^M=\active % these lines must end with %
  \gdef\obeylines{\catcode`\^^M\active \let^^M\par}%
  \global\let^^M\par} % this is in case ^^M appears in a \write
\def\obeyspaces{\catcode`\ \active}
\def\space{ }
{\obeyspaces\global\let =\space}

 

\def\@cr{\ifmmode
           \def\do@cr{\char`\\}%
         \else
           \def\do@cr{\@ifnextchar[{\@cra}{\@crb}}%
         \fi
         \advance\msirow by 1\relax
         \do@cr}


\def\@cra[#1]{\cr}%
\def\@crb{\cr}%


\def\tabularnewline{\\}


\def\MSITabularStarEnvironment#1{%
  \@namedef{msi@begin@#1}##1{%
      \@ifnextchar[{\msitabularstarA{##1}}%
                   {\msitabularstarB{##1}}}%
  \@namedef{msi@end@#1}{%
        \crcr
        \egroup % the \halign
     \endgroup % to do the closetabular
     \ifmathtabular\char`\}\fi
  \endgroup}
}

\def\msitabularstarA#1[#2]#3{%
   \begingroup
      \ifmmode
        \string\msipassthru\char`\{% 
        \mathtabulartrue
        \msihmode
        \catcode`\&=4%
      \else
        \mathtabularfalse
      \fi
      \msihmode\msiopentag{table}{<table opt="#2" _moz_resizing="true"  cellpadding="4" border="1">}%
      \msi@br\msiopentag{tbody}{<tbody>}%
      \begingroup
        \aftergroup\msi@closetabular
        \let\\=\@cr
        \halign\bgroup&##\cr
}

\def\msitabularstarB#1#2{%
   \begingroup
      \msi@dollar=`\$%
      \ifmmode
        \string\msipassthru\char`\{% 
        \mathtabulartrue
        \msihmode
        \catcode`\&=4%
      \else
        \mathtabularfalse
      \fi
      \msihmode
      \msiopentag{table}{<table cellspacing="2" cellpadding="2" border="1">}%
      \msiopentag{tbody}{<tbody>}%
      \begingroup
        \aftergroup\msi@closetabular
        \let\\=\@cr
        \halign\bgroup&##\cr
}
 
\def\msi@begin@tabular{%
   \@ifnextchar[{\msi@begin@tabularA}{\msi@begin@tabularA[]}}

\expandafter\def\csname msi@begin@tabular*\endcsname{%
   \@ifnextchar[{\msi@begin@tabularA}{\msi@begin@tabularA[]}}


\newif\ifmathtabular
\mathtabulartrue
\newcount\msirow=0

\def\msi@begin@tabularA[#1]#2{%
   \begingroup
      \msi@dollar=`\$%
      \ifmmode
        \string\msipassthru\char`\{% 
        \mathtabulartrue
        \msihmode
        \catcode`\&=4%
      \else
        \mathtabularfalse
        \msihmode
      \fi
      \msiopentag{table}{<table cols="#2" cellpadding="4" border="1" req="tabulary" _moz_resizing="true">}%
      \msiopentag{tbody}{<tbody>}%
      \begingroup
        \msirow=1
        \aftergroup\msi@closetabular
        \let\\=\@cr
        \def\tabularnewline{\\}%
        \halign\bgroup&##\cr}


\def\msi@closetabular{%
  \msiclosetag{tbody}{</tbody>}%
  \msiclosetag{table}{</table>}%
}
%  \msitag{<br type="_moz"/>}
%  \ifmathtabular
%    \msiclosetag{bodyText}{</bodyText>}%
%  \fi}


\def\msi@end@tabular{% 
        \crcr
        \egroup % the \halign
     \endgroup % to do the closetabular
     \ifmathtabular\char`\}\fi
  \endgroup}

\expandafter\def\csname msi@end@tabular*\endcsname{%
        \crcr
        \egroup % the \halign
     \endgroup % to do the closetabular
     \ifmathtabular\char`\}\fi
  \endgroup}


\def\cline#1{\msitag{<cline arg="#1"/>}}
\def\hline{\msitag{<hline row="}\the\msirow\msitag{"/>}}

%\def\hline{}
%\def\cline#1{}


%%\def\multicolumn#1#2#3{%
%%   \count0=#1\relax
%%   \ifnum\count0=1
%%     #3
%%   \else
%%      %\msitag{ Unimplemented multicol }
%%      \msikilllasttag  % the td
%%      \msikilllasttag  % the body text
%%      \msiopentag{td}{<td rowspan="1" colspan="#1">}%
%%        \msiopentag{bodyText}{<bodyText>}%
%%        #3
%%        %\msiclosetag{bodyText}{</bodyText>}%
%%      %\msiclosetag{td}{</td>}%
%%   \fi
%%}

\def\multicolumn#1#2#3#{%
   \count0=#1\relax
   \ifnum\count0=1\relax
   \else
      \msikilllasttag{<td>}{<td rowspan="1" colspan="#1">}%
   \fi}

\def\endbrace{\char`\}}
\chardef\other=12
\chardef\active=13
\bgroup
   \catcode`&=\active
   \gdef&{\ifmmode\char`\&\else\char`\&amp;\fi}%
\egroup
\bgroup
   \catcode`#=\active
   \gdef#{\char`\#}%
\egroup
\bgroup
   \catcode` =\active
   \gdef {\char`\ }%
\egroup

\begingroup
  \catcode`\<=\active
  \gdef<{\ifmmode\char`\<\else\char`\&lt;\fi}
\endgroup



\def\out@verbatim#1{%
  \msi@br\msihmode
  \msiopentag{verbatim}{<verbatim>}%
  \msiopentag{bodyText}{<bodyText>}%
   #1%
  \msiclosetag{bodyText}{</bodyText>}
  \msiclosetag{verbatim}{</verbatim>}}%

\def\verb{%
   \msihmode
   \bgroup
      \catcode`\\=\other
      \catcode`{=\other
      \catcode`}=\other
      \catcode`&=\active
      \catcode`$=\other
      \catcode`##=\other
      \catcode`\<=\active
      \msiopentag{verb}{<verb>}%
      \@sverb
}

\def\out@boxedverbatim#1{%
  \msi@br\msihmode
  \msiopentag{boxedverbatim}{<boxedverbatim>}%
  \msiopentag{bodyText}{<bodyText>}%
   #1%
  \msiclosetag{bodyText}{</bodyText>}
  \msiclosetag{boxedverbatim}{</boxedverbatim>}}%


\def\verb@egroup{\msiclosetag{verb}{</verb>}\egroup}
\def\@sverb#1{%
  \catcode`#1\active
  \lccode`\~`#1\relax
  \lowercase{\let~\verb@egroup}%
}

%\def\fbox#1{\ifmmode\temptoksa={#1}\string\fbox\char`\{\the\temptoksa\char`\}\fi}


\def\{{\ifmmode\string\{\else\char123\relax\fi}
\def\}{\ifmmode\string\}\else\char125\relax\fi}

\let\@hbox\hbox
\def\hbox{%
   \def\@after{}
   \ifmmode
      \string\hbox\char`\{\aftergroup\endbrace
   \else
      \def\@after{\@hbox}%
   \fi
   \@after
}


\def\mathbf#1{\string\mathbf\char123#1\char125}
\def\mathfrak#1{\string\mathfrak\char123#1\char125}


\def\TeX{\msihmode\msitag{<texlogo xmlns="http://www.w3.org/1999/xhtml" name="tex">T<sub>E</sub>X</texlogo>}}
\def\LaTeX{\msihmode\msitag{<texlogo xmlns="http://www.w3.org/1999/xhtml" name="latex">L<sup>A</sup>T<sub>E</sub>X</texlogo>}}
\def\BibTeX{\msihmode\msitag{<texlogo xmlns="http://www.w3.org/1999/xhtml" name="bibtex">B<sup>i</sup>bT<sub>E</sub>X</texlogo>}}

% Operators

{\catcode`\<=\active
\gdef\TEXTsymbol#1{%
  \ifmmode
    \temptoksa={\TEXTsymbol{#1}}%
    \the\temptoksa
  \else
    \ifx#1\backslash
      \char`\\%
    \fi
    \ifx#1\vert
      \char`\|%
    \fi
    \ifx#1<
      \char`\&lt;%
    \fi
    \ifx#1>
      \char`\&gt;%
    \fi
  \fi
}}

\msi@pass{\backslash}

\def\textemdash{\@hex{2014}}



{\catcode`\$=11\relax
 \gdef\${\ifmmode\msitag{\$}\else\char`\${}\fi}
 \catcode`\_=11\relax
 \gdef\_{\ifmmode\msitag{\_}\else\char`\_{}\fi}
 \catcode`\%=11\relax
 \gdef\%{\ifmmode\msitag{\%}\else\char`\%{}\fi}
}




\def\@hex#1{\char38\char35x#1;}


\newif\if@inaccent
\@inaccentfalse

\def\@cdced@{\@hex{327}} % combining cedilla
\def\@cdk@{\@hex{328}}   % combining ogonek
\def\@cdd@{\@hex{323}}   % combining dot below
\def\@cdb@{\@hex{331}}   % combining macron below
\def\@cdr@{\@hex{30A}}   % combining ring above
\def\@cdcaret@{\@hex{302}} % combining circumflex
\def\@cdv@{\@hex{30C}}   % combining caron
\def\@cdacute@{\@hex{301}} % combining acute
\def\@cdgrave@{\@hex{300}} % combining grave
\def\@cdtilde@{\@hex{303}} % ~
\def\@cdu@{\@hex{306}} % breve
\def\@cdbar@{\@hex{304}}   % combining macron bar
\def\@cdH@{\@hex{30B}}     % double acute
\def\@cddot@{\@hex{307}}   % combining dot above
\def\@cdumlat@{\@hex{308}} % umlat




\def\@accent#1#2{%
    \if@inaccent
      \@@accent{#1}{#2}%
    \else
      %\@inaccenttrue
      \edef\@acc{\csname\@@accent{#1}{#2}\endcsname}%
      \expandafter\ifx\@acc\relax
        #1\csname @cd#2@\endcsname
      \else
        \@acc
      \fi
      %\@inaccentfalse
    \fi
}

\def\@@accent#1#2{%
    \ifx#1\i
      dotlessi-#2%
    \else\ifx#1\j
      dotlessj-#2%
    \else 
      #1-#2%
  \fi\fi\fi
}

\def\`#1{\@accent{#1}{grave}}

\def\'#1{\@accent{#1}{acute}}

\def\~#1{\@accent{#1}{tilde}}

\def\^#1{\@accent{#1}{caret}}

\def\=#1{\@accent{#1}{bar}}

\def\"#1{\@accent{#1}{umlat}}

\def\c#1{\@accent{#1}{ced}}

\def\.#1{\@accent{#1}{dot}}

\def\r#1{\@accent{#1}{r}}

\def\H#1{\@accent{#1}{H}}

\def\v#1{\@accent{#1}{v}}

\def\u#1{\@accent{#1}{u}}

\def\k#1{\@accent{#1}{k}}

\def\d#1{\@accent{#1}{d}}

\def\b#1{\@accent{#1}{b}}



\def\t#1{oo}

%% \'{\^{e}}

\def\MSIUnicode#1#2{\@namedef{#1}{\ifmmode\expandafter\string\csname#1 \endcsname\else\@hex{#2}\fi}}


%% Spacing



\def\ {%
  \ifmmode
    \char`\\\char`\ %
  \else
    \msihmode\msitag{<hspace type="requiredSpace">&nbsp;</hspace>}%
  \fi}

\def~{%
  \ifmmode
    \char`\~
  \else
    \msihmode\msitag{<nbspace>~</nbspace>}%
  \fi}

\def\/{%
  \ifmmode
    \char`\\\char`\/%
  \else
    \msihmode\msitag{<hspace dim="0.083en" type="italicCorrectionSpace">}\@hex{200a}\msitag{</hspace>}%
  \fi}

\def\noindent{\par\msitag{<hspace dim="0.0em" type="noIndent"></hspace>}}

\def\thinspace{%
   \ifmmode
      \string\thinspace
   \else
      \msihmode\msitag{<hspace type="thinSpace"/>}%
   \fi
}



\def\thickspace{%
   \ifmmode
      \string\thickspace
   \else
      \msihmode\msitag{<hspace type="thickSpace"/>}%
   \fi
}

\let\:\medspace


\def\negthinspace{%
   \ifmmode
      \string\negthinspace
   \else
      \msihmode\msitag{<hspace type="negativeThinSpace" />}%
   \fi
}



\def\hrulefill{%
   \ifmmode
     %\string\hrulefill
     \msitag{\msipassthru{<mspace width="thickmathspace" type="customSpace" class="stretchySpace" flex="3" fillWith="line" atEnd="true"/>}}%
   \else
     <hspace fillWith="line" flex="1" class="stretchySpace" type="stretchySpace"></hspace>
   \fi}



\def\pagebreak{\msitag{<msibr type="pageBreak"><newPageRule/></msibr>}}
\def\newpage{\msitag{<msibr type="newPage"><newPageRule/></msibr>}}

\def\hfil{<hspace type="customSpace" class="stretchySpace" flex="1" atEnd="true"/>}
\def\hfill{<hspace type="customSpace" class="stretchySpace" flex="2" atEnd="true"/>}

%\msi@rfi{\msi@hook@hfil}{\hfil}
%\msi@rfi{\msi@hook@hfill}{\hfill}

%\msi@rfi{\msi@hook@vfil}{\vfil}
%\msi@rfi{\msi@hook@vfill}{\vfill}

\def\msi@hook@hss{\msitag{\hss}}
\def\msi@hook@vss{\msitag{\vss}}

\def\label{%
   \ifmmode\else\msihmode\fi
   \begingroup\attrib@text
   \@label}

\def\@label#1{%
  \endgroup
   \ifmmode
      %\msitag{\msipassthru}\char`\{%
      %\msihmode\msitag{<a name="}#1\msitag{" key="}#1\msitag{" id="}#1\msitag{"/>}%
      %\char`\}%
      \msitag{\label}\char`\{#1\char`\}%
   \else
      \msihmode\msitag{<a name="}#1\msitag{" key="}#1\msitag{" id="}#1\msitag{"/>}%
   \fi}

\def\frac#1#2{\string\frac\@brace{#1}\@brace{#2}}
\def\dfrac#1#2{\string\dfrac\@brace{#1}\@brace{#2}}
\def\tfrac#1#2{\string\tfrac\@brace{#1}\@brace{#2}}

\def\unit#1{\string\unit\@brace{#1}}

\def\mathbf#1{\string\mathbf\@brace{#1}}
\def\mathit#1{\string\mathit\@brace{#1}}
\def\mathsf#1{\string\mathsf\@brace{#1}}
\def\mathtt#1{\string\mathtt\@brace{#1}}
\def\mathrm#1{\string\mathrm\@brace{#1}}


\def\Big#1{\ifmmode\string\big#1\fi}
\def\big#1{\ifmmode\string\big#1\fi}
\def\Bigg#1{\ifmmode\string\big#1\fi}
\def\bigg#1{\ifmmode\string\big#1\fi}
\def\Bigl#1{\ifmmode\string\big#1\fi}
\def\bigl#1{\ifmmode\string\big#1\fi}
\def\Bigm#1{\ifmmode\string\big#1\fi}
\def\bigm#1{\ifmmode\string\big#1\fi}
\def\Bigr#1{\ifmmode\string\big#1\fi}
\def\bigr#1{\ifmmode\string\big#1\fi}
\def\Biggl#1{\ifmmode\string\big#1\fi}
\def\biggl#1{\ifmmode\string\big#1\fi}
\def\Biggm#1{\ifmmode\string\big#1\fi}
\def\biggm#1{\ifmmode\string\big#1\fi}
\def\Biggr#1{\ifmmode\string\big#1\fi}
\def\biggr#1{\ifmmode\string\big#1\fi}



%\def\func#1{\string\func\@brace{#1}}
\def\@limfunc#1#2{\string\limfunc\@brace{#2}}
\def\operatorname{\@ifnextchar*{\@limfunc}{\func}}


\def\sqrt{%
  \@ifnextchar[%
    {\@nthroot}%
    {\@sqrt}}

\def\@sqrt#1{%
   \string\sqrt\@brace{#1}%
}

\def\@nthroot[#1]#2{%
   \string\sqrt[#1]\@brace{#2}%
}


%\msi@mathmlpass{\thinspace}



\def\mathpass@arg#1{%
  \def#1##1{\string#1\@brace{##1}}
}

\def\mathpass@arg@arg#1{%
  \def#1##1##2{\string#1\@brace{##1}\@brace{##2}}
}
\def\mathpass@arg@arg@arg#1{%
  \def#1##1##2##3{\string#1\@brace{##1}\@brace{##2}\@brace{##3}}
}
\def\mathpass@arg@arg@arg@arg#1{%
  \def#1##1##2##3##4{\string#1\@brace{##1}\@brace{##2}\@brace{##3}\@brace{##4}}
}

\def\mathpass@arg@arg@arg@arg@arg#1{%
  \def#1##1##2##3##4##5{\string#1\@brace{##1}\@brace{##2}\@brace{##3}\@brace{##4}\@brace{##5}}
}



%\mathpass{\tag}
%\mathpass{\TCItag}
\let\tag\msidefaultb



\mathpass@arg{\mathbb}
\mathpass@arg{\mathcal}


\mathpass@arg@arg{\binom}
\mathpass@arg@arg{\dbinom}
\mathpass@arg@arg{\tbinom}

\mathpass@arg@arg{\QATOP}
\mathpass@arg@arg{\QTATOP}
\mathpass@arg@arg{\QDATOP}

\mathpass@arg@arg@arg{\QABOVE}
\mathpass@arg@arg@arg{\QTABOVE}
\mathpass@arg@arg@arg{\QDABOVE}
\mathpass@arg@arg@arg@arg{\QOVERD}
\mathpass@arg@arg@arg@arg{\QTOVERD}
\mathpass@arg@arg@arg@arg{\QDOVERD}
\mathpass@arg@arg@arg@arg{\QATOPD}
\mathpass@arg@arg@arg{\QTATOPD}
\mathpass@arg@arg@arg{\QDATOPD}

\def\unit#1{\string\unit\@brace{#1}}



\newif\if@mathintext
\@mathintextfalse 

{\catcode`\$=\active
 \gdef${%
   \if@mathintext
      \@mathintextfalse
      \msitag{\msipassthru}%
      \char`\{
      \msiopentag{mtext}{<mml:mtext>}
   \else
      \@mathintexttrue
      \msiclosetag{mtext}{</mml:mtext>}%
      \char`\}%
   \fi}
}


\def\text{\@mathintextfalse\@text}

\def\@text#1{%
  \begingroup
    \msi@dollar=`\$%
    \msitag{\msipassthru}%
    \char`\{%
    \msiopentag{mrow}{<mml:mrow>}%
    \msitag{<mml:mspace width="thickmathspace"/>}%
    \msiopentag{mtext}{<mml:mtext>}%
    \begingroup \msihmode #1\endgroup%
    \msiclosetag{mtext}{</mml:mtext>}%
    \msitag{<mml:mspace width="thickmathspace"/>}%
    \msiclosetag{mrow}{</mml:mrow>}%
    \char`\}
  \endgroup
  \catcode`\$=3\relax}


\bgroup
  \catcode`\%=12
  \gdef\strut{\msitag{<vspace type="strut" lineHt="100%"/>}}
\egroup

%\bgroup
%  \catcode`\%=12
%  \gdef\mathstrut{\ifmmode\string\mathstrut\else\msitag{<vspace type="mathStrut" lineHt="100%"/>\fi}}
%\egroup

\def\vspace{%
   \leavevmode
   \@ifnextchar*{\@vspacestar}{\@vspace}}

\def\@vspace#1{%
   \ifmmode
     \string\vspace\@brace{#1}%
   \else
     \msitag{<vspace type="customSpace" dim="#1" atEnd="false"></vspace>}%
   \fi
}

\def\@vspacestar#1#2{%
   \ifmmode
     \string\vspace\@brace{#2}%
   \else
     \msihmode\msitag{<vspace type="customSpace" dim="#2" atEnd="true"></vspace>}%
   \fi
}

\def\MSICounter#1{%
  \expandafter\newcount\csname c@#1\endcsname}

\def\&{\ifmmode\char`\\\char`\&\else\char`\&amp;\fi}
\def\|{\Vert}

\def\textcurrency{\msitag{<textcurrency req="textcomp"/>}}

\def\out@hyperref#1#2#3#4{%
   \msiopentag{a}{<a href="#4" req="hyperref">}#1\msiclosetag{a}{</a>}\ignorespaces}

\newif\if@in@texttag
\@in@texttagfalse

\def\QTP#1{%
  \par
  \msihmode
  \msiopentag{QTP}{<QTP type="#1">}%
  \begingroup
    \def\par{\msiclosetag{QTP}{</QTP>}\endgroup}%
}

\def\@ttclose#1{\msiclosetag{#1}{</#1>}}
\def\@closelasttexttag{\expandafter\@ttclose\expandafter{\@lasttexttag}}

\def\MSIFontSwitch{%
  \@ifnextchar[{\msi@font@switchA}{\msi@font@switchB}}


\def\msi@font@switchA[#1]#2{%
   \expandafter\def\csname#2\endcsname{%
      \ifmmode
        \expandafter\string\csname #2\endcsname\char`\  %
        %%\char`\\\msitag{#2}%
      \else
        \msihmode
        \ifnum\msi@inpreamble=1
           \msihmode\expandafter\msitopreambleextra\expandafter{\csname #2\endcsname}%
        \else
           \if@in@texttag
              \expandafter\@ttclose\expandafter{\@lasttexttag}%
           \else
              \@in@texttagtrue
           \fi
           \gdef\@lasttexttag{#1}%
           \msiopentag{#1}{<#1>}%
           \aftergroup\@closelasttexttag
        \fi
      \fi
   }
}

\def\msi@font@switchB#1{\msi@font@switchA[#1]{#1}}


\def\MSITextElement{%
  \@ifnextchar[{\msi@text@elmA}{\msi@text@elmB}}


\def\msi@text@elmA[#1]#2{%
  \expandafter\def\csname #2\endcsname{%
     \let\@next\relax
     \ifmmode
       \expandafter\string\csname #2\endcsname%
     \else
          \msihmode\msiopentag{#1}{<#1>}%
          \def\endelem{\msiclosetag{#1}{</#1>}}%
          \bgroup
             \aftergroup\endelem
             \let\@next\@gobbleit
     \fi
     \@next}}


\def\msi@text@elmB#1{\msi@text@elmA[#1]{#1}}


\def\MSISimpleElement{%
  \@ifnextchar[{\msi@simple@elmA}{\msi@simple@elmB}}

\def\msi@simple@elmA[#1]#2{%
  \expandafter\def\csname #2\endcsname##1{%
     \ifmmode
       \expandafter\string\csname #2\endcsname{##1}%
     \else
       \msitag{^^0a}\msihmode\msiopentag{#1}{<#1>}##1\msiclosetag{#1}{</#1>}%
     \fi}}

\def\msi@simple@elmB#1{\msi@simple@elmA[#1]{#1}}



\def\HTMLButton#1#2{\msitag{<htmlfield name="#1">#2</htmlfield>}}

\def\CDATA#1{#1}

\def\TeXButton{%
  \begingroup
  \begingroup
  \attrib@text
  \@TeXButton}

\def\@TeXButton#1{
    \endgroup
    \@@TeXButton{#1}}
    
\def\@@TeXButton#1#2{%
   \ifmmode
     \msitag{\msipassthru\char`\{<texb xmlns="http://www.w3.org/1999/xhtml" enc="1" name="}#1\msitag{\char`\}"><![CDATA[#2]]></texb>}
   \else
     \msihmode
     \msitag{<texb xmlns="http://www.w3.org/1999/xhtml" enc="1" name="}\msiattrib{#1}\msitag{"><![CDATA[#2]]></texb>}%
   \fi
   \endgroup}

\def\ensure@attrib#1{%
   
}


%% Create TeX Buttons

\def\MSITeXButton#1{%
    \expandafter\def\csname#1\endcsname
       \else
          \msihmode 
       \fi
       \msitag{<texb xmlns="http://www.w3.org/1999/xhtml" }
         \ifnum\msi@inpreamble=1
           pre="1"
         \fi 
         \msitag{enc="1" name="}#1\msitag{">}%
                <![CDATA[\expandafter\string\csname#1\endcsname]]>%
        \msitag{</texb>}%
       \ifmmode
         \char`\}%}
       \fi
     }}



\def\MSITeXButtonReq#1{%
    \expandafter\def\csname#1\endcsname##1{%
       \@temptokena={\msitag{##1}}%
       \ifmmode
         \string\msipassthru\char`\{%}%
       \else
         \msihmode
       \fi
       \msitag{<texb xmlns="http://www.w3.org/1999/xhtml" enc="1" name="}#1\msitag{">}%
           <![CDATA[\expandafter\string\csname#1\endcsname
           \ifmmode
             {\the\@temptokena}]]>%
           \else
             \{\the\@temptokena\}]]>%
           \fi
       \msitag{</texb>}%
       \ifmmode
         \char`\}%}
       \fi}}

\def\MSITeXButtonReqReq#1{%
    \expandafter\def\csname#1\endcsname##1##2{%
       \@temptokena={\msitag{##1}}%
       \@temptokenb={\msitag{##2}}%
       \msihmode
       \msitag{<texb xmlns="http://www.w3.org/1999/xhtml" enc="1" name="#1">}%
          <![CDATA[\expandafter\string\csname#1\endcsname\{\the\@temptokena\}\{\the\@temptokenb\}]]>%
       \msitag{</texb>}}}


\def\fbox#1{\ifmmode
              \msitag{\msipassthru{<menclose notation='box' type='fbox'>}}%
              \msitag{\msipassthru}\char`\{\begingroup \msihmode #1\endgroup\char`\}%
              \msitag{\msipassthru{</menclose>}}%
            \else
              \msihmode\msiopentag{fbox}{<smallbox>}#1\msiclosetag{fbox}{</smallbox>}%
            \fi}

\def\frame#1{\ifmmode
%               \msitag{\msipassthru{<smallbox>#1</smallbox>}}%
              \msitag{\msipassthru{<menclose notation='box' type='frame'>}}%
              \msitag{\msipassthru}{\msihmode #1}%
              \msitag{\msipassthru{</menclose>}}%
             \else
               \msihmode\msiopentag{frame}{<smallbox>}#1\msiclosetag{frame}{</smallbox>}%
             \fi}



%\def\TCItag{\@ifnextchar*{\@TCItagstar}{\@TCItag}}
%\def\tag{\@ifnextchar*{\@TCItagstar}{\@TCItag}}

\def\@TCItag#1{\msitag{\TCItag{#1}}}
\def\@TCItagstar*#1{\string\TCItag\msitag{*{#1}}}

%{\catcode`\'=\active
% \gdef'{^{\prime}}}


%\def\@TCItag#1{%
%    \global\tag@true
%    \global\def\@taggnum{(#1)}%
%    \global\def\@currentlabel{#1}}

%\def\@TCItagstar*#1{%
%    \global\tag@true
%    \global\def\@taggnum{#1}%
%    \global\def\@currentlabel{#1}}




\def\msi@cdr#1#2xxxx{#2}

%\def\MSILocalDef#1{\def#1{\string #1}}



\bgroup
     \@make@braces@other
     \catcode `|=0
     \catcode`\(=1  % use parens for braces
     \catcode`\)=2
     \catcode`\\=12
     |gdef|MSITeXButtonEnvironment#1(%
          |expandafter|gdef|csname msi@begin@#1|endcsname
             (|bgroup
                 |@make@braces@other
                 |catcode`|\=12
                 |catcode`|&=12
                 |aftergroup|@make@braces@normal
                 |csname scan@#1|endcsname)

          |expandafter|gdef|csname scan@#1|endcsname##1\end{#1}%
             (|msitag(<texb xmlns="http://www.w3.org/1999/xhtml" enc="1" name="#1">%
                        <![CDATA[\begin{#1}##1\end{#1}]]>%
                      </texb>)%
              |egroup)
     )%
|egroup



%% Various note type things

\def\@start@notewrapper{%
    \msiopentag{notewrapper}{<notewrapper xmlns="http://www.w3.org/1999/xhtml" type='footnote'>}}

\def\@end@notewrapper{%
    \msiclosetag{notewrapper}{</notewrapper>}}

\def\@end@footnote{\msiclosetag{note}{</note>}\@end@notewrapper}
\def\@innote{\bgroup\aftergroup\@end@footnote}


\newif\ifmathfootnote
\mathfootnotefalse

\def\footnote{%
  \mathfootnotefalse
  \ifmmode
       \string\msipassthru\char`\{%
       \mathfootnotetrue
  \fi
  \@ifnextchar[{\footnotex}{\footnotea}}


\def\footnotex[#1]{%
  \msi@br\@start@notewrapper
  \msiopentag{note}{<note type="footnote" footnoteNumber="#1">}%
  \msiopentag{bodyText}{<bodyText>}%
  % Next couple lines read the brace following \footnote and then does \footnoteb
  \afterassignment\@footnoteb
  \let\junk}


\def\footnotea{%
  \msi@br\@start@notewrapper
  \msiopentag{note}{<note type="footnote">}%
  \msiopentag{bodyText}{<bodyText>}%
  % Next couple lines read the brace following \footnote and then does \footnoteb
  \afterassignment\@footnoteb
  \let\junk}

\def\@footnoteb{%
    \begingroup
    \msihmode
    \def\par{\msiclosetag{bodyText}{</bodyText>}\msiopentag{bodyText}{<bodyText>}}
    \aftergroup\@footnotec}

%% \def\@footnoteb{%
%%     \msi@br
%%     \msiopentag{notewrapper}{<notewrapper type='footnote'>}
%%     \msiopentag{note}{<note type="footnote" hide='false'>}%
%%     \msiopentag{bodyText}{<bodyText>}%
%%     % Next couple lines read the brace following \footnote and then does \footnoteb
%%     \afterassignment\@footnotec
%%     \let\junk}

%% \def\@footnotec{%
%%     \bgroup
%%     \def\par{\msiclosetag{bodyText}{</bodyText>}\msiopentag{bodyText}{<bodyText>}}
%%     \aftergroup\@footnoted
%% }

\def\@footnotec{%
    \msiclosetag{bodyText}{</bodyText>}%
    \msiclosetag{note}{</note>}%
    \@end@notewrapper
    \ifmathfootnote
      \char`\}%
    \fi}

\def\out@CustomNoteA[#1]#2#3{%
    \msiopentag{notewrapper}{<notewrapper xmlns="http://www.w3.org/1999/xhtml">}%
    \msiopentag{note}{<note type="#2">}%
    \bgroup
      \@in@texttagfalse% turn this off so we don't close pending tags if we nest a font switch. 
      #3%
    \egroup
    \msiclosetag{note}{</note>}%
    \msiclosetag{notewrapper}{</notewrapper>}%
}

\def\out@CustomNoteB#1#2{%
    \msiopentag{notewrapper}{<notewrapper xmlns="http://www.w3.org/1999/xhtml">}%
    \msiopentag{note}{<note type="#1">}%
    \bgroup
      \@in@texttagfalse% turn this off so we don't close pending tags if we nest a font switch. 
      #2%
    \egroup
    \msiclosetag{note}{</note>}%
    \msiclosetag{notewrapper}{</notewrapper>}%
}


%\MSIMathQuote{\\} %% may need to hard-code this one:

\def\\{\@ifnextchar*{\@linebrk}{\@linebrk+}}


\def\@linebrk#1{%
   \@ifnextchar[{\@linebrkA{#1}}{\@linebrkB{#1}}}

\def\@linebrkA#1[#2]{%
  \ifmmode\string\\\else\msitag{<msibr type="customNewLine" dim="#2" style="vertical-align: -#2" depth="#2"/>}\fi}

\def\@linebrkB#1{\ifmmode\string\\\else\msitag{<msibr type="newLine" invisDisplay="&##x21b5;"/>}\fi}

%% Temporary until prince can read xml

\newtoks\msi@extrapi

\def\msi@opendocument{%
   \msitag{<?xml version="1.0" encoding="UTF-8"?>^^0a}
   \msitag{<?xml-stylesheet href="css/my.css" type="text/css"?>^^0a}
   \msitag{<?sw-tagdefs href="resource://app/res/tagdefs/latexdefs.xml" type="text/xml" ?>^^0a}
   \the\msi@extrapi
   \msitag{<!DOCTYPE html PUBLIC "-//W3C//DTD XHTML 1.1 plus MathML 2.0//EN" "http://www.w3.org/Math/DTD/mathml2/xhtml-math11-f.dtd">}\msi@br
   \msitag{<html xmlns="http://www.w3.org/1999/xhtml" 
             xmlns:mml="http://www.w3.org/1998/Math/MathML"
             xmlns:sw="http://www.sciword.com/namespaces/sciword">}
   \msi@br\msitag{<head>}
   \msi@br\msiopentag{preamble}{<preamble>}
   \expandafter\msitag\expandafter{\the\msi@preamble}
   \def\makeatother{%
       \msihmode
       \ifnum\msi@inpreamble=1
         \ifnum\msi@output=0
           \expandafter\msi@preamble\expandafter{\the\msi@preamble^^0a\makeatother}
         \fi
       \else
           \msitag{<texb xmlns="http://www.w3.org/1999/xhtml" enc="1" name="makeatother">}%
              <![CDATA[\string\makeatother]]>%
           \msitag{</texb>}%
       \fi
       \catcode`\@=12}

   \def\makeatletter{%
       \msihmode
       \ifnum\msi@inpreamble=1
          \ifnum\msi@output=0
            \expandafter\msi@preamble\expandafter{\the\msi@preamble^^0a\makeatletter}%
          \fi
       \else
         \msitag{<texb xmlns="http://www.w3.org/1999/xhtml" enc="1" name="makeatletter">}%
           <![CDATA[\string\makeatletter]]>%
         \msitag{</texb>}%
       \fi
       \catcode`\@=11\relax}

   \msi@inpreamble=1
}


\def\@gobbleit{\let\@temptok}

\futurelet\@bracetok\@gobbleit{
\futurelet\@brackettok\@gobbleit[
\futurelet\@startok\@gobbleit*

\begingroup
  \@make@braces@other
  \catcode`\[=1
  \catcode`\]=2
  \gdef\msi@lbrace[{]
  \gdef\msi@rbrace[}]
\endgroup

\def\@inbrace#1{%
   \ifnum\msi@inpreamble=1
     \expandafter\msitopreambleextra\expandafter{\msi@lbrace}%
     \msitopreambleextra{#1}%
     \expandafter\msitopreambleextra\expandafter{\msi@rbrace}%
   \else
     \@temptokena={\msitag{#1}}%
     \msi@lbrace\the\@temptokena\msi@rbrace
   \fi
   \@gobbleargs}

\def\@inbracket[#1]{%
   \ifnum\msi@inpreamble=1
     \msitopreambleextra{[#1]}%
   \else
     \@temptokena={\msitag{[#1]}}%
     \the\@temptokena
   \fi
   \@gobbleargs}

\def\@astar*{*\@gobbleargs}

\def\@gobbleargs{\futurelet\atok\@gobblemore}

\def\@endgobble{\@endtexb\relax}

\def\@gobblemore{%
  \begingroup
     \msi@inpreamble=0
     \ifx\atok\@bracetok
       \global\let\@donext\@inbrace
     \else
       \ifx\atok\@brackettok
         \global\let\@donext\@inbracket
       \else
         \ifx\atok\@startok
           \global\let\@donext\@astar
         \else
           \global\let\@donext\@endgobble
         \fi
       \fi
     \fi
   \endgroup
   \@donext}



% Here #1 = some unknown control sequence 

\def\msidefault#1{%
  \ifmmode
    \msitag{\msipassthru}%
    \char`\{
    \msiopentag{texb}{<texb xmlns="http://www.w3.org/1999/xhtml" enc="1" name="#1">}%
       \msitag{<![CDATA[#1}
    \let\@next\@gobbleargs
  \else
    \def\@next{\expandafter\msidefaulta\expandafter{\string #1}}%
  \fi
  \@next}

\def\msidefaulta#1{%
  \expandafter\msidefaultb\expandafter{\msi@cdr#1xxxx}}


\def\msidefaultb#1{%
    \msihmode
    \ifnum\msi@inpreamble=1
      %%% \temptoksa={<texb xmlns="http://www.w3.org/1999/xhtml" enc="1" name="#1">}%
      %%% \expandafter\msitopreambleextra\expandafter{\the\temptoksa}
      %%% \msiopentag{texb}{}%
      %%% \msitopreambleextra{<![CDATA[}
      \expandafter\temptoksa\expandafter{\csname#1\endcsname}
      \msitopreambleextra{^^0a}%
      \expandafter\msitopreambleextra\expandafter{\the\temptoksa}
    \else
      \msiopentag{texb}{<texb xmlns="http://www.w3.org/1999/xhtml" enc="1" name="#1">}%
         \msitag{<![CDATA[}\expandafter\string\csname#1\endcsname
    \fi
    \@gobbleargs}

\def\@endtexb{% 
    \ifnum\msi@inpreamble=1
       %%% \msitopreambleextra{]]></texb>}
       %%% \msiclosetag{texb}{}
    \else
         \msitag{]]>}%
         \msiclosetag{texb}{</texb>}%
         \ifmmode
           \char`\} % to close off msipassthru
         \fi
    \fi}


%\let\renewcommand\msidefault   

\def\msi@begin@figure{%
  \msihmode\@ifnextchar[{\msi@begin@figureA}{\msi@begin@figureB}}


\def\msi@begin@figureA[#1]{%
   \msiopentag{msiframe}{<msiframe frametype="figure" pos="floating"
   placement="f" ltxfloat="#1" captionloc="bottom" >}}

\def\msi@begin@figureB{%
   \msiopentag{msiframe}{<msiframe frametype="figure" pos="floating"
   placement="f" captionloc="bottom">}}

\def\msi@end@figure{%
   \msiclosetag{msiframe}{</msiframe>}}



\def\msi@begin@table{%
  \msihmode\@ifnextchar[{\msi@begin@tableA}{msi@begin@tableB}}


\def\msi@begin@tableA[#1]{%
   \msiopentag{msiframe}{<msiframe frametype="table" pos="floating" ltxfloat="#1">}}

\def\msi@begin@tableB{%
   \msiopentag{msiframe}{<msiframe frametype="table" pos="floating">}}

\def\msi@end@table{%
   \msiclosetag{msiframe}{</msiframe>}}


\def\caption{\msi@caption}

\def\msi@caption#1{%
  \msiopentag{caption}{<caption align="bottom">}%
  #1%
  \msiclosetag{caption}{</caption>}}

\bgroup
  \@make@braces@other
  \catcode `|=0
  \catcode`\[=1
  \catcode`\]=2
  \catcode`\\=12
  |gdef|msi@scan@filecontents[|catcode`|^^M=12|catcode`|^^J=12|relax|msi@@scan@filecontents]
  |gdef|msi@@scan@filecontents#1\end{filecontents}[%
     |out@filecontents[#1]%
     |end[filecontents]%
     |@make@braces@normal
     |ignorespaces]%
|egroup

\def\msi@begin@filecontents{%
  \@make@braces@other
  \catcode `|=0
  \catcode`\[=1
  \catcode`\]=2
  \catcode`\\=12
  \msi@scan@filecontents}

\def\out@filecontents#1{%
  \global\msi@preamble{<filecontents><![CDATA[#1]]></filecontents>}}




\input latex-default.tex
\input tcilatex.tex
\catcode`\<=\active
%\catcode`\&=12


\input latex-default.tex
\input tcilatex.tex
\catcode`\<=\active
