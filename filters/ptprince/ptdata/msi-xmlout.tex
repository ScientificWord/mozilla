\makeatletter

\newcounter{MaxMatrixCols}

\msi@inpreamble = 0
\msi@insubequations=0




%%\def\msiToPreamble#1{\expandafter\preambleToks\expandafter{\the\preambleToks #1}}

\def\msi@rfi#1#2{\def#1{\msitag{<requestimplementation>#2</requestimplementation>}}}

\begingroup\catcode`\"=\active\relax
\gdef"{'}
\endgroup

\def\href{
    \begingroup
    \catcode`\"=\active    
    \@href}


\def\@href#1#2{%
   \ifmmode
      \msitag{\msipassthru{<a href="#1">#2</a>}}%
   \else
      \msiopentag{a}{<a href="}#1\msitag{">}#2\msiclosetag{a}{</a>}%
   \fi
   \endgroup}


% This writes the css file my.css. 
% Do this at the end because it happens that we don't know what css are required. 
 
\def\msimakecss{
   \def\css{2}	  % the css output file
   \openout\css={css/my.css} %% the CSS style file for this document
   \write\css{@import url("resource://app/res/css/latex_internal.css");}
   \write\css{@import url("resource://app/res/css/baselatex.css");}
   \write\css{@import url("resource://app/res/css/latex.css");}
   \ifx\classcss\@empty
     \write\css{@import url("resource://app/res/css/article.css");}
   \else
     \write\css{\classcss}
   \fi
   \write\css{\newtheorems}
   \write\css{body{}}

   \write\css{sectiontitle{}}
   \write\css{part>sectiontitle{}}
   \write\css{part>sectiontitle:before{}}
   \write\css{chapter>sectiontitle{}}
   \write\css{chapter>sectiontitle:before{}}
   \write\css{section>sectiontitle{}}  
   \write\css{section>sectiontitle:before{}}
   \write\css{subsection>sectiontitle{}}
   \write\css{subsection>sectiontitle:before{}}
   \write\css{subsubsection>sectiontitle{}}
   \write\css{subsubsection>sectiontitle:before{}}
   \write\css{paragraph>sectiontitle{}}
   \write\css{paragraph>sectiontitle:before{}}
   \write\css{bodyMath{}}
   \write\css{bodyText{}}
   \write\css{caption{}}
   \write\css{epigraph{}}
   \write\css{flushleft{}}
   \write\css{flushright{}}
   \write\css{frontmatter{}}
   \write\css{longQuotation{}}
   \write\css{plotcaption{}}
   \write\css{preface{}}
   \write\css{rtlBodyText{}}
   \write\css{shortQuote{}}
   \write\css{table{}}
   \write\css{verbatim{}}
   \write\css{blackboardBold{}}
   \write\css{bold{}}
   \write\css{calligraphic{}}
   \write\css{emphasized{}}
   \write\css{footnotesize{}}
   \write\css{fraktur{}}
   \write\css{Huge{}}
   \write\css{huge{}}
   \write\css{italics{}}
   \write\css{LARGE{}}
   \write\css{Large{}}
   \write\css{large{}}
   \write\css{phantom{}}
   \write\css{pre{}}
   \write\css{roman{}}
   \write\css{rtl{}}
   \write\css{sansSerif{}}
   \write\css{scriptsize{}}
   \write\css{slanted{}}
   \write\css{smallCaps{}}
   \write\css{small{}}
   \write\css{tiny{}}
   \write\css{typewriter{}}
   \write\css{bulletlist bulletlistItem{}}
   \write\css{bulletlist bulletlist bulletlistItem{}}
   \write\css{bulletlist bulletlist bulletlist bulletlistItem{}}
   \write\css{bulletlist bulletlist bulletlist bulletlist bulletlistItem{}}
   \write\css{numberedlist numberedlistItem{}}
   \write\css{numberedlist numberedlist numberedlistItem{}}
   \write\css{numberedlist numberedlist numberedlist numberedlistItem{}}
   \write\css{numberedlist numberedlist numberedlist numberedlist bulletlistItem{}}
   \write\css{numberedlist numberedlistItem:before{}}
   \write\css{numberedlist numberedlist numberedlistItem:before{}}
   \write\css{numberedlist numberedlist numberedlist numberedlistItem:before{}}
   \write\css{numberedlist numberedlist numberedlist numberedlist numberedlistItem:before{}}
   \write\css{address{}}
   \write\css{author{}}
   \write\css{date{}}
   \write\css{title{}}
   \write\css{acknowledgement{}}  
   \write\css{algorithm{}}
   \write\css{assertion{}}
   \write\css{assumption{}}
   \write\css{axiom{}}
   \write\css{case{}}
   \write\css{claim{}}
   \write\css{conclusion{}}
   \write\css{condition{}}
   \write\css{conjecture{}}
   \write\css{corollary{}}
   \write\css{criterion{}}
   \write\css{definition{}}
   \write\css{example{}}
   \write\css{exercise{}}
   \write\css{hypothesis{}}
   \write\css{lemma{}}
   \write\css{notation{}}
   \write\css{problem{}}
   \write\css{proof{}}
   \write\css{property{}}
   \write\css{proposition{}}
   \write\css{question{}}
   \write\css{remark{}}
   \write\css{summary{}}
   \write\css{theorem{}}
   \write\css{acknowledgement>*:first-child:before{}}  
   \write\css{algorithm>*:first-child:before{}}
   \write\css{assertion>*:first-child:before{}}
   \write\css{assumption>*:first-child:before{}}
   \write\css{axiom>*:first-child:before{}}
   \write\css{case>*:first-child:before{}}
   \write\css{claim>*:first-child:before{}}
   \write\css{conclusion>*:first-child:before{}}
   \write\css{condition>*:first-child:before{}}
   \write\css{conjecture>*:first-child:before{}}
   \write\css{corollary>*:first-child:before{}}
   \write\css{criterion>*:first-child:before{}}
   \write\css{definition>*:first-child:before{}}
   \write\css{example>*:first-child:before{}}
   \write\css{exercise>*:first-child:before{}}
   \write\css{hypothesis>*:first-child:before{}}
   \write\css{lemma>*:first-child:before{}}
   \write\css{notation>*:first-child:before{}}
   \write\css{problem>*:first-child:before{}}
   \write\css{proof>*:first-child:before{}}
   \write\css{property>*:first-child:before{}}
   \write\css{proposition>*:first-child:before{}}
   \write\css{question>*:first-child:before{}}
   \write\css{remark>*:first-child:before{}}
   \write\css{summary>*:first-child:before{}}
   \write\css{theorem>*:first-child:before{}}
   \closeout\css
}

\def\classcss{}
\def\newtheorems{}

\msi@dollar=`\$

\def\msi@pass#1{\def#1{\msitag{<texb xmlns="http://www.w3.org/1999/xhtml">#1</texb>}}}
\def\msi@pass@newline#1{\def#1{\msi@br\msihmode\msitag{<texb xmlns="http://www.w3.org/1999/xhtml">#1</texb>}}}

\msi@pass@newline{\@settodim}
\msi@pass@newline{\settoheight}
\msi@pass@newline{\settodepth}
\msi@pass@newline{\settowidth}

%\def\leftline#1{\@@line{#1\hss}}
%\def\rightline#1{\@@line{\hss#1}}
%\def\centerline#1{#1}


\def\MSIMathQuote#1{%
   %\expandafter\def\csname#1\endcsname{\char`\\#1\char32\relax}}
   \expandafter\def\csname#1\endcsname{\expandafter\string\csname#1\endcsname\char32\relax}}


\def\left#1{\string\left#1}
\def\right#1{\string\right#1}


% \newcounter{#1}[#2]

\newtoks\preambleToks


%% redefine these to produce output

\def\newcounter#1{%
  \msihmode
  \msiopentag{texb}{<texb xmlns="http://www.w3.org/1999/xhtml" enc="1" name="newcounter">}
       \msitag{<![CDATA[\newcounter{#1}]]>}%
  \msiclosetag{texb}{</texb>}%
}

\let\setcounter\msidefaultb

%% \def\setcounter#1#2{%
%%     \ifnum\msi@inpreamble
%%        
%%     \else
%%        \expandafter\global\csname c@#1\endcsname#2\relax
%%        \msihmode
%%        \msiopentag{texb}{<texb xmlns="http://www.w3.org/1999/xhtml" enc="1" name="setcounter">}
%%           \msitag{<![CDATA[\setcounter{#1}{#2}]]>}%
%%        \msiclosetag{texb}{</texb>}%
%%     \fi}

\def\addtocounter#1#2{%
    \expandafter\global\advance\csname c@#1\endcsname #2\relax
    \msihmode
    \msiopentag{texb}{<texb xmlns="http://www.w3.org/1999/xhtml" enc="1" name="addtocounter">}
       \msitag{<![CDATA[\addtocounter{#1}{#2}]]>}%
    \msiclosetag{texb}{</texb>}}


\msi@inline@proc={jbm}
\msi@display@proc{jbm}


\msi@everyrowstart{\msiopentag{tr}{<tr>}}
\msi@everyrowend{\msiclosetag{tr}{</tr>}}


\msi@everycellstart{%
   \msiclosetag{td}{</td>}%
   \msiopentag{td}{<td>}%
   \ifmmode\else\msiopentag{bodyText}{<bodyText>}\fi}


\msi@everycellend{%
   \ifmmode
   \else
      \msitag{<br type="_moz"/>}%
   \fi
   \msiclosetag{td}{</td>}}

\def\msi@stylesheets{\msitag{<?xml-stylesheet href="css/my.css" type="text/css"?>^^0a}}


\def\@brace#1{\char123\relax#1\char125\relax}

\def\msi@br{\ifvmode\msitag{^^0a}\fi}


%% \documentclass.
\newif\ifmsi@firstopt
\msi@firstopttrue

% ["pgorient", "papersize", "sides", "qual", "columns", "titlepage", "textsize", "eqnnopos", "eqnpos", "bibstyle"];

\expandafter\def\csname 10ptAttrib\endcsname{textsize="10pt"}
\expandafter\def\csname 11ptAttrib\endcsname{textsize="11pt"}
\expandafter\def\csname 12ptAttrib\endcsname{textsize="12pt"}

%\expandafter\def\csname portAttrib\endcsname{pgorient="port"}
\expandafter\def\csname landscapeAttrib\endcsname{pgorient="landscape"}

\expandafter\def\csname letterpaperAttrib\endcsname{papersize="letter"}
\expandafter\def\csname a4paperAttrib\endcsname{papersize="a4paper"}
\expandafter\def\csname a5paperAttrib\endcsname{papersize="a5paper"}
\expandafter\def\csname b5paperAttrib\endcsname{papersize="b5paper"}
\expandafter\def\csname legalpaperAttrib\endcsname{papersize="legalpaper"}
\expandafter\def\csname executivepaperAttrib\endcsname{papersize="executivepaper"}

\expandafter\def\csname twosideAttrib\endcsname{sides="twoside"}
\expandafter\def\csname onesideAttrib\endcsname{sides="oneside"}

\expandafter\def\csname finalAttrib\endcsname{qual="final"}
\expandafter\def\csname draftAttrib\endcsname{qual="draft"}

\expandafter\def\csname onecolumnAttrib\endcsname{columns="1"}
\expandafter\def\csname twocolumnAttrib\endcsname{columns="2"}

\expandafter\def\csname leqnoAttrib\endcsname{eqnnopos="leqno"}
\expandafter\def\csname reqnoAttrib\endcsname{eqnnopos="reqno"}

\expandafter\def\csname fleqnAttrib\endcsname{eqnpos="fleqn"}

\newtoks\temptoksb

\def\msiappend#1MSIXYZZY{%
   \expandafter\temptoksa\expandafter{\the\temptoksa #1}}

\def\msi@nextopt#1,{%
   \def\MSIXYZZY{MSIXYZZY}%
   \def\@Opt{#1}%
   \ifx\@Opt\MSIXYZZY%
      \let\@next\relax
   \else
      \ifmsi@firstopt
         \msi@firstoptfalse
      \else
         \expandafter\temptoksa\expandafter{\the\temptoksa\ }%
      \fi
      \edef\attrib{\csname #1Attrib\endcsname}%
      \expandafter\temptoksb\expandafter{\attrib}
      \expandafter\msiappend\the\temptoksb MSIXYZZY
      \let\@next\msi@nextopt
   \fi
   \@next}

\def\out@documentclassA[#1]#2{%
   %\msi@firstopttrue
   %\temptoksa={}%
   %\msi@nextopt#1,MSIXYZZY,%
   %\edef\colist{\the\temptoksa}%
   \msi@opendocument
   \temptoksa{#1}
   \ifx\temptoksa\relax
     \msi@br\msitag{<documentclass class="#2"/>}%
   \else
     \msi@br\msitag{<documentclass class="#2" options="#1"/>}%
   \fi}

\def\out@documentclassB#1{%
   \msi@opendocument
   \msi@br\msitag{<documentclass class="#1"/>}}

\newcount\preambleid

\def\out@preamble@tex#1{%
   \advance\preambleid by 1
   \msitag{^^0a<preambleTeX id="}\the\preambleid\msitag{"><![CDATA[#1}\msipreambleextra\msitag{]]></preambleTeX>}%
   \preambleToks={}}


%% \document and \enddocument

\def\out@begin@document{%
   %\br\msiopentag{docformat}{<docformat>}
   %   \msitag{<colist }\colist\msitag{/>}
   %\br\msiclosetag{docformat}{</docformat>}
   \expandafter\preambleToks\expandafter{\the\preambleToks^^0a\usepackage{bm}}%
   \expandafter\preambleToks\expandafter{\the\preambleToks^^0a\usepackage{tabulary}}%
%   \@ifundefined{@TCILATEX}
%     {\relax}
%     {\expandafter\preambleToks\expandafter{\the\preambleToks^^0a% Macros for Scientific Word and Scientific WorkPlace 5.5 documents saved with the LaTeX filter.
% Copyright (C) 2005 Mackichan Software, Inc.

\typeout{TCILATEX Macros for Scientific Word and Scientific WorkPlace 5.5 <06 Oct 2005>.}
\typeout{NOTICE:  This macro file is NOT proprietary and may be 
freely copied and distributed.}
%
\makeatletter

%%%%%%%%%%%%%%%%%%%%%
% pdfTeX related.
\ifx\pdfoutput\relax\let\pdfoutput=\undefined\fi
\newcount\msipdfoutput
\ifx\pdfoutput\undefined
\else
 \ifcase\pdfoutput
 \else 
    \msipdfoutput=1
    \ifx\paperwidth\undefined
    \else
      \ifdim\paperheight=0pt\relax
      \else
        \pdfpageheight\paperheight
      \fi
      \ifdim\paperwidth=0pt\relax
      \else
        \pdfpagewidth\paperwidth
      \fi
    \fi
  \fi  
\fi

%%%%%%%%%%%%%%%%%%%%%
% FMTeXButton
% This is used for putting TeXButtons in the 
% frontmatter of a document. Add a line like
% \QTagDef{FMTeXButton}{101}{} to the filter 
% section of the cst being used. Also add a
% new section containing:
%     [f_101]
%     ALIAS=FMTexButton
%     TAG_TYPE=FIELD
%     TAG_LEADIN=TeX Button:
%
% It also works to put \defs in the preamble after 
% the \input tcilatex
\def\FMTeXButton#1{#1}
%
%%%%%%%%%%%%%%%%%%%%%%
% macros for time
\newcount\@hour\newcount\@minute\chardef\@x10\chardef\@xv60
\def\tcitime{
\def\@time{%
  \@minute\time\@hour\@minute\divide\@hour\@xv
  \ifnum\@hour<\@x 0\fi\the\@hour:%
  \multiply\@hour\@xv\advance\@minute-\@hour
  \ifnum\@minute<\@x 0\fi\the\@minute
  }}%

%%%%%%%%%%%%%%%%%%%%%%
% macro for hyperref and msihyperref
%\@ifundefined{hyperref}{\def\hyperref#1#2#3#4{#2\ref{#4}#3}}{}

\def\x@hyperref#1#2#3{%
   % Turn off various catcodes before reading parameter 4
   \catcode`\~ = 12
   \catcode`\$ = 12
   \catcode`\_ = 12
   \catcode`\# = 12
   \catcode`\& = 12
   \catcode`\% = 12
   \y@hyperref{#1}{#2}{#3}%
}

\def\y@hyperref#1#2#3#4{%
   #2\ref{#4}#3
   \catcode`\~ = 13
   \catcode`\$ = 3
   \catcode`\_ = 8
   \catcode`\# = 6
   \catcode`\& = 4
   \catcode`\% = 14
}

\@ifundefined{hyperref}{\let\hyperref\x@hyperref}{}
\@ifundefined{msihyperref}{\let\msihyperref\x@hyperref}{}




% macro for external program call
\@ifundefined{qExtProgCall}{\def\qExtProgCall#1#2#3#4#5#6{\relax}}{}
%%%%%%%%%%%%%%%%%%%%%%
%
% macros for graphics
%
\def\FILENAME#1{#1}%
%
\def\QCTOpt[#1]#2{%
  \def\QCTOptB{#1}
  \def\QCTOptA{#2}
}
\def\QCTNOpt#1{%
  \def\QCTOptA{#1}
  \let\QCTOptB\empty
}
\def\Qct{%
  \@ifnextchar[{%
    \QCTOpt}{\QCTNOpt}
}
\def\QCBOpt[#1]#2{%
  \def\QCBOptB{#1}%
  \def\QCBOptA{#2}%
}
\def\QCBNOpt#1{%
  \def\QCBOptA{#1}%
  \let\QCBOptB\empty
}
\def\Qcb{%
  \@ifnextchar[{%
    \QCBOpt}{\QCBNOpt}%
}
\def\PrepCapArgs{%
  \ifx\QCBOptA\empty
    \ifx\QCTOptA\empty
      {}%
    \else
      \ifx\QCTOptB\empty
        {\QCTOptA}%
      \else
        [\QCTOptB]{\QCTOptA}%
      \fi
    \fi
  \else
    \ifx\QCBOptA\empty
      {}%
    \else
      \ifx\QCBOptB\empty
        {\QCBOptA}%
      \else
        [\QCBOptB]{\QCBOptA}%
      \fi
    \fi
  \fi
}
\newcount\GRAPHICSTYPE
%\GRAPHICSTYPE 0 is for TurboTeX
%\GRAPHICSTYPE 1 is for DVIWindo (PostScript)
%%%(removed)%\GRAPHICSTYPE 2 is for psfig (PostScript)
\GRAPHICSTYPE=\z@
\def\GRAPHICSPS#1{%
 \ifcase\GRAPHICSTYPE%\GRAPHICSTYPE=0
   \special{ps: #1}%
 \or%\GRAPHICSTYPE=1
   \special{language "PS", include "#1"}%
%%%\or%\GRAPHICSTYPE=2
%%%  #1%
 \fi
}%
%
\def\GRAPHICSHP#1{\special{include #1}}%
%
% \graffile{ body }                                  %#1
%          { contentswidth (scalar)  }               %#2
%          { contentsheight (scalar) }               %#3
%          { vertical shift when in-line (scalar) }  %#4

\def\graffile#1#2#3#4{%
%%% \ifnum\GRAPHICSTYPE=\tw@
%%%  %Following if using psfig
%%%  \@ifundefined{psfig}{\input psfig.tex}{}%
%%%  \psfig{file=#1, height=#3, width=#2}%
%%% \else
  %Following for all others
  % JCS - added BOXTHEFRAME, see below
    \bgroup
	   \@inlabelfalse
       \leavevmode
       \@ifundefined{bbl@deactivate}{\def~{\string~}}{\activesoff}%
        \raise -#4 \BOXTHEFRAME{%
           \hbox to #2{\raise #3\hbox to #2{\null #1\hfil}}}%
    \egroup
}%
%
% A box for drafts
\def\draftbox#1#2#3#4{%
 \leavevmode\raise -#4 \hbox{%
  \frame{\rlap{\protect\tiny #1}\hbox to #2%
   {\vrule height#3 width\z@ depth\z@\hfil}%
  }%
 }%
}%
%
\newcount\@msidraft
\@msidraft=\z@
\let\nographics=\@msidraft
\newif\ifwasdraft
\wasdraftfalse

%  \GRAPHIC{ body }                                  %#1
%          { draft name }                            %#2
%          { contentswidth (scalar)  }               %#3
%          { contentsheight (scalar) }               %#4
%          { vertical shift when in-line (scalar) }  %#5
\def\GRAPHIC#1#2#3#4#5{%
   \ifnum\@msidraft=\@ne\draftbox{#2}{#3}{#4}{#5}%
   \else\graffile{#1}{#3}{#4}{#5}%
   \fi
}
%
\def\addtoLaTeXparams#1{%
    \edef\LaTeXparams{\LaTeXparams #1}}%
%
% JCS -  added a switch BoxFrame that can 
% be set by including X in the frame params.
% If set a box is drawn around the frame.

\newif\ifBoxFrame \BoxFramefalse
\newif\ifOverFrame \OverFramefalse
\newif\ifUnderFrame \UnderFramefalse

\def\BOXTHEFRAME#1{%
   \hbox{%
      \ifBoxFrame
         \frame{#1}%
      \else
         {#1}%
      \fi
   }%
}


\def\doFRAMEparams#1{\BoxFramefalse\OverFramefalse\UnderFramefalse\readFRAMEparams#1\end}%
\def\readFRAMEparams#1{%
 \ifx#1\end%
  \let\next=\relax
  \else
  \ifx#1i\dispkind=\z@\fi
  \ifx#1d\dispkind=\@ne\fi
  \ifx#1f\dispkind=\tw@\fi
  \ifx#1t\addtoLaTeXparams{t}\fi
  \ifx#1b\addtoLaTeXparams{b}\fi
  \ifx#1p\addtoLaTeXparams{p}\fi
  \ifx#1h\addtoLaTeXparams{h}\fi
  \ifx#1X\BoxFrametrue\fi
  \ifx#1O\OverFrametrue\fi
  \ifx#1U\UnderFrametrue\fi
  \ifx#1w
    \ifnum\@msidraft=1\wasdrafttrue\else\wasdraftfalse\fi
    \@msidraft=\@ne
  \fi
  \let\next=\readFRAMEparams
  \fi
 \next
 }%
%
%Macro for In-line graphics object
%   \IFRAME{ contentswidth (scalar)  }               %#1
%          { contentsheight (scalar) }               %#2
%          { vertical shift when in-line (scalar) }  %#3
%          { draft name }                            %#4
%          { body }                                  %#5
%          { caption}                                %#6


\def\IFRAME#1#2#3#4#5#6{%
      \bgroup
      \let\QCTOptA\empty
      \let\QCTOptB\empty
      \let\QCBOptA\empty
      \let\QCBOptB\empty
      #6%
      \parindent=0pt
      \leftskip=0pt
      \rightskip=0pt
      \setbox0=\hbox{\QCBOptA}%
      \@tempdima=#1\relax
      \ifOverFrame
          % Do this later
          \typeout{This is not implemented yet}%
          \show\HELP
      \else
         \ifdim\wd0>\@tempdima
            \advance\@tempdima by \@tempdima
            \ifdim\wd0 >\@tempdima
               \setbox1 =\vbox{%
                  \unskip\hbox to \@tempdima{\hfill\GRAPHIC{#5}{#4}{#1}{#2}{#3}\hfill}%
                  \unskip\hbox to \@tempdima{\parbox[b]{\@tempdima}{\QCBOptA}}%
               }%
               \wd1=\@tempdima
            \else
               \textwidth=\wd0
               \setbox1 =\vbox{%
                 \noindent\hbox to \wd0{\hfill\GRAPHIC{#5}{#4}{#1}{#2}{#3}\hfill}\\%
                 \noindent\hbox{\QCBOptA}%
               }%
               \wd1=\wd0
            \fi
         \else
            \ifdim\wd0>0pt
              \hsize=\@tempdima
              \setbox1=\vbox{%
                \unskip\GRAPHIC{#5}{#4}{#1}{#2}{0pt}%
                \break
                \unskip\hbox to \@tempdima{\hfill \QCBOptA\hfill}%
              }%
              \wd1=\@tempdima
           \else
              \hsize=\@tempdima
              \setbox1=\vbox{%
                \unskip\GRAPHIC{#5}{#4}{#1}{#2}{0pt}%
              }%
              \wd1=\@tempdima
           \fi
         \fi
         \@tempdimb=\ht1
         %\advance\@tempdimb by \dp1
         \advance\@tempdimb by -#2
         \advance\@tempdimb by #3
         \leavevmode
         \raise -\@tempdimb \hbox{\box1}%
      \fi
      \egroup%
}%
%
%Macro for Display graphics object
%   \DFRAME{ contentswidth (scalar)  }               %#1
%          { contentsheight (scalar) }               %#2
%          { draft label }                           %#3
%          { name }                                  %#4
%          { caption}                                %#5
\def\DFRAME#1#2#3#4#5{%
  \vspace\topsep
  \hfil\break
  \bgroup
     \leftskip\@flushglue
	 \rightskip\@flushglue
	 \parindent\z@
	 \parfillskip\z@skip
     \let\QCTOptA\empty
     \let\QCTOptB\empty
     \let\QCBOptA\empty
     \let\QCBOptB\empty
	 \vbox\bgroup
        \ifOverFrame 
           #5\QCTOptA\par
        \fi
        \GRAPHIC{#4}{#3}{#1}{#2}{\z@}%
        \ifUnderFrame 
           \break#5\QCBOptA
        \fi
	 \egroup
  \egroup
  \vspace\topsep
  \break
}%
%
%Macro for Floating graphic object
%   \FFRAME{ framedata f|i tbph x F|T }              %#1
%          { contentswidth (scalar)  }               %#2
%          { contentsheight (scalar) }               %#3
%          { caption }                               %#4
%          { label }                                 %#5
%          { draft name }                            %#6
%          { body }                                  %#7
\def\FFRAME#1#2#3#4#5#6#7{%
 %If float.sty loaded and float option is 'h', change to 'H'  (gp) 1998/09/05
  \@ifundefined{floatstyle}
    {%floatstyle undefined (and float.sty not present), no change
     \begin{figure}[#1]%
    }
    {%floatstyle DEFINED
	 \ifx#1h%Only the h parameter, change to H
      \begin{figure}[H]%
	 \else
      \begin{figure}[#1]%
	 \fi
	}
  \let\QCTOptA\empty
  \let\QCTOptB\empty
  \let\QCBOptA\empty
  \let\QCBOptB\empty
  \ifOverFrame
    #4
    \ifx\QCTOptA\empty
    \else
      \ifx\QCTOptB\empty
        \caption{\QCTOptA}%
      \else
        \caption[\QCTOptB]{\QCTOptA}%
      \fi
    \fi
    \ifUnderFrame\else
      \label{#5}%
    \fi
  \else
    \UnderFrametrue%
  \fi
  \begin{center}\GRAPHIC{#7}{#6}{#2}{#3}{\z@}\end{center}%
  \ifUnderFrame
    #4
    \ifx\QCBOptA\empty
      \caption{}%
    \else
      \ifx\QCBOptB\empty
        \caption{\QCBOptA}%
      \else
        \caption[\QCBOptB]{\QCBOptA}%
      \fi
    \fi
    \label{#5}%
  \fi
  \end{figure}%
 }%
%
%
%    \FRAME{ framedata f|i tbph x F|T }              %#1
%          { contentswidth (scalar)  }               %#2
%          { contentsheight (scalar) }               %#3
%          { vertical shift when in-line (scalar) }  %#4
%          { caption }                               %#5
%          { label }                                 %#6
%          { name }                                  %#7
%          { body }                                  %#8
%
%    framedata is a string which can contain the following
%    characters: idftbphxFT
%    Their meaning is as follows:
%             i, d or f : in-line, display, or floating
%             t,b,p,h   : LaTeX floating placement options
%             x         : fit contents box to contents
%             F or T    : Figure or Table. 
%                         Later this can expand
%                         to a more general float class.
%
%
\newcount\dispkind%

\def\makeactives{
  \catcode`\"=\active
  \catcode`\;=\active
  \catcode`\:=\active
  \catcode`\'=\active
  \catcode`\~=\active
}
\bgroup
   \makeactives
   \gdef\activesoff{%
      \def"{\string"}%
      \def;{\string;}%
      \def:{\string:}%
      \def'{\string'}%
      \def~{\string~}%
      %\bbl@deactivate{"}%
      %\bbl@deactivate{;}%
      %\bbl@deactivate{:}%
      %\bbl@deactivate{'}%
    }
\egroup

\def\FRAME#1#2#3#4#5#6#7#8{%
 \bgroup
 \ifnum\@msidraft=\@ne
   \wasdrafttrue
 \else
   \wasdraftfalse%
 \fi
 \def\LaTeXparams{}%
 \dispkind=\z@
 \def\LaTeXparams{}%
 \doFRAMEparams{#1}%
 \ifnum\dispkind=\z@\IFRAME{#2}{#3}{#4}{#7}{#8}{#5}\else
  \ifnum\dispkind=\@ne\DFRAME{#2}{#3}{#7}{#8}{#5}\else
   \ifnum\dispkind=\tw@
    \edef\@tempa{\noexpand\FFRAME{\LaTeXparams}}%
    \@tempa{#2}{#3}{#5}{#6}{#7}{#8}%
    \fi
   \fi
  \fi
  \ifwasdraft\@msidraft=1\else\@msidraft=0\fi{}%
  \egroup
 }%
%
% This macro added to let SW gobble a parameter that
% should not be passed on and expanded. 

\def\TEXUX#1{"texux"}

%
% Macros for text attributes:
%
\def\BF#1{{\bf {#1}}}%
\def\NEG#1{\leavevmode\hbox{\rlap{\thinspace/}{$#1$}}}%
%
%%%%%%%%%%%%%%%%%%%%%%%%%%%%%%%%%%%%%%%%%%%%%%%%%%%%%%%%%%%%%%%%%%%%%%%%
%
%
% macros for user - defined functions
\def\limfunc#1{\mathop{\rm #1}}%
\def\func#1{\mathop{\rm #1}\nolimits}%
% macro for unit names
\def\unit#1{\mathord{\thinspace\rm #1}}%

%
% miscellaneous 
\long\def\QQQ#1#2{%
     \long\expandafter\def\csname#1\endcsname{#2}}%
\@ifundefined{QTP}{\def\QTP#1{}}{}
\@ifundefined{QEXCLUDE}{\def\QEXCLUDE#1{}}{}
\@ifundefined{Qlb}{\def\Qlb#1{#1}}{}
\@ifundefined{Qlt}{\def\Qlt#1{#1}}{}
\def\QWE{}%
\long\def\QQA#1#2{}%
\def\QTR#1#2{{\csname#1\endcsname {#2}}}%
  %	Add aliases for the ulem package
  \let\QQQuline\uline
  \let\QQQsout\sout
  \let\QQQuuline\uuline
  \let\QQQuwave\uwave
  \let\QQQxout\xout
\long\def\TeXButton#1#2{#2}%
\long\def\QSubDoc#1#2{#2}%
\def\EXPAND#1[#2]#3{}%
\def\NOEXPAND#1[#2]#3{}%
\def\PROTECTED{}%
\def\LaTeXparent#1{}%
\def\ChildStyles#1{}%
\def\ChildDefaults#1{}%
\def\QTagDef#1#2#3{}%

% Constructs added with Scientific Notebook
\@ifundefined{correctchoice}{\def\correctchoice{\relax}}{}
\@ifundefined{HTML}{\def\HTML#1{\relax}}{}
\@ifundefined{TCIIcon}{\def\TCIIcon#1#2#3#4{\relax}}{}
\if@compatibility
  \typeout{Not defining UNICODE  U or CustomNote commands for LaTeX 2.09.}
\else
  \providecommand{\UNICODE}[2][]{\protect\rule{.1in}{.1in}}
  \providecommand{\U}[1]{\protect\rule{.1in}{.1in}}
  \providecommand{\CustomNote}[3][]{\marginpar{#3}}
\fi

\@ifundefined{lambdabar}{
      \def\lambdabar{\errmessage{You have used the lambdabar symbol. 
                      This is available for typesetting only in RevTeX styles.}}
   }{}

%
% Macros for style editor docs
\@ifundefined{StyleEditBeginDoc}{\def\StyleEditBeginDoc{\relax}}{}
%
% Macros for footnotes
\def\QQfnmark#1{\footnotemark}
\def\QQfntext#1#2{\addtocounter{footnote}{#1}\footnotetext{#2}}
%
% Macros for indexing.
%
\@ifundefined{TCIMAKEINDEX}{}{\makeindex}%
%
% Attempts to avoid problems with other styles
\@ifundefined{abstract}{%
 \def\abstract{%
  \if@twocolumn
   \section*{Abstract (Not appropriate in this style!)}%
   \else \small 
   \begin{center}{\bf Abstract\vspace{-.5em}\vspace{\z@}}\end{center}%
   \quotation 
   \fi
  }%
 }{%
 }%
\@ifundefined{endabstract}{\def\endabstract
  {\if@twocolumn\else\endquotation\fi}}{}%
\@ifundefined{maketitle}{\def\maketitle#1{}}{}%
\@ifundefined{affiliation}{\def\affiliation#1{}}{}%
\@ifundefined{proof}{\def\proof{\noindent{\bfseries Proof. }}}{}%
\@ifundefined{endproof}{\def\endproof{\mbox{\ \rule{.1in}{.1in}}}}{}%
\@ifundefined{newfield}{\def\newfield#1#2{}}{}%
\@ifundefined{chapter}{\def\chapter#1{\par(Chapter head:)#1\par }%
 \newcount\c@chapter}{}%
\@ifundefined{part}{\def\part#1{\par(Part head:)#1\par }}{}%
\@ifundefined{section}{\def\section#1{\par(Section head:)#1\par }}{}%
\@ifundefined{subsection}{\def\subsection#1%
 {\par(Subsection head:)#1\par }}{}%
\@ifundefined{subsubsection}{\def\subsubsection#1%
 {\par(Subsubsection head:)#1\par }}{}%
\@ifundefined{paragraph}{\def\paragraph#1%
 {\par(Subsubsubsection head:)#1\par }}{}%
\@ifundefined{subparagraph}{\def\subparagraph#1%
 {\par(Subsubsubsubsection head:)#1\par }}{}%
%%%%%%%%%%%%%%%%%%%%%%%%%%%%%%%%%%%%%%%%%%%%%%%%%%%%%%%%%%%%%%%%%%%%%%%%
% These symbols are not recognized by LaTeX
\@ifundefined{therefore}{\def\therefore{}}{}%
\@ifundefined{backepsilon}{\def\backepsilon{}}{}%
\@ifundefined{yen}{\def\yen{\hbox{\rm\rlap=Y}}}{}%
\@ifundefined{registered}{%
   \def\registered{\relax\ifmmode{}\r@gistered
                    \else$\m@th\r@gistered$\fi}%
 \def\r@gistered{^{\ooalign
  {\hfil\raise.07ex\hbox{$\scriptstyle\rm\text{R}$}\hfil\crcr
  \mathhexbox20D}}}}{}%
\@ifundefined{Eth}{\def\Eth{}}{}%
\@ifundefined{eth}{\def\eth{}}{}%
\@ifundefined{Thorn}{\def\Thorn{}}{}%
\@ifundefined{thorn}{\def\thorn{}}{}%
% A macro to allow any symbol that requires math to appear in text
\def\TEXTsymbol#1{\mbox{$#1$}}%
\@ifundefined{degree}{\def\degree{{}^{\circ}}}{}%
%
% macros for T3TeX files
\newdimen\theight
\@ifundefined{Column}{\def\Column{%
 \vadjust{\setbox\z@=\hbox{\scriptsize\quad\quad tcol}%
  \theight=\ht\z@\advance\theight by \dp\z@\advance\theight by \lineskip
  \kern -\theight \vbox to \theight{%
   \rightline{\rlap{\box\z@}}%
   \vss
   }%
  }%
 }}{}%
%
\@ifundefined{qed}{\def\qed{%
 \ifhmode\unskip\nobreak\fi\ifmmode\ifinner\else\hskip5\p@\fi\fi
 \hbox{\hskip5\p@\vrule width4\p@ height6\p@ depth1.5\p@\hskip\p@}%
 }}{}%
%
\@ifundefined{cents}{\def\cents{\hbox{\rm\rlap c/}}}{}%
\@ifundefined{tciLaplace}{\def\tciLaplace{\ensuremath{\mathcal{L}}}}{}%
\@ifundefined{tciFourier}{\def\tciFourier{\ensuremath{\mathcal{F}}}}{}%
\@ifundefined{textcurrency}{\def\textcurrency{\hbox{\rm\rlap xo}}}{}%
\@ifundefined{texteuro}{\def\texteuro{\hbox{\rm\rlap C=}}}{}%
\@ifundefined{euro}{\def\euro{\hbox{\rm\rlap C=}}}{}%
\@ifundefined{textfranc}{\def\textfranc{\hbox{\rm\rlap-F}}}{}%
\@ifundefined{textlira}{\def\textlira{\hbox{\rm\rlap L=}}}{}%
\@ifundefined{textpeseta}{\def\textpeseta{\hbox{\rm P\negthinspace s}}}{}%
%
\@ifundefined{miss}{\def\miss{\hbox{\vrule height2\p@ width 2\p@ depth\z@}}}{}%
%
\@ifundefined{vvert}{\def\vvert{\Vert}}{}%  %always translated to \left| or \right|
%
\@ifundefined{tcol}{\def\tcol#1{{\baselineskip=6\p@ \vcenter{#1}} \Column}}{}%
%
\@ifundefined{dB}{\def\dB{\hbox{{}}}}{}%        %dummy entry in column 
\@ifundefined{mB}{\def\mB#1{\hbox{$#1$}}}{}%   %column entry
\@ifundefined{nB}{\def\nB#1{\hbox{#1}}}{}%     %column entry (not math)
%
\@ifundefined{note}{\def\note{$^{\dag}}}{}%
%
\def\newfmtname{LaTeX2e}
% No longer load latexsym.  This is now handled by SWP, which uses amsfonts if necessary
%
\ifx\fmtname\newfmtname
  \DeclareOldFontCommand{\rm}{\normalfont\rmfamily}{\mathrm}
  \DeclareOldFontCommand{\sf}{\normalfont\sffamily}{\mathsf}
  \DeclareOldFontCommand{\tt}{\normalfont\ttfamily}{\mathtt}
  \DeclareOldFontCommand{\bf}{\normalfont\bfseries}{\mathbf}
  \DeclareOldFontCommand{\it}{\normalfont\itshape}{\mathit}
  \DeclareOldFontCommand{\sl}{\normalfont\slshape}{\@nomath\sl}
  \DeclareOldFontCommand{\sc}{\normalfont\scshape}{\@nomath\sc}
\fi

%
% Greek bold macros
% Redefine all of the math symbols 
% which might be bolded	 - there are 
% probably others to add to this list

\def\alpha{{\Greekmath 010B}}%
\def\beta{{\Greekmath 010C}}%
\def\gamma{{\Greekmath 010D}}%
\def\delta{{\Greekmath 010E}}%
\def\epsilon{{\Greekmath 010F}}%
\def\zeta{{\Greekmath 0110}}%
\def\eta{{\Greekmath 0111}}%
\def\theta{{\Greekmath 0112}}%
\def\iota{{\Greekmath 0113}}%
\def\kappa{{\Greekmath 0114}}%
\def\lambda{{\Greekmath 0115}}%
\def\mu{{\Greekmath 0116}}%
\def\nu{{\Greekmath 0117}}%
\def\xi{{\Greekmath 0118}}%
\def\pi{{\Greekmath 0119}}%
\def\rho{{\Greekmath 011A}}%
\def\sigma{{\Greekmath 011B}}%
\def\tau{{\Greekmath 011C}}%
\def\upsilon{{\Greekmath 011D}}%
\def\phi{{\Greekmath 011E}}%
\def\chi{{\Greekmath 011F}}%
\def\psi{{\Greekmath 0120}}%
\def\omega{{\Greekmath 0121}}%
\def\varepsilon{{\Greekmath 0122}}%
\def\vartheta{{\Greekmath 0123}}%
\def\varpi{{\Greekmath 0124}}%
\def\varrho{{\Greekmath 0125}}%
\def\varsigma{{\Greekmath 0126}}%
\def\varphi{{\Greekmath 0127}}%

\def\nabla{{\Greekmath 0272}}
\def\FindBoldGroup{%
   {\setbox0=\hbox{$\mathbf{x\global\edef\theboldgroup{\the\mathgroup}}$}}%
}

\def\Greekmath#1#2#3#4{%
    \if@compatibility
        \ifnum\mathgroup=\symbold
           \mathchoice{\mbox{\boldmath$\displaystyle\mathchar"#1#2#3#4$}}%
                      {\mbox{\boldmath$\textstyle\mathchar"#1#2#3#4$}}%
                      {\mbox{\boldmath$\scriptstyle\mathchar"#1#2#3#4$}}%
                      {\mbox{\boldmath$\scriptscriptstyle\mathchar"#1#2#3#4$}}%
        \else
           \mathchar"#1#2#3#4% 
        \fi 
    \else 
        \FindBoldGroup
        \ifnum\mathgroup=\theboldgroup % For 2e
           \mathchoice{\mbox{\boldmath$\displaystyle\mathchar"#1#2#3#4$}}%
                      {\mbox{\boldmath$\textstyle\mathchar"#1#2#3#4$}}%
                      {\mbox{\boldmath$\scriptstyle\mathchar"#1#2#3#4$}}%
                      {\mbox{\boldmath$\scriptscriptstyle\mathchar"#1#2#3#4$}}%
        \else
           \mathchar"#1#2#3#4% 
        \fi     	    
	  \fi}

\newif\ifGreekBold  \GreekBoldfalse
\let\SAVEPBF=\pbf
\def\pbf{\GreekBoldtrue\SAVEPBF}%
%

\@ifundefined{theorem}{\newtheorem{theorem}{Theorem}}{}
\@ifundefined{lemma}{\newtheorem{lemma}[theorem]{Lemma}}{}
\@ifundefined{corollary}{\newtheorem{corollary}[theorem]{Corollary}}{}
\@ifundefined{conjecture}{\newtheorem{conjecture}[theorem]{Conjecture}}{}
\@ifundefined{proposition}{\newtheorem{proposition}[theorem]{Proposition}}{}
\@ifundefined{axiom}{\newtheorem{axiom}{Axiom}}{}
\@ifundefined{remark}{\newtheorem{remark}{Remark}}{}
\@ifundefined{example}{\newtheorem{example}{Example}}{}
\@ifundefined{exercise}{\newtheorem{exercise}{Exercise}}{}
\@ifundefined{definition}{\newtheorem{definition}{Definition}}{}


\@ifundefined{mathletters}{%
  %\def\theequation{\arabic{equation}}
  \newcounter{equationnumber}  
  \def\mathletters{%
     \addtocounter{equation}{1}
     \edef\@currentlabel{\theequation}%
     \setcounter{equationnumber}{\c@equation}
     \setcounter{equation}{0}%
     \edef\theequation{\@currentlabel\noexpand\alph{equation}}%
  }
  \def\endmathletters{%
     \setcounter{equation}{\value{equationnumber}}%
  }
}{}

%Logos
\@ifundefined{BibTeX}{%
    \def\BibTeX{{\rm B\kern-.05em{\sc i\kern-.025em b}\kern-.08em
                 T\kern-.1667em\lower.7ex\hbox{E}\kern-.125emX}}}{}%
\@ifundefined{AmS}%
    {\def\AmS{{\protect\usefont{OMS}{cmsy}{m}{n}%
                A\kern-.1667em\lower.5ex\hbox{M}\kern-.125emS}}}{}%
\@ifundefined{AmSTeX}{\def\AmSTeX{\protect\AmS-\protect\TeX\@}}{}%
%

% This macro is a fix to eqnarray
\def\@@eqncr{\let\@tempa\relax
    \ifcase\@eqcnt \def\@tempa{& & &}\or \def\@tempa{& &}%
      \else \def\@tempa{&}\fi
     \@tempa
     \if@eqnsw
        \iftag@
           \@taggnum
        \else
           \@eqnnum\stepcounter{equation}%
        \fi
     \fi
     \global\tag@false
     \global\@eqnswtrue
     \global\@eqcnt\z@\cr}


\def\TCItag{\@ifnextchar*{\@TCItagstar}{\@TCItag}}
\def\@TCItag#1{%
    \global\tag@true
    \global\def\@taggnum{(#1)}%
    \global\def\@currentlabel{#1}}
\def\@TCItagstar*#1{%
    \global\tag@true
    \global\def\@taggnum{#1}%
    \global\def\@currentlabel{#1}}
%
%%%%%%%%%%%%%%%%%%%%%%%%%%%%%%%%%%%%%%%%%%%%%%%%%%%%%%%%%%%%%%%%%%%%%
%
\def\QATOP#1#2{{#1 \atop #2}}%
\def\QTATOP#1#2{{\textstyle {#1 \atop #2}}}%
\def\QDATOP#1#2{{\displaystyle {#1 \atop #2}}}%
\def\QABOVE#1#2#3{{#2 \above#1 #3}}%
\def\QTABOVE#1#2#3{{\textstyle {#2 \above#1 #3}}}%
\def\QDABOVE#1#2#3{{\displaystyle {#2 \above#1 #3}}}%
\def\QOVERD#1#2#3#4{{#3 \overwithdelims#1#2 #4}}%
\def\QTOVERD#1#2#3#4{{\textstyle {#3 \overwithdelims#1#2 #4}}}%
\def\QDOVERD#1#2#3#4{{\displaystyle {#3 \overwithdelims#1#2 #4}}}%
\def\QATOPD#1#2#3#4{{#3 \atopwithdelims#1#2 #4}}%
\def\QTATOPD#1#2#3#4{{\textstyle {#3 \atopwithdelims#1#2 #4}}}%
\def\QDATOPD#1#2#3#4{{\displaystyle {#3 \atopwithdelims#1#2 #4}}}%
\def\QABOVED#1#2#3#4#5{{#4 \abovewithdelims#1#2#3 #5}}%
\def\QTABOVED#1#2#3#4#5{{\textstyle 
   {#4 \abovewithdelims#1#2#3 #5}}}%
\def\QDABOVED#1#2#3#4#5{{\displaystyle 
   {#4 \abovewithdelims#1#2#3 #5}}}%
%
% Macros for text size operators:
%

\def\tint{\msi@int\textstyle\int}%
\def\tiint{\msi@int\textstyle\iint}%
\def\tiiint{\msi@int\textstyle\iiint}%
\def\tiiiint{\msi@int\textstyle\iiiint}%
\def\tidotsint{\msi@int\textstyle\idotsint}%
\def\toint{\msi@int\textstyle\oint}%


\def\tsum{\mathop{\textstyle \sum }}%
\def\tprod{\mathop{\textstyle \prod }}%
\def\tbigcap{\mathop{\textstyle \bigcap }}%
\def\tbigwedge{\mathop{\textstyle \bigwedge }}%
\def\tbigoplus{\mathop{\textstyle \bigoplus }}%
\def\tbigodot{\mathop{\textstyle \bigodot }}%
\def\tbigsqcup{\mathop{\textstyle \bigsqcup }}%
\def\tcoprod{\mathop{\textstyle \coprod }}%
\def\tbigcup{\mathop{\textstyle \bigcup }}%
\def\tbigvee{\mathop{\textstyle \bigvee }}%
\def\tbigotimes{\mathop{\textstyle \bigotimes }}%
\def\tbiguplus{\mathop{\textstyle \biguplus }}%
%
%
%Macros for display size operators:
%

\newtoks\temptoksa
\newtoks\temptoksb
\newtoks\temptoksc


\def\msi@int#1#2{%
 \def\@temp{{#1#2\the\temptoksc_{\the\temptoksa}^{\the\temptoksb}}}%   
 \futurelet\@nextcs
 \@int
}

\def\@int{%
   \ifx\@nextcs\limits
      \typeout{Found limits}%
      \temptoksc={\limits}%
	  \let\@next\@intgobble%
   \else\ifx\@nextcs\nolimits
      \typeout{Found nolimits}%
      \temptoksc={\nolimits}%
	  \let\@next\@intgobble%
   \else
      \typeout{Did not find limits or no limits}%
      \temptoksc={}%
      \let\@next\msi@limits%
   \fi\fi
   \@next   
}%

\def\@intgobble#1{%
   \typeout{arg is #1}%
   \msi@limits
}


\def\msi@limits{%
   \temptoksa={}%
   \temptoksb={}%
   \@ifnextchar_{\@limitsa}{\@limitsb}%
}

\def\@limitsa_#1{%
   \temptoksa={#1}%
   \@ifnextchar^{\@limitsc}{\@temp}%
}

\def\@limitsb{%
   \@ifnextchar^{\@limitsc}{\@temp}%
}

\def\@limitsc^#1{%
   \temptoksb={#1}%
   \@ifnextchar_{\@limitsd}{\@temp}%   
}

\def\@limitsd_#1{%
   \temptoksa={#1}%
   \@temp
}



\def\dint{\msi@int\displaystyle\int}%
\def\diint{\msi@int\displaystyle\iint}%
\def\diiint{\msi@int\displaystyle\iiint}%
\def\diiiint{\msi@int\displaystyle\iiiint}%
\def\didotsint{\msi@int\displaystyle\idotsint}%
\def\doint{\msi@int\displaystyle\oint}%

\def\dsum{\mathop{\displaystyle \sum }}%
\def\dprod{\mathop{\displaystyle \prod }}%
\def\dbigcap{\mathop{\displaystyle \bigcap }}%
\def\dbigwedge{\mathop{\displaystyle \bigwedge }}%
\def\dbigoplus{\mathop{\displaystyle \bigoplus }}%
\def\dbigodot{\mathop{\displaystyle \bigodot }}%
\def\dbigsqcup{\mathop{\displaystyle \bigsqcup }}%
\def\dcoprod{\mathop{\displaystyle \coprod }}%
\def\dbigcup{\mathop{\displaystyle \bigcup }}%
\def\dbigvee{\mathop{\displaystyle \bigvee }}%
\def\dbigotimes{\mathop{\displaystyle \bigotimes }}%
\def\dbiguplus{\mathop{\displaystyle \biguplus }}%

\if@compatibility\else
  % Always load amsmath in LaTeX2e mode
  \RequirePackage{amsmath}
\fi

\def\ExitTCILatex{\makeatother\endinput}

\bgroup
\ifx\ds@amstex\relax
   \message{amstex already loaded}\aftergroup\ExitTCILatex
\else
   \@ifpackageloaded{amsmath}%
      {\if@compatibility\message{amsmath already loaded}\fi\aftergroup\ExitTCILatex}
      {}
   \@ifpackageloaded{amstex}%
      {\if@compatibility\message{amstex already loaded}\fi\aftergroup\ExitTCILatex}
      {}
   \@ifpackageloaded{amsgen}%
      {\if@compatibility\message{amsgen already loaded}\fi\aftergroup\ExitTCILatex}
      {}
\fi
\egroup

%Exit if any of the AMS macros are already loaded.
%This is always the case for LaTeX2e mode.


%%%%%%%%%%%%%%%%%%%%%%%%%%%%%%%%%%%%%%%%%%%%%%%%%%%%%%%%%%%%%%%%%%%%%%%%%%
% NOTE: The rest of this file is read only if in LaTeX 2.09 compatibility
% mode. This section is used to define AMS-like constructs in the
% event they have not been defined.
%%%%%%%%%%%%%%%%%%%%%%%%%%%%%%%%%%%%%%%%%%%%%%%%%%%%%%%%%%%%%%%%%%%%%%%%%%
\typeout{TCILATEX defining AMS-like constructs in LaTeX 2.09 COMPATIBILITY MODE}
%%%%%%%%%%%%%%%%%%%%%%%%%%%%%%%%%%%%%%%%%%%%%%%%%%%%%%%%%%%%%%%%%%%%%%%%
%  Macros to define some AMS LaTeX constructs when 
%  AMS LaTeX has not been loaded
% 
% These macros are copied from the AMS-TeX package for doing
% multiple integrals.
%
\let\DOTSI\relax
\def\RIfM@{\relax\ifmmode}%
\def\FN@{\futurelet\next}%
\newcount\intno@
\def\iint{\DOTSI\intno@\tw@\FN@\ints@}%
\def\iiint{\DOTSI\intno@\thr@@\FN@\ints@}%
\def\iiiint{\DOTSI\intno@4 \FN@\ints@}%
\def\idotsint{\DOTSI\intno@\z@\FN@\ints@}%
\def\ints@{\findlimits@\ints@@}%
\newif\iflimtoken@
\newif\iflimits@
\def\findlimits@{\limtoken@true\ifx\next\limits\limits@true
 \else\ifx\next\nolimits\limits@false\else
 \limtoken@false\ifx\ilimits@\nolimits\limits@false\else
 \ifinner\limits@false\else\limits@true\fi\fi\fi\fi}%
\def\multint@{\int\ifnum\intno@=\z@\intdots@                          %1
 \else\intkern@\fi                                                    %2
 \ifnum\intno@>\tw@\int\intkern@\fi                                   %3
 \ifnum\intno@>\thr@@\int\intkern@\fi                                 %4
 \int}%                                                               %5
\def\multintlimits@{\intop\ifnum\intno@=\z@\intdots@\else\intkern@\fi
 \ifnum\intno@>\tw@\intop\intkern@\fi
 \ifnum\intno@>\thr@@\intop\intkern@\fi\intop}%
\def\intic@{%
    \mathchoice{\hskip.5em}{\hskip.4em}{\hskip.4em}{\hskip.4em}}%
\def\negintic@{\mathchoice
 {\hskip-.5em}{\hskip-.4em}{\hskip-.4em}{\hskip-.4em}}%
\def\ints@@{\iflimtoken@                                              %1
 \def\ints@@@{\iflimits@\negintic@
   \mathop{\intic@\multintlimits@}\limits                             %2
  \else\multint@\nolimits\fi                                          %3
  \eat@}%                                                             %4
 \else                                                                %5
 \def\ints@@@{\iflimits@\negintic@
  \mathop{\intic@\multintlimits@}\limits\else
  \multint@\nolimits\fi}\fi\ints@@@}%
\def\intkern@{\mathchoice{\!\!\!}{\!\!}{\!\!}{\!\!}}%
\def\plaincdots@{\mathinner{\cdotp\cdotp\cdotp}}%
\def\intdots@{\mathchoice{\plaincdots@}%
 {{\cdotp}\mkern1.5mu{\cdotp}\mkern1.5mu{\cdotp}}%
 {{\cdotp}\mkern1mu{\cdotp}\mkern1mu{\cdotp}}%
 {{\cdotp}\mkern1mu{\cdotp}\mkern1mu{\cdotp}}}%
%
%
%  These macros are for doing the AMS \text{} construct
%
\def\RIfM@{\relax\protect\ifmmode}
\def\text{\RIfM@\expandafter\text@\else\expandafter\mbox\fi}
\let\nfss@text\text
\def\text@#1{\mathchoice
   {\textdef@\displaystyle\f@size{#1}}%
   {\textdef@\textstyle\tf@size{\firstchoice@false #1}}%
   {\textdef@\textstyle\sf@size{\firstchoice@false #1}}%
   {\textdef@\textstyle \ssf@size{\firstchoice@false #1}}%
   \glb@settings}

\def\textdef@#1#2#3{\hbox{{%
                    \everymath{#1}%
                    \let\f@size#2\selectfont
                    #3}}}
\newif\iffirstchoice@
\firstchoice@true
%
%These are the AMS constructs for multiline limits.
%
\def\Let@{\relax\iffalse{\fi\let\\=\cr\iffalse}\fi}%
\def\vspace@{\def\vspace##1{\crcr\noalign{\vskip##1\relax}}}%
\def\multilimits@{\bgroup\vspace@\Let@
 \baselineskip\fontdimen10 \scriptfont\tw@
 \advance\baselineskip\fontdimen12 \scriptfont\tw@
 \lineskip\thr@@\fontdimen8 \scriptfont\thr@@
 \lineskiplimit\lineskip
 \vbox\bgroup\ialign\bgroup\hfil$\m@th\scriptstyle{##}$\hfil\crcr}%
\def\Sb{_\multilimits@}%
\def\endSb{\crcr\egroup\egroup\egroup}%
\def\Sp{^\multilimits@}%
\let\endSp\endSb
%
%
%These are AMS constructs for horizontal arrows
%
\newdimen\ex@
\ex@.2326ex
\def\rightarrowfill@#1{$#1\m@th\mathord-\mkern-6mu\cleaders
 \hbox{$#1\mkern-2mu\mathord-\mkern-2mu$}\hfill
 \mkern-6mu\mathord\rightarrow$}%
\def\leftarrowfill@#1{$#1\m@th\mathord\leftarrow\mkern-6mu\cleaders
 \hbox{$#1\mkern-2mu\mathord-\mkern-2mu$}\hfill\mkern-6mu\mathord-$}%
\def\leftrightarrowfill@#1{$#1\m@th\mathord\leftarrow
\mkern-6mu\cleaders
 \hbox{$#1\mkern-2mu\mathord-\mkern-2mu$}\hfill
 \mkern-6mu\mathord\rightarrow$}%
\def\overrightarrow{\mathpalette\overrightarrow@}%
\def\overrightarrow@#1#2{\vbox{\ialign{##\crcr\rightarrowfill@#1\crcr
 \noalign{\kern-\ex@\nointerlineskip}$\m@th\hfil#1#2\hfil$\crcr}}}%
\let\overarrow\overrightarrow
\def\overleftarrow{\mathpalette\overleftarrow@}%
\def\overleftarrow@#1#2{\vbox{\ialign{##\crcr\leftarrowfill@#1\crcr
 \noalign{\kern-\ex@\nointerlineskip}$\m@th\hfil#1#2\hfil$\crcr}}}%
\def\overleftrightarrow{\mathpalette\overleftrightarrow@}%
\def\overleftrightarrow@#1#2{\vbox{\ialign{##\crcr
   \leftrightarrowfill@#1\crcr
 \noalign{\kern-\ex@\nointerlineskip}$\m@th\hfil#1#2\hfil$\crcr}}}%
\def\underrightarrow{\mathpalette\underrightarrow@}%
\def\underrightarrow@#1#2{\vtop{\ialign{##\crcr$\m@th\hfil#1#2\hfil
  $\crcr\noalign{\nointerlineskip}\rightarrowfill@#1\crcr}}}%
\let\underarrow\underrightarrow
\def\underleftarrow{\mathpalette\underleftarrow@}%
\def\underleftarrow@#1#2{\vtop{\ialign{##\crcr$\m@th\hfil#1#2\hfil
  $\crcr\noalign{\nointerlineskip}\leftarrowfill@#1\crcr}}}%
\def\underleftrightarrow{\mathpalette\underleftrightarrow@}%
\def\underleftrightarrow@#1#2{\vtop{\ialign{##\crcr$\m@th
  \hfil#1#2\hfil$\crcr
 \noalign{\nointerlineskip}\leftrightarrowfill@#1\crcr}}}%
%%%%%%%%%%%%%%%%%%%%%

\def\qopnamewl@#1{\mathop{\operator@font#1}\nlimits@}
\let\nlimits@\displaylimits
\def\setboxz@h{\setbox\z@\hbox}


\def\varlim@#1#2{\mathop{\vtop{\ialign{##\crcr
 \hfil$#1\m@th\operator@font lim$\hfil\crcr
 \noalign{\nointerlineskip}#2#1\crcr
 \noalign{\nointerlineskip\kern-\ex@}\crcr}}}}

 \def\rightarrowfill@#1{\m@th\setboxz@h{$#1-$}\ht\z@\z@
  $#1\copy\z@\mkern-6mu\cleaders
  \hbox{$#1\mkern-2mu\box\z@\mkern-2mu$}\hfill
  \mkern-6mu\mathord\rightarrow$}
\def\leftarrowfill@#1{\m@th\setboxz@h{$#1-$}\ht\z@\z@
  $#1\mathord\leftarrow\mkern-6mu\cleaders
  \hbox{$#1\mkern-2mu\copy\z@\mkern-2mu$}\hfill
  \mkern-6mu\box\z@$}


\def\projlim{\qopnamewl@{proj\,lim}}
\def\injlim{\qopnamewl@{inj\,lim}}
\def\varinjlim{\mathpalette\varlim@\rightarrowfill@}
\def\varprojlim{\mathpalette\varlim@\leftarrowfill@}
\def\varliminf{\mathpalette\varliminf@{}}
\def\varliminf@#1{\mathop{\underline{\vrule\@depth.2\ex@\@width\z@
   \hbox{$#1\m@th\operator@font lim$}}}}
\def\varlimsup{\mathpalette\varlimsup@{}}
\def\varlimsup@#1{\mathop{\overline
  {\hbox{$#1\m@th\operator@font lim$}}}}

%
%Companion to stackrel
\def\stackunder#1#2{\mathrel{\mathop{#2}\limits_{#1}}}%
%
%
% These are AMS environments that will be defined to
% be verbatims if amstex has not actually been 
% loaded
%
%
\begingroup \catcode `|=0 \catcode `[= 1
\catcode`]=2 \catcode `\{=12 \catcode `\}=12
\catcode`\\=12 
|gdef|@alignverbatim#1\end{align}[#1|end[align]]
|gdef|@salignverbatim#1\end{align*}[#1|end[align*]]

|gdef|@alignatverbatim#1\end{alignat}[#1|end[alignat]]
|gdef|@salignatverbatim#1\end{alignat*}[#1|end[alignat*]]

|gdef|@xalignatverbatim#1\end{xalignat}[#1|end[xalignat]]
|gdef|@sxalignatverbatim#1\end{xalignat*}[#1|end[xalignat*]]

|gdef|@gatherverbatim#1\end{gather}[#1|end[gather]]
|gdef|@sgatherverbatim#1\end{gather*}[#1|end[gather*]]

|gdef|@gatherverbatim#1\end{gather}[#1|end[gather]]
|gdef|@sgatherverbatim#1\end{gather*}[#1|end[gather*]]


|gdef|@multilineverbatim#1\end{multiline}[#1|end[multiline]]
|gdef|@smultilineverbatim#1\end{multiline*}[#1|end[multiline*]]

|gdef|@arraxverbatim#1\end{arrax}[#1|end[arrax]]
|gdef|@sarraxverbatim#1\end{arrax*}[#1|end[arrax*]]

|gdef|@tabulaxverbatim#1\end{tabulax}[#1|end[tabulax]]
|gdef|@stabulaxverbatim#1\end{tabulax*}[#1|end[tabulax*]]


|endgroup
  

  
\def\align{\@verbatim \frenchspacing\@vobeyspaces \@alignverbatim
You are using the "align" environment in a style in which it is not defined.}
\let\endalign=\endtrivlist
 
\@namedef{align*}{\@verbatim\@salignverbatim
You are using the "align*" environment in a style in which it is not defined.}
\expandafter\let\csname endalign*\endcsname =\endtrivlist




\def\alignat{\@verbatim \frenchspacing\@vobeyspaces \@alignatverbatim
You are using the "alignat" environment in a style in which it is not defined.}
\let\endalignat=\endtrivlist
 
\@namedef{alignat*}{\@verbatim\@salignatverbatim
You are using the "alignat*" environment in a style in which it is not defined.}
\expandafter\let\csname endalignat*\endcsname =\endtrivlist




\def\xalignat{\@verbatim \frenchspacing\@vobeyspaces \@xalignatverbatim
You are using the "xalignat" environment in a style in which it is not defined.}
\let\endxalignat=\endtrivlist
 
\@namedef{xalignat*}{\@verbatim\@sxalignatverbatim
You are using the "xalignat*" environment in a style in which it is not defined.}
\expandafter\let\csname endxalignat*\endcsname =\endtrivlist




\def\gather{\@verbatim \frenchspacing\@vobeyspaces \@gatherverbatim
You are using the "gather" environment in a style in which it is not defined.}
\let\endgather=\endtrivlist
 
\@namedef{gather*}{\@verbatim\@sgatherverbatim
You are using the "gather*" environment in a style in which it is not defined.}
\expandafter\let\csname endgather*\endcsname =\endtrivlist


\def\multiline{\@verbatim \frenchspacing\@vobeyspaces \@multilineverbatim
You are using the "multiline" environment in a style in which it is not defined.}
\let\endmultiline=\endtrivlist
 
\@namedef{multiline*}{\@verbatim\@smultilineverbatim
You are using the "multiline*" environment in a style in which it is not defined.}
\expandafter\let\csname endmultiline*\endcsname =\endtrivlist


\def\arrax{\@verbatim \frenchspacing\@vobeyspaces \@arraxverbatim
You are using a type of "array" construct that is only allowed in AmS-LaTeX.}
\let\endarrax=\endtrivlist

\def\tabulax{\@verbatim \frenchspacing\@vobeyspaces \@tabulaxverbatim
You are using a type of "tabular" construct that is only allowed in AmS-LaTeX.}
\let\endtabulax=\endtrivlist

 
\@namedef{arrax*}{\@verbatim\@sarraxverbatim
You are using a type of "array*" construct that is only allowed in AmS-LaTeX.}
\expandafter\let\csname endarrax*\endcsname =\endtrivlist

\@namedef{tabulax*}{\@verbatim\@stabulaxverbatim
You are using a type of "tabular*" construct that is only allowed in AmS-LaTeX.}
\expandafter\let\csname endtabulax*\endcsname =\endtrivlist

% macro to simulate ams tag construct


% This macro is a fix to the equation environment
 \def\endequation{%
     \ifmmode\ifinner % FLEQN hack
      \iftag@
        \addtocounter{equation}{-1} % undo the increment made in the begin part
        $\hfil
           \displaywidth\linewidth\@taggnum\egroup \endtrivlist
        \global\tag@false
        \global\@ignoretrue   
      \else
        $\hfil
           \displaywidth\linewidth\@eqnnum\egroup \endtrivlist
        \global\tag@false
        \global\@ignoretrue 
      \fi
     \else   
      \iftag@
        \addtocounter{equation}{-1} % undo the increment made in the begin part
        \eqno \hbox{\@taggnum}
        \global\tag@false%
        $$\global\@ignoretrue
      \else
        \eqno \hbox{\@eqnnum}% $$ BRACE MATCHING HACK
        $$\global\@ignoretrue
      \fi
     \fi\fi
 } 

 \newif\iftag@ \tag@false
 
 \def\TCItag{\@ifnextchar*{\@TCItagstar}{\@TCItag}}
 \def\@TCItag#1{%
     \global\tag@true
     \global\def\@taggnum{(#1)}%
     \global\def\@currentlabel{#1}}
 \def\@TCItagstar*#1{%
     \global\tag@true
     \global\def\@taggnum{#1}%
     \global\def\@currentlabel{#1}}

  \@ifundefined{tag}{
     \def\tag{\@ifnextchar*{\@tagstar}{\@tag}}
     \def\@tag#1{%
         \global\tag@true
         \global\def\@taggnum{(#1)}}
     \def\@tagstar*#1{%
         \global\tag@true
         \global\def\@taggnum{#1}}
  }{}

\def\tfrac#1#2{{\textstyle {#1 \over #2}}}%
\def\dfrac#1#2{{\displaystyle {#1 \over #2}}}%
\def\binom#1#2{{#1 \choose #2}}%
\def\tbinom#1#2{{\textstyle {#1 \choose #2}}}%
\def\dbinom#1#2{{\displaystyle {#1 \choose #2}}}%

% Do not add anything to the end of this file.  
% The last section of the file is loaded only if 
% amstex has not been.
\makeatother
\endinput
}}
   \expandafter\out@preamble@tex\expandafter{\the\preambleToks}
   \msiclosetag{preamble}{</preamble>}%
   \msi@inpreamble = 0
   \msitag{</head>}\msitag{^^0a}%
   \msiopentag{body}{<body showexpanders="true" showshort="true">}
   \@paragraphs
   \msivmode
}

\def\out@end@document{%
   %%\br\msiclosetag{docbody}{</docbody>}%
   \msi@br\msiclosetag{body}{</body>}%
   \msi@br\msitag{</html>}%
   \msimakecss
   \endinput}


%% \usepackage
\def\out@usepackageA[#1]#2{%
   \msitag{<usepackage req="#2" opt="}#1\msitag{"/>}}

\def\out@usepackageB#1{%
   \msitag{^^0a}\msitag{<usepackage req="#1"/>}}

%% \RequirePackage
\def\out@reqpackageA[#1]#2{%
   \msitag{^^0a}\msitag{<usepackage requirepackage="true" req="#2" opt="}#1\msitag{"/>}}

\def\out@reqpackageB#1{%
   \msitag{^^0a}\msitag{<usepackage requirepackage="true"  req="#1"/>}}



%% Frontmatter

\def\title{%
  \msitag{^^0a}%
  \@ifnextchar[{\@titlea}{\@titleb}}

\def\@titlea[#1]#2{%
  \msiopentag{title}{<title>}%
    \msiopentag{short}{<shortTitle>}%
      #1%
    \msiclosetag{short}{</shortTitle>}%
      #2%
  \msiclosetag{title}{</title>}%
}

\def\@titleb#1{%
   \msiopentag{title}{<title>}#1\msiclosetag{title}{</title>}}

\def\author#1{\gdef\@author{#1}\out@author{#1}}

\def\date{\@ifnextchar[{\dateA}{\dateB}}
\def\dateA[#1]#2{%
   \gdef\@date{#2}\out@dateA{#1}{#2}}
\def\dateB#1{\gdef\@date{#1}\out@date{#1}}

\def\thanks#1{\out@thanks{#1}}
\def\and{\out@and}

%\def\frontmatter{\out@frontmatter}
%\def\mainmatter{\out@mainmatter}

\def\out@title#1{%
   {\everypar={}\msitag{^^0a}\msiopentag{title}{<title>}#1\msiclosetag{title}{</title>}}}


\def\out@thanks#1{%
   \msitag{%
      <notewrapper type="footnote" markOrText="undefined">
        <note style="padding: 10px;" type="footnote" hide="false">
          <bodyText>}%
   #1
   \msitag{%
          </bodyText>
        </note>
      </notewrapper>}}

\def\out@and{\msiclosetag{author}{</author>}\msitag{^^0a}\msiopentag{author}{<author>}}%

\def\out@author#1{%
  \begingroup
   \def\\{\msiclosetag{author}{</author>}%
          \msitag{^^0a}%
          \msiopentag{address}{<address>}%
          \msiopentag{bodyText}{<bodyText>}%
          \def\\{\par}%
          \ignorespaces}
          
   \def\and{\msiclosetag{address}{</address>}%
            \msiclosetag{author}{</author>}\msitag{^^0a}\msiopentag{author}{<author>}}%
   \msitag{^^0a}%
   \msiopentag{author}{<author>}#1\msiclosetag{address}{</address>}\msiclosetag{author}{</author>}%
  \endgroup
}

\def\out@date#1{%
   \msihmode\msi@br\msiopentag{date}{<date>}#1\msiclosetag{date}{</date>}}

\def\out@dateA#1#2{%
   \msihmode\msi@br
   \msiopentag{date}{<date>}%
      #2%
      \msiopentag{option}{<option>}%
         #1%
      \msiclosetag{option}{</option>}%
   \msiclosetag{date}{</date>}}

\def\MSIEmptyElement{%
  \@ifnextchar[{\msi@empty@elmA}{\msi@empty@elmB}}

\def\msi@empty@elmA[#1]#2#3{%
     \expandafter\def\csname #2\endcsname{%
        \ifmmode
           %\msitag{\msipassthru{<#1/>}}%
           \expandafter\msitag\expandafter{\csname#2\endcsname}%
        \else
           \if#3T\relax%
             ^^0a%
           \else              
           \fi           
           \msitag{<#1/>}%
        \fi}}

\def\msi@empty@elmB#1{\msi@empty@elmA[#1]{#1}}


\def\MSISimpleEnvironment{%
  \@ifnextchar[{\msi@simple@envA}{\msi@simple@envB}}

\def\msi@simple@envA[#1]#2{%
  \expandafter\def\csname msi@begin@#2\endcsname{\msiopentag{#1}{<#1>}\msihmode\ignorespaces}%
  \expandafter\def\csname msi@end@#2\endcsname{\msiclosetag{#1}{</#1>}}}

\def\msi@simple@envB#1{\msi@simple@envA[#1]{#1}}



%% newtheorem

\def\theoremstyle#1{%
  \def\msi@currentthmstyle{#1}%
  \msitag{^^0a<theoremstyle style="#1"/>}}
\theoremstyle{plain}


% should check for second optional
\def\out@newtheorema#1[#2]#3{%
  \endgroup% started in \newtheorem
  %\msitag{^^0a<newtheorem }%
  %\if@newtheoremstar\msitag{star="1" }\fi%
  %\msitag{name="#1" style = "}\msi@currentthmstyle\msitag{" counter="#2" label="#3"/>}%
  \expandafter\preambleToks\expandafter{\the\preambleToks^^0a\newtheorem{#1}[#2]{#3}}
  \@ifundefined{msi@begin@#1}%
    {\@namedef{msi@begin@#1}{\msi@br\msiopentag{#1}{<#1>}\msihmode}%
     \@namedef{msi@end@#1}{\msi@br\msiclosetag{#1}{</#1>}}}%
    {}}

\def\out@newtheoremb#1#2{%
  \endgroup% started in \newtheorem
  \@ifundefined{msi@begin@#1}%
    {%\msitag{^^0a<newtheorem }%
     %\if@newtheoremstar\msitag{star="1" }\fi% 
     %\msitag{name="#1" style = "}\msi@currentthmstyle\msitag{" label="#2"/>}%
     \expandafter\preambleToks\expandafter{\the\preambleToks^^0a\newtheorem{#1}{#2}}
     \@namedef{msi@begin@#1}{\msi@br\msiopentag{#1}{<#1>}\msihmode}%
     \@namedef{msi@end@#1}{\msi@br\msiclosetag{#1}{</#1>}}}%
    {}}

%% recheck this one
\def\out@newtheoremc#1#2[#3]{%
  \endgroup% started in \newtheorem
  %\msitag{^^0a<newtheorem }
  %\if@newtheoremstar\msitag{star="1" }\fi% 
  %\msitag{name="#1" style = "}\msi@currentthmstyle\msitag{" label="#2"/>}%
  \expandafter\preambleToks\expandafter{\the\preambleToks^^0a\newtheorem{#1}{#2}[#3]}%
  \@ifundefined{msi@begin@#1}%
    {\@namedef{msi@begin@#1}{\msi@br\msiopentag{#1}{<#1>}\msihmode}%
     \@namedef{msi@end@#1}{\msi@br\msiclosetag{#1}{</#1>}}}%
    {}}



% Forms of the \newenvironment command
\def\newenvironment{%
   \@ifnextchar*{\@newenvironment{*}}{\@newenvironment{}}}

 
\def\@newenvironment#1#2{%
   \@ifnextchar[{\@newenv@a{#1}{#2}}{\@newenv@a{#1}{#2}[]}}


\def\@newenv@a#1#2[#3]{%
   \@ifnextchar[{\@newenv@b{#1}{#2}[#3]}{\@newenv@b{#1}{#2}[#3][]}}

\def\@newenv@b#1#2[#3][#4]#5#6{%
    \expandafter\preambleToks\expandafter{\the\preambleToks^^0a\newenvironment#1{#2}}%
    \def\msi@tmp{#3}%
    \ifx\msi@tmp\@empty
    \else
       \expandafter\preambleToks\expandafter{\the\preambleToks[#3]}
    \fi
    \def\msi@tmp{#4}%
    \ifx\msi@tmp\@empty
    \else
       \expandafter\preambleToks\expandafter{\the\preambleToks[#4]}
    \fi
    \expandafter\preambleToks\expandafter{\the\preambleToks{#5}{#6}}
    \@ifundefined{msi@begin@#2}%
      {\@namedef{msi@begin@#2}{\msi@open@element<#2>}%
       \@namedef{msi@end@#2}{\msi@close@element</#2>}}
      {}%
}


%%\def\numberwithin#1#2{\msitag{\numberwithin{#1}{#2}}}
%% %% Newcommand and renewcommand
 
 \def\newcommand{\@ifnextchar*{\@newcommand}{\@newcommand{}}}
 \def\renewcommand{\@ifnextchar*{\@renewcommand}{\@renewcommand{}}}
 \let\providecommand\newcommand
 
 
 %%% #1 = optional star, #2 = command name
 \def\@newcommand#1#2{\@ifnextchar[{\@newcommandA{#1}{#2}}{\@newcommandA{#1}{#2}[]}}
 
 \def\@newcommandA#1#2[#3]{\@ifnextchar[{\@newcommandB{#1}{#2}[#3]}{\@newcommandB{#1}{#2}[#3][]}}

 \def\@newcommandB#1#2[#3][#4]#5{%
   \expandafter\preambleToks\expandafter{\the\preambleToks^^0a\newcommand#1{#2}}%
   \if x#3x
   \else
     \expandafter\preambleToks\expandafter{\the\preambleToks[#3]}%
   \fi
   \if x#4x
   \else
     \expandafter\preambleToks\expandafter{\the\preambleToks[#4]}\fi
   \expandafter\preambleToks\expandafter{\the\preambleToks{#5}}}%

 %%% #1 = optional star, #2 = command name
 \def\@renewcommand#1#2{\@ifnextchar[{\@renewcommandA{#1}{#2}}{\@renewcommandA{#1}{#2}[]}}
 
 \def\@renewcommandA#1#2[#3]{\@ifnextchar[{\@renewcommandB{#1}{#2}[#3]}{\@renewcommandB{#1}{#2}[#3][]}}

 \def\@renewcommandB#1#2[#3][#4]#5{%
   \expandafter\preambleToks\expandafter{\the\preambleToks^^0a\renewcommand#1{#2}}
   \if x#3x
   \else
     \expandafter\preambleToks\expandafter{\the\preambleToks[#3]}%
   \fi
   \if x#4x
   \else
     \expandafter\preambleToks\expandafter{\the\preambleToks[#4]}\fi
   \expandafter\preambleToks\expandafter{\the\preambleToks{#5}}}%
 

%   \MSILocalDef{#2}}



%% Citations and cross refs

\def\attrib@text{
  \catcode`\&=\active
  \catcode`\<=\active
  \catcode`\>=\active
}

\def\cite{%
  \ifvmode\msihmode\fi
  \@ifnextchar[{\@tempswatrue\begingroup\attrib@text\out@citex}{\@tempswafalse\begingroup\attrib@text\out@citey}}


%\def\@citey#1#{%
%  \attrib@text\out@citey
%}


\def\out@citex[#1]#2{%
   \ifmmode
     \msitag{\msipassthru{<citation xmlns="http://www.w3.org/1999/xhtml" citekey="#2"  hasRemark="true">
                            <biblabel class="remark" xmlns="http://www.w3.org/1999/xhtml">#1</biblabel>
                           </citation>}}%
   \else
     \msiopentag{citation}{<citation xmlns="http://www.w3.org/1999/xhtml" 
                              citekey="}#2\msitag{" 
                              hasRemark="true">}%
          \msiopentag{biblabel}{<biblabel class="remark" xmlns="http://www.w3.org/1999/xhtml">}%
              #1%
          \msiclosetag{biblabel}{</biblabel>}%
     \msiclosetag{citation}{</citation>}%
   \fi
  \endgroup}

\def\out@citey#1{%
  \ifmmode
    \msitag{\msipassthru{<citation xmlns="http://www.w3.org/1999/xhtml" citekey="#1" hasRemark="false"></citation>}}%
  \else
    \msiopentag{citation}{<citation xmlns="http://www.w3.org/1999/xhtml" 
                                    citekey="}#1\msitag{" hasRemark="false">}%
    \msiclosetag{citation}{</citation>}%
  \fi
  \endgroup}


%\MSITextElement[a]{ref}
%%<xref key="mykey" reftype="page"/>
%\MSITextElement{pageref}

\def\ref#1{%
  \ifmmode
    \msitag{\msipassthru{<xref key="#1" href="#1" reftype="obj"/>}}%
  \else
    \msihmode\msitag{<xref key="}#1\msitag{" href="}#1\msitag{" reftype="obj"/>}%
  \fi}

\def\pageref#1{%
  \ifmmode
    \msitag{\msipassthru{<xref key="#1" href="#1" reftype="page"/>}}%
  \else
    \msihmode\msitag{<xref key="}#1\msitag{" href="}#1\msitag{" reftype="page"/>}%
  \fi}



\def\msi@pass#1{\def#1{\msitag{#1}}}

\def\@descr{descriptionListItem}
\def\@enumerate{numberedListItem}
\def\@bullet{bulletListItem}
\newif\if@first@item

\def\out@itema[#1]{
   \if@first@item
     \@first@itemfalse
   \else
     \def\enditemtag##1{\msiclosetag{##1}{</##1>}}%
     \expandafter\enditemtag\expandafter{\itemprefix}%
   \fi
   \def\startitemtag##1{\msi@br\msihmode\msiopentag{##1}{<##1>}\msiopentag{bodyText}{<bodyText>}}%
   \expandafter\startitemtag\expandafter{\itemprefix}%
   \ifx\itemprefix\@descr
     \msitag{<descriptionLabel>}#1\msitag{</descriptionLabel>}%
   \else
     \ifx\itemprefix\@enumerate
       \msitag{<numberedLabel>}#1\msitag{</numberedLabel>}%
	 \else
       \ifx\itemprefix\@bullet
         \msitag{<bulletLabel>}#1\msitag{</bulletLabel>}%
	   \else
         \msitag{<explicit-item>}#1\msitag{</explicit-item>}%
	   \fi
	 \fi
   \fi
}

\def\out@itemb{
   \if@first@item
     \@first@itemfalse
   \else
     \def\enditemtag##1{\msiclosetag{##1}{</##1>}}%
     \expandafter\enditemtag\expandafter{\itemprefix}%
   \fi
   \def\startitemtag##1{\msi@br\msihmode\msiopentag{##1}{<##1>}\msiopentag{bodyText}{<bodyText>}}%
   \expandafter\startitemtag\expandafter{\itemprefix}%
}


\def\out@bibitemA[#1]#2{%
   \if@first@item
     \global\@first@itemfalse
   \else
     \msiclosetag{bodyText}{</bodyText>}%
     \msiclosetag{bibitem}{</bibitem>}%
   \fi
   \msi@br\msihmode\msiopentag{bibitem}{}<bibitem hasLabel="true">%
   \msitag{<biblabel class="bibitemlabel">}#1\msitag{</biblabel>}%
   \msitag{<bibkey>}#2\msitag{</bibkey>}%
   \msi@br\msiopentag{bodyText}{<bodyText>}%
   \endgroup
}

\def\out@bibitemB#1{%
   \if@first@item
     \global\@first@itemfalse
   \else
     \msiclosetag{bodyText}{</bodyText>}%
     \msiclosetag{bibitem}{</bibitem>}%
   \fi
   \msi@br\msiopentag{bibitem}{<bibitem hasLabel="false">}%
   \msitag{<bibkey>}#1\msitag{</bibkey>}%
   \msi@br\msiopentag{bodyText}{<bodyText>}%
   \endgroup%
}


\def\@closeitem{\msiclosetag{li}{</li>}}
\def\@openitem{\msiopentag{li}{<li>}}


% A default item name
\def\itemprefix{item}

% #1=xml name, #2=latex env name, #3=xml item name
\def\MSIListEnvironment#1#2#3{%
   \expandafter\def\csname msi@begin@#2\endcsname{%
      \def\itemprefix{#3}%
      \msihmode\msitag{^^0a^^0a}\@first@itemtrue\msiopentag{#1}{<#1>}}

   \expandafter\def\csname msi@end@#2\endcsname{%
      \msi@br\msiclosetag{#1}{</#1>}}
}



\def\msi@begin@thebibliography#1{%
  \everypar={}%
  \msihmode\msitag{^^0a^^0a}%
  \@first@itemtrue
  \msiopentag{bibliography}{<bibliography>}}

\def\msi@end@thebibliography{%
  \msi@br\msiclosetag{bibliography}{</bibliography>}}


\def\msi@thm@like#1{}



%\input msiframe.tex



%% \def\[#1\]{%
%%   \msitag{<mml:math>}
%%   \msiextproc{jbm}{$#1$}%
%%   \msitag{</mml:math>}
%% }
%% 


\newcount\seclevel
\seclevel=0

\newif\if@starred
\def\@gobble#1{}


%% Sectioning

% We always use <sectiontitle> rather than #4
\def\newsection#1#2#3#4{%
  \expandafter\newcount\csname#1number\endcsname
  \csname#1number\endcsname=0
  \expandafter\def\csname the#1\endcsname{#3}%
  \expandafter\def\csname #1\endcsname{%
     \@ifnextchar*{\@starredtrue\csname @#1\endcsname}%
                  {\@starredfalse\csname @#1\endcsname *}%
  }
  \expandafter\def\csname @#1\endcsname*{%
     \@ifnextchar[{\csname @@#1\endcsname}%
                  {\csname @@#1\endcsname[]}%
  }
  \expandafter\def\csname @@#1\endcsname[##1]##2{%
     \seclevel=#2%
     \expandafter\advance\csname #1number\endcsname by 1%
     \let\protect\relax
     \msiclosetag{#1}{</#1>}%
     \msitag{^^0a}%
     \if@starred
       \msiopentag{#1}{<#1 nonum="true">}%
     \else 
       \msiopentag{#1}{<#1>}%
     \fi
     {\everypar={}\msitag{^^0a}%
      \def\short{##1}%
      \msiopentag{sectiontitle}{<sectiontitle>}%
      \ifx\short\@empty
      \else
        \msiopentag{shortTitle}{<shortTitle>}%
          ##1%
        \msiclosetag{shortTitle}{</shortTitle>}%
      \fi
      ##2%
      \msiclosetag{sectiontitle}{</sectiontitle>}%
      \msitag{^^0a}%
     }%
     \par
  }%
}




\newsection{part}{1}{Part \the\sectionnumber}{sectiontitle}%
\newsection{chapter}{2}{Chapter \the\sectionnumber}{sectiontitle}%
\newsection{section}{2}{Section \the\sectionnumber}{sectiontitle}%
\newsection{subsection}{3}{\the\sectionnumber.\the\subsectionnumber}{sectiontitle}%
\newsection{subsubsection}{4}{\thesectionnumber.\thesubsectionnumber.\the\subsubsectionnumber}{sectiontitle}%
\newsection{paragraph}{5}{\thesectionnumber.\thesubsectionnumber.\the\subsubsectionnumber.\the\paragraphnumber}{sectiontitle}%
\newsection{subparagraph}{6}{\thesectionnumber.\thesubsectionnumber.\the\subsubsectionnumber.\the\paragraphnumber}{sectiontitle}%


\def\out@marginparA#1#2{%
  \@start@notewrapper
  \msiopentag{note}{<note type="marginpar" opt="#1">}%
    \msiopentag{bodyText}{<bodyText>}%
      #2%
    \msiclosetag{bodyText}{</bodyText>}%
  \msiclosetag{note}{</note>}%
  \@end@notewrapper
}

\def\out@marginparB#1{%
  \@start@notewrapper
  \msiopentag{note}{<note type="marginpar">}%
    \msiopentag{bodyText}{<bodyText>}%
      #1%
    \msiclosetag{bodyText}{</bodyText>}%
  \msiclosetag{note}{</note>}%
  \@end@notewrapper
}

\def\rule{%
  \@ifnextchar[{\@rulea}{\@ruleb}}%

\def\@rulea[#1]#2#3{%
  \ifmmode
     %\string\rule[#1]\@brace{#2}\@brace{#3}%
    \msitag{\msipassthru{<msirule lift="#1" width="#2" height="#3" %
                     style="vertical-align: #1; width: #2; height: #3; background-color: rgb(0, 0, 0);"/>}}%
  \else
    \msitag{<msirule lift="#1" width="#2" height="#3" %
                     style="vertical-align: #1; width: #2; height: #3; background-color: rgb(0, 0, 0);"/>}%
  \fi
}
\def\@ruleb#1#2{%
  \ifmmode
    \msitag{\msipassthru{<msirule lift="0pt" width="#1" height="#2" %
                     style="vertical-align: 0; width: #1; height: #2; background-color: rgb(0, 0, 0);"/>}}%
  \else
    \msitag{<msirule lift="0pt" width="#1" height="#2" %
                     style="vertical-align: 0; width: #1; height: #2; background-color: rgb(0, 0, 0);"/>}%
  \fi
}

\def\hspace{%
   \@ifnextchar*{\@hspacestar}{\@hspace}}

\def\@hspace#1{%
   \ifmmode
     \string\hspace\@brace{#1}%
   \else
     \msitag{<hspace dim="#1" type="customSpace" atEnd="false"></hspace>}%
   \fi
}

\def\@hspacestar#1#2{%
   \ifmmode
     \string\hspace\@brace{#2}%
   \else
     \msitag{<hspace dim="#2" type="customSpace" atEnd="true"></hspace>}%
   \fi
}

\def\@paragraphs{%
  \let\@par\par
  \def\par{\msiclosetag{bodyText}{</bodyText>}\@par}%
  \everypar{^^0a\msihmode\msiopentag{bodyText}{<bodyText>}}%
}


\everymath{%
   %\catcode`\&=13 %
   %\catcode`\'=\active
   \aftergroup\endmath
}
\def\endmath{}

\everydisplay{%
  %\catcode`\&=13 %
  %\catcode`\'=\active
  \aftergroup\enddisplay
}
\def\enddisplay{}


\def\@make@braces@other{%
  \catcode`\{=12%
  \catcode`\}=12%
  \relax
}
\def\@make@braces@normal{%
  \catcode`\{=1%
  \catcode`\}=2%
  \relax
}

\bgroup
  \catcode`\^^M=13\relax
  \gdef^^M{\char13\char10\relax}%
\egroup


%  \def\@make@linebreak@other{%
%    \catcode`\^^M=13\relax
%  }
%
%  \def\@make@linebreak@normal{%
%    \catcode`\^^M=5\relax
%  }


\def\linebreak{\msitag{<msibr type="lineBreak" invisDisplay="&##x21b5;"><br/></msibr>}}

\def\newline{%
  \@ifnextchar[{\newlinea}{\newlineb}}

\def\newlinea[#1]{\msitag{<msibr type="newLine" invisDisplay="&##x21b5;"><br/></msibr>}}

\def\newlineb{\msitag{<msibr type="newLine" invisDisplay="&##x21b5;"><br/></msibr>}}

\newtoks\temptoksa


\bgroup
  \@make@braces@other
  \catcode`\[=1
  \catcode`\]=2

  \gdef\@scan@align@star#1\end{align*}[%
     \egroup
     \@make@braces@normal
     \msiopentag[mml][<mml:math class="displayedmathml" mode="display" display="block">]%
     \msiextproc[jbm][\begin{align*}#1\end{align*}]%
     \msiclosetag[mml][</mml:math>]
  ]
  \gdef\@scan@align#1\end{align}[%
     \egroup
     \@make@braces@normal
     \msiopentag[mml][<mml:math class="displayedmathml" mode="display" display="block">]%
     \msiextproc[jbm][\begin{align}#1\end{align}]%
     \msiclosetag[mml][</mml:math>]
  ]

  \gdef\@scan@bmatrix#1\end{bmatrix}[%
     \egroup
     \@make@braces@normal
     \msitag[\begin{bmatrix}#1\end{bmatrix}]
     %\temptoksa=[\begin{bmatrix}#1\end{bmatrix}]%
     %\the\temptoksa
  ]
  \gdef\@scan@Bmatrix#1\end{Bmatrix}[%
     \egroup
     \@make@braces@normal
     \msitag[\begin{Bmatrix}#1\end{Bmatrix}]
     %\temptoksa=[\begin{Bmatrix}#1\end{Bmatrix}]%
     %\the\temptoksa
  ]
  \gdef\@scan@smallmatrix#1\end{smallmatrix}[%
     \egroup
     \@make@braces@normal
     \msitag[\begin{smallmatrix}#1\end{smallmatrix}]
  ]

  \gdef\@scan@vmatrix#1\end{vmatrix}[%
     \egroup
     \@make@braces@normal
     \msitag[\begin{vmatrix}#1\end{vmatrix}]
     %\temptoksa=[\begin{vmatrix}#1\end{vmatrix}]%
     %\the\temptoksa
  ]
  \gdef\@scan@Vmatrix#1\end{Vmatrix}[%
     \egroup
     \@make@braces@normal
     \msitag[\begin{Vmatrix}#1\end{Vmatrix}]
     %\temptoksa=[\begin{Vmatrix}#1\end{Vmatrix}]%
     %\the\temptoksa
  ]
  \gdef\@scan@pmatrix#1\end{pmatrix}[%
     \egroup
     \@make@braces@normal
     \msitag[\begin{pmatrix}#1\end{pmatrix}]
     %\temptoksa=[\begin{pmatrix}#1\end{pmatrix}]%
     %\the\temptoksa
  ]
\egroup


%\@namedef{msi@begin@multline*}{\ \@make@braces@other\relax\@scan@multline@star}



\@namedef{msi@begin@flalign}{\bgroup\msi@dollar=0\relax$$\string\begin{flalign}}
\@namedef{msi@end@flalign}{\string\end{flalign}$$\egroup}

\@namedef{msi@begin@flalign*}{\bgroup\msi@dollar=0\relax$$\string\begin{flalign*}}
\@namedef{msi@end@flalign*}{\string\end{flalign*}$$\egroup}

\@namedef{msi@begin@alignat}{\bgroup\msi@dollar=0\relax$$\string\begin{alignat}}
\@namedef{msi@end@alignat}{\string\end{alignat}$$\egroup}

\def\msi@begin@math{\bgroup\msi@dollar=0\relax$\string\begin{math}}
\def\msi@end@math{\string\end{math}$\egroup}

\def\msi@begin@subequations{\msi@insubequations=1}
\def\msi@end@subequations{\msi@insubequations=0}

\def\msi@begin@equation{\bgroup\msi@dollar=0\relax$$\string\begin{equation}}
\def\msi@end@equation{\string\end{equation}$$\egroup}


\@namedef{msi@begin@equation*}{\bgroup\msi@dollar=0\relax$$\string\begin{equation*}}
\@namedef{msi@end@equation*}{\string\end{equation*}$$\egroup}

\@namedef{msi@begin@displaymath}{\bgroup\msi@dollar=0\relax$$\string\begin{equation*}}
\@namedef{msi@end@displaymath}{\string\end{equation*}$$\egroup}


\def\msi@begin@array{\begingroup\def\\{\char`\\\char`\\}\catcode`\&=12\string\begin{array}}
\def\msi@end@array{\string\end{array}\endgroup}


%\def\@label#1{\msikilllasttag{<tr>}{<tr marker="#1" style="">}}%

\def\MSIEqnarrayLikeEnvironment#1{%
  \@namedef{msi@begin@#1}{%
     \begingroup
        \def\\{\char`\\\char`\\}%
%        \let\label\@label
        \catcode`\&=12\msi@dollar=0\relax$$%
        \string\begin{#1}}%
  \@namedef{msi@end@#1}{\string\end{#1}$$\endgroup}}



\def\MSISplitLikeEnvironment#1{%
  \@namedef{msi@begin@#1}{
    \bgroup
      \def\\{\char`\\\char`\\}%
      \ifmmode
      \else
         \msi@dollar=0\relax$$
      \fi
      \string\begin{#1}\catcode`\&=12}%
  \@namedef{msi@end@#1}{%
      \string\end{#1}%
      \ifmmode
      \else
         $$
      \fi
    \egroup}}




\@namedef{msi@begin@align*}{\ \@make@braces@other\relax\@scan@align@star}
\@namedef{msi@begin@align}{\ \@make@braces@other\relax\@scan@align}



\def\msi@begin@bmatrix{\@make@braces@other\relax\@scan@bmatrix} 
\def\msi@begin@Bmatrix{\@make@braces@other\relax\@scan@Bmatrix} 
\def\msi@begin@smallmatrix{\@make@braces@other\relax\@scan@smallmatrix} 
\def\msi@begin@vmatrix{\@make@braces@other\relax\@scan@vmatrix} 
\def\msi@begin@Vmatrix{\@make@braces@other\relax\@scan@Vmatrix} 
\def\msi@begin@pmatrix{\@make@braces@other\relax\@scan@pmatrix}



{\catcode`\^^M=\active % these lines must end with %
  \gdef\obeylines{\catcode`\^^M\active \let^^M\par}%
  \global\let^^M\par} % this is in case ^^M appears in a \write
\def\obeyspaces{\catcode`\ \active}
\def\space{ }
{\obeyspaces\global\let =\space}

 

\def\@cr{\ifmmode
           \def\do@cr{\char`\\}%
         \else
           \def\do@cr{\@ifnextchar[{\@cra}{\@crb}}%
         \fi
         \advance\msirow by 1\relax
         \do@cr}


\def\@cra[#1]{\cr}%
\def\@crb{\cr}%


\def\tabularnewline{\\}


\def\MSITabularStarEnvironment#1{%
  \@namedef{msi@begin@#1}##1{%
      \@ifnextchar[{\msitabularstarA{##1}}%
                   {\msitabularstarB{##1}}}%
  \@namedef{msi@end@#1}{%
        \crcr
        \egroup % the \halign
     \endgroup % to do the closetabular
     \ifmathtabular\char`\}\fi
  \endgroup}
}

\def\msitabularstarA#1[#2]#3{%
   \begingroup
      \ifmmode
        \string\msipassthru\char`\{% 
        \mathtabulartrue
        \msihmode
        \catcode`\&=4%
      \else
        \mathtabularfalse
      \fi
      \msihmode\msiopentag{table}{<table opt="#2" _moz_resizing="true"  cellpadding="4" border="1">}%
      \msi@br\msiopentag{tbody}{<tbody>}%
      \begingroup
        \aftergroup\msi@closetabular
        \let\\=\@cr
        \halign\bgroup&##\cr
}

\def\msitabularstarB#1#2{%
   \begingroup
      \msi@dollar=`\$%
      \ifmmode
        \string\msipassthru\char`\{% 
        \mathtabulartrue
        \msihmode
        \catcode`\&=4%
      \else
        \mathtabularfalse
      \fi
      \msihmode
      \msiopentag{table}{<table cellspacing="2" cellpadding="2" border="1">}%
      \msiopentag{tbody}{<tbody>}%
      \begingroup
        \aftergroup\msi@closetabular
        \let\\=\@cr
        \halign\bgroup&##\cr
}
 
\def\msi@begin@tabular{%
   \@ifnextchar[{\msi@begin@tabularA}{\msi@begin@tabularA[]}}

\expandafter\def\csname msi@begin@tabular*\endcsname{%
   \@ifnextchar[{\msi@begin@tabularA}{\msi@begin@tabularA[]}}


\newif\ifmathtabular
\mathtabulartrue
\newcount\msirow=0

\def\msi@begin@tabularA[#1]#2{%
   \begingroup
      \msi@dollar=`\$%
      \ifmmode
        \string\msipassthru\char`\{% 
        \mathtabulartrue
        \msihmode
        \catcode`\&=4%
      \else
        \mathtabularfalse
        \msihmode
      \fi
      \msiopentag{table}{<table cols="#2" cellpadding="4" border="1" req="tabulary" _moz_resizing="true">}%
      \msiopentag{tbody}{<tbody>}%
      \begingroup
        \msirow=1
        \aftergroup\msi@closetabular
        \let\\=\@cr
        \def\tabularnewline{\\}%
        \halign\bgroup&##\cr}


\def\msi@closetabular{%
  \msiclosetag{tbody}{</tbody>}%
  \msiclosetag{table}{</table>}%
}
%  \msitag{<br type="_moz"/>}
%  \ifmathtabular
%    \msiclosetag{bodyText}{</bodyText>}%
%  \fi}


\def\msi@end@tabular{% 
        \crcr
        \egroup % the \halign
     \endgroup % to do the closetabular
     \ifmathtabular\char`\}\fi
  \endgroup}

\expandafter\def\csname msi@end@tabular*\endcsname{%
        \crcr
        \egroup % the \halign
     \endgroup % to do the closetabular
     \ifmathtabular\char`\}\fi
  \endgroup}


\def\cline#1{\msitag{<cline arg="#1"/>}}
\def\hline{\msitag{<hline row="}\the\msirow\msitag{"/>}}

%\def\hline{}
%\def\cline#1{}


%%\def\multicolumn#1#2#3{%
%%   \count0=#1\relax
%%   \ifnum\count0=1
%%     #3
%%   \else
%%      %\msitag{ Unimplemented multicol }
%%      \msikilllasttag  % the td
%%      \msikilllasttag  % the body text
%%      \msiopentag{td}{<td rowspan="1" colspan="#1">}%
%%        \msiopentag{bodyText}{<bodyText>}%
%%        #3
%%        %\msiclosetag{bodyText}{</bodyText>}%
%%      %\msiclosetag{td}{</td>}%
%%   \fi
%%}

\def\multicolumn#1#2#3#{%
   \count0=#1\relax
   \ifnum\count0=1\relax
   \else
      \msikilllasttag{<td>}{<td rowspan="1" colspan="#1">}%
   \fi}

\def\endbrace{\char`\}}
\chardef\other=12
\chardef\active=13
\bgroup
   \catcode`&=\active
   \gdef&{\ifmmode\char`\&\else\char`\&amp;\fi}%
\egroup
\bgroup
   \catcode`#=\active
   \gdef#{\char`\#}%
\egroup



\def\out@verbatim#1{%
  \msi@br\msihmode
  \msiopentag{verbatim}{<verbatim>}%
  \msiopentag{bodyText}{<bodyText>}%
   #1%
  \msiclosetag{bodyText}{</bodyText>}
  \msiclosetag{verbatim}{</verbatim>}}%

\begingroup
  \catcode`\<=\active
  \gdef<{\ifmmode\char`\<\else\char`\&lt;\fi}
\endgroup

\def\verb{%
   \msihmode
   \bgroup
      \catcode`\\=\other
      \catcode`{=\other
      \catcode`}=\other
      \catcode`&=\active
      \catcode`$=\other
      \catcode`##=\other
      \catcode`\<=\active
      \msiopentag{verb}{<verb>}%
      \@sverb
}

\def\verb@egroup{\msiclosetag{verb}{</verb>}\egroup}
\def\@sverb#1{%
  \catcode`#1\active
  \lccode`\~`#1\relax
  \lowercase{\let~\verb@egroup}%
}

%\def\fbox#1{\ifmmode\temptoksa={#1}\string\fbox\char`\{\the\temptoksa\char`\}\fi}


\def\{{\ifmmode\string\{\else\char123\relax\fi}
\def\}{\ifmmode\string\}\else\char125\relax\fi}

\let\@hbox\hbox
\def\hbox{%
   \def\@after{}
   \ifmmode
      \string\hbox\char`\{\aftergroup\endbrace
   \else
      \def\@after{\@hbox}%
   \fi
   \@after
}


\def\mathbf#1{\string\mathbf\char123#1\char125}
\def\mathfrak#1{\string\mathfrak\char123#1\char125}


\def\TeX{\msihmode\msitag{<texlogo xmlns="http://www.w3.org/1999/xhtml" name="tex">T<sub>E</sub>X</texlogo>}}
\def\LaTeX{\msihmode\msitag{<texlogo xmlns="http://www.w3.org/1999/xhtml" name="latex">L<sup>A</sup>T<sub>E</sub>X</texlogo>}}
\def\BibTeX{\msihmode\msitag{<texlogo xmlns="http://www.w3.org/1999/xhtml" name="bibtex">B<sup>i</sup>bT<sub>E</sub>X</texlogo>}}

% Operators

{\catcode`\<=\active
\gdef\TEXTsymbol#1{%
  \ifmmode
    \temptoksa={\TEXTsymbol{#1}}%
    \the\temptoksa
  \else
    \ifx#1\backslash
      \char`\\%
    \fi
    \ifx#1\vert
      \char`\|%
    \fi
    \ifx#1<
      \char`\&lt;%
    \fi
    \ifx#1>
      \char`\&gt;%
    \fi
  \fi
}}

\msi@pass{\backslash}

\def\textemdash{\@hex{2014}}



{\catcode`\$=11\relax
 \gdef\${\ifmmode\msitag{\$}\else\char`\${}\fi}
 \catcode`\_=11\relax
 \gdef\_{\ifmmode\msitag{\_}\else\char`\_{}\fi}
 \catcode`\%=11\relax
 \gdef\%{\ifmmode\msitag{\%}\else\char`\%{}\fi}
}




\def\@hex#1{\char38\char35x#1;}


\newif\if@inaccent
\@inaccentfalse


\def\@accent#1#2{%
    \if@inaccent
      \@@accent{#1}{#2}%
    \else
      %\@inaccenttrue
      \expandafter\csname \@@accent{#1}{#2}\endcsname
      %\@inaccentfalse
    \fi
}

\def\@@accent#1#2{%
    \ifx#1\i
      dotlessi-#2%
    \else\ifx#1\j
      dotlessj-#2%
    \else 
      #1-#2%
  \fi\fi\fi
}

\def\`#1{\@accent{#1}{grave}}

\def\'#1{\@accent{#1}{acute}}

\def\~#1{\@accent{#1}{tilde}}

\def\^#1{\@accent{#1}{caret}}

\def\=#1{\@accent{#1}{bar}}

\def\"#1{\@accent{#1}{umlat}}

\def\c#1{\@accent{#1}{ced}}

\def\.#1{\@accent{#1}{dot}}

\def\r#1{\@accent{#1}{r}}

\def\H#1{\@accent{#1}{H}}

\def\v#1{\@accent{#1}{v}}

\def\u#1{\@accent{#1}{u}}

\def\k#1{\@accent{#1}{k}}

\def\d#1{\@accent{#1}{d}}

\def\b#1{\@accent{#1}{b}}



\def\t#1{oo}

%% \'{\^{e}}

\def\MSIUnicode#1#2{\@namedef{#1}{\ifmmode\expandafter\string\csname#1 \endcsname\else\@hex{#2}\fi}}


%% Spacing



\def\ {%
  \ifmmode
    \char`\\\char`\ %
  \else
    \msihmode\msitag{<hspace type="requiredSpace">&nbsp;</hspace>}%
  \fi}

\def~{%
  \ifmmode
    \char`\~
  \else
    \msihmode\msitag{<hspace type="nonBreakingSpace">}\@hex{a0}\msitag{</hspace>}%
  \fi}

\def\/{%
  \ifmmode
    \char`\\\char`\/%
  \else
    \msihmode\msitag{<hspace dim="0.083en" type="italicCorrectionSpace">}\@hex{200a}\msitag{</hspace>}%
  \fi}

\def\noindent{\par\msitag{<hspace dim="0.0em" type="noIndent"></hspace>}}

\def\thinspace{%
   \ifmmode
      \string\thinspace
   \else
      \msihmode\msitag{<hspace type="thinSpace"/>}%
   \fi
}



\def\thickspace{%
   \ifmmode
      \string\thickspace
   \else
      \msihmode\msitag{<hspace type="thickSpace"/>}%
   \fi
}

\let\:\medspace


\def\negthinspace{%
   \ifmmode
      \string\negthinspace
   \else
      \msihmode\msitag{<hspace type="negativeThinSpace" />}%
   \fi
}



\def\hrulefill{%
   <hspace fillWith="line" flex="1" class="stretchySpace" type="stretchySpace"></hspace>}



\def\pagebreak{\msitag{<msibr type="pageBreak"><newPageRule/></msibr>}}
\def\newpage{\msitag{<msibr type="newPage"><newPageRule/></msibr>}}

\def\hfil{<hspace type="customSpace" class="stretchySpace" flex="1" atEnd="true"/>}
\def\hfill{<hspace type="customSpace" class="stretchySpace" flex="2" atEnd="true"/>}

%\msi@rfi{\msi@hook@hfil}{\hfil}
%\msi@rfi{\msi@hook@hfill}{\hfill}

%\msi@rfi{\msi@hook@vfil}{\vfil}
%\msi@rfi{\msi@hook@vfill}{\vfill}

\def\msi@hook@hss{\msitag{\hss}}
\def\msi@hook@vss{\msitag{\vss}}


\def\label#1{%
   \ifmmode
      \msitag{\label{#1}}%
   \else
      \msihmode\msitag{<a name="}#1\msitag{" key="}#1\msitag{" id="}#1\msitag{"/>}%
   \fi}

\def\frac#1#2{\string\frac\@brace{#1}\@brace{#2}}
\def\dfrac#1#2{\string\dfrac\@brace{#1}\@brace{#2}}
\def\tfrac#1#2{\string\tfrac\@brace{#1}\@brace{#2}}

\def\unit#1{\string\unit\@brace{#1}}

\def\mathbf#1{\string\mathbf\@brace{#1}}
\def\mathit#1{\string\mathit\@brace{#1}}
\def\mathsf#1{\string\mathsf\@brace{#1}}
\def\mathtt#1{\string\mathtt\@brace{#1}}
\def\mathrm#1{\string\mathrm\@brace{#1}}


\def\Big#1{\ifmmode\string\big#1\fi}
\def\big#1{\ifmmode\string\big#1\fi}
\def\Bigg#1{\ifmmode\string\big#1\fi}
\def\bigg#1{\ifmmode\string\big#1\fi}
\def\Bigl#1{\ifmmode\string\big#1\fi}
\def\bigl#1{\ifmmode\string\big#1\fi}
\def\Bigm#1{\ifmmode\string\big#1\fi}
\def\bigm#1{\ifmmode\string\big#1\fi}
\def\Bigr#1{\ifmmode\string\big#1\fi}
\def\bigr#1{\ifmmode\string\big#1\fi}
\def\Biggl#1{\ifmmode\string\big#1\fi}
\def\biggl#1{\ifmmode\string\big#1\fi}
\def\Biggm#1{\ifmmode\string\big#1\fi}
\def\biggm#1{\ifmmode\string\big#1\fi}
\def\Biggr#1{\ifmmode\string\big#1\fi}
\def\biggr#1{\ifmmode\string\big#1\fi}



\def\func#1{\string\func\@brace{#1}}
\def\limfunc#1{\string\limfunc\@brace{#1}}


\def\sqrt{%
  \@ifnextchar[%
    {\@nthroot}%
    {\@sqrt}}

\def\@sqrt#1{%
   \string\sqrt\@brace{#1}%
}

\def\@nthroot[#1]#2{%
   \string\sqrt[#1]\@brace{#2}%
}


%\msi@mathmlpass{\thinspace}



\def\mathpass@arg#1{%
  \def#1##1{\string#1\@brace{##1}}
}

\def\mathpass@arg@arg#1{%
  \def#1##1##2{\string#1\@brace{##1}\@brace{##2}}
}
\def\mathpass@arg@arg@arg#1{%
  \def#1##1##2##3{\string#1\@brace{##1}\@brace{##2}\@brace{##3}}
}
\def\mathpass@arg@arg@arg@arg#1{%
  \def#1##1##2##3##4{\string#1\@brace{##1}\@brace{##2}\@brace{##3}\@brace{##4}}
}

\def\mathpass@arg@arg@arg@arg@arg#1{%
  \def#1##1##2##3##4##5{\string#1\@brace{##1}\@brace{##2}\@brace{##3}\@brace{##4}\@brace{##5}}
}



\mathpass@arg{\tag}


\mathpass@arg{\mathbb}
\mathpass@arg{\mathcal}


\mathpass@arg@arg{\binom}
\mathpass@arg@arg{\dbinom}
\mathpass@arg@arg{\tbinom}

\mathpass@arg@arg{\QATOP}
\mathpass@arg@arg{\QTATOP}
\mathpass@arg@arg{\QDATOP}

\mathpass@arg@arg@arg{\QABOVE}
\mathpass@arg@arg@arg{\QTABOVE}
\mathpass@arg@arg@arg{\QDABOVE}
\mathpass@arg@arg@arg@arg{\QOVERD}
\mathpass@arg@arg@arg@arg{\QTOVERD}
\mathpass@arg@arg@arg@arg{\QDOVERD}
\mathpass@arg@arg@arg@arg{\QATOPD}
\mathpass@arg@arg@arg{\QTATOPD}
\mathpass@arg@arg@arg{\QDATOPD}

\def\unit#1{\string\unit\@brace{#1}}



\newif\if@mathintext
\@mathintextfalse 

{\catcode`\$=\active
 \gdef${%
   \if@mathintext
      \@mathintextfalse
      \msitag{\msipassthru}%
      \char`\{
      \msiopentag{mtext}{<mtext>}
   \else
      \@mathintexttrue
      \msiclosetag{mtext}{</mtext>}%
      \char`\}%
   \fi}
}


\def\text{\@mathintextfalse\catcode`\$=\active\@text}

\def\@text#1{%
  \begingroup
    \msitag{\msipassthru}%
    \char`\{%
    \msitag{<mrow>}%
    \msitag{<mspace width="thickmathspace"/>}%
    \msiopentag{mtext}{<mtext>}%
    \begingroup \msihmode #1\endgroup%
    \msiclosetag{mtext}{</mtext>}%
    \msitag{<mspace width="thickmathspace"/>}%
    \msitag{</mrow>}%
    \char`\}
  \endgroup
  \catcode`\$=3\relax}


\bgroup
  \catcode`\%=12
  \gdef\strut{\msitag{<vspace type="strut" lineHt="100%"/>}}
\egroup

\bgroup
  \catcode`\%=12
  \gdef\mathstrut{\msitag{<vspace type="mathStrut" lineHt="100%"/>}}
\egroup

\def\vspace{%
  \@ifnextchar*{\@vspacestar}{\@vspace}}

\def\@vspace#1{%
   \ifmmode
     \string\vspace\@brace{#1}%
   \else
     \msitag{<vspace type="customSpace" dim="#1" atEnd="false"></vspace>}%
   \fi
}

\def\@vspacestar#1#2{%
   \ifmmode
     \string\vspace\@brace{#2}%
   \else
     \msitag{<vspace type="customSpace" dim="#2" atEnd="true"></vspace>}%
   \fi
}

\def\MSICounter#1{%
  \expandafter\newcount\csname c@#1\endcsname}

\def\&{\char`\&amp;}
\def\|{\Vert}

\def\textcurrency{\msitag{<textcurrency req="textcomp"/>}}

\def\out@hyperref#1#2#3#4{%
   \msiopentag{a}{<a href="#4" req="hyperref">}#1\msiclosetag{a}{</a>}\ignorespaces}

\newif\if@in@texttag
\@in@texttagfalse

\def\QTP#1{%
  \par
  \msihmode
  \msiopentag{QTP}{<QTP type="#1">}%
  \begingroup
    \def\par{\msiclosetag{QTP}{</QTP>}\endgroup}%
}

\def\@ttclose#1{\msiclosetag{#1}{</#1>}}
\def\@closelasttexttag{\expandafter\@ttclose\expandafter{\@lasttexttag}}

\def\MSIFontSwitch{%
  \@ifnextchar[{\msi@font@switchA}{\msi@font@switchB}}


\def\msi@font@switchA[#1]#2{%
   \expandafter\def\csname#2\endcsname{%
      \ifmmode
        \expandafter\string\csname #2\endcsname\char`\  %
        %%\char`\\\msitag{#2}%
      \else
        \msihmode
        \if@in@texttag
           \expandafter\@ttclose\expandafter{\@lasttexttag}%
        \else
           \@in@texttagtrue
        \fi
        \gdef\@lasttexttag{#1}%
        \msiopentag{#1}{<#1>}%
        \aftergroup\@closelasttexttag
      \fi
   }
}

\def\msi@font@switchB#1{\msi@font@switchA[#1]{#1}}


\def\MSITextElement{%
  \@ifnextchar[{\msi@text@elmA}{\msi@text@elmB}}


\def\msi@text@elmA[#1]#2{%
  \expandafter\def\csname #2\endcsname{%
     \let\@next\relax
     \ifmmode
       \expandafter\string\csname #2\endcsname%
     \else
       \msihmode\msiopentag{#1}{<#1>}%
       \def\endelem{\msiclosetag{#1}{</#1>}}
       \bgroup
          \aftergroup\endelem
          \let\@next\@gobbleit
     \fi
     \@next}}


\def\msi@text@elmB#1{\msi@text@elmA[#1]{#1}}


\def\MSISimpleElement{%
  \@ifnextchar[{\msi@simple@elmA}{\msi@simple@elmB}}

\def\msi@simple@elmA[#1]#2{%
  \expandafter\def\csname #2\endcsname##1{%
     \ifmmode
       \expandafter\string\csname #2\endcsname{##1}%
     \else
       \msitag{^^0a}\msihmode\msiopentag{#1}{<#1>}##1\msiclosetag{#1}{</#1>}%
     \fi}}

\def\msi@simple@elmB#1{\msi@simple@elmA[#1]{#1}}




\def\TeXButton#1#2{%
   \ifmmode
     %\string\TeXButton{#1}{#2}%
     \msitag{\msipassthru{<texb xmlns="http://www.w3.org/1999/xhtml" enc="1" name="#1"><![CDATA[#2]]></texb>}}
   \else
     \msihmode\msitag{<texb xmlns="http://www.w3.org/1999/xhtml" enc="1" name="}\@wrap{#1}\msitag{"><![CDATA[#2]]></texb>}%
   \fi}

%% Create TeX Buttons

\def\MSITeXButton#1{%
    \expandafter\def\csname#1\endcsname
       \else
          \msihmode 
       \fi
       \msitag{<texb xmlns="http://www.w3.org/1999/xhtml" }
         \ifnum\msi@inpreamble=1
           pre="1"
         \fi 
         \msitag{enc="1" name="}#1\msitag{">}%
                <![CDATA[\expandafter\string\csname#1\endcsname]]>%
        \msitag{</texb>}%
       \ifmmode
         \char`\}%}
       \fi
     }}



\def\MSITeXButtonReq#1{%
    \expandafter\def\csname#1\endcsname##1{%
       \@temptokena={\msitag{##1}}%
       \ifmmode
         \string\msipassthru\char`\{%}%
       \else
         \msihmode
       \fi
       \msitag{<texb xmlns="http://www.w3.org/1999/xhtml" enc="1" name="}#1\msitag{">}%
           <![CDATA[\expandafter\string\csname#1\endcsname
           \ifmmode{\else\{\fi\the\@temptokena\ifmmode}\else\}\fi]]>%
       \msitag{</texb>}%
       \ifmmode
         \char`\}%}
       \fi}}

\def\MSITeXButtonReqReq#1{%
    \expandafter\def\csname#1\endcsname##1##2{%
       \@temptokena={\msitag{##1}}%
       \@temptokenb={\msitag{##2}}%
       \msihmode
       \msitag{<texb xmlns="http://www.w3.org/1999/xhtml" enc="1" name="#1">}%
          <![CDATA[\expandafter\string\csname#1\endcsname\{\the\@temptokena\}\{\the\@temptokenb\}]]>%
       \msitag{</texb>}}}


\def\fbox#1{\ifmmode
              \msitag{\msipassthru{<menclose notation='box' type='fbox'>}}%
              \msitag{\msipassthru}{\msihmode #1}%
              \msitag{\msipassthru{</menclose>}}%
            \else
              \msihmode\msiopentag{fbox}{<smallbox>}#1\msiclosetag{fbox}{</smallbox>}%
            \fi}

\def\frame#1{\ifmmode
%               \msitag{\msipassthru{<smallbox>#1</smallbox>}}%
              \msitag{\msipassthru{<menclose notation='box' type='frame'>}}%
              \msitag{\msipassthru}{\msihmode #1}%
              \msitag{\msipassthru{</menclose>}}%
             \else
               \msihmode\msiopentag{frame}{<smallbox>}#1\msiclosetag{frame}{</smallbox>}%
             \fi}



\def\TCItag{\@ifnextchar*{\@TCItagstar}{\@TCItag}}

\def\@TCItag#1{\msitag{\TCItag{#1}}}
\def\@TCItagstar*#1{\string\TCItag\msitag{*{#1}}}

%{\catcode`\'=\active
% \gdef'{^{\prime}}}


%\def\@TCItag#1{%
%    \global\tag@true
%    \global\def\@taggnum{(#1)}%
%    \global\def\@currentlabel{#1}}

%\def\@TCItagstar*#1{%
%    \global\tag@true
%    \global\def\@taggnum{#1}%
%    \global\def\@currentlabel{#1}}




\def\msi@cdr#1#2xxxx{#2}

%\def\MSILocalDef#1{\def#1{\string #1}}



\bgroup
     \@make@braces@other
     \catcode `|=0
     \catcode`\(=1  % use parens for braces
     \catcode`\)=2
     \catcode`\\=12
     |gdef|MSITeXButtonEnvironment#1(%
          |expandafter|gdef|csname msi@begin@#1|endcsname
             (|bgroup
                 |@make@braces@other
                 |catcode`|\=12
                 |catcode`|&=12
                 |aftergroup|@make@braces@normal
                 |csname scan@#1|endcsname)

          |expandafter|gdef|csname scan@#1|endcsname##1\end{#1}%
             (|msitag(<texb xmlns="http://www.w3.org/1999/xhtml" enc="1" name="#1">%
                        <![CDATA[\begin{#1}##1\end{#1}]]>%
                      </texb>)%
              |egroup)
     )%
|egroup



%% Various note type things

\def\@start@notewrapper{%
    \msiopentag{notewrapper}{<notewrapper xmlns="http://www.w3.org/1999/xhtml" type='footnote'>}}

\def\@end@notewrapper{%
    \msiclosetag{notewrapper}{</notewrapper>}}

\def\@end@footnote{\msiclosetag{note}{</note>}\@end@notewrapper}
\def\@innote{\bgroup\aftergroup\@end@footnote}

%%<note type="#2">


\def\footnote{%
    \msi@br\@start@notewrapper
    \msiopentag{note}{<note type="footnote">}%
    \msiopentag{bodyText}{<bodyText>}%
    % Next couple lines read the brace following \footnote and then does \footnoteb
    \afterassignment\@footnoteb
    \let\junk}

\def\@footnoteb{%
    \bgroup
    \def\par{\msiclosetag{bodyText}{</bodyText>}\msiopentag{bodyText}{<bodyText>}}
    \aftergroup\@footnotc
    \let\junk}

\def\@footnoteb{%
    \msi@br
    \msiopentag{notewrapper}{<notewrapper type='footnote'>}
    \msiopentag{note}{<note type="footnote" hide='false'>}%
    \msiopentag{bodyText}{<bodyText>}%
    % Next couple lines read the brace following \footnote and then does \footnoteb
    \afterassignment\@footnotec
    \let\junk}

\def\@footnotec{%
    \bgroup
    \def\par{\msiclosetag{bodyText}{</bodyText>}\msiopentag{bodyText}{<bodyText>}}
    \aftergroup\@footnoted
}

\def\@footnoted{%
    \msiclosetag{bodyText}{</bodyText>}%
    \msiclosetag{note}{</note>}%
    \@end@notewrapper}

\def\out@CustomNoteA[#1]#2#3{%
    \msiopentag{notewrapper}{<notewrapper xmlns="http://www.w3.org/1999/xhtml">}%
    \msiopentag{note}{<note type="#2">}%
    \bgroup
      \@in@texttagfalse% turn this off so we don't close pending tags if we nest a font switch. 
      #3%
    \egroup
    \msiclosetag{note}{</note>}%
    \msiclosetag{notewrapper}{</notewrapper>}%
}

\def\out@CustomNoteB#1#2{%
    \msiopentag{notewrapper}{<notewrapper xmlns="http://www.w3.org/1999/xhtml">}%
    \msiopentag{note}{<note type="#1">}%
    \bgroup
      \@in@texttagfalse% turn this off so we don't close pending tags if we nest a font switch. 
      #2%
    \egroup
    \msiclosetag{note}{</note>}%
    \msiclosetag{notewrapper}{</notewrapper>}%
}


%\MSIMathQuote{\\} %% may need to hard-code this one:

\def\\{\@ifnextchar*{\@linebrk}{\@linebrk+}}


\def\@linebrk#1{%
   \@ifnextchar[{\@linebrkA{#1}}{\@linebrkB{#1}}}

\def\@linebrkA#1[#2]{%
  \ifmmode\string\\\else\msitag{<msibr type="customNewLine" dim="#2" style="vertical-align: -#2" depth="#2"/>}\fi}

\def\@linebrkB#1{\ifmmode\string\\\else\msitag{<msibr type="newLine" invisDisplay="&##x21b5;"/>}\fi}

%% Temporary until prince can read xml

\newtoks\msi@extrapi

\def\msi@opendocument{%
   \msitag{<?xml version="1.0" encoding="UTF-8"?>^^0a}
   \msitag{<?xml-stylesheet href="css/my.css" type="text/css"?>^^0a}
   \msitag{<?sw-tagdefs href="resource://app/res/tagdefs/latexdefs.xml" type="text/xml" ?>^^0a}
   \the\msi@extrapi
   \msitag{<!DOCTYPE html PUBLIC "-//W3C//DTD XHTML 1.1 plus MathML 2.0//EN" "http://www.w3.org/Math/DTD/mathml2/xhtml-math11-f.dtd">}\msi@br
   \msitag{<html xmlns="http://www.w3.org/1999/xhtml" 
             xmlns:mml="http://www.w3.org/1998/Math/MathML"
             xmlns:sw="http://www.sciword.com/namespaces/sciword">}
   \msi@br\msitag{<head>}
   \msi@br\msiopentag{preamble}{<preamble>}
   %\expandafter\out@preamble@tex\expandafter{\the\preambleToks}
   \def\makeatother{%
       \msihmode
       \ifnum\msi@inpreamble=1
         \ifnum\msi@output=0
           \expandafter\preambleToks\expandafter{\the\preambleToks^^0a\makeatother}
         \fi
       \else
           \msitag{<texb xmlns="http://www.w3.org/1999/xhtml" enc="1" name="makeatother">}%
              <![CDATA[\string\makeatother]]>%
           \msitag{</texb>}%
       \fi
       \catcode`\@=12}

   \def\makeatletter{%
       \msihmode
       \ifnum\msi@inpreamble=1
          \ifnum\msi@output=0
            \expandafter\preambleToks\expandafter{\the\preambleToks^^0a\makeatletter}
          \fi
       \else
         \msitag{<texb xmlns="http://www.w3.org/1999/xhtml" enc="1" name="makeatletter">}%
           <![CDATA[\string\makeatletter]]>%
         \msitag{</texb>}%
       \fi
       \catcode`\@=11}

   \msi@inpreamble=1
}


\def\@gobbleit{\let\@temptok}

\futurelet\@bracetok\@gobbleit{
\futurelet\@brackettok\@gobbleit[
\futurelet\@startok\@gobbleit*

\begingroup
  \@make@braces@other
  \catcode`\[=1
  \catcode`\]=2
  \gdef\msi@lbrace[{]
  \gdef\msi@rbrace[}]
\endgroup

\def\@inbrace#1{%
   \ifnum\msi@inpreamble=1
     \expandafter\msitopreambleextra\expandafter{\msi@lbrace}%
     \msitopreambleextra{#1}%
     \expandafter\msitopreambleextra\expandafter{\msi@rbrace}%
   \else
     \@temptokena={\msitag{#1}}%
     \msi@lbrace\the\@temptokena\msi@rbrace
   \fi
   \@gobbleargs}

\def\@inbracket[#1]{%
   \ifnum\msi@inpreamble=1
     \msitopreambleextra{[#1]}%
   \else
     \@temptokena={\msitag{[#1]}}%
     \the\@temptokena
   \fi
   \@gobbleargs}

\def\@astar*{*\@gobbleargs}

\def\@gobbleargs{\futurelet\atok\@gobblemore}

\def\@endgobble{\@endtexb\relax}

\def\@gobblemore{%
  \begingroup
     \msi@inpreamble=0
     \ifx\atok\@bracetok
       \global\let\@donext\@inbrace
     \else
       \ifx\atok\@brackettok
         \global\let\@donext\@inbracket
       \else
         \ifx\atok\@startok
           \global\let\@donext\@astar
         \else
           \global\let\@donext\@endgobble
         \fi
       \fi
     \fi
   \endgroup
   \@donext}



% Here #1 = some unknown control sequence 

\def\msidefault#1{%
  \ifmmode
    \msitag{\msipassthru}%
    \char`\{
    \msiopentag{texb}{<texb xmlns="http://www.w3.org/1999/xhtml" enc="1" name="#1">}%
       \msitag{<![CDATA[#1}
    \let\@next\@gobbleargs
  \else
    \def\@next{\expandafter\msidefaulta\expandafter{\string #1}}%
  \fi
  \@next}

\def\msidefaulta#1{%
  \expandafter\msidefaultb\expandafter{\msi@cdr#1xxxx}}


\def\msidefaultb#1{%
    \msihmode
    \ifnum\msi@inpreamble=1
      %%% \temptoksa={<texb xmlns="http://www.w3.org/1999/xhtml" enc="1" name="#1">}%
      %%% \expandafter\msitopreambleextra\expandafter{\the\temptoksa}
      %%% \msiopentag{texb}{}%
      %%% \msitopreambleextra{<![CDATA[}
      \expandafter\temptoksa\expandafter{\csname#1\endcsname}
      \msitopreambleextra{^^0a}%
      \expandafter\msitopreambleextra\expandafter{\the\temptoksa}
    \else
      \msiopentag{texb}{<texb xmlns="http://www.w3.org/1999/xhtml" enc="1" name="#1">}%
         \msitag{<![CDATA[}\expandafter\string\csname#1\endcsname
    \fi
    \@gobbleargs}

\def\@endtexb{% 
    \ifnum\msi@inpreamble=1
       %%% \msitopreambleextra{]]></texb>}
       %%% \msiclosetag{texb}{}
    \else
         \msitag{]]>}%
         \msiclosetag{texb}{</texb>}%
         \ifmmode
           \char`\} % to close off msipassthru
         \fi
    \fi}


\def\msi@begin@figure{%
  \@ifnextchar[{\msi@begin@figureA}{\msi@begin@figureB}}


\def\msi@begin@figureA[#1]{%
   \msiopentag{msiframe}{<msiframe frametype="figure" pos="float" placement="f" ltxfloat="#1" >}%
}

\def\msi@begin@figureB{%
   \msiopentag{msiframe}{<msiframe frametype="figure" pos="float" placement="f">}%
}

\def\msi@end@figure{%
   \msiclosetag{msiframe}{</msiframe>}%
}



\def\msi@begin@table{%
  \@ifnextchar[{\msi@begin@tableA}{msi@begin@tableB}}


\def\msi@begin@tableA[#1]{%
   \msiopentag{msiframe}{<msiframe frametype="table" pos="float" ltxfloat="#1">}%
}

\def\msi@begin@tableB{%
   \msiopentag{msiframe}{<msiframe frametype="table" pos="float">}%
}

\def\msi@end@table{%
   \msiclosetag{msiframe}{</msiframe>}%
}


\def\caption{\msi@caption}

\def\msi@caption#1{%
  \msiopentag{caption}{<caption align="bottom">}%
  #1%
  \msiclosetag{caption}{</caption>}%
}

\bgroup
  \@make@braces@other
  \catcode `|=0
  \catcode`\[=1
  \catcode`\]=2
  \catcode`\\=12
  |gdef|msi@scan@filecontents#1\end{filecontents}[%
     |out@filecontents[#1]%
     |end[filecontents]%
     |@make@braces@normal%
  ] 
|egroup

\def\msi@begin@filecontents{%
  \@make@braces@other
  \catcode `|=0
  \catcode`\[=1
  \catcode`\]=2
  \catcode`\\=12
  \msi@scan@filecontents}

\def\out@filecontents#1{%
   \expandafter\global\expandafter\preambleToks\expandafter{\the\preambleToks^^0a\begin{filecontents}#1\end{filecontents}}}


% A hack to fool the jbm filter into handling \dots gracefully
%\def\dots{%
%  \ifmmode
%    \msitag{\U{2026}}%
%  \else
%    }



\input latex-default.tex
\input tcilatex.tex
\catcode`\<=\active
%\catcode`\&=12


\input latex-default.tex
\input tcilatex.tex
\catcode`\<=\active
