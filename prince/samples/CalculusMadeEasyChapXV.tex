%2multibyte Version: 5.50.0.2960 CodePage: 1252
\documentclass[12pt,openany]{book}%
\usepackage{amsmath}
\usepackage{amssymb}
\usepackage{hyperref}
\usepackage{amsfonts}
\usepackage{graphicx}
\usepackage{geometry}%
\setcounter{MaxMatrixCols}{30}
%TCIDATA{OutputFilter=latex2.dll}
%TCIDATA{Version=5.50.0.2960}
%TCIDATA{Codepage=1252}
%TCIDATA{LastRevised=Wednesday, August 10, 2016 12:02:22}
%TCIDATA{<META NAME="GraphicsSave" CONTENT="32">}
%TCIDATA{<META NAME="SaveForMode" CONTENT="1">}
%TCIDATA{BibliographyScheme=Manual}
%BeginMSIPreambleData
\providecommand{\U}[1]{\protect\rule{.1in}{.1in}}
%EndMSIPreambleData
\def\tb{\begin{center}\rule{1.5in}{0.5pt}\end{center}}
\renewcommand{\thechapter}{\Roman{chapter}}
\setcounter{chapter}{14}
\geometry{margin=1in}
\begin{document}
%
%TCIMACRO{\TeXButton{B titlepage}{\begin{titlepage}}}%
%BeginExpansion
\begin{titlepage}%
%EndExpansion


\begin{center}
(excerpted from)

\bigskip

\textbf{{}{\Huge CALCULUS MADE EASY:}{}}\vspace*{0.67in}

{\large BEING A VERY-SIMPLEST INTRODUCTION TO}\medskip

{\large THOSE BEAUTIFUL METHODS OF RECKONING}\medskip

{\large WHICH ARE GENERALLY CALLED BY THE}\medskip

{\large TERRIFYING NAMES OF THE}\vspace*{0.5in}

{\Large DIFFERENTIAL CALCULUS}\bigskip

{\large AND THE}\bigskip

{\Large INTEGRAL CALCULUS.}\vspace*{0.33in}

{\large BY}\bigskip

{\LARGE F. R. S.}\vspace*{0.5in}

{\Large SECOND EDITION, ENLARGED}%

%TCIMACRO{\FRAME{itbpF}{5.0548in}{1.8334in}{0in}{}{}{mcmillianlogo.png}%
%{\special{ language "Scientific Word";  type "GRAPHIC";
%maintain-aspect-ratio TRUE;  display "USEDEF";  valid_file "F";
%width 5.0548in;  height 1.8334in;  depth 0in;  original-width 5.028in;
%original-height 1.8057in;  cropleft "0";  croptop "1";  cropright "1";
%cropbottom "0";  filename 'McMillianLogo.png';file-properties "XNPEU";}} }%
%BeginExpansion
{\includegraphics[
natheight=1.805700in,
natwidth=5.028000in,
height=1.8334in,
width=5.0548in
]%
{C:/George/Prince2/mozilla/prince/samples/CalculusMadeEasyChapXV__1.jpg}%
}
%EndExpansion


{\large MACMILLAN AND CO., LIMITED}\medskip

{\large ST. MARTIN'S STREET, LONDON}\medskip

{\large 1914}
\end{center}%

%TCIMACRO{\TeXButton{E titlepage}{\end{titlepage}}}%
%BeginExpansion
\end{titlepage}%
%EndExpansion


\chapter[SINES AND COSINES]{How to deal with Sines and Cosines}

\setcounter{page}{162}

GREEK letters being usual to denote angles, we will take as the usual letter
for any variable angle the letter~$\theta$ (``theta'').

Let us consider the function
\[
y= \sin\theta.
\]
%

%TCIMACRO{\FRAME{dtbpF}{2.7233in}{2.5278in}{0pt}{}{}{calculusmadeeasyfig43.jpg}%
%{\special{ language "Scientific Word";  type "GRAPHIC";
%maintain-aspect-ratio TRUE;  display "USEDEF";  valid_file "F";
%width 2.7233in;  height 2.5278in;  depth 0pt;  original-width 2.6809in;
%original-height 2.4863in;  cropleft "0";  croptop "1";  cropright "1";
%cropbottom "0";
%filename 'CalculusMadeEasyFig43.jpg';file-properties "XNPEU";}}}%
%BeginExpansion
\begin{center}
\includegraphics[
natheight=2.486300in,
natwidth=2.680900in,
height=2.5278in,
width=2.7233in
]%
{CalculusMadeEasyFig43.jpg}%
\end{center}
%EndExpansion


What we have to investigate is the value of $\dfrac{d(\sin\theta)}{d \theta}$;
or, in other words, if the angle~$\theta$ varies, we have to find the relation
between the increment of the sine and the increment of the angle, both
increments being indefinitely small in themselves. Examine Fig 43, wherein, if
the radius of the circle is unity, the height of~$y$ is the sine, and $\theta$
is the angle. Now, if $\theta$ is supposed to increase by the addition to it
of the small angle $d \theta$---an element of angle---the height of~$y$, the
sine, will be increased by a small element~$dy$. The new height~$y + dy$ will
be the sine of the new angle $\theta+ d \theta$, or, stating it as an
equation,
\[
y+dy = \sin(\theta+ d \theta);
\]
and subtracting from this the first equation gives
\[
dy = \sin(\theta+ d \theta)- \sin\theta.
\]


The quantity on the right-hand side is the difference between two sines, and
books on trigonometry tell us how to work this out. For they tell us that if
$M$ and~$N$ are two different angles,
\[
\sin M - \sin N = 2 \cos\frac{M+N}{2}\cdot\sin\frac{M-N}{2}.
\]


If, then, we put $M= \theta+ d \theta$ for one angle, and $N= \theta$ for the
other, we may write
\begin{align*}
dy  &  =2\cos\frac{\theta+d\theta+\theta}{2}\textperiodcentered\sin
\frac{\theta+d\theta-\theta}{2},\\
\text{or,\qquad\qquad}dy  &  =2\cos(\theta+\tfrac{1}{2}d\theta
)\textperiodcentered\sin\tfrac{1}{2}d\theta.
\end{align*}


But if we regard $d \theta$ as indefinitely small, then in the limit we may
neglect~$\frac{1}{2} d \theta$ by comparison with~$\theta$, and may also take
$\sin\frac{1}{2} d \theta$ as being the same as~$\frac{1}{2} d \theta$. The
equation then becomes:
\begin{align*}
dy  &  = 2 \cos\theta\times\tfrac{1}{2} d \theta;\\
dy  &  = \cos\theta\cdot d \theta,\\
\text{and, finally,\qquad\qquad}\dfrac{dy}{d \theta}  &  = \cos\theta.
\end{align*}


The accompanying curves, Figs 44 and 45, show, plotted to scale, the values of
$y=\sin\theta$, and $\dfrac{dy}{d\theta}=\cos\theta$, for the corresponding
values of~$\theta$.%

%TCIMACRO{\FRAME{dtbpF}{4.6207in}{2.5261in}{0pt}{}{}{calculusmadeeasyfig44.wmf}%
%{\special{ language "Scientific Word";  type "GRAPHIC";
%maintain-aspect-ratio TRUE;  display "USEDEF";  valid_file "F";
%width 4.6207in;  height 2.5261in;  depth 0pt;  original-width 4.5688in;
%original-height 2.4855in;  cropleft "0";  croptop "1";  cropright "1";
%cropbottom "0";
%filename 'CalculusMadeEasyFig44.wmf';file-properties "XNPEU";}}}%
%BeginExpansion
\begin{center}
\includegraphics[
natheight=2.485500in,
natwidth=4.568800in,
height=2.5261in,
width=4.6207in
]%
{C:/George/Prince2/mozilla/prince/samples/CalculusMadeEasyChapXV__2.pdf}%
\end{center}
%EndExpansion
%

%TCIMACRO{\FRAME{dtbpF}{4.619in}{2.5175in}{0pt}{}{}{calculusmadeeasyfig45.eps}%
%{\special{ language "Scientific Word";  type "GRAPHIC";
%maintain-aspect-ratio TRUE;  display "USEDEF";  valid_file "F";
%width 4.619in;  height 2.5175in;  depth 0pt;  original-width 4.5671in;
%original-height 2.4768in;  cropleft "0";  croptop "1";  cropright "1";
%cropbottom "0";
%filename 'CalculusMadeEasyFig45.eps';file-properties "XNPEU";}}}%
%BeginExpansion
\begin{center}
\includegraphics[
height=2.5175in,
width=4.619in
]%
{CalculusMadeEasyFig45.eps}%
\end{center}
%EndExpansion


\begin{center}
\rule{1.5in}{0.5pt}
\end{center}

Take next the cosine.

Let $y=\cos\theta$.

Now $\cos\theta=\sin\left(  \dfrac{\pi}{2}-\theta\right)  $.

Therefore%
\begin{align*}
&
\begin{array}
[c]{r}%
dy=d\left(  \sin\left(  \frac{\pi}{2}-\theta\right)  \right)  =\cos\left(
\frac{\pi}{2}-\theta\right)  \times d(-\theta),\\
=\cos\left(  \frac{\pi}{2}-\theta\right)  \times(-d\theta),
\end{array}
\\
&  \frac{dy}{d\theta}=-\cos\left(  \frac{\pi}{2}-\theta\right)  .
\end{align*}


And it follows that%
\[
\frac{dy}{d\theta}=-\sin\theta.
\]


\begin{center}
\rule{1.5in}{0.5pt}
\end{center}

Lastly, take the tangent.
\begin{align*}
\text{Let\qquad\qquad\qquad\qquad\qquad}y  &  =\tan\theta,\\
dy  &  =\tan(\theta+d\theta)-\tan\theta.
\end{align*}


Expanding, as shown in books on trigonometry%

\begin{align*}
\tan(\theta+d\theta)  &  =\frac{\tan\theta+\tan d\theta}{1-\tan\theta\cdot\tan
d\theta};\\
\text{whence\qquad\qquad\qquad}dy  &  =\frac{\tan\theta+\tan d\theta}%
{1-\tan\theta\cdot\tan d\theta}-\tan\theta\\
&  =\frac{(1+\tan^{2}\theta)\tan d\theta}{1-\tan\theta\cdot\tan d\theta}.
\end{align*}


Now remember that if $d\theta$ is indefinitely diminished, the value of~$\tan
d\theta$ becomes identical with~$d\theta$, and $\tan\theta\cdot d\theta$ is
negligibly small compared with~$1$, so that the expression reduces to
\begin{align*}
dy  &  = \frac{(1+\tan^{2} \theta)\, d\theta}{1},\\
\text{so that\qquad\qquad} \frac{dy}{d\theta}  &  = 1 + \tan^{2}\theta,\\
\text{or\qquad\qquad} \frac{dy}{d\theta}  &  = \sec^{2} \theta.
\end{align*}


Collecting these results, we have:

\begin{center}%
\begin{tabular}
[c]{|c|c|}\hline
$\rule[-12pt]{0pt}{32pt}y$ & $\dfrac{dy}{d\theta}$\\\hline
$\sin\theta$ & $\cos\theta$\\
$\cos\theta$ & $-\sin\theta$\\
$\tan\theta$ & $\sec^{2}\theta$\\\hline
\end{tabular}



\end{center}

Sometimes, in mechanical and physical questions, as, for example, in simple
harmonic motion and in wave-motions, we have to deal with angles that increase
in proportion to the time. Thus, if $T$ be the time of one complete
\emph{period}, or movement round the circle, then, since the angle all round
the circle is $2\pi$~radians, or~$360%
%TCIMACRO{\U{b0}}%
%BeginExpansion
{{}^\circ}%
%EndExpansion
$, the amount of angle moved through in time~$t$, will be
\begin{align*}
\theta &  = 2\pi\frac{t}{T},\quad\text{in radians,}\\
\text{or\qquad\qquad} \theta &  = 360\frac{t}{T},\quad\text{in degrees.}%
\end{align*}


If the \emph{frequency}, or number of periods per second, be denoted by~$n$,
then $n = \dfrac{1}{T}$, and we may then write:
\[
\theta=2\pi nt.
\]
Then we shall have
\[
y = \sin2\pi nt.
\]


If, now, we wish to know how the sine varies with respect to time, we must
differentiate with respect, not to~$\theta$, but to~$t$. For this we must
resort to the artifice explained in Chapter~IX., and put
\[
\frac{dy}{dt} = \frac{dy}{d\theta} \cdot\frac{d\theta}{dt}.
\]


Now $\dfrac{d\theta}{dt}$ will obviously be~$2\pi n$; so that
\begin{align*}
\frac{dy}{dt}  &  =\cos\theta\times2\pi n\\
&  =2\pi n\cdot\cos2\pi nt.
\end{align*}
Similarly, it follows that%
\[
\frac{d(\cos2\pi nt)}{dt}=-2\pi n\cdot\sin2\pi nt.
\]


\section*{Second Differential Coefficient of Sine or Cosine.}

We have seen that when $\sin\theta$ is differentiated with respect to~$\theta$
it becomes $\cos\theta$; and that when $\cos\theta$ is differentiated with
respect to~$\theta$ it becomes $-\sin\theta$; or, in symbols,
\[
\frac{d^{2}(\sin\theta)}{d\theta^{2}} = -\sin\theta.
\]


So we have this curious result that we have found a function such that if we
differentiate it twice over, we get the same thing from which we started, but
with the sign changed from $+$~to~$-$.

The same thing is true for the cosine; for differentiating $\cos\theta$ gives
us $-\sin\theta$, and differentiating $-\sin\theta$ gives us $-\cos\theta$; or
thus:
\[
\frac{d^{2}(\cos\theta)}{d\theta^{2}} = -\cos\theta.
\]


\emph{Sines and cosines are the only functions of which the second
differential coefficient is equal \emph{(and of opposite sign to)} the
original function.}

\begin{center}
\rule{1.5in}{0.5pt}
\end{center}

\subsection*{Examples.}

With what we have so far learned we can now differentiate expressions of a
more complex nature.

(1) $y=\arcsin x$.

If $y$ is the arc whose sine is~$x$, then $x = \sin y$.
\[
\frac{dx}{dy}=\cos y.
\]


Passing now from the inverse function to the original one, we get
\begin{align*}
\frac{dy}{dx}  &  = \frac{1}{\;\dfrac{dx}{dy}\;} = \frac{1}{\cos y}.\\
\text{Now\qquad\qquad} \cos y  &  = \sqrt{1-\sin^{2} y}=\sqrt{1-x^{2}};\\
\text{hence\qquad\qquad\qquad\qquad} \frac{dy}{dx}  &  = \frac{1}%
{\sqrt{1-x^{2}}},
\end{align*}
a rather unexpected result.

(2) $y=\cos^{3} \theta$.

This is the same thing as $y=(\cos\theta)^{3}$.

Let $\cos\theta=v$;\quad then $y=v^{3}$;\quad$\dfrac{dy}{dv}=3v^{2}$.
\begin{align*}
\frac{dv}{d\theta}  &  = -\sin\theta.\\
\frac{dy}{d\theta}  &  = \frac{dy}{dv} \times\frac{dv}{d\theta} = -3 \cos^{2}
\theta\sin\theta.
\end{align*}


(3) $y=\sin(x+a)$.

Let $x+a=v$;\quad then $y=\sin v$.
\[
\frac{dy}{dv}=\cos v;\qquad\frac{dv}{dx}=1 \quad\text{and}\quad\frac{dy}%
{dx}=\cos(x+a).
\]


(4) $y=\log_{\epsilon}\sin\theta$.

Let $\sin\theta=v$;\quad$y=\log_{\epsilon}v$.
\begin{align*}
\frac{dy}{dv}  &  = \frac{1}{v};\quad\frac{dv}{d\theta}=\cos\theta;\\
\frac{dy}{d\theta}  &  = \frac{1}{\sin\theta} \times\cos\theta= \cot\theta.
\end{align*}


(5) $y=\cot\theta=\dfrac{\cos\theta}{\sin\theta}$.
\begin{align*}
\frac{dy}{d\theta}  &  = \frac{-\sin^{2}\theta- \cos^{2} \theta}{\sin^{2}
\theta}\\
&  = -(1+\cot^{2} \theta) = -\csc^{2} \theta.
\end{align*}


(6) $y=\tan3\theta$.

Let $3\theta=v$;\quad$y=\tan v$;\quad$\dfrac{dy}{dv}=\sec^{2} v$.
\[
\frac{dv}{d\theta}=3;\quad\frac{dy}{d\theta}=3 \sec^{2} 3\theta.
\]


(7) $y = \sqrt{1+3\tan^{2}\theta}$;\quad$y=(1+3 \tan^{2} \theta)^{\frac{1}{2}%
}$.

Let $3\tan^{2}\theta=v$.
\begin{align*}
y  &  = (1+v)^{\frac{1}{2}};\quad\frac{dy}{dv} = \frac{1}{2\sqrt{1+v}}\\
\frac{dv}{d\theta}  &  = 6\tan\theta\sec^{2} \theta\\
\text{(for, if $\tan\theta= u$,\qquad\qquad} v  &  = 3u^{2};\quad\frac{dv}{du}
= 6u;\quad\frac{du}{d\theta} = \sec^{2} \theta;\\
\text{hence\qquad\qquad} \frac{dv}{d\theta}  &  = 6 (\tan\theta\sec^{2}
\theta)\\
\text{hence\qquad\qquad} \frac{dy}{d\theta}  &  = \frac{6\tan\theta\sec
^{2}\theta}{2\sqrt{1 + 3\tan^{2}\theta}}.
\end{align*}


(8) $y=\sin x\cos x$.
\begin{align*}
\frac{dy}{dx} &  =\sin x(-\sin x)+\cos x\times\cos x\\
&  =\cos^{2}x-\sin^{2}x.
\end{align*}


\newpage%

%TCIMACRO{\FRAME{dtbpF}{2.8202in}{2.2623in}{0pt}{}{}{project_gutenberg-1.bmp}%
%{\special{ language "Scientific Word";  type "GRAPHIC";
%maintain-aspect-ratio TRUE;  display "USEDEF";  valid_file "F";
%width 2.8202in;  height 2.2623in;  depth 0pt;  original-width 2.7778in;
%original-height 2.2226in;  cropleft "0";  croptop "1";  cropright "1";
%cropbottom "0";  filename 'project_gutenberg-1.bmp';file-properties "XNPEU";}%
%}}%
%BeginExpansion
\begin{center}
\includegraphics[
natheight=2.222600in,
natwidth=2.777800in,
height=2.2623in,
width=2.8202in
]%
{C:/George/Prince2/mozilla/prince/samples/CalculusMadeEasyChapXV__3.jpg}%
\end{center}
%EndExpansion
This sample document was derived from
\textit{\href{http://www.gutenberg.org/3/3/2/8/33283/}{\textit{Calculus Made
Easy, 2nd Edition} by Silvanus Thompson}}, from
\href{http://www.gutenberg.org/}{Project Gutenberg}. See the Project Gutenberg
licnesing information
\href{http://www.gutenberg.org/wiki/Gutenberg:The_Project_Gutenberg_License}{*here*}%
.

This document demonstrates typical mathematical text, plus shows graphics
using five different formats:

\begin{enumerate}
\item .png, the logo on the first page

\item .jpg, Fig. 43

\item .wmf, Fig. 44

\item .eps, Fig 45

\item .bmp, the logo on the last page
\end{enumerate}


\end{document}