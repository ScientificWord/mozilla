%%%%%%%%%%%%%%%%%%%%%%%%%%%%%%%%%%%%%%%%%%%%%%%%%%%%%%%
%                   File: OEtemp2.tex                 %
%                   VERSION: 1.08                     %
%                   Date: April 3, 1998               %
% LaTeX template file for use with OSA OPTICS EXPRESS %
%                                                     %
% This file requires a substyle file under the LaTeX  %
% Article style, opex.sty .                           %
%               FOR LATEX 2E USE                      %
%         \documentclass[10pt]{article}               %
%         \usepackage{opex}                           %
%                                                     %
%              FOR LATEX 2.09 USE                     %
%         \documentstyle[opex]{article}               %
%                                                     %
%              FOR REVTeX 3.0 or 3.1 USE              %
%         \documentstyle[osa,opex]{revtex}            %
%                                                     %
% Copyright 1997, The Optical Society of America      %
%%%%%%%%%%%%%%%%%%%%%%%%%%%%%%%%%%%%%%%%%%%%%%%%%%%%%%%
%           FOR COLORED WWW LINKS USE                 %
%     \documentclass[10pt]{article}                   %
%     \usepackage{opex}                               %
%     \usepackage{color}%             (LATEX 2E)      %
%                                                     %
%           FOR COLOR WITH LATEX 2.09 BE              %
%           CERTAIN YOU HAVE color.sty AND USE        %
%     \documentstyle[opex,color]{article}             %
%                                                     %
%           FOR COLOR WITH REVTeX 3.0 or 3.1 BE       %
%           CERTAIN YOU HAVE color.sty AND USE        %
%     \documentstyle[osa,opex,color]{revtex}          %
%                                                     %
% Copyright 1997, The Optical Society of America      %
%%%%%%%%%%%%%%%%%%%%%%%%%%%%%%%%%%%%%%%%%%%%%%%%%%%%%%%
\documentstyle[opex]{article}
%\documentstyle[osa,opex]{revtex}
%\documentclass[10pt]{article}
%\usepackage{opex}
\begin{document}

\title{Short template for preparation of \\
Optics Express manuscripts}

\author{Frank E. Harris}
\address{Optical Society of America, Washington, DC 20036-1023 USA}
\email{fharri@osa.org}

\begin{abstract}%
This short template outlines the basic LaTeX macros 
needed for preparing a compuscript 
(i.e., manuscript in electronic form) for the journal 
Optics Express, using the opex.sty style file.  
Specific guidelines are given for essential 
compuscript elements (such as title, author, address, 
abstract, OCIS codes, references, equations, 
figures, and  tables) to achieve 
optimal typographic quality with LaTeX.
\end{abstract}
\ocis{(020.0020) Atomic and molecular physics; (140.3320) Laser cooling}

% The commands, \begin{OEReferences} and \end{OEReferences}
% format the References section according to OpEx standard
% style, showing the title "References".
%
% The commands, \begin{OERefLinks} and \end{OERefLinks}
% format the References section according to OpEx standard
% style, if the references also include URLs or other
% unreviewed links.  In this case the title of the section
% is "References and unreviewed links".
%
\begin{OEReferences}
\item  Frank E. Harris, ``Instructions 
for the preparation of a manuscript for Optics Express''
{\it in press} (1998)
\item  P. J. Harshman,  T. K. Gustafson, P. Kelley,
 ``Strong field theory of low loss optical switching and
  wave mixing in a semiconductor quantum well,''  \opex 
  {\bf 3}, 166-168  (1999).
\item  C. van Trigt,  \josaa
 {\bf 14}, 741-755 (1997).
\end{OEReferences}
%\end{OERefLinks}

\noindent
As describe in a previous article,$^1$
{\it Optics Express} takes advantage of 
some of the unique opportunities the electronic 
medium offers to allow authors to include 
3-D scenes and QuickTime movies 
as illustrations within their papers.  
However, this template will address
only the essential commands needed to produce a 
short letter to  {\it Optics Express}.
This template is intended to provide the macros 
unique to opex.sty, so that a person familiar
with LaTeX can copy and paste text, equations, 
and figure file names into this template to
produce an acceptable manuscript. Print out the
accompanying style guide for more detailed  
explanations.

The Instructions to 
Authors for Optics Express that can be found at the 
Web site itself (http://www.osa.org/admin/OpEx/). 

Authors should use the command 
\verb+\title{}+  to achieve the correct 
title format.  Authors should
place their names within the curly brackets of the 
\verb+\author{}+ command.  Authors should place 
their affiliation within the curly brackets of \\ 
\verb+\address{}+.


Enter the e-mail address of the corresponding author 
directly below the corresponding author's affiliation.  
Authors should place the 
e-mail address within the curly brackets of the 
\verb+\email{}+ command.
Multiple use of the \verb+\author{}+, 
\verb+\address{}+, and \verb+\email{}+ commands 
are permitted and produce clear association of authors
with their institutions.

Abstracts should be one paragraph, and limited to 100 
words. Authors should place the abstract between the 
\verb+\begin{abstract}+ and \verb+\end{abstract}+ 
commands to achieve the correct format.  The abstract 
commands adds the word {\bf Abstract:} to the front and the 
words, {\bf \copyright 1998 Optical Society of America}  
at the end of the abstract.

Optics Classification and Indexing Scheme (OCIS) codes 
should be included at the end of the abstract, on a 
separate line that begins with the words {\bf OCIS codes:} 
in bold. OCIS codes should be plain text.  Use 8pt type 
for this line.  Authors should
place the OCIS codes within the curly brackets of the 
\verb+\ocis{}+ command to achieve the correct 
format.

References should appear at the top of the article and 
below the abstract, in the order in which they are 
referenced in the body text. The standard LaTeX and REVTeX
commands for formatting references can be made to produce 
acceptable references for Optics Express articles, and 
may be used in place of the commands listed below.

The commands, \verb+\begin{OEReferences}+ 
and \verb+\end{OEReferences}+ format the {\bf References} 
section according to OpEx standard style, showing the
title {\bf References}.  Each reference will
be numbered consecutively.  Use the \verb+\item+ command
to start each reference. 

Optics Express uses numerals in brackets[2] or 
superscripts for bibliographic citations.  When using the 
\verb+\ref{}+ command, labels must be placed in the references:
\begin{verbatim}
   \item\label{paper2}  C. van Trigt, \josaa {\bf 14}, 741-755 (1997).
\end{verbatim}
Then the citation command should be of the form: 
\verb+[\ref{paper2}}]+.


To assist authors with journal abbreviations in references, 
standard abbreviations for 31 commonly cited journals 
have been included as macros within opex.sty.  The abbreviations 
for a few OSA, APS, and AIP journals are shown in Table 1.


\begin{table}
\center{Table 1. Standard abbreviations for 14 commonly cited journals}
\begin{center}
\begin{tabular}{lp{1.7in}|lp{1.7in}}\hline 
Macro & Abbreviation & Macro & Abbreviation \\ \hline 
\verb+\ao+ & Appl.\  Opt.\  &
   \verb+\prl+ & Phys.\ Rev.\ Lett.\ \\
\verb+\josa+ & J.\ Opt.\ Soc.\ Am.\  &
   \verb+\pra+ & Phys.\ Rev.\ A   \\
\verb+\josaa+ & J.\ Opt.\ Soc.\ Am.\ A  &
   \verb+\prb+ & Phys.\ Rev.\ B   \\
\verb+\josab+ & J.\ Opt.\ Soc.\ Am.\ B &
   \verb+\prc+ & Phys.\ Rev.\ C   \\
\verb+\ol+ & Opt.\ Lett.\   &
   \verb+\prd+ & Phys.\ Rev.\ D   \\
\verb+\opex+ & Opt.\ Express   &
    \verb+\pre+ & Phys.\ Rev.\ E     \\   
\verb+\apl+ & Appl.\ Phys.\ Lett.\  &
    \verb+\rmp+ & Rev.\ Mod.\ Phys.\  \\
 \hline
\end{tabular}
\end{center}
\end{table}


Author names, affiliations, e-mail addresses, and 
references may be made into active World Wide Web (WWW)
links. Use of the LaTeX packages HyperRef and 
color.sty is encouraged, but not required.  See each
package's documentation.  Linked text should be colored
blue, if possible.  Underlined text to indicate links
may be used in place of color.  Links in the References
section should show the URL after the reference. 
Include any special instructions concerning links in the 
cover message accompanying submission.

Equation numbers should 
appear at the right hand margin, in parentheses. 
Summations and integrals that appear 
within text such as  $\frac{1}{2}{\Sigma } 
 _{n=1}^{n=\infty} (n^2 - 2n)^{-1}$ 
should have limits placed to the 
right of the symbol to reduce white space.
\begin{equation}
H = \frac{1}{2m}(p_x^2 + p_y^2) + \frac{1}{2} M{\Omega}^2
     (x^2 + y^2) + \omega (xp_y - yp_x)
\end{equation}

Place figures as close as 
possible to where they are mentioned in the text. 
Authors should use PostScript or EPS figures. Place 
the file name of the figure within the 
curly brackets of the 
\verb+\epsfbox{}+ command. To resize figures, 
\verb+\epsfxsize=+  may be used before \verb+\epsfbox{}+.
Short figure captions should be centered.
Use the caption command with long figure captions.

\begin{figure}
%  \pagebreak \epsfbox{filename.ps}
\epsfxsize=4.2in \epsfbox{OpExF1.EPS}
\begin{center} 
{\footnotesize Fig. 1. Sample figure.}
\end{center}
%\caption{Use the caption command with figure captions 
%longer than 1 line.}
\end{figure}

After proofreading the compuscript, authors may print  
PostScript to a file using BaKoMa fonts and submit the 
PostScript file to the journal.  Authors who prefer to
submit LaTeX files must use tar and gzip compression
to combine the manuscript and figures.  Video should
always be submitted as separate files. Bibtex may not 
be used, except as described in the OSA bibtex 
instructions accompanying REVTeX 3.1.


Enter the requested 
information into the OpEx online submission system at 
http://www.osa.org, and upload the file.  For 
instructions, please see the OpEx system Help.
\end{document}
