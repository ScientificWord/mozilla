% This file (sw2bjour.tex) contains commands for adapting SW3.x syntax to the
% syntax in the Academic Press Journal Style B (bjour.cls) Document Class

% Filename: sw2bjour.tex
% Usage:    % This file (sw2bjour.tex) contains commands for adapting SW3.x syntax to the
% syntax in the Academic Press Journal Style B (bjour.cls) Document Class

% Filename: sw2bjour.tex
% Usage:    % This file (sw2bjour.tex) contains commands for adapting SW3.x syntax to the
% syntax in the Academic Press Journal Style B (bjour.cls) Document Class

% Filename: sw2bjour.tex
% Usage:    % This file (sw2bjour.tex) contains commands for adapting SW3.x syntax to the
% syntax in the Academic Press Journal Style B (bjour.cls) Document Class

% Filename: sw2bjour.tex
% Usage:    \input{sw2bjour.tex} BEFORE \input{tcilatex} in document preamble
%           For use with documents using bjour.cls; store with *.cls files.
% Author:   Melissa Harrison
% Correspondence to: styles@mackichan.com
% Last Rev: 15 May 2000 mlh
% Correspondence to: styles@mackichan.com
% Updates:  See the Free Updates section of the Technical Support area at 
%           www.mackichan.com for updates of this and other files. 


% Command Synonyms
% ----------------

% author and address are renamed to override SW standard usage for these 
% environments which is inconsistent with Academic Press usage.

\let\APauthor=\author
\let\xemail=\email

% \keywords, \draft, \and, and \qed are renamed to avoid conflicts 
% and/or bugs with SW.
\let\xkeywords=\keywords
\let\xdraft=\draft
\let\xand=\and
\let\QED=\qed

% Syntax Changes
% --------------

% SW uses \QTR{tagname}{text} for tagged runs. This line changes that syntax
% to more standard tex syntax, namely: \tagname{text}. Similarly for the \QTP 
% syntax used for structure tags.

\def\QTP#1{\csname #1\endcsname}
\def\QTSN#1{\csname #1\endcsname}


% unnumbered section definitions
% ------------------------------

\def\unsection#1\par{\section*{#1}}
\def\unsubsection#1\par{\subsection*{#1}}
\def\unsubsubsection#1\par{\subsubsection*{#1}}


% zero section definition
% -----------------------

\def\zerosect#1\par{\zerosection{#1}}


% change mathbf to boldmath
% -------------------------

%\renewcommand{\mathbf}[1]{\mbox{\boldmath$#1$\unboldmath}}


% demo environment
% ----------------

\def\xdemo{\futurelet\next\lookfortype}
\def\lookfortype{\ifx\next[\long\def\go[##1]##2\end##3{\endgroup\demo{##1}{##2}}
\else\long\def\go##1\end##2{\endgroup\demo{demo}{##1}}\fi\go}

% proclaim environment
% --------------------

\def\xproclaim{\futurelet\next\lookforptype}
\def\lookforptype{\ifx\next[\long\def\go[##1]##2\end##3{\endgroup\proclaim{##1}{##2}}
\else\long\def\go##1\end##2{\endgroup\proclaim{proclaim}{##1}}\fi\go}

% hyperlinks
% ----------

% remove the \ref{#4} to prevent ?? in typeset hyperlinks.
\def\hyperref#1#2#3#4{#2#3}



 BEFORE \input{tcilatex} in document preamble
%           For use with documents using bjour.cls; store with *.cls files.
% Author:   Melissa Harrison
% Correspondence to: styles@mackichan.com
% Last Rev: 15 May 2000 mlh
% Correspondence to: styles@mackichan.com
% Updates:  See the Free Updates section of the Technical Support area at 
%           www.mackichan.com for updates of this and other files. 


% Command Synonyms
% ----------------

% author and address are renamed to override SW standard usage for these 
% environments which is inconsistent with Academic Press usage.

\let\APauthor=\author
\let\xemail=\email

% \keywords, \draft, \and, and \qed are renamed to avoid conflicts 
% and/or bugs with SW.
\let\xkeywords=\keywords
\let\xdraft=\draft
\let\xand=\and
\let\QED=\qed

% Syntax Changes
% --------------

% SW uses \QTR{tagname}{text} for tagged runs. This line changes that syntax
% to more standard tex syntax, namely: \tagname{text}. Similarly for the \QTP 
% syntax used for structure tags.

\def\QTP#1{\csname #1\endcsname}
\def\QTSN#1{\csname #1\endcsname}


% unnumbered section definitions
% ------------------------------

\def\unsection#1\par{\section*{#1}}
\def\unsubsection#1\par{\subsection*{#1}}
\def\unsubsubsection#1\par{\subsubsection*{#1}}


% zero section definition
% -----------------------

\def\zerosect#1\par{\zerosection{#1}}


% change mathbf to boldmath
% -------------------------

%\renewcommand{\mathbf}[1]{\mbox{\boldmath$#1$\unboldmath}}


% demo environment
% ----------------

\def\xdemo{\futurelet\next\lookfortype}
\def\lookfortype{\ifx\next[\long\def\go[##1]##2\end##3{\endgroup\demo{##1}{##2}}
\else\long\def\go##1\end##2{\endgroup\demo{demo}{##1}}\fi\go}

% proclaim environment
% --------------------

\def\xproclaim{\futurelet\next\lookforptype}
\def\lookforptype{\ifx\next[\long\def\go[##1]##2\end##3{\endgroup\proclaim{##1}{##2}}
\else\long\def\go##1\end##2{\endgroup\proclaim{proclaim}{##1}}\fi\go}

% hyperlinks
% ----------

% remove the \ref{#4} to prevent ?? in typeset hyperlinks.
\def\hyperref#1#2#3#4{#2#3}



 BEFORE \input{tcilatex} in document preamble
%           For use with documents using bjour.cls; store with *.cls files.
% Author:   Melissa Harrison
% Correspondence to: styles@mackichan.com
% Last Rev: 15 May 2000 mlh
% Correspondence to: styles@mackichan.com
% Updates:  See the Free Updates section of the Technical Support area at 
%           www.mackichan.com for updates of this and other files. 


% Command Synonyms
% ----------------

% author and address are renamed to override SW standard usage for these 
% environments which is inconsistent with Academic Press usage.

\let\APauthor=\author
\let\xemail=\email

% \keywords, \draft, \and, and \qed are renamed to avoid conflicts 
% and/or bugs with SW.
\let\xkeywords=\keywords
\let\xdraft=\draft
\let\xand=\and
\let\QED=\qed

% Syntax Changes
% --------------

% SW uses \QTR{tagname}{text} for tagged runs. This line changes that syntax
% to more standard tex syntax, namely: \tagname{text}. Similarly for the \QTP 
% syntax used for structure tags.

\def\QTP#1{\csname #1\endcsname}
\def\QTSN#1{\csname #1\endcsname}


% unnumbered section definitions
% ------------------------------

\def\unsection#1\par{\section*{#1}}
\def\unsubsection#1\par{\subsection*{#1}}
\def\unsubsubsection#1\par{\subsubsection*{#1}}


% zero section definition
% -----------------------

\def\zerosect#1\par{\zerosection{#1}}


% change mathbf to boldmath
% -------------------------

%\renewcommand{\mathbf}[1]{\mbox{\boldmath$#1$\unboldmath}}


% demo environment
% ----------------

\def\xdemo{\futurelet\next\lookfortype}
\def\lookfortype{\ifx\next[\long\def\go[##1]##2\end##3{\endgroup\demo{##1}{##2}}
\else\long\def\go##1\end##2{\endgroup\demo{demo}{##1}}\fi\go}

% proclaim environment
% --------------------

\def\xproclaim{\futurelet\next\lookforptype}
\def\lookforptype{\ifx\next[\long\def\go[##1]##2\end##3{\endgroup\proclaim{##1}{##2}}
\else\long\def\go##1\end##2{\endgroup\proclaim{proclaim}{##1}}\fi\go}

% hyperlinks
% ----------

% remove the \ref{#4} to prevent ?? in typeset hyperlinks.
\def\hyperref#1#2#3#4{#2#3}



 BEFORE \input{tcilatex} in document preamble
%           For use with documents using bjour.cls; store with *.cls files.
% Author:   Melissa Harrison
% Correspondence to: styles@mackichan.com
% Last Rev: 15 May 2000 mlh
% Correspondence to: styles@mackichan.com
% Updates:  See the Free Updates section of the Technical Support area at 
%           www.mackichan.com for updates of this and other files. 


% Command Synonyms
% ----------------

% author and address are renamed to override SW standard usage for these 
% environments which is inconsistent with Academic Press usage.

\let\APauthor=\author
\let\xemail=\email

% \keywords, \draft, \and, and \qed are renamed to avoid conflicts 
% and/or bugs with SW.
\let\xkeywords=\keywords
\let\xdraft=\draft
\let\xand=\and
\let\QED=\qed

% Syntax Changes
% --------------

% SW uses \QTR{tagname}{text} for tagged runs. This line changes that syntax
% to more standard tex syntax, namely: \tagname{text}. Similarly for the \QTP 
% syntax used for structure tags.

\def\QTP#1{\csname #1\endcsname}
\def\QTSN#1{\csname #1\endcsname}


% unnumbered section definitions
% ------------------------------

\def\unsection#1\par{\section*{#1}}
\def\unsubsection#1\par{\subsection*{#1}}
\def\unsubsubsection#1\par{\subsubsection*{#1}}


% zero section definition
% -----------------------

\def\zerosect#1\par{\zerosection{#1}}


% change mathbf to boldmath
% -------------------------

%\renewcommand{\mathbf}[1]{\mbox{\boldmath$#1$\unboldmath}}


% demo environment
% ----------------

\def\xdemo{\futurelet\next\lookfortype}
\def\lookfortype{\ifx\next[\long\def\go[##1]##2\end##3{\endgroup\demo{##1}{##2}}
\else\long\def\go##1\end##2{\endgroup\demo{demo}{##1}}\fi\go}

% proclaim environment
% --------------------

\def\xproclaim{\futurelet\next\lookforptype}
\def\lookforptype{\ifx\next[\long\def\go[##1]##2\end##3{\endgroup\proclaim{##1}{##2}}
\else\long\def\go##1\end##2{\endgroup\proclaim{proclaim}{##1}}\fi\go}

% hyperlinks
% ----------

% remove the \ref{#4} to prevent ?? in typeset hyperlinks.
\def\hyperref#1#2#3#4{#2#3}



