%% 
\def\fileversion{v2.2f}
\def\filedate{1994/06/21}

\DeclareFontFamily{T1}{ntr}{}
\DeclareFontShape{T1}{ntr}{m}{n}
   {  <5> <6> <7> <8> <9> <10> <12> 
      <10.95> 
      <14.4>  
      <17.28><20.74><24.88> times}{}
\DeclareFontShape{T1}{ntr}{m}{sl}
    {
      <5><6><7><8><9>
      <10><10.95><12><14.4><17.28><20.74><24.88> timesi%
      }{}
\DeclareFontShape{T1}{ntr}{m}{it}
    {
      <5><6><7><8><9>
      <10><10.95>
      <12><14.4><17.28><20.74><24.88>timesi%
      }{}
\DeclareFontShape{T1}{ntr}{m}{sc}
    {
      <5><6><7><8><9><10><10.95><12>
      <14.4><17.28><20.74><24.88> cmcsc10
      }{}
% Warning: please note that the upright shape below is
%          used for the \pounds symbol of LaTeX. So this
%          font definition shouldn't be removed.
%
\DeclareFontShape{T1}{cmr}{m}{ui}
   {
      <5><6><7><8><9><10><10.95><12>%
      <14.4><17.28><20.74><24.88>cmu10%
      }{}
%%%%%%% bold series
\DeclareFontShape{T1}{ntr}{bx}{n}
     {
      <5><6><7><8><9><10><10.95><12>
      <14.4><17.28><20.74><24.88> timesbd}{}
%%%%%%%% bold extended series
\DeclareFontShape{T1}{cmr}{bx}{n}
   {
      <5> <6> <7> <8> <9> gen * cmbx
      <10><10.95> cmbx10
      <12><14.4><17.28><20.74><24.88>cmbx12
      }{}
\DeclareFontShape{T1}{cmr}{bx}{sl}
      {
      <5> <6> <7> <8> <9>
      <10> <10.95> <12> <14.4> <17.28> <20.74> <24.88> cmbxsl10
      }{}
\DeclareFontShape{T1}{cmr}{bx}{it}
      {
      <5> <6>  <7>  <8> <9>
      <10> <10.95> <12> <14.4> <17.28> <20.74> <24.88> cmbxti10
      }{}
% Again this is necessary for a correct \pounds symbol in
% the cmr fonts Hopefully the dc/ec font layout will take
% over soon.
%
\DeclareFontShape{T1}{cmr}{bx}{ui}
      {<->ssub * cmr/m/ui}{}





\DeclareFontFamily{T1}{ari}{}
\DeclareFontShape{T1}{ari}{m}{n}
   {  <5> <6> <7> <8> <9> <10> <12> 
      <10.95> 
      <14.4>  
      <17.28><20.74><24.88> arial}{}
\DeclareFontShape{T1}{ari}{m}{sl}
    {
      <5><6><7><8><9>
      <10><10.95><12><14.4><17.28><20.74><24.88> ariali%
      }{}
\DeclareFontShape{T1}{ari}{m}{it}
    {
      <5><6><7><8><9>
      <10><10.95>
      <12><14.4><17.28><20.74><24.88>ariali%
      }{}
\DeclareFontShape{T1}{ntr}{m}{sc}
    {
      <5><6><7><8><9><10><10.95><12>
      <14.4><17.28><20.74><24.88> cmcsc10
      }{}
\DeclareFontShape{T1}{ari}{bx}{n}
   {  <5> <6> <7> <8> <9> <10> <12> 
      <10.95> 
      <14.4>  
      <17.28><20.74><24.88> arialbd}{}
\DeclareFontShape{T1}{ari}{bx}{sl}
    {
      <5><6><7><8><9>
      <10><10.95><12><14.4><17.28><20.74><24.88> arialbi%
      }{}
\DeclareFontShape{T1}{ari}{bx}{it}
    {
      <5><6><7><8><9>
      <10><10.95>
      <12><14.4><17.28><20.74><24.88>arialbi%
      }{}
\DeclareFontShape{T1}{ntr}{bx}{sc}
    {
      <5><6><7><8><9><10><10.95><12>
      <14.4><17.28><20.74><24.88> cmcsc10
      }{}



\DeclareFontFamily{T1}{cou}{}
\DeclareFontShape{T1}{cou}{m}{n}
   {  <5> <6> <7> <8> <9> <10> <12> 
      <10.95> 
      <14.4>  
      <17.28><20.74><24.88> cour}{}
\DeclareFontShape{T1}{cou}{m}{sl}
    {
      <5><6><7><8><9>
      <10><10.95><12><14.4><17.28><20.74><24.88> couri%
      }{}
\DeclareFontShape{T1}{cou}{m}{it}
    {
      <5><6><7><8><9>
      <10><10.95>
      <12><14.4><17.28><20.74><24.88>couri%
      }{}
\DeclareFontShape{T1}{cou}{m}{sc}
    {
      <5><6><7><8><9><10><10.95><12>
      <14.4><17.28><20.74><24.88> cmcsc10
      }{}
\DeclareFontShape{T1}{cou}{bx}{n}
   {  <5> <6> <7> <8> <9> <10> <12> 
      <10.95> 
      <14.4>  
      <17.28><20.74><24.88> courbd}{}
\DeclareFontShape{T1}{cou}{bx}{sl}
    {
      <5><6><7><8><9>
      <10><10.95><12><14.4><17.28><20.74><24.88> courbi%
      }{}
\DeclareFontShape{T1}{cou}{bx}{it}
    {
      <5><6><7><8><9>
      <10><10.95>
      <12><14.4><17.28><20.74><24.88>courbi%
      }{}
\DeclareFontShape{T1}{cou}{bx}{sc}
    {
      <5><6><7><8><9><10><10.95><12>
      <14.4><17.28><20.74><24.88> cmcsc10
      }{}

\renewcommand{\familydefault}{ntr}
\renewcommand{\rmdefault}{ntr}
\renewcommand{\sfdefault}{ari}
\renewcommand{\ttdefault}{cou}


\endinput
%% 
